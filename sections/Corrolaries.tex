
\subsection{Extending To Product of Blowups}
%%% %%% %%% %%% %%% %%% %%% %%% %%% %%% %%% %%% %%% %%% %%% %%%
%       Behavior of Ramification under Product of Blowups
%%% %%% %%% %%% %%% %%% %%% %%% %%% %%% %%% %%% %%% %%% %%% %%%
\subsubsection{Behavior of Ramification under Product of Blowups}\label{subsection:ram_prod_analysis}
Let $X$ and $Y$ be smooth schemes over a field $k$, and let $x \in X$ and $y \in Y$ be closed points. 
We denote the blowups of these schemes at the given points by 
$\pi_X: \Bl_x(X) \to X$ and $\pi_Y: \Bl_y(Y) \to Y$. Furthermore, let $\pi_{X \times Y}: \Bl_{(x,y)}(X \times_k Y) \to X \times_k Y$ 
be the blowup of the product scheme at the point $(x,y)$.
We denote by $E_X, E_Y$, and $E_{X \times Y}$ the respective exceptional divisors, and let $\eta_X, \eta_Y$, and $\eta_{X \times Y}$ be their generic points.

In this section, we establish the following result concerning the stability of ramification bounds under the external product of torsors.

\begin{prop}\label{prop:ramification-external-product}
Let $G$ be a finite abelian group. Suppose $\cG_X$ and $\cG_Y$ are $G$-torsors defined on open subsets $U_X \subset \Bl_x(X)$ and $U_Y \subset \Bl_y(Y)$ that are disjoint 
from the exceptional divisors. 
If the ramification of $\cG_X$ at $\eta_X$ and $\cG_Y$ at $\eta_Y$ is bounded by $r$, then the external product torsors 
\[
\cG_{X \times Y} := \text{pr}_1^{-1} \cG_X \otimes \text{pr}_2^{-1} \cG_Y
\]
has ramification bounded by $r$ at the generic point $\eta_{X\times Y}$ of the exceptional divisor in the product blowup.
\end{prop}

The proposition is purely local in nature, it suffices to consider the case where $X$ and $Y$ are affine. 
More precisely, by the smoothness of $X$ and $Y$, we may restrict our attention to open neighborhoods of $x$ and $y$ that are isomorphic to affine spaces. 
The rest of this section treats that case.

\subsubsection*{The Affine Case}
Let $X = \mathbb{A}_k^n$ be the affine $n$-space over a field $k$, and let $0 \in X$ be the origin. 
Let $\tilde{X} = \text{Bl}_0(X)$ be the blowup of $X$ at the origin. 
Recall that $\tilde{X} \subset X \times_k \mathbb{P}^{n-1}_k$ is defined by the equations $x_i u_j = x_j u_i$, where $[u_1 : \dots : u_n]$ are the homogeneous coordinates of 
$\mathbb{P}^{n-1}_k$.
The exceptional divisor $E \subset \tilde{X}$ is the fiber over the origin, $E = \{ (0, [u_1 : \dots : u_n]) \}$, which is of codimension 1 in $\tilde{X}$. 
Let $\eta \in E$ be the generic point of $E$, and let $R = \mathcal{O}_{\tilde{X}, \eta}$ be the associated local ring. 
This ring $R$ is a discrete valuation ring (DVR) with fraction field $K = K(X) = k(x_1, \dots, x_n)$.

On the affine chart $U_1$ where $u_1 \neq 0$, we have $x_i = \frac{u_i}{u_1} x_1$. 
The coordinate ring is:$$\mathcal{O}_{\tilde{X}}(U_1) = k\left[x_1, \frac{u_2}{u_1}, \dots, \frac{u_n}{u_1}\right]$$
In this chart, the generic point $\eta$ corresponds to the prime ideal $\mathfrak{p}_1 = (x_1)$. Thus, the local ring is $R = k[x_1, \frac{u_2}{u_1}, \dots, \frac{u_n}{u_1}]_{(x_1)}$. 
The residue field is $\kappa(\eta) = k(\frac{u_2}{u_1}, \dots, \frac{u_n}{u_1})$, and the completion of $R$ with respect to its maximal ideal is:
$$\hat{R} = \kappa(\eta)[[x_1]]$$
In this local ring, $x_1$ is a uniformizer. Note that any $x_i$ (for $i > 1$) can also serve as a uniformizer, 
as $x_i = (\frac{u_i}{u_1}) x_1$ and $\frac{u_i}{u_1}$ is a unit in $R$.

For any monomial $M = x_1^{a_1} \dots x_n^{a_n} \in K$, we can write:
$$M = x_1^{\sum a_i} \left( \frac{u_2}{u_1} \right)^{a_2} \dots \left( \frac{u_n}{u_1} \right)^{a_n}$$
Since the term in parentheses is a unit in $R$, the valuation $\nu_E$ associated with $E$ satisfies:
$$\nu_E(M) = \sum a_i = \deg M$$
Consequently, for any polynomial $f = f_d + f_{d+1} + \dots + f_l$, where $f_i$ is the homogeneous part of degree $i$, we have $\nu_E(f) = d$ (the order of vanishing at the origin).

\subsubsection*{The Product Case}
Now, let $X = \mathbb{A}^n$ and $Y = \mathbb{A}^m$ with origins $x=0$ and $y=0$. 
As before, $\text{Bl}_0(X) \subset X \times \mathbb{P}^{n-1}$ and $\text{Bl}_0(Y) \subset Y \times \mathbb{P}^{m-1}$ have exceptional divisors $E_X$ and $E_Y$ respectively.
Consider the product $X \times_k Y \cong \mathbb{A}_k^{n+m}$. 
The blowup of the product at the origin $(0,0)$, denoted $\text{Bl}_{(0,0)}(X \times_k Y)$, is a subscheme of $(X \times Y) \times \mathbb{P}^{n+m-1}$ defined by:

$$\begin{cases} 
x_i w_j = x_j w_i & 1 \le i, j \le n \\
y_k w_{n+l} = y_l w_{n+k} & 1 \le k, l \le m \\
x_i w_{n+j} = y_j w_i & 1 \le i \le n, \,\, 1 \le j \le m 
\end{cases}$$
where $[w_1 : \dots : w_{n+m}]$ are the homogeneous coordinates of $\mathbb{P}^{n+m-1}$. The exceptional divisor $E_{X \times Y}$ is isomorphic to $\mathbb{P}^{n+m-1}$.

\subsubsection*{Comparison of Blowups}

Both $\text{Bl}_0(X) \times_k \text{Bl}_0(Y)$ and $\text{Bl}_{(0,0)}(X \times_k Y)$ are birational to $X \times_k Y$. 
They share a common dense open set $\tilde{U}$ defined by the condition that neither the $X$-coordinates nor the $Y$-coordinates vanish simultaneously in the projective space:
$$\tilde{U} = \{ ((x,y), [w_1: \dots : w_{n+m}]) \mid (w_1, \dots, w_n) \neq 0 \text{ and } (w_{n+1}, \dots, w_{n+m}) \neq 0 \}$$This yields a diagram of open immersions:
$$ 
   \begin{tikzcd}
        & \tilde{U} \arrow[dl, "f_1"'] \arrow[dr, "f_2", hook] & \\
        \text{Bl}_0(X) \times_k \text{Bl}_0(Y) & & \text{Bl}_{(0,0)}(X \times_k Y)
    \end{tikzcd}
$$
    
    where $f_1$ maps the coordinates to the respective projectivizations $[w_1: \dots : w_n]$ and $[w_{n+1}: \dots : w_{n+m}]$.
    Also note that the generic point $\eta_{X \times Y}$  of $E_{X \times_k Y}$ is inside $\tilde{U}$

\subsubsection*{Extensions of DVRs}
Let $S$ be the local ring of the generic point $\eta_{X\times Y}$ of $E_{X \times Y}$ in $\tilde{U}$. 
On the chart where $w_1 \neq 0$ and $w_{n+1} \neq 0$, we have 
$x_1 = (\frac{w_1}{w_{n+1}}) y_1$. 
Since $\frac{w_1}{w_{n+1}}$ is a unit in this chart, $x_1$ and $y_1$ are equivalent as uniformizers.
We have:
$$
S=k\left[x_1, \frac{w_2}{w_1} \cdots, \frac{w_n}{w_1}, \frac{w_{n+1}}{w_1} \cdots \frac{w_{n+m}}{w_1}\right]_{(x_1)} = 
k\left[\frac{w_1}{w_{n+1}}, \frac{w_2}{w_{n+1}} \cdots \frac{w_n}{w_{n+1}}, y_1 \cdots \frac{w_{n+m}}{w_{n+1}}\right]_{(y_1)}
$$

$$k(\eta_{X \times Y}) = k\left(\frac{w_2}{w_1} \cdots, \frac{w_n}{w_1}, \frac{w_{n+1}}{w_1} \cdots \frac{w_{n+m}}{w_1}\right)$$

Let $R_X$ be the local ring of the exceptional divisor $E_X$ in $\text{Bl}_0(X)$. 
The pullback of $E_X \times_k \text{Bl}_0(Y)$ along $f_1$ induces an extension of DVRs $R_X \hookrightarrow S$. Which is:
\begin{enumerate}
    \item Weakly Unramified: $x_1$ is a uniformizer in both $R_X$ and $S$, so the ramification index is $e=1$.
    \item Residually Transcendental: The residue field extension $\kappa(\eta_X) \subset \kappa(\eta)$ is:
    $$k\left(\frac{w_2}{w_1}, \dots, \frac{w_n}{w_1}\right) \subset k\left(\frac{w_2}{w_1}, \dots, \frac{w_n}{w_1}, \frac{w_{n+1}}{w_1}, \dots, \frac{w_{n+m}}{w_1}\right)$$
    Hence separable.
\end{enumerate}
Since this extension is generated by transcendental elements, it is separable and formally smooth at the maximal ideal (\stackstag{09E7}).

\subsubsection*{Ramification of $G$-Torsors}
Let $G$ be a finite abelian group. 
Let $\mathcal{Q}$ be a $G$-torsor on an open $U \subset \text{Bl}_0(X)$ disjoint from $E_X$. 
%Suppose $\mathcal{Q}$ has ramification bounded by $r$ at the generic point $\eta_X$ of $E_X$.
Let $V \subset \text{Bl}_0(Y)$ be an open subscheme, and let $\pi_X: \text{Bl}_0(X) \times_k V \to \text{Bl}_0(X)$ 
be the projection onto the first factor. 
By restricting this projection to $U \times_k V$, we obtain the pullback $G$-torsor:
$$\pi_X^{-1}(\mathcal{Q}) \cong \mathcal{Q} \times_k V$$
which is defined on the open subset $U \times_k V \subset \text{Bl}_0(X) \times_k \text{Bl}_0(Y)$.
The extension of local rings $R_X \to S$ is weakly unramified (the ramification index $e=1$) and residually transcendental with seprable residue field exntesion. 
Under these conditions the ramification filtration is preserved. 
Therefore, the pullback $\pi_X^{-1}(\mathcal{Q})$ has ramification bounded by $r$ at the generic point $\eta_{X \times Y}$ of the exceptional divisor $E_{X \times Y}$ if and only if 
the original torsor $\mathcal{Q}$ has ramification bounded by $r$ at the generic point $\eta_X$ of $E_X$

And we finish by \Cref{lemma:bounding_ramification_of_contracted_product}.



