\textcolor{red}{do we assume here $S=k$?}
The main theorem of this section is
\begin{restatable}{theorem}{SymmetricPowerOfSheafIsTamelyRamified}\label{theorem:SymmetricPowerOfSheafIsTamelyRamified}
     Let $\Lambda$ be a finite ring of cardinality invertible in $k$, and let $\cF$ be an \'etale sheaf of $\Lambda$-modules, locally free of rank 1 on $U$, with ramification bounded by $\mdls$. 
     Considering $U^{(d)}$ as an open subscheme of the blowup $\tilde{C}^{(d)}_\mdls$ of $C^{(d)}$, we have that
     for sufficiently large integer $d$, $\cF^{(d)}$ is tamely ramified on $H= \tilde{C}^{(d)}_\mdls \setminus U^{(d)}=E_0 \times_{C^{(d)}} \tilde{C}^{(d)}_\mdls $.
\end{restatable}

Following the notation of \Cref{section:BlowupOfSymmetricPowerOfCurves},
For any modulus $\ndls \subset \mdls$, we define $Z_\ndls$ as the closed subscheme of $C^{(\deg \ndls)}$ 
defined by $\ndls$ as a point of $C^{(\deg \ndls)}$ .

We then define $X_\ndls$ as the blowup of $C^{(\deg \ndls)}$ at $Z_\ndls$,
and we denote by $E_\ndls = Z_\ndls \times_{C^{(d)}} X_\ndls$ the exceptional divisor of this blowup, it is irreducible of codimension 1.
We denote by $\eta_\ndls$ the generic point of $E_\ndls$. 
Diagrammatically:
\[\begin{tikzcd}
\overline{\{\eta_\ndls \}} =  E_\ndls \arrow[r, hook] \arrow[d] & X_\ndls \arrow[d, "\pi_\ndls"] \\
Z_\ndls \arrow[r, hook]           & C^{(\deg \ndls)}          
\end{tikzcd}\] 


% Recall the defintions from \Cref{section:BlowupOfSymmetricPowerOfCurves},
% $\ndls \subset \mdls$ a submodulus, $Z_\ndls$ is the closed subscheme of 
% $C^{(\deg \ndls)}$ defined by $\ndls$ as a point.
% $X_\ndls$ as the blowup of $C^{(\deg \ndls)}$ at 
% $Z_\ndls$, and $E_\ndls = Z_\ndls \times_{C^{(d)}} X_\ndls$ is exceptional divisor of this blowup with 
% $\eta_\ndls$ its generic point.

\Cref{theorem:SymmetricPowerOfSheafIsTamelyRamified} easily follows from: 
\begin{theorem}\label{theorem:SymmetricPowerOfSheavesIsTamelyRamifiedReduction}
     Let $\cF$ be a local system on $U$ with ramification at $P$ bounded by $\ndls = k_P P \subset \mdls$.
     Then $\cF^{(\deg \ndls)}$ is tamely ramified at $\eta_\ndls$ of $E_\ndls$
\end{theorem}

In the upcoming section, we perform the reduction and derive \Cref{theorem:SymmetricPowerOfSheafIsTamelyRamified} from 
\Cref{theorem:SymmetricPowerOfSheavesIsTamelyRamifiedReduction}. 
We then prove \Cref{theorem:SymmetricPowerOfSheavesIsTamelyRamifiedReduction} in the section that follows.

\subsection{Reduction Lemmas}

The first lemma is from \cite{takeuchi2019blow}, we include their proof for the convenience of the reader.
\begin{lemma}[\cite{takeuchi2019blow}, Lemma 4.1]
     Let $C$ be a projective smooth geometrically connected curve over a perfect field $k$. 
     Let $\mdls = \sum_{i=1}^{r} k_i P_i$ where $P_1, ..., P_r$ are disctinct closed points of $\mdls$. Let $U$ be the comlement of $\mdls$ in $C$.
     And let $d_i= \deg Pi$. Take $d \geq \mdls$. Then:
     The morphism $\pi : C^{(n_1 d_1)} \times_k \dots \times_k C^{(n_r d_r)} \times_k C^{(d - \deg \mathfrak{m})} \to C^{(d)}$, taking the sum, is étale at the generic point of the closed subvariety $\{n_1 P_1\} \times \dots \times \{n_r P_r\} \times C^{(d - \deg \mathfrak{m})}$ of $C^{(n_1 d_1)} \times_k \dots \times_k C^{(n_r d_r)} \times_k C^{(d - \deg \mathfrak{m})}$.
\begin{proof}
     We may assume that $k$ is algebraically closed (hence $d_i = 1$ for all $i$). Since the map $\pi : C^{(n_1)} \times_k \dots \times_k C^{(n_r)} \times_k C^{(d - \deg \mathfrak{m})} \to C^{(d)}$ is finite flat, it is enough to show that there exists a closed point $Q$ of $n_1 P_1 + \dots n_r P_r + C^{(d - \deg \mathfrak{m})}$ over which there are $\deg \pi$ points on $C^{(n_1)} \times_k \dots \times_k C^{(n_r)} \times_k C^{(d - \deg \mathfrak{m})}$. Choose $Q$ as a point corresponding to a divisor
     $n_1 P_1 + \dots n_r P_r + P_{r+1} + \dots + P_{r+d - \deg \mathfrak{m}}$, where $P_1, \dots, P_{r+d - \deg \mathfrak{m}}$ are distinct points of $U(k)$.
\end{proof}     
\end{lemma}

The following lemma is adapted from \cite{takeuchi2019blow} (Lemma 4.1)
\begin{lemma}\label{lemma:takeuchi_lemma}
     Let $C$ be a projective, smooth, and geometrically connected curve over a perfect field $k$. 
     Let $\mathfrak{m} = \sum_{i=1}^{r} k_i P_i$ be an effective divisor where $P_1, \dots, P_r$ are distinct closed points. 
     Let $U = C \setminus \mathfrak{m}$ and let $d \geq \deg \mathfrak{m}$.

     Suppose $\mathfrak{n}_1, \dots, \mathfrak{n}_l$ are pairwise coprime submoduli of $\mathfrak{m}$ such that $\mathfrak{m} = \sum_{j=1}^l \mathfrak{n}_j$. 
     Consider the summation morphism:
     \[
     \pi : C^{(\deg \mathfrak{n}_1)} \times_k \dots \times_k C^{(\deg \mathfrak{n}_l)} \times_k C^{(d - \deg \mathfrak{m})} \longrightarrow C^{(d)}
     \]
     defined by $(D_1, \dots, D_l, D_{extra}) \mapsto \sum_{j=1}^l D_j + D_{extra}$.

     Then $\pi$ is étale at the generic point of the closed subvariety 
     \[
     V = \{\mathfrak{n}_1\} \times_k \dots \times_k \{\mathfrak{n}_l\} \times_k C^{(d - \deg \mathfrak{m})}
     \]
     inside the domain $C^{(\deg \mathfrak{n}_1)} \times_k \dots \times_k C^{(\deg \mathfrak{n}_l)} \times_k C^{(d - \deg \mathfrak{m})}$.

\begin{proof}
     We may assume that $k$ is algebraically closed (hence $\deg P_i = 1$ for all $i$). By miracle flatness $\pi$ is finite flat. 
     Thus, it is enough to show that there exists a closed point $Q$ of $\ndls_1 + \dots \ndls_l + C^{(d - \deg \mathfrak{m})} \subset C^{(d)}$ 
     over which there are $\deg \pi$ points on $C^{(\ndls_1)} \times_k \dots \times_k C^{(\ndls_l)} \times_k C^{(d - \deg \mathfrak{m})}$. (Because it will be unramified 
     at this point and thus also at the generic point of $V$.)
     Choose $Q$ as a point corresponding to a divisor
     $\ndls_1 + \dots + \ndls_l + P_{r+1} + \dots + P_{r+d - \deg \mathfrak{m}}$, where $P_1, \dots, P_{r+d - \deg \mathfrak{m}}$ are distinct points of $U(k)$.
\end{proof}     
\end{lemma}


\textbf{Plan for Corollary}
\begin{enumerate}
          \item State precisly. 
          \item Make sure all the notions are well defined.
          \item Prove.
          \item State in maximum generality as in previous lemma. 
\end{enumerate}

\textcolor{red}{
\begin{cor}
    Suppose $\cF^{(\deg \mdls)}$ is tamely ramified at $\eta$.
    Then  $\Fbox{\deg \mdls}{d - \deg \mdls}$ is tamely ramified at the generic point $\theta$ of $E_{\mdls} \times_k C^{(d-\deg \mdls)} \subset C^{(\deg \mdls)} \times_k C^{d - \deg \mdls}$
    and thus $\cF^{(d)}$ is taemly ramified at the generic point ? which one.
\end{cor}
\begin{proof}
     \textcolor{red}{Compete and make precise.}
\end{proof}
}

\begin{lemma}
     Let $\ndls_1, \ndls_2 \subset \mdls$ be two moduli of the form $\ndls_1 = k_1 P_1$, $\ndls_2 = k_2 P_2$ where $P_1, P_2$ are distinct points.
     Assume $\cF ^{(\deg \ndls_1)}$, $\cF ^{(\deg \ndls_2)}$ are at most tamely ramified at $\eta_{\ndls_1}$, $\eta_{\ndls_2}$ respectively.
     Then $\cF^{(\deg \ndls_1 + \deg \ndls_2)}$ is at most tamely tamified at $\eta_{\ndls_1 + \ndls_2}$ .
\end{lemma}
\begin{proof}
     \textcolor{red}{Complete}
\end{proof}

\begin{lemma}\label{lemma:moduli_reduction}
     Let $\ndls, \ndls' \subset \mdls$ be coprime sub moduli of $\mdls$ where $\ndls' = k_P P$.
     Assume $\cF ^{(\deg \ndls)}$, $\cF ^{(\deg \ndls')}$ are at most tamely ramified at $\eta_{\ndls}$, $\eta_{\ndls'}$ respectively.
     Then $\cF^{(\deg \ndls + \deg \ndls')}$ is at most tamely tamified at $\eta_{\ndls + \ndls'}$ .
\end{lemma}

\begin{proof}
     \textcolor{red}{Complete}
\end{proof}


\subsection{\texorpdfstring{Proof of \Cref{theorem:SymmetricPowerOfSheafIsTamelyRamified}}{Proof of Theorem X}}
\begin{proof}[Proof of \Cref{theorem:SymmetricPowerOfSheafIsTamelyRamified}]
     Let $\cF$ be as in \Cref{theorem:SymmetricPowerOfSheafIsTamelyRamified}, $\mdls = \sum_{i=1}^n k_P P$ with $\deg P = d_P$  
     Then by \Cref{theorem:SymmetricPowerOfSheavesIsTamelyRamifiedReduction} for every $\ndls \subset \mdls$ of the form $\ndls = k_P P$, $\cF^{(\deg \ndls)}$ is at most tamely ramified at $\eta_\ndls$.
     By \Cref{lemma:moduli_reduction}, $\cF^{(\deg \mdls)}$ is then at most tamely ramified at $\eta_\mdls$. And thus by lemma \Cref{lemma:takeuchi_lemma} 
     (\textcolor{red}{maybe do another step here directly, in addition to the lemma?})
     $\cF^{(d)}$ is tamely ramified at the generic point of $H$.
\end{proof}


\subsection{\texorpdfstring{Proof of \Cref{theorem:SymmetricPowerOfSheavesIsTamelyRamifiedReduction}}{Proof of Theorem Y}}
In this section we prove \Cref{theorem:SymmetricPowerOfSheavesIsTamelyRamifiedReduction}
\textcolor{red}{complete - this is not finished}

We will work this out along an example:
Let $X=\bG_m=\spec R[t, t^{-1}]$ and Let $\cP=\bG_m \xrightarrow[]{(\cdot)^{n}} G_m$ be the $n$'th power map.
It is a $G=\Z/n\Z$ torsor.
The ring map is $R[t, t^{-1}] \xrightarrow[]{t \mapsto t^n} R[t, t^{-1}]$
which corresponds to ring extension: $R[t, t^{-1}] \to R[t^{\frac{1}{n}}, t^{-\frac{1}{n}}]$.
The field of fractions of $\bG_m$ is $K(t)$ and the corresponding map between fields of $\cP \to X$
is $K(t) \xrightarrow[]{t \to t^n} K(t)$. which corresponds to the field extension:
$K(t) \hookrightarrow K(t^{\frac{1}{n}}) = K(t)[X]/(X^n -t)$.
Where $K = Frac R$

The points $0, \infty \in \bP^1$ correspond to the local rings
$\Oo_{0} = K[t]_{(t)}$ and $\Oo_{\infty} = K[\frac{1}{t}]_{(\frac{1}{t})}$ of $K(t)$,
which are DVRs. The corresponding valuations of $K(t)$ are given by:
\[
    v_0\left(\frac{f}{g}\right) = \text{maximal exponent } n \text{ s.t. } t^{n} \mid \frac{f}{g}, \quad v_\infty\left(\frac{f}{g}\right) = \deg g - \deg f
\]

So the diagram of the G-torsor $\cP=\bG_m$ over $\bG_m$ is:
\begin{equation}
    \begin{tikzcd}
        & \cP=\mathbb{G}_m \ar[d, "(\cdot)^n"] \\
        \bP^1 & \arrow[l] \bG_m
    \end{tikzcd}
\end{equation}

The bounded ramification condition is given by:
\begin{equation}
    \text{ram } P_{\eta} \leq k
\end{equation}

We wish to understand


In the general case of a $G$ torsor $P \to C$ we have similarly:
$K(C) \cong K(t)$ the function ring of $C$ for some variable $t$ and a finite field extension $K/\F_p(t')$.
If the basefield $K$ contains $n$'th roots of unity, then the torsor is the same... and continue here:
$\Oo_{P_1}$
the same..
and continue.

