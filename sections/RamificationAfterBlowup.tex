\textcolor{red}{do we assume here $S=k$?}
The main theorem of this section is
\begin{restatable}{theorem}{SymmetricPowerOfSheafIsTamelyRamified}\label{theorem:SymmetricPowerOfSheafIsTamelyRamified}
     Let $\Lambda$ be a finite ring of cardinality invertible in $k$, and let $\cF$ be an \'etale sheaf of $\Lambda$-modules, locally free of rank 1 on $U$, with ramification bounded by $\mdls$. 
     Considering $U^{(d)}$ as an open subscheme of the blowup $\tilde{C}^{(d)}_\mdls$ of $C^{(d)}$, we have that
     for sufficiently large integer $d$, $\cF^{(d)}$ is tamely ramified on $H= \tilde{C}^{(d)}_\mdls \setminus U^{(d)}=E_0 \times_{C^{(d)}} \tilde{C}^{(d)}_\mdls $.
\end{restatable}

Following the notation of \Cref{section:BlowupOfSymmetricPowerOfCurves},
For any modulus $\ndls \subset \mdls$, we define $Z_\ndls$ as the closed subscheme of $C^{(\deg \ndls)}$ 
defined by $\ndls$ as a point of $C^{(\deg \ndls)}$ .

We then define $X_\ndls$ as the blowup of $C^{(\deg \ndls)}$ at $Z_\ndls$,
and we denote by $E_\ndls = Z_\ndls \times_{C^{(d)}} X_\ndls$ the exceptional divisor of this blowup, it is irreducible of codimension 1.
We denote by $\eta_\ndls$ the generic point of $E_\ndls$. 
Diagrammatically:
\begin{equation}\label{equation:blowup_diagram_ndls}
  \begin{tikzcd}
\overline{\{\eta_\ndls \}} =  E_\ndls \arrow[r, hook] \arrow[d] & X_\ndls \arrow[d, "\pi_\ndls"] \\
Z_\ndls \arrow[r, hook]           & C^{(\deg \ndls)}          
\end{tikzcd}   
\end{equation}



% Recall the defintions from \Cref{section:BlowupOfSymmetricPowerOfCurves},
% $\ndls \subset \mdls$ a submodulus, $Z_\ndls$ is the closed subscheme of 
% $C^{(\deg \ndls)}$ defined by $\ndls$ as a point.
% $X_\ndls$ as the blowup of $C^{(\deg \ndls)}$ at 
% $Z_\ndls$, and $E_\ndls = Z_\ndls \times_{C^{(d)}} X_\ndls$ is exceptional divisor of this blowup with 
% $\eta_\ndls$ its generic point.

\Cref{theorem:SymmetricPowerOfSheafIsTamelyRamified} easily follows from: 

/theorem:SymmetricPowerOfSheavesIsTamelyRamifiedReduction
\theorem:SymmetricPowerOfSheavesIsTamelyRamifiedReduction
% \begin{theorem}\label{theorem:SymmetricPowerOfSheavesIsTamelyRamifiedReduction}
%      Let $\cF$ be a local system on $U$ with ramification at $P$ bounded by $\ndls = k_P P \subset \mdls$.
%      Then $\cF^{(\deg \ndls)}$ is tamely ramified at $\eta_\ndls$ of $E_\ndls$
% \end{theorem}

In the upcoming section, we perform the reduction and derive \Cref{theorem:SymmetricPowerOfSheafIsTamelyRamified} from 
\Cref{theorem:SymmetricPowerOfSheavesIsTamelyRamifiedReduction}. 
We then prove \Cref{theorem:SymmetricPowerOfSheavesIsTamelyRamifiedReduction} in the section that follows.

\subsection{Reduction Lemmas}

The following lemma is adapted from \cite{takeuchi2019blow} (Lemma 4.1)
\begin{lemma}\label{lemma:takeuchi_lemma}
     Let $C$ be a projective, smooth, and geometrically connected curve over a perfect field $k$. 
     Let $\mathfrak{m} = \sum_{i=1}^{r} k_i P_i$ be an effective divisor where $P_1, \dots, P_r$ are distinct closed points. 
     Let $U = C \setminus \mathfrak{m}$ and let $d \geq \deg \mathfrak{m}$.

     Suppose $\mathfrak{n}_1, \dots, \mathfrak{n}_l$ are pairwise coprime submoduli of $\mathfrak{m}$ such that $\mathfrak{m} = \sum_{j=1}^l \mathfrak{n}_j$. 
     Consider the summation morphism:
     \[
     \pi : C^{(\deg \mathfrak{n}_1)} \times_k \dots \times_k C^{(\deg \mathfrak{n}_l)} \times_k C^{(d - \deg \mathfrak{m})} \longrightarrow C^{(d)}
     \]
     defined by $(D_1, \dots, D_l, D_{extra}) \mapsto \sum_{j=1}^l D_j + D_{extra}$.

     Then $\pi$ is étale at the generic point of the closed subvariety 
     \[
     V = \{\mathfrak{n}_1\} \times_k \dots \times_k \{\mathfrak{n}_l\} \times_k C^{(d - \deg \mathfrak{m})}
     \]
     inside the domain $C^{(\deg \mathfrak{n}_1)} \times_k \dots \times_k C^{(\deg \mathfrak{n}_l)} \times_k C^{(d - \deg \mathfrak{m})}$.

\begin{proof}
     We may assume that $k$ is algebraically closed (hence $\deg P_i = 1$ for all $i$). By miracle flatness $\pi$ is flat, it is quasi-finite and projective as a map between projecive spaces.
     so we condlude $\pi$ is finite and flat. 
     It is enough to show that there exists a closed point $Q$ of $\ndls_1 + \dots \ndls_l + C^{(d - \deg \mathfrak{m})} \subset C^{(d)}$ 
     over which there are $\deg \pi$ points on $C^{(\ndls_1)} \times_k \dots \times_k C^{(\ndls_l)} \times_k C^{(d - \deg \mathfrak{m})}$. (Because it will be unramified 
     at this point and thus also at the generic point of $V$.)
     Choose $Q$ as a point corresponding to a divisor
     $\ndls_1 + \dots + \ndls_l + P_{r+1} + \dots + P_{r+d - \deg \mathfrak{m}}$, where $P_1, \dots, P_{r+d - \deg \mathfrak{m}}$ are distinct points of $U(k)$.
\end{proof}     
\end{lemma}

\begin{cor}\label{cor:etale_at_generic_point}
     The morphism $C^{(\deg \mdls)} \times_k C^{(d - \deg \mdls)} \xrightarrow[]{\pi} C^{(d)}$
     is finite flat everywhere, and étale at the generic point of the closed subvariety
     $Z_\mdls \times_k C^{(d - \deg \mdls)} \subset C^{(\deg \mdls)} \times_k C^{(d - \deg \mdls)}$.     
\end{cor}

Following from this, we look at the following diagram, coming from the flat base change

$C^{(d - \deg \mdls)} \to \Spec(k)$ (\Cref{prop:symmetric_power_flat_base_change}) of \Cref{equation:blowup_diagram_ndls}:
(\textcolor{red}{Is this even smooth? })

\[\begin{tikzcd}
E_{\mdls} \times_k C^{(d-\deg \mdls)} \arrow[r] \arrow[d] & X_{\mdls} \times_k C^{(d-\deg \mdls)} \arrow[d] & \cF^{(\deg \mdls)} \times_k C^{(d-\deg \mdls)} \arrow[d, color=purple] \\
Z_\mdls \times_k C^{(d-\deg \mdls)} \arrow[r] & C^{(\deg \mdls)} \times_k C^{(d-\deg \mdls)} & U^{(\deg \mdls)} \times_k C^{(d-\deg \mdls)} \arrow[lu]
\end{tikzcd}\]

Note that $U^{(\deg \mdls)} \times_k C^{(d-\deg \mdls)}$ is dense open subscheme of $X_{\mdls} \times_k C^{(d-\deg \mdls)}$,
And $E_{\mdls} \times_k C^{(d-\deg \mdls)}$ is a prime divisor of $X_{\mdls} \times_k C^{(d-\deg \mdls)}$.
Hence it is well defined question according to \Cref{definition:tamely_ramified_in_codim_1} to ask whether
$\cF^{(\deg \mdls)} \times_k C^{(d-\deg \mdls)}$ is tamely ramified at the generic point $\theta$ of
$E_{\mdls} \times_k C^{(d-\deg \mdls)}$.

\begin{lemma}\label{lemma:tame_ramification_persists}
     If $\cF^{(\deg \mdls)}$ is tamely ramified at $\eta_\mdls$, then
     $\cF^{(\deg \mdls)} \times_k C^{(d-\deg \mdls)}$ is tamely ramified at the generic point $\theta$ of
     $E_{\mdls} \times_k C^{(d-\deg \mdls)}$.
\end{lemma}
\begin{proof}
    This follows from \stackstag{0EYD}
    \textcolor{red}{make adjusments to definition and lemma, to only require that some prime divisors are with desired properties.}
\end{proof}


Replacing $C^{(d - \deg \mdls)}$ with the dense open subscheme $U^{(d - \deg \mdls)} \subset C^{(d - \deg \mdls)}$,
we get that the $G$-torsor $p_1^{-1} \cP^{(\deg \mdls)}$ ($\cP$ correspodns to $\cF$ under \Cref{prop:torsor_module_equivalence})
is tamely ramified at $\theta$ the generic point of
$E_{\mdls} \times_k U^{(d-\deg \mdls)} \subset U^{(\deg \mdls)} \times_k U^{(d - \deg \mdls)}$,
where $p_1: U^{(\deg \mdls)} \times_k U^{(d - \deg \mdls)} \to U^{(\deg \mdls)}$ is the projection to the first factor.

Looking at the second projection $p_2:X_{\mdls} \times_k U^{(d - \deg \mdls)} \to U^{(d - \deg \mdls)}$, and the fact that 
$\cP^{(d - \deg \mdls)}$ is \'etale on $U^{(d - \deg \mdls)}$ we get that 
$p_2^{-1} \cP^{(d - \deg \mdls)}= X_{\mdls} \times_k \cP^{(d - \deg \mdls)}$ is \'etale on $X_{\mdls} \times_k U^{(d - \deg \mdls)}$.
Hence, its restriction to $U^{(\deg \mdls)} \times_k U^{(d - \deg \mdls)}$ is unramified at $\theta$.

Thus, by the following lemma, we conclude that $\boxP{\cP^{(\deg \mdls)}}{\cP^{(d - \deg \mdls)}} = p_1^{-1} \cP^{(\deg \mdls)} \wedge^G p_2^{-1} \cP^{(d - \deg \mdls)}$
     is tamely ramified at $\theta$ the generic point of $E_{\mdls} \times_k U^{(d-\deg \mdls)}$.

\begin{lemma}
     Let $X \to \spec k$ be a scheme over a field $k$, and let $\cP_1, \cP_2$ be two $G$-torsors on $U_{et}$.
     Let $\xi$ be the generic point of a prime divisor $D \subset X$.
     If $\cP_1$ is tamely ramified at $\xi$, and $\cP_2$ is unramified at $\xi$,
     then the contracted product $\cP_1 \wedge^G \cP_2$ is tamely ramified at $\xi$.
\end{lemma}
\begin{proof}
     This follows from \Cref{lemma:bounding_ramification_of_contracted_product}.
\end{proof}

Combining this with \Cref{cor:etale_at_generic_point} we get

\begin{cor}\label{cor:tame-ramification-symmetric-product}
     Let $\mdls$ be a modulus as above, and let $\eta_\mdls$ be the generic point of $E_\mdls$.
     Let $\cP$ be a $G$-torsor on $U_{et}$ with ramification bounded by $\mdls$.
     Assume $\cP^{(\deg \mdls)}$ is tamely ramified at $\eta_\mdls$.
     Then $\boxP{\cP^{(\deg \mdls)}}{\cP^{(d - \deg \mdls)}}$ is tamely ramified at the generic point $\theta$ of 
     $E_{\mdls} \times_k C^{(d-\deg \mdls)} \subset C^{(\deg \mdls)} \times_k C^{(d - \deg \mdls)}$,
     and $\cP^{(d)}$ is tamely ramified at the generic point $\eta_0=\eta_{\mdls,d}$ of $E_0=E_{\mdls,d}$
\end{cor}
\begin{proof} 
The first assertion, that $\mathcal{P}^{(\deg \mathfrak{m})} \boxtimes \mathcal{P}^{(d - \deg \mathfrak{m})}$ is tamely ramified at $\theta$, follows from the preceding discussion. Thus, it remains to show that $\mathcal{P}^{(d)}$ is tamely ramified at $\eta_0$.

Consider the blowup diagram defining $X_{\mathfrak{m}, d}$:
\[
\begin{tikzcd}
\overline{\{\eta_0 \}} =  E_0 \arrow[r] \arrow[d] & X_{\mdls, d} \arrow[d, "\pi"] \\
Z_0 \arrow[r, "c.i"]           & C^{(d)}          
\end{tikzcd}
\] 
By performing a base change along the flat addition map $+: C^{(\deg \mathfrak{m})} \times_k U^{(d-\deg \mathfrak{m})} \to C^{(d)}$, we obtain the following commutative diagram:

\[
\begin{tikzcd}
&  \left(C^{(\deg \mdls)} \times_k U^{(d-\deg \mdls)} \right) \times_{C^{(d)}} E_0 \arrow[r] \arrow[d] & \overline{\{\eta_0 \}} =  E_0 \arrow[d] \\
& \left(C^{(\deg \mdls)} \times_k U^{(d-\deg \mdls)} \right) \times_{C^{(d)}} X_{\mdls, d} \arrow[d] \arrow[r] & X_{\mdls, d} \arrow[d, "\pi"] \\
\mdls \times_k U^{(d-\deg \mdls)} \arrow[r, "c.i"] & C^{(\deg \mdls)} \times_k U^{(d-\deg \mdls)} \arrow[r, "+"]  & C^{(d)}          
\end{tikzcd}
\] 
Since blowups commute with flat base change, and the inverse image of the center $Z_0$ under the map $+$ is $\mathfrak{m} \times_k U^{(d-\deg \mathfrak{m})}$, the scheme $\left(C^{(\deg \mathfrak{m})} \times_k U^{(d-\deg \mathfrak{m})} \right) \times_{C^{(d)}} X_{\mathfrak{m}, d}$ is the blowup of $C^{(\deg \mathfrak{m})} \times_k U^{(d-\deg \mathfrak{m})}$ along $\mathfrak{m} \times_k U^{(d-\deg \mathfrak{m})}$. 
This, in turn, is isomorphic to the base change of the blowup $X_{\mathfrak{m}}$ (of $C^{(\deg \mathfrak{m})}$ along $\mathfrak{m}$) via the (flat) projection $C^{(\deg \mathfrak{m})} \times_k U^{(d - \deg \mathfrak{m})} \to C^{(\deg \mathfrak{m})}$.

Assembling these facts, we obtain the following Cartesian square:
\[
\begin{tikzcd}
&  E_{\mdls}\times U^{(d - \deg \mdls)}  \arrow[r, "\tilde{+}"] \arrow[d] & \overline{\{\eta_0 \}} =  E_0 \arrow[d] \\
& X_{\mdls} \times_k U^{(d - \deg \mdls)} \arrow[d] \arrow[r, "\tilde{+}"] & X_{\mdls, d} \arrow[d, "\pi"] \\
\mdls \times_k U^{(d-\deg \mdls)} \arrow[r, "c.i"] & C^{(\deg \mdls)} \times_k U^{(d-\deg \mdls)} \arrow[r, "+"]  & C^{(d)}          
\end{tikzcd}
\] 

The point $\theta$ defined in the Corollary is the generic point of $E_{\mathfrak{m}} \times_k U^{(d-\deg \mathfrak{m})}$.
 Let $\eta$ be the generic point of $\mathfrak{m} \times_k U^{(d-\deg \mathfrak{m})}$. 
 By \Cref{cor:etale_at_generic_point}, the map $+$ is étale at $\eta$. 
Consequently, the lifted map $\tilde{+}$ is étale at $\theta$. 
Given the isomorphism $(+^{-1})(\cP^{(d)}) \cong \cP ^{(\deg \mdls)} \boxtimes \cP ^{(d - \deg \mdls)}$ from \Cref{section:symmetric_powers_on_curves},
the tame ramification of the box product at $\theta$ descends to the tame ramification of $\mathcal{P}^{(d)}$ at $\eta_0$ by applying \Cref{lemma:descend_ramification_along_etale}.

\end{proof}

% \begin{lemma}
%      Let $\ndls_1, \ndls_2 \subset \mdls$ be two moduli of the form $\ndls_1 = k_1 P_1$, $\ndls_2 = k_2 P_2$ where $P_1, P_2$ are distinct points.
%      Assume $\cF ^{(\deg \ndls_1)}$, $\cF ^{(\deg \ndls_2)}$ are at most tamely ramified at $\eta_{\ndls_1}$, $\eta_{\ndls_2}$ respectively.
%      Then $\cF^{(\deg \ndls_1 + \deg \ndls_2)}$ is at most tamely tamified at $\eta_{\ndls_1 + \ndls_2}$.
% \end{lemma}
% \begin{proof}
%      \textcolor{red}{Complete}
% \end{proof}

\begin{lemma}\label{lemma:moduli_reduction}
     Let $\ndls_1, \ndls_2 \subset \mdls$ be two coprime sub moduli of $\mdls$.
     Assume $\cP ^{(\deg \ndls_1)}$, $\cP ^{(\deg \ndls_2)}$ are at most tamely ramified at $\eta_{\ndls_1}$, $\eta_{\ndls_2}$ respectively.
     Then $\cP^{(\deg \ndls_1 + \deg \ndls_2)}$ is at most tamely tamified at $\eta_{\ndls_1 + \ndls_2}$.
\end{lemma}

\begin{proof}
     It follows from \Cref{prop:ramification-external-product} and \Cref{section:symmetric_powers_on_curves}
\end{proof}

\subsection{\texorpdfstring{Proof of \Cref{theorem:SymmetricPowerOfSheafIsTamelyRamified}}{Proof of Theorem X}}
\begin{proof}[Proof of \Cref{theorem:SymmetricPowerOfSheafIsTamelyRamified}]
     Let $\cF$ be as in \Cref{theorem:SymmetricPowerOfSheafIsTamelyRamified}, $\mdls = \sum_{i=1}^n k_P P$ with $\deg P = d_P$  
     Then by \Cref{theorem:SymmetricPowerOfSheavesIsTamelyRamifiedReduction} for every $\ndls \subset \mdls$ of the form $\ndls = k_P P$, $\cF^{(\deg \ndls)}$ is at most tamely ramified at $\eta_\ndls$.
     By \Cref{lemma:moduli_reduction}, $\cF^{(\deg \mdls)}$ is then at most tamely ramified at $\eta_\mdls$. And thus by \Cref{cor:tame-ramification-symmetric-product}
     $\cF^{(d)}$ is tamely ramified at the generic point $\eta_0$  of $E_0$
\end{proof}


\subsection{\texorpdfstring{Proof of \Cref{theorem:SymmetricPowerOfSheavesIsTamelyRamifiedReduction}}{Proof of Theorem Y}}

Let $G$ be a finite abelian group. 
Unless otherwise stated, assume that $C=\bP_k^1$, $\mdls = d \cdot 0$ and 
$\bG_m = U \subset U' = C \setminus \mdls$. Then $\deg \mdls = d$
We also assume $k$ is algebraicly closed. (We can etale base change, and this doesn't change ramification.)
Our first result is:

\begin{theorem}
     Let $\cP \to \bG_m \subset \bP_k^1$ be a $G$ torsor which is either
     \begin{enumerate}
          \item tamely ramified at $0$ 
          \item wildly ramified at $0$ with $G=\Z/p\Z$ and ramificaiton bounded by $d$.
     \end{enumerate}
     Then the ramification of the $G$-torsor $\cP^{(d)} \to \bG_m^{(d)}$ at $\eta_\mdls$ the generic point of $E_\mdls \subset X_{\mdls}$ is  
     \begin{enumerate}
          \item tamely ramified if $\cP$ was tamely ramified
          \item unramified if $\cP$ was wildly ramified with $G=\Z/p\Z$ and ramification bounded by $d$
     \end{enumerate}
\end{theorem}
\begin{proof}
     Recall that in \Cref{subsection:ram_prod_analysis}, we saw that the local ring at the generic point of the excptional divisor
     of the blowup of the affine space at 0 point is
     $R = k[x_d, \frac{u_1}{u_d}, \dots, \frac{u_{d-1}}{u_d}]_{(x_d)}$. 
     Where for every  $i < d$, we have $x_i = \frac{u_i}{u_d} x_d$ are all uniformizers.
     The residue field was $\kappa(\eta) = k(\frac{u_1}{u_d}, \dots, \frac{u_{d-1}}{u_d})$
     and the completion of $R$ with respect to its maximal ideal is:
     $$\hat{R} = \kappa(\eta)[[x_d]]$$
     In our situation, when we take symmetric product of the affine space, the situation is similiar with different coordinates
     if we let $e_1, \dots, e_d$ be the symmetric polynomials in $x_1, \dots x_d$ then:
     The local ring is 
     $R = k[e_d, \frac{u_2}{u_d}, \dots, \frac{u_{d-1}}{u_d}]_{(e_d)}$. 
     $e_i = \frac{u_i}{u_d} e_d$ are all uniformizers.
     The residue field being:
     $\kappa(\eta) = k(\frac{u_1}{u_d}, \dots, \frac{u_{d-1}}{u_d})$
     and the completion: $\hat{R} = \kappa(\eta)[[s_d]]$
     Note that from  $e_i = \frac{u_i}{u_d} e_d$
     We get  $\frac{e_i}{e_d} = \frac{u_i}{u_d}$ in the fracion field. hence

     \begin{equation}\label{eq:complete_field_at_generic_point}
          \hat{K}= k(\frac{e_1}{e_d}, \dots, \frac{e_{d-1}}{u_d})((e_d))
     \end{equation}

     We compute directly the extesntion of complete valued fields over the complete valued field at the generic point. 
     Note that by  \Cref{theorem:kummer_ramification} and \Cref{theorem:artin_schreier_ramification} 
     We can assume $\cP = \Spec k[x, x^{-1}][X]/(X^n - a)$ for $a \in k[x, x^{-1}]$ or $\cP = \Spec k[x, x^{-1}][X]/(X^p + X - f(x, x^{-1}))$
     where $f(x,x^{-1}) = c x^{-m} + a_{-m + 1} x^{-m + 1} + \cdots a_{-1}x^{-1} + a_0 = c x^{-m} + f_{-m+1}(x^{-1})$ where $m < d$.
     
     We deal with each case separately. 
     
     \textbf{Artin-Schreier Extensions:}

     Set $R=k[x, x^{-1}]$, and $S = \Spec k[x, x^{-1}][X]/(X^p + X - f(x^{-1}))$
     we have 
     $$
          \begin{aligned}
          R^{\otimes_k d} &= k[x_1, x_1^{-1}, \dots, x_d, x_d^{-1}] \\
          S^{\otimes_k d} &= k[x_1, x_1^{-1}, \dots, x_d, x_d^{-1}][X_1, \dots, X_d] / (X_1^p - X_1 - f(x_1), \dots, X_d^p - X_d - f(x_d)) \\
          &= R^{\otimes_k d}[X_1, \dots, X_d] / (X_1^p - X_1 - f(x_1), \dots, X_d^p - X_d - f(x_d))
          \end{aligned}
     $$

     Next, we want to understand the ring corresponding to $p_1^{-1}(\cP) \otimes \dots \otimes p_d^{-1}(\cP)$ on $C^{d}$ - 
     the $d$'th-contracted product of the $G=\Z/p\Z$-torsors  $p_1^{-1}(\cP), \dots, p_d^{-1}(\cP)$ on $U^{d}$. 
     It correspond to qoutient:
     $\left(p_1^{-1}(\cP) \times \dots \times p_d^{-1}(\cP)\right) / G^{d-1}$ where the action of $G^{d-1}$ on the product is:
     $$(g_1, \dots, g_{d-1}) \cdot (p_1, p_2, \dots, p_{d-1}, p_{d}) = (g_1(p_1), g_1^{-1}g_2(p_2), \dots, 
     g_{d-2}^{-1}g_{d-1}(p_{d-1}), g_{d-1}^{-1}(p_d))$$
     
     The affine ring corrsponding to the contracted product is $\left(S^{\otimes_k d}\right)^{G^{d-1}}$
    
     Recall that the action of $g \in G=\Z/p\Z$ on $X$ is $g(X) = X + g$ ($g$ correspond to a number $0 \leq g \leq p-1)$.
     So, the action of $(g_1, \dots, g_{d-1})$ on the generators $(X_1, X_2, \dots, X_{d-1}, X_{d})$
     Is $X_1 \mapsto X_1 + g_1$, $X_i \mapsto X_i - g_{i-1} + g_i$ for $1<i<d$ and $X_d \mapsto X_d - g_{d-1}$.
     So we see that $Y=X_1 + \dots + X_d$ is invariant.
     Moreover $Y^P - Y - \sum_{i=1}^d f(x_i)=0$ is irreducible degree $p$ equation for $Y$,
     Since we are quotienting a rank $p^{d}$ extesntion by a group of order $p^{d-1}$ the resulting invaraint subring must
     have rank $p$ over $R^{\otimes_k d}$, So we conclude:
     \[
          \left(S^{\otimes_k d}\right)^{G^{d-1}} \cong R^{\otimes_k d}[Y] / (Y^p - Y - \sum_{i=1}^d f(x_i))
     \]  
     
     The group $S_d$ acts on 
     $ R^{\otimes_k d} = k[x_1^{\pm 1}, x_2^{\pm 1}, \dots, x_d^{\pm 1}]$
      by permuting the variables $\{x_i\}_{i=1}^d$.
     And since $Y=\sum_i^d X_i$ it leaves $Y$ invaraint. 
     The invaraint subring $(R^{\otimes_k d})^{S_d}$ is simply $k[e_1, e_2, \dots, e_d, e_d^{-1}]$
     where $\{ e_i \}$  are the symmetric polynomials in $x_1, \dots, x_d$ e.g. $e_1 = x_1 + \dots + x_d$
     \textcolor{red}{give general defintion here...}
     and $e_d = x_1 x_2 \ldots x_d$.
     
     To find $\left(R^{\otimes_k d}[Y] / (Y^p - Y - \sum_{i=1}^d f(x_i))\right)^{S_d}$ its enough to express
     $\sum_{i=1}^d f(x_i)$ in $e_1, \dots, e_d$, this can be done with the newton polynomials, moreover, we claim the following:

     \begin{lemma}
          Let $f(x) = c x^{-m} + a_{-m + 1} x^{-m + 1} + \cdots a_{-1}x^{-1} + a_0$, 
          define  $\alpha(x_1, ..., x_d) = \sum_1^d f(x_i)$, and deonte by $e_1, ..., e_d$ the elementary symmetric polynomials 
          in $x_1, ..., x_d$. If $m < d$ then $\alpha(x_1, ..., x_d) \in k(e_1/e_d, e_2/e_d, ..., e_{d-1}/e_d)$
     \end{lemma}
     \begin{proof}
          Changing variables $y_i=x_i^{-1}$ for each $i \in \{1, \dots, d\}$
          We get 
          $$\alpha = \sum_{i=1}^d f(x_i) = \sum_{i=1}^d \left( c y_i^m + a_{-m+1} y_i^{m-1} + \dots + a_{-1} y_i + a_0 \right)$$
          Rearranging the sums, we get
          $$\alpha = c \sum_{i=1}^d y_i^m + a_{-m+1} \sum_{i=1}^d y_i^{m-1} + \dots + a_{-1} \sum_{i=1}^d y_i + d a_0$$
          Let $p_k(y_1, \dots, y_d) = \sum_{i=1}^d y_i^k$ be the $k$-th power sum symmetric polynomial. 
          The expression for $\alpha$ is a linear combination of these power sums:
          $$\alpha = c p_m(y) + a_{-m+1} p_{m-1}(y) + \dots + a_{-1} p_1(y) + d a_0$$
         
          According to the \textit{Fundamental Theorem of Symmetric Polynomials}, any symmetric polynomial in $y_1, \dots, y_d$ 
          can be expressed as a polynomial in the elementary symmetric polynomials $e_k(y_1, \dots, y_d)$. 
          Since $m < d$, $\alpha$ is a polynomial in $e_1(y), e_2(y), \dots, e_m(y)$. ($y=(y_1, \dots, y_d)$)
          The elementary symmetric polynomials in $y_i = 1/x_i$ are related to the elementary symmetric polynomials in $x_i$ as follows: 
          $$e_k(y_1, \dots, y_d) = \sum_{1 \le i_1 < \dots < i_k \le d} \frac{1}{x_{i_1} \dots x_{i_k}} = \frac{\sum_{1 \le j_1 < \dots < j_{d-k} \le d} x_{j_1} \dots x_{j_{d-k}}}{x_1 x_2 \dots x_d}$$
          Thus,
          $$e_k(y_1, \dots, y_d) = \frac{e_{d-k}(x_1, \dots, x_d)}{e_d(x_1, \dots, x_d)}$$
          which concludes the proof.
     \end{proof}

Finally, restricting $\cP^{(d)}$ to $\spec {\hat{K}}$ we get by \Cref{eq:complete_field_at_generic_point}
and \Cref{theorem:artin_schreier_ramification} the result. (that $\cP$ is unramified at the generic point of the exceptional divisor of the blowup).

\textbf{Kummer Extensions:}
Few things are different in that case,

     Set $R=k[x, x^{-1}]$, and $S=R[X]/(X^n - f)$ where $f= f(x, 1/x) \in R$
     In this case we have $char k = p$ and $gcd(p, n)=1$.

     % Let $g$ be a generator of $G = \mathbb{Z}/n\mathbb{Z}$
     % The action of $g$ on an element $s \in S$ is determined by its action on the generator $X$. 
     % Since $X^n = f$ in $S$, any $R$-automorphism $\sigma$ must satisfy:
     % $$\sigma(X)^n = \sigma(X^n) = \sigma(f) = f$$
     % This implies that $\sigma(X)$ must also be a root of the polynomial $Z^n - f$. 
     % Therefore, $\sigma(X) = \zeta^i X$ for some $i \in \{0, \dots, n-1\}$.
     
     % The standard action for the generator $g \in G$ is:
     % $$\sigma_g(X) = \zeta X$$
     % By linearity and the property of $R$-algebras, the action on any element 
     % $s = \sum_{j=0}^{n-1} r_j X^j$ (where $r_j \in R$) is:
     % $$\sigma_g \left( \sum_{j=0}^{n-1} r_j X^j \right) = \sum_{j=0}^{n-1} r_j (\zeta X)^j = \sum_{j=0}^{n-1} r_j \zeta^j X^j$$


     We have 
     $$
          \begin{aligned}
          R^{\otimes_k d} &= k[x_1, x_1^{-1}, \dots, x_d, x_d^{-1}] \\
          S^{\otimes_k d} &= k[x_1, x_1^{-1}, \dots, x_d, x_d^{-1}][X_1, \dots, X_d] / (X_1^n - f_1, \dots, X_d^n - f_d) \\
          &= R^{\otimes_k d}[X_1, \dots, X_d] / (X_1^n - f_1, \dots, X_d^n - f_d)
          \end{aligned}
     $$
     Where $f_i = f(x_i, x_i^{-1})$

     Next, we want to figure out 
      $\left(S^{\otimes_k d}\right)^{G^{d-1}}$
    

     Recall that the action of $g \in G=\Z/n\Z$ on $X$ is $g(X) = \zeta^{g} X$ ($g$ correspond to a number $0 \leq g \leq n-1)$.
     So, the action of $(g_1, \dots, g_{d-1})$ on the generators $(X_1, X_2, \dots, X_{d-1}, X_{d})$
     Is $X_1 \mapsto \zeta^{g_1} X_1$, $X_i \mapsto \zeta^{g_i - g_{i-1}}X_i$ for $1<i<d$ and $X_d \mapsto \zeta^{-g_{d-1}} X_d$.

     So we see that $Y=X_1 X_2 \dots X_d$ is invariant.
     And $Y^n - \prod_{i=1}^d f_i$ is irreducible degree $n$ equation for $Y$,
     So, like before, we conclude:
     \[
          \left(S^{\otimes_k d}\right)^{G^{d-1}} \cong R^{\otimes_k d}[Y] / (Y^n - \prod_{i=1}^d f_i)
     \]  
     
     The group $S_d$ acts on $R^{\otimes_k d}[Y] / (Y^n - \prod_{i=1}^d f_i)$ by permuting the indices.
     on the variables $x_i$, 
     On $Y=\prod_1^d X_i$ it is invaraint. 
     The invaraint subring $(R^{\otimes_k d})^{S_d}$ is simply $k[e_1, e_2, \dots, e_d, e_d^{-1}]$ like before.
     

     The polynomial $F=\prod_{i=1}^d f_i$ is symmetric in $\{x_i\}_1^d$ so it can be expressed as a polynomial
     $\tilde{F}(e_1, \dots, e_d)$ in the elementary symmetric variables. 
     Hence the qoutient ring is:
     $$\left( \frac{k[x_1^{\pm 1}, \dots, t_d^{\pm 1}][Y]}{(Y^n - \prod_{i=1}^d f_i)} \right)^{S_d} \cong 
     \frac{k[e_1, \dots, e_d, e_d^{-1}][Y]}{(Y^n - \tilde{F}(e_1, \dots, e_d))}$$
     So we see again, that  restricting $\cP^{(d)}$ to $\spec {\hat{K}}$ we get by \Cref{eq:complete_field_at_generic_point} 
     and \Cref{theorem:kummer_ramification}, that $\cP$ is tamely ramified at the generic point of the exceptional divisor of the blowup.

     
\end{proof}

Now, $G$ is fintie abelian. So by the Structure Theorem for Finite Abelian Groups we have a descending sequence of subgroups:
$$G = H_0 \supset H_1 supset H_2 \supset \dots \supset H_l \supset {1}$$
Where for all $i < l$ we have 