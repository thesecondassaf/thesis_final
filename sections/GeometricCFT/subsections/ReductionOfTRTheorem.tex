We do a reduction theorem \ref{theorem:SymmetricPowerOfSheafIsTamelyRamified} to the case 
$\mdls=dP$. 

Throughout this section Let $C \to S$ be a smooth morphism of schemes of relative dimension 1, with connected geometric fibers of genus $g$, which is
\textcolor{red}{OR}

Let $C$ be a projective smooth geometrically connected curve over a perfect field $k$. Let $\mathfrak{m}$ be a modulus on $C$ and write $\mathfrak{m} = n_1 P_1 + \dots + n_r P_r$, where $P_1, \dots, P_r$ are distinct closed points of $\mathfrak{m}$. Denote the complement of $\mathfrak{m}$ in $C$ by $U$. Let $d_i := \deg P_i$. Take a positive integer $d$ so that $d \ge \deg \mathfrak{m}$.

Zariski-locally projective over $S$.

The reduction is followed by 3 lemmas:
\begin{lemma}
The morphism $\pi : C^{(n_1 d_1)} \times_k \dots \times_k C^{(n_r d_r)} \times_k C^{(d - \deg \mathfrak{m})} \to C^{(d)}$, taking the sum, is étale at the generic point of the closed subvariety $\{n_1 P_1\} \times \dots \times \{n_r P_r\} \times C^{(d - \deg \mathfrak{m})}$ of $C^{(n_1 d_1)} \times_k \dots \times_k C^{(n_r d_r)} \times_k C^{(d - \deg \mathfrak{m})}$.
\end{lemma}

\begin{proof}
We may assume that $k$ is algebraically closed (hence $d_i = 1$ for all $i$). Since the map $\pi : C^{(n_1)} \times_k \dots \times_k C^{(n_r)} \times_k C^{(d - \deg \mathfrak{m})} \to C^{(d)}$ is finite flat, it is enough to show that there exists a closed point $Q$ of $n_1 P_1 + \dots n_r P_r + C^{(d - \deg \mathfrak{m})}$ over which there are $\deg \pi$ points on $C^{(n_1)} \times_k \dots \times_k C^{(n_r)} \times_k C^{(d - \deg \mathfrak{m})}$. Choose $Q$ as a point corresponding to a divisor
$n_1 P_1 + \dots n_r P_r + P_{r+1} + \dots + P_{r+d - \deg \mathfrak{m}}$, where $P_1, \dots, P_{r+d - \deg \mathfrak{m}}$ are distinct points of $U(k)$.
\end{proof}


\begin{lemma} Denote $\mdls_1=n_1P_1$ and $\mdls_2=n_2P_2 + ... + n_rP_r$. Let $X_{\mdls_1}, X_{\mdls_2}$ be the blowups of $C^{(\deg \mdls_1)}, C^{(\deg \mdls_2)}$ by $\mdls_1$, $\mdls_2$ respectively.  
Let $E_1, E_2$ be the respective exceptional divisors (\textcolor{red}{which are irreducibele of codim 1}), and let $\eta_1$, $\eta_2$ be their generic points respectively.
Assume $\cF^{(\deg \mdls_1)}$, $\cF^{(\deg \mdls_2)}$ are tamely ramified (\textcolor{red}{is bounded ramification here good enough?}) on $\eta_1, \eta_2$  respectively.
Then $\cF^{(\deg \mdls)}$ is tamely ramified (\textcolor{red}{or any bounded ramification?}) at $\eta$ - the generic point of the exceiptional divisor of the blowup $X_\mdls$ of $C^{(\deg \mdls)}$ by $\mdls$
\end{lemma}

\begin{lemma}
    In the notations of the previous lemma, suppose $\cF^{(\deg \mdls)}$ is tamely ramified at $\eta$.
    Then  $\Fbox{\deg \mdls}{n - \deg \mdls}$ is tamely ramified at the generic point $\theta$ of $E \times_k C^{(n-\deg \mdls)}$
\end{lemma}

Combining the above we get:
\begin{prop}
    If for every $1 \leq i \leq r$, $\cF^{(d_i n_i)}$ is tamely ramified at $\eta_i$ then $\cF^{(n)}$ is tamely ramified at $\theta$
    \textcolor{red}{Make that precise and everything. notation wise etc...}
\end{prop}

\textcolor{red}{
\begin{enumerate}
    \item The first lemma copied from takeuchi, should we explain its proof? give another proof? exclude its proof and refer?
    \item Which of the above definition of $C$ are we going to use? (over $k$ or $s$)
    \item In the second lemmas, add/explain why are the exceptioanl divisors are irreudcible of codmin 1
    \item Say what is E - the exceiptional divisor of the blowup.
\end{enumerate}
}

