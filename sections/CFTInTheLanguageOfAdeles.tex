\section{Class Field Theory In the language of Adeles and Ideles}
\textcolor{red}{Can we already say we are only considering function fields here?}
The modern formulation of Global Class Field Theory is given in terms of the adele and idele groups of a global field. In this chapter we will define these objects and state the main theorems of Class Field Theory in this language.
\subsection{Adeles and Ideles}
Let $K$ be a global field. For each place $v$ of $K$, we denote:
\begin{enumerate}
    \item $K_v$ = the completion of $K$ at $v$
    \item $\pp_v$ = the corresponding prime ideal in the ring of integers $\Oo_K$ of $K$
    \item $\Oo_v$ = the ring of integers of $K_v$ 
    \item $\hat{\pp}_v$ = the completion of $\pp_v$ = the maximal ideal of $\Oo_v$
\end{enumerate}
 We define the \textbf{adele ring} of $K$ as the restricted direct product
\[\A_K = \prod_v' K_v
\]
where the restriction is taken with respect to the rings of integers $\Oo_v$ of $K_v$ for all \textcolor{red}{non-archimedean (IS IT NECESSEARY TO STATE HERE? WE WORK OVER P ANYWAY)}  places $v$. In other words, an adele is a tuple $(x_v)_v$ with $x_v \in K_v$ such that $x_v \in \Oo_v$ for all but finitely many non-archimedean places $v$.
The \textbf{idele group} of $K$ is defined as the group of units of the adele ring:
\[\I_K = \A_K^\times = \prod_v' K_v^\times
\]
where the restriction is taken with respect to the unit groups $\Oo_v^\times$ of the rings of integers $\Oo_v$ for all non-archimedean places $v$. An idele is thus a tuple $(x_v)_v$ with $x_v \in K_v^\times$ such that $x_v \in \Oo_v^\times$ for all but finitely many non-archimedean places $v$.  

The field $K$ embeds diagonally into $\mathbb{A}_K$, and thus $K^\times$ embeds diagonally into $\mathbb{I}_K$ as the subgroup of principal ideles. The \textbf{idele class group} $\fC_K$ is the quotient:
\[\fC_K = \I_K / K^\times\]

There is a natural isomorphism between certain quotients of the idele group and the ideal group of $K$, which ultimately follows by understanding
ideles as thickening of ideals:
There is a canonical surjective homomophism $\id$:

\begin{align*}
    &\id: \I_K \to I_K \\
    &(x_v)_v \mapsto \prod_v \pp_v^{v(x_v)}
\end{align*}

Thus, composing with $I_K \to C$ gives a surjective homomorphism $\I_K \to C$, noting that $K^\times \to \I_K \to C$ is 0,
we realize $C=I_K/i(K^\times)$ as a quotient of $\fC_K=\I_K/K^\times$.

The same thing is true for $C_\mdls$:
Let $\mdls=\sum n_v \pp_v$ be a modulus of $K$, set: 


\[W_{\mdls}(v) = \begin{cases}
\Oo_v^\times & v \notin \Supp(\mdls)  \\
1 + \hat{\pp_v}^{n_v} & v \in \Supp(\mdls)
\end{cases}
\]

And define 
\[
\I_{\mdls} = \left(\prod_{v \notin \Supp(\mdls)}^{}K_v^\times \times \prod_{v \in \Supp(\mdls)}^{}
 W_{\mdls}(v)  \right) \bigcap \I_K
\]

And \[
\bO_{\mdls}^\times = \prod_v W_{\mdls}(v)
\]
 
Note that:
\[
K_{\mdls,1} = K^\times \cap \prod_{v \in \mdls}^{} W_{\mdls}(v) \qquad \text{Intersection inside } \prod_{v \in \mdls}^{} K_v^\times
\]

and that 
\[
K_{\mdls,1} = K^\times \cap \I_{\mdls} \qquad \text{Intersection inside } \I_K
\]

Milne \textcolor{red}{shows} the following proposition:
\begin{prop} Let $\mdls$ be a modulus of $K$.
\begin{enumerate}
    \item The map $\id: \I_\mdls \to I^{S(\mdls)}_K$ defines an isomorphism
    \[ \I_\mdls / K_{\mdls,1} \bO_\mdls^\times \isomto I^{S(\mdls)}_K / i(K_{\mdls,1}) = C_\mdls \]
    \item The inclusion $\I_\mdls \hookrightarrow \I_K$ defines an isomorphism
    \[ \I_\mdls / K_{\mdls,1} \isomto \I_K / K^\times \]
\end{enumerate}
\end{prop}

\textcolor{red}{Taking the qoutient into character form?}


\subsubsection{Topology on Adeles and Ideles}
We state quickly the topology on the adele ring and the idele group. More can be found in \textcolor{red}{add reference}%\cite{milneCFT}.
Recall that, for all $v$, $K_v$ is locally compact more over, $\Oo_v$ is a compact neighborhood of 0. Similarly 
$K_v^\times$ is locally compact, in fact:
\[
1 + \hat{\pp}_v \supset 1 + \hat{\pp}_v^2 \supset 1 + \hat{\pp}_v^3\ldots
\]
is a fundamental system of neighborhoods of 1 consisting of compact open subgroups of $K_v^\times$.

For every finite set $S$ of places of $K$, define:
\[
\I_S = \prod_{v \in S} K_v^\times \times \prod_{v \notin S} \Oo_v^\times
\]
with the product topology. $\I_S$ is locally compact and as sets we have:
\[\I_K = \bigcup_{S} \I_S
\]
where the union is taken over all finite sets of places of $K$.
We define a topology on $I_K$ by giving a basis for the open sets $\prod_v V_v$ with 
$V_v \subseteq K_v^\times$ open for all $v$ and and $V_v = \Oo_v^\times$ for all but finitely many $v$.
This makes $\I_K$ a locally compact topological group, such that each $\I_S$ is open in $\I_K$, and inherits 
the product topology. The following sets form a fundamental system of neighborhoods of 1: for each 
finite set of primes $S$ and $n > 0$, define 
\[
U_{S,n} = \left\{ (x_v)_v \in \I_K \mid v(x_v - 1) > n \text{ for all } v \in S, x_v \in \Oo_v^\times \text{ for } v \notin S \right\}    
\]

Note that the embedding $K^\times \to \I_K$ is discrete and thus the idele class group $\fC_K = \I_K/K^\times$ is a locally compact topological group as well.
Moreover the canonical injective homomorphism 
\begin{align}
    &K_v^\times \to \I_K \\
    & x \mapsto (1,\ldots,1,x,1,\ldots, 1) \qquad \text{($x$ in the $v$-th position)}
\end{align}
is a topological embedding for each place $v$ of $K$.

\subsubsection{Characters of ideals and of ideles}
in \cite{milneCFT}, Milne proves:
\begin{prop}\label{prop:local-global-characters}
    Let $G$ be a finite abelian group.
    If $\psi: I^S \to G$ admits a modulus, then there exists a unique homomorphism $\phi: \mathbb{I} \to G$ such that
    \begin{enumerate}
        \item $\phi$ is continuous ($G$ with the discrete topology)
        \item $\phi(K^\times) = 1$;
        \item $\phi(\mathbf{a}) = \psi(\text{id}(\mathbf{a}))$, all $\mathbf{a} \in \mathbb{I}^S \stackrel{\text{def}}{=} \{\mathbf{a} \mid a_v = 1 \text{ all } v \in S\}$.
    \end{enumerate}
    Moreover, every continuous homomorphism $\phi: \mathbb{I} \to G$ satisfying (2) arises from a $\psi$.
    More over, $\phi$ and $\psi$ fit in the following chain:
\begin{equation}\label{eq:local-global-characters-diagram}
\begin{tikzcd}
& I^\mathfrak{m} \arrow[r] & C_\mathfrak{m} \arrow[r, "\psi"] & G \\
& \mathbb{I}_\mathfrak{m} / K_{\mathfrak{m},1} \arrow[r] \arrow[d, "\cong"] & \mathbb{I}_\mathfrak{m} / K_{\mathfrak{m},1} \bO^\times_\mdls \arrow[u, "\cong"] & \\
\mathbb{I} \arrow[r] \arrow[uurrr, bend right=50, "\phi"] & \mathbb{I}_K / K^\times & &
\end{tikzcd}
\end{equation}

\end{prop}

\subsubsection{Norms of ideles}
Let $L$ be a finite extension of the number field $K$.

For an idèle $\mathbf{a} = (a_w) \in \mathbb{I}_L$, define $\text{Nm}_{L/K}(\mathbf{a})$ to be the idèle $\mathbf{b} \in \mathbb{I}_K$ with $b_v = \prod_{w|v} \text{Nm}_{L_w/K_v} a_w$. 
Then, one can show that the following diagram commutes:
\[
\begin{tikzcd}
L^\times \arrow[r] \arrow[d, "\text{Nm}_{L/K}"] & \mathbb{I}_L \arrow[r, "\text{id}"] \arrow[d, "\text{Nm}_{L/K}"] & I_L \arrow[d, "\text{Nm}_{L/K}"] \\
K^\times \arrow[r] & \mathbb{I}_K \arrow[r, "\text{id}"] & I_K.
\end{tikzcd}
\]
Thus getting a commutative diagram:
\[
\begin{tikzcd}
\mathbf{C}_L \arrow[r] \arrow[d, "\text{Nm}_{L/K}"] & C_L \arrow[d, "\text{Nm}_{L/K}"] \\
\mathbf{C}_K \arrow[r] & C_K
\end{tikzcd}
\]
(where $C_L, C_K$ are the ideal class groups of $L$ and $K$ respectively).

\subsection{The main theorems}
The theory establishes a fundamental connection between the idele class group $\fC_K$ and the Galois group of the maximal abelian extension of $K$, denoted $K^{ab}$.
\begin{theorem}[Reciprocity Law]\label{theorem:idele-reciprocity-law}
    There exists a unique continious homomorphism $\phi_K: \I_K \to \text{Gal}(K^{ab}/K)$ called the \textbf{Artin map} with the following properties:
    \begin{enumerate}
        \item $\phi_K(K^\times) = 1$;
        \item For every finite abelian extension $L/K$, $\phi_K$ defines an isomorphism:
        \[ 
        \phi_{L/K}: \I_K/(K^\times \cdot \text{Nm}_{L/K}(\I_L)) \isomto \text{Gal}(L/K)
        \]
        or, equivalently, an isomorphism:
        \[
        \fC_K / \text{Nm}_{L/K}(\fC_L) \isomto \text{Gal}(L/K)
        \]
        \item $\phi_{L/K} $ arises from the global $\I_K \to \Gal(L/K)$ coming from the ideal-theoetic global artin map, as in \Cref{prop:local-global-characters}. 
    \end{enumerate}    
\end{theorem}

\begin{theorem}[The Existence Theorem]\label{theorem:idele-existence-theorem}
    There is a one-to-one, inclusion-reversing correspondence between the set of finite abelian extensions of $K$ and the set of open subgroups of finite index in the idele class group $\fC_K$.
    $$
    \begin{Bmatrix}
    \text{Finite abelian} \\
    \text{extensions } L/K
    \end{Bmatrix}
    \quad \longleftrightarrow \quad
    \begin{Bmatrix}
    \text{Open subgroups } H \subseteq \fC_K \\
    \text{of finite index}
    \end{Bmatrix}
    $$
    Under this correspondence, an extension $L$ corresponds to the subgroup $H = N_{L/K}(\fC_L)$.
    
\end{theorem}

\begin{theorem}
    Ideal-Theoetic and Idele-Theoetic formulations of CFT are equivilant through \Cref{prop:local-global-characters}. 
\end{theorem}

\textcolor{red}{
    \begin{enumerate}
        \item Above is finite adelic formulation of CFT, state somethign about the infinite extension CFT theorem
        \item State something about the topology on the idele class group, Say how the fintie implies the infinte by taking inverse limits. 
        \item Find sources for the above. for exampel milne? maybe other?
        \item The restriction is for the non-archimedean, are you sure?
        \item What is the topology of the "continious" homomorphism?
    \end{enumerate}
    }
