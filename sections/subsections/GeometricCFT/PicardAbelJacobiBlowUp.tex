\subsection{Generalized Picard Scheme}
We copy from \cite{Guignard2018} \cite{takeuchi2019blow}.
Let $S$ be a scheme, $C$ be a projective smooth $S$-scheme whose geometric fibers are connected and of dimension 1. Let $\mathfrak{m}$ be an effective Cartier divisor of $C/S$, i.e. a closed subscheme of $C$ which is finite flat of finite presentation over $S$. We also call $\mathfrak{m}$ a modulus. Let us denote, for $S$-schemes $T$, the projections $C \times_S T \to T$ by the same symbol $\text{pr}$. In this section, we recall and study the notion of generalized jacobian varieties. Let $d$ be an integer and $\mathfrak{m}$ be a modulus. Let $T$ be an $S$-scheme. Consider a datum $(\mathcal{L}, \psi)$ such that
\begin{itemize}
    \item $\mathcal{L}$ is an invertible sheaf of $\text{deg} = d$ on $C_T$.
    \item $\psi$ is an isomorphism $\mathcal{O}_{m_T} \to \mathcal{L}|_{m_T}$.
\end{itemize}
We say that two such data $(\mathcal{L}, \psi)$ and $(\mathcal{L}', \psi')$ are isomorphic if there exists an isomorphism of invertible sheaves $f : \mathcal{L} \to \mathcal{L}'$ making the following diagram commutes
\[
\begin{tikzcd}
& \mathcal{O}_{m_T} \arrow[dl, "\psi'"] \arrow[dr, "\psi"] & \\
\mathcal{L}'|_{m_T} \arrow[rr, "f|_{m_T}"] & & \mathcal{L}|_{m_T}
\end{tikzcd}
\]
For an $S$-scheme $T$, define a set
\[
\text{Pic}_{C, \mathfrak{m}}^{d, \text{pre}}(T) := \{\text{the isomorphism class of } (\mathcal{L}, \psi) \text{ defined as above}\}.
\]
$\text{Pic}_{C, \mathfrak{m}}^{d, \text{pre}}$ extends in an obvious way to a presheaf on $\text{Sch}/S$, which we denote by $\text{Pic}_{C, \mathfrak{m}}^{d, \text{pre}}$ also. Define $\text{Pic}_{C, \mathfrak{m}}^{d}$ as the \'etale sheafification of $\text{Pic}_{C, \mathfrak{m}}^{d, \text{pre}}$. Their fundamental properties which we use without proofs are:
\begin{itemize}
    \item $\text{Pic}_{C, \mathfrak{m}}^{d}$ are represented by $S$-schemes. When $\mathfrak{m}$ is faithfully flat over $S$, $\text{Pic}_{C, \mathfrak{m}}^{d, \text{pre}}$ are already \'etale sheaves.
    \item $\text{Pic}_{C, \mathfrak{m}}^{0}$ is a smooth commutative group $S$-scheme with geometrically connected fibers.
    \item \item $\text{Pic}_{C, \mathfrak{m}}^{d}$ are $\text{Pic}_{C, \mathfrak{m}}^{0}$-torsors.
\end{itemize}

$\text{Pic}_{C, \mathfrak{m}}^{0}$ is called the generalized jacobian variety of $C$ with modulus $\mathfrak{m}$. When $\mathfrak{m} = 0$,
 this is the jacobian variety of $C$. 
 In this case, we also denote $\text{Pic}_C^d$ for $\text{Pic}_{C, \mathfrak{m}}^d$. 

 \textcolor{red}{maybe here, instead of saying what he says, just summarize, and refer to it/don't actually include it}
For a finite flat $S$-scheme of finite presentation $D$, define a presheaf $\mathcal{O}_D^\times$ on $\text{Sch}/S$ by sending an $S$-scheme $T$ to the multiplicative group $\Gamma(T, \mathcal{O}_{D \times_S T}^\times)$, which is called the Weil restriction of $\mathbb{G}_{m, D}$ to $S$. This is an \'etale sheaf, and represented by a smooth group $S$-scheme. When $D = S$, this is $\mathbb{G}_{m, S}$. Define a map $\mathbb{G}_{m, S} \to \mathcal{O}_D^\times$ from the map of $S$-schemes $D \to S$. When $\text{deg } D$ is strictly positive everywhere on $S$, this is an injection of \'etale sheaves.

Consider a map $\mathcal{O}_\mathfrak{m}^\times \to \text{Pic}_{C, \mathfrak{m}}^0$ sending $s \in \mathcal{O}_\mathfrak{m}^\times$ to the pair $(\mathcal{O}_C, \mathcal{O}_\mathfrak{m} \to \mathcal{O}_\mathfrak{m})$. The image of this map coincides with the kernel of the map $\text{Pic}_{C, \mathfrak{m}}^0 \to \text{Pic}_C^0$, and the kernel of the map $\mathcal{O}_\mathfrak{m}^\times \to \text{Pic}_{C, \mathfrak{m}}^0$ is the image of $\mathbb{G}_{m, S} \to \mathcal{O}_\mathfrak{m}^\times$ induced by the morphism of $S$-schemes $\mathfrak{m} \to S$. In summary, if $\text{deg } \mathfrak{m}$ is everywhere strictly positive, we have a short exact sequence:
\[
0 \to \mathcal{O}_\mathfrak{m}^\times / \mathbb{G}_{m, S} \to \text{Pic}_{C, \mathfrak{m}}^0 \to \text{Pic}_C^0 \to 0.
\]
In particular, when $C \to S$ has a section $P : S \to C$, $\text{Pic}_{C, P}^0$ is isomorphic to $\text{Pic}_C^0$. In this case, $\text{Pic}_C^d$ has an expression as a sheaf which does not depend on the choice of $P$. 
$T$ be an $S$-scheme, and $\mathcal{L}_1$ and $\mathcal{L}_2$ are invertible sheaves of $\text{deg} = d$ on $C_T$. Define an equivalence relation on $\text{Pic}_C^{d, \text{pre}}$ such that $\mathcal{L}_1$ and $\mathcal{L}_2$ are equivalent if and only if there exists an invertible sheaf $M$ on $T$ such that $\mathcal{L}_1 \cong \mathcal{L}_2 \otimes \text{pr}^* M$. If $C \to S$ has a section, the quotient presheaf of $\text{Pic}_C^{d, \text{pre}}$ by this equivalence relation is an \'etale sheaf and coincides with the \'etale sheafification of $\text{Pic}_C^{d, \text{pre}}$ via the natural surjection. In particular, the identity map $\text{Pic}_C^d \to \text{Pic}_C^d$ corresponds to an equivalence class of invertible sheaves on $C \times_S \text{Pic}_C^d$. In this paper, we call this class the universal class of invertible sheaves of $\text{deg} = d$.

From now on we fix a modulus $\mdls$ which is everywhere strictly positive. Then, $\text{Pic}_{C, \mathfrak{m}}^d$ has an explicit expression as a sheaf, as explained before.

Denote the genus of $C$ by $g$. This is a locally constant function on $S$. We consider a condition on an integer $d$ as below:
\begin{equation}\label{equation:degree-condition}
d \geq \max\{2g - 1 + \deg {\mathfrak{m}}, \deg {\mathfrak{m}}\}.
\end{equation}
When $S$ is quasi-compact, such a $d$ always exists. 

Fix an integer $d$ satisfying the condition above. Let $T$ be an $S$-scheme and $\mathcal{L}$ be an invertible sheaf of $\deg = d$ on $C_T$. 
One can show that $\text{pr}_* \mathcal{L}(-\tilde{\mathfrak{m}})$ and $\text{pr}_* \mathcal{L}$ are locally free sheaves and their formations commute 
with any base change, i.e. for any morphism of $S$-schemes $f : T' \to T$, the base change morphisms $f^* \text{pr}_* \mathcal{L} \to \text{pr}_* f^* \mathcal{L}$ 
and $f^* \text{pr}_* (\mathcal{L}(-\tilde{\mathfrak{m}})) \to \text{pr}_* f^* (\mathcal{L}(-\tilde{\mathfrak{m}}))$ are isomorphisms. 

Moreover  (from \cite{Guignard2018} ) one can show that if $\cL$ is invertible $\cO_C$-module with degree $d$ on each fiber of $f$ 
Then, the $\cO_S$-module $pr_* \cL$ is locally free of rank $d-g+1$


\textcolor{red}{
    \begin{enumerate}
        \item It may be helpful to consult Milne's "Abelian Varieties" for further background and confidence in these constructions. https://www.jmilne.org/math/CourseNotes/AV.pdf
    \end{enumerate}
}

\subsection{Abel-Jacobi Map}
We copy from \cite{Toth2011}
We assume $d$ as in \cref{equation:degree-condition}
\begin{definition}[Abel-Jacobi Map of degree d]
    
\end{definition}
The \textit{Abel-Jacobi map of degree d}
\[
\Phi_d: \text{Div}_C^d \to \text{Pic}_C^d
\]
is defined for a scheme $T$ over $\text{Spec}(k)$ and for a relative effective Cartier divisor $D$ of degree $d$ on $(C \times_{\text{Spec}(k)} T)/T$ by

\[
\Phi_d(T)(D) := [\mathcal{O}(D)]
\]
where $[\mathcal{O}(D)]$ is the class of the invertible sheaf $\mathcal{O}(D)$ on $(C \times_{\text{Spec}(k)} T)/T$. Equivalently if $D$ is represented by the pair $(\mathcal{G}, s)$ (1.2.6), then the Abel-Jacobi map is given by
\[
\Phi_d{A}_d(T)((\mathcal{G}, s)) := [\mathcal{G}].
\]
A theorem in milne \cite{milneAV} shows that over a field $k$, for any $d\geq 1$ we have that 
the functor $\text{Div}_C^d$ is representable by the $d$-th symmetric power $C^{(d)}$. 

\subsubsection{Generalized Effective Cartier Divisors}
We copy from \cite{Guignard2018}
Let $f : X \to S$ be as in 4.1, 
and let $i : Y \to X$ be a closed subscheme of $X$, which is finite locally free over $S$ of degree $N \ge 1$, 
and let $U = X \setminus Y$ be its complement. A \textbf{Y-trivial effective Cartier divisor of degree d on X} is a pair $(\mathcal{L}, \sigma)$ such that $\mathcal{L}$ is a locally free $\mathcal{O}_X$-module of rank 1 and $\sigma : \mathcal{O}_X \to \mathcal{L}$ is an injective homomorphism such that $i^*\sigma$ is an isomorphism and such that the closed subscheme $V(\sigma)$ of $X$ defined by the vanishing of the ideal $\sigma\mathcal{L}^{-1}$ of $\mathcal{O}_X$ is finite locally free of rank $d$ over $S$. Two $Y$-trivial effective divisors $(\mathcal{L}, \sigma)$ and $(\mathcal{L}', \sigma')$ are \textbf{equivalent} if there is an isomorphism $\beta : \mathcal{L} \to \mathcal{L}'$ of $\mathcal{O}_X$-modules such that $\beta\sigma = \sigma'$. As in 4.7, if such an isomorphism exists then it is unique.


\begin{enumerate}
    \item The notation here from giroud of $f$ is as $pr$ in takeuchi, so we need to make it precise in coheret
    \item Some places they work over a family of curves, and some places over $k$. - make this uniform as well. 
    \item Here gioured use the term $Y$, we need to formulate it like $\mdls$
\end{enumerate}
\subsection{The Blowup}


\subsection{Other}
\textcolor{red}{add reference or proof} $\cF^{(d)}$ can be extended to a sheaf on this compactification with tame ramification along the boundary.
The compactification, denoted by $\tilde{C}^{(d)}_\mdls$ is fibered over $\PicCm[d]$ with fibers isomorphic to projective spaces. 
Hence, \textcolor{red}{show/add ref/add explanation} 

Here we will show 
The compactification has  

blow-up $X_\mdls$ of $C^{(d)}$ along a the closed subscheme $Z_0 \subset C^{(d)}$ defined by the map:







$$
\begin{aligned}
\phi \colon C^{(d - \deg \mdls)} &\to C^{(d)} \\
E &\mapsto E + \mdls
\end{aligned}
$$

He shows the existence of a commutative diagram:
\begin{equation}
\begin{tikzcd}
    % Row 1
    & \tilde{C}_m^{(d_m)} \arrow[r] \arrow[d] \arrow[dr, phantom, "\square"]
    & \text{Pic}_{C,m}^{d_m} \arrow[d, "(3.7)"'] \\
    % Row 2
    X_m \arrow[r, "\cong"] \arrow[dr]
    & \mathbb{P}(\mathcal{E}_m) \arrow[r] \arrow[d, "(3.5)"']
    & P_m^{d_m} \arrow[d, "(3.2)"'] \\
    % Row 3
    & C^{(d_m)}
    & \text{Pic}_C^{d_m}
\end{tikzcd}
\end{equation}

\textcolor{red}{
    \begin{enumerate}
        \item Here, C is defined over S, a family of curves, usually?
        \item Maybe add a section about Symmetric Powers of a curve? probably should.
    \end{enumerate}
}
