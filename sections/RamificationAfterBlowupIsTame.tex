\section{Ramification After Blowup Is Tame}\label{section:ramification_after_blowup_is_tame}
Throughout this section let $C$ be a projective smooth geometrically connected curve over $k$.
Let $\mdls = \sum n_i P_i$ be a fixed modulus. Let $U=C \setminus \mdls$.
For any modulus $\ndls \subset \mdls$, we define $Z_\ndls$ as the closed subscheme of $C^{(\deg \ndls)} = \Sym_{k}^{\deg \ndls}(C)$ 
defined by $\ndls$ as a point of $C^{(\deg \ndls)}$
We then define $X_\ndls$ as the blowup of $C^{(\deg \ndls)}$ at $Z_\ndls$,
and we denote by $E_\ndls = Z_\ndls \times_{C^{(d)}} X_\ndls$ the exceptional divisor of this blowup, it is irreducible of codimension 1.
We denote by $\eta_\ndls$ the generic point of $E_\ndls$. 

Diagrammatically:
\begin{equation}\label{equation:blowup_diagram_ndls}
  \begin{tikzcd}
\overline{\{\eta_\ndls \}} =  E_\ndls \arrow[r, hook] \arrow[d] & X_\ndls \arrow[d, "\pi_\ndls"] \\
Z_\ndls \arrow[r, hook]           & C^{(\deg \ndls)}          
\end{tikzcd}   
\end{equation}

Our main goal in this section is to prove \Cref{theorem:SymmetricPowerOfSheavesIsTamelyRamifiedReduction}:
\begin{theorem}\label{theorem:torsors_after_blowup_is_tame}
    Let $G$ be a finite abelian group. And let $\cP \to U$ be a $G$-torsor in $U_{\text{\'Et}}$ with ramification bounded by $\mdls = \sum n_i P_i$.
    Let $\ndls = n_i P_i \subset \mdls$ be a sub modulus of $\mdls$. Then $\cP^{(\deg \ndls)}$ is tamely ramified at $\eta_\ndls$.
\end{theorem}

Our approach proceeds in three stages. 
First, we establish the result for the cyclic case 
$G = \mathbb{Z}/p\mathbb{Z}$ and for groups $G$ where $\gcd(|G|, p) = 1$.
Next, we extend the proof to cyclic $p$-groups $G = \mathbb{Z}/p^r\mathbb{Z}$. 
Finally, for a general finite abelian group $G$, we conclude by applying the structure theorem for finite abelian groups 
and \Cref{prop:quotient_torsors} to decompose the torsor into these constituent cases.

\subsection{\texorpdfstring{Ramification of $G$-Torsors}{Ramification of G-Torsors}}

Generally speaking, assume we are given a locally Noetherian scheme $X$, 
a dense open $U \subset X$, a finite \'etale morphsm $f: Y \to U$ and a prime divisor $Z \subset X$ such that $Z \cap U = \emptyset$
and the local ring $\Oo_{X, \xi}$ of $X$ at the generic point $\xi$ of $Z$ is a discrete valuation ring. 
Setting $K_\xi = \Frac(\Oo_{X, \xi})$ we obtain a cartesian square
\[ \xymatrix{ \mathop{\mathrm{Spec}}(K_\xi ) \ar[r] \ar[d] & U \ar[d] \\ \mathop{\mathrm{Spec}}(\mathcal{O}_{X, \xi }) \ar[r] & X } \]
of schemes. 
In particular, we see that $Y \times _ U \mathop{\mathrm{Spec}}(K_\xi )$ is the spectrum of a finite separable algebra $L_\xi /K_\xi $. 
\begin{definition}\label{definition:tamely_ramified_in_codim_1}
We say: 
\begin{enumerate}
    \item $Y$ is \textit{unramified} over $X$ at $(Z, \xi)$ resp. $Y$ is \textit{tamely ramified} over $X$ at $(Z, \xi)$
        if $L_\xi /K_\xi $ is unramified, resp. tamely ramified with respect to $\mathcal{O}_{X, \xi }$
        (\Cref{definition:tamely_ramified_dvrs})
    
    \item $Y$ is unramified over $X$ in codimension $1$, resp. $Y$ is tamely ramified over $X$ in codimension $1$ if $L_\xi /K_\xi $ is unramified, resp. tamely ramified 
    with respect to $\mathcal{O}_{X, \xi }$ for every $(Z, \xi )$ as above. 
\end{enumerate}

More precisely, we decompose $L_\xi $ into a product of finite separable field extensions of $K_\xi $ and we require each of these to be unramified, resp. tamely ramified with 
respect to $\mathcal{O}_{X, \xi }$.

\end{definition}

Now back to our orginial assumptions, $C$ is projective smooth geometrically connected curve
over a field $k$ and $U=C\setminus \mdls$. Every $(Z, \xi)$ is now a closed point $P$ and $\Oo_{C, P}$ is a discrete valuation ring. 
Assume $f: Y \to U$ is actually $G$-torsor $\cP \to U$ in $U_{\text{\'Et}}$ for $G$ finite abelian group.

By passing to the completion $\widehat{\Oo}_{C,P}$, we obtain the complete local field $\widehat{K}_P$.
Restricting the torsor $\cP$ to the punctured formal neighborhood $\Spec(\widehat{K}_P)$
 allows us to study the ramification in a strictly local context. 
 This restriction yields a $G$-torsor over a field, which necessarily decomposes into a disjoint union of spectra of finite separable field extensions
 $$ \cP|_{\Spec(\widehat{K}_P)} \cong \bigsqcup \Spec(F) $$
 such that:
 \begin{enumerate}
    \item These extensions $F/\widehat{K}_P$ are all isomorphic and Galois. 
    \item The Galois group $H = \Gal(F/\widehat{K}_P)$ identifies with the stabilizer of any connected component of the pullback $\cP|_{\Spec(\widehat{K}_P)}$, 
     which is a subgroup of $G$. 
    \item  The number of such components is then given by the index $[G:H]$.
 \end{enumerate}

 \begin{definition}\label{definition:local_ramification_boundness}
    We say that the extension $F/\widehat{K}_P$ \textit{has ramification bounded by $r$} if the ramification group $H^r$ is trivial. 
    We say \textit{$\cP$ has ramification bounded by $r$ at $P$} if the corresponding local field extensions $F/\widehat{K}_P$  
    satisfy this condition.
 \end{definition}


\begin{definition}
    Let $\mdls = \sum n_i P_i$ be a modulus on $C$. We say the $G$-torsor $\cP \to U$ has 
    \textbf{ramification bounded by $\mdls$} if, for each point $P_i$, the local restriction 
    $\cP|_{\Spec(\widehat{K}_{P_i})}$ has ramification bounded by the coefficient $n_i$.
\end{definition}

%%%%%%%%%%%%%%%%%%%%%%%%%%%%%%%%%%%%%%%
This local property remains invariant under base change to the separable closure.
Let $\bar{s} = \Spec(\bar{k}) \to \Spec(k)$ 
be a geometric point. 
The higher ramification groups for $\cP|_{\Spec(\widehat{K}_P)}$ are isomorphic to those of 
the geometric restriction $\cP_{\bar{k}}$ at any point $\bar{x}$ lying over $P$. 
Thus an equivalent defintion:
\begin{definition}
    Let $\mdls = \sum n_i P_i$ be a modulus on $C$. 
    The $G$-torsor $\cP \to U$ has \textbf{ramification bounded by $\mdls$} 
    if for every geometric point $\bar{x}$ of $\mdls$ (with image $\bar{s}$ in $\Spec(k)$), 
    the restriction of $\cP$ to$$ \Spec(\widehat{\mathcal{O}}_{C_{\bar{k}},\bar{x}}) \times_{C_{\bar{k}}} U_{\bar{k}} $$
    has ramification bounded by the multiplicity of $\mdls_{\bar{s}}$ at $\bar{x}$ in the sense of 
    \Cref{definition:local_ramification_boundness}.
\end{definition}
These definitions are equivalent because $\widehat{\mathcal{O}}_{C_{\bar{k}},\bar{x}} \cong \widehat{\mathcal{O}}_{C,P} \otimes_k \bar{k}$. Furthermore, the notions of tame ramification and unramifiedness derived here coincide with the codimension-$1$ criteria in \Cref{definition:tamely_ramified_in_codim_1}.

There is another equiviliant formulation through the language of characters.
The group $G$, being a finite abelian group over $k$, is \'etale. 
Consequently, the correspondence between $G$-torsors and the fundamental group allows us to describe $G$-torsors
 via characters. Specifically, there is an isomorphism: (\Cref{cor:abelian_equivilance_fundamental_torsors})
\[ \mathrm{Hom}_{\mathrm{cont.}}(\pi_1^{\text{\'et}}(U, \overline{x}), G) \xrightarrow{\cong} \mathrm{Tors}(U_{\text{\'et}}, G) \]

In our local setting, where we consider the restriction to the complete local field $\widehat{K}_P$, 
the fundamental group identifies with the absolute Galois group 
$G_{\widehat{K}_P} := \mathrm{Gal}({\widehat{K}_P}^{\mathrm{sep}}/{\widehat{K}_P})$. 
Thus, the restricted $G$-torsor $\mathcal{P}|_{\mathrm{Spec}(\widehat{K}_P)}$ corresponds to a unique continuous homomorphism:
\[ \rho : G_{\widehat{K}_P} \to G \]

Under this correspondence, the torsor $\mathcal{P}$ has ramification bounded by $r$ at $P$ 
if and only if the homomorphism $\rho$ trivializes the $r$-th ramification group in the upper numbering:
\[ \rho(G_{\widehat{K}_P}^r) = \{ 1 \} \]

\subsubsection*{Basic Properties of Ramification of \texorpdfstring{$G$-Torsors}{G-Torsors}}
We now prove some elementary lemmas regarding the ramification of $G$-torsors.
\textcolor{red}{fix this if you have time}
\begin{lemma}\label{lemma:bounding_ramification_of_contracted_product}
    Let $G$ be a finite abelian group and $X$ be a locally Noetherian scheme over a field $k$. 
    Let $U \subset X$ be a dense open subset and let $(Z, \xi)$ be a prime divisor in the complement $X \setminus U$.
    Assume $\cP_1$ and $\cP_2$ are two $G$-torsors on $U_{\text{ét}}$, Such that $\cP_1$ has ramification bounded by $r_1$ at $(Z, \xi)$, and $\cP_2$ has ramification bounded by $r_2$ at $(Z, \xi)$.
    Then their contracted product $\cP_1 \times^G \cP_2$ has ramification bounded by $max(r_1, r_2)$ at $(Z, \xi)$.
\end{lemma}
\begin{proof}
    Let $A=\widehat{\Oo_{X, \xi}}$ and let $K=Frac(A)$.
    Let $\rho_1, \rho_2: G_K \to G$ be the associated continious homomorphisms corresponding to the $G$-torsors 
    $\cP_1|_{\Spec(K)}$, $\cP_2|_{\Spec(K)}$.
    We finish by noticing that the associated character of $(\cP_1 \times^G \cP_2)|_{\Spec(K)}$ is $\rho=\rho_1 + \rho_2$.    
\end{proof}

\begin{lemma}\label{lemma:descend_ramification_along_etale}

Let $f: X \to Y$ be a morphism of locally Noetherian schemes. Let $U_X \subset X$ and $U_Y \subset Y$ be dense open subschemes such that $f^{-1}(U_Y) \subset U_X$. 

Let $(Z_X, \xi_X)$ and $(Z_Y, \xi_Y)$ be prime divisors of $X$ and $Y$ such that $Z_X \cap U_X = \emptyset$ and $Z_Y \cap U_Y = \emptyset$. 
Suppose $f(\xi_X) = \xi_Y$ and that $f$ is étale at $\xi_X$.
Let $\mathcal{P}$ be a $G$-torsor over $U_Y$, and let $f^{-1} \mathcal{P}$ be its pullback to $f^{-1}(U_Y) \subset U_X$. 
Then, the ramification of $f^{-1} \mathcal{P}$ at $\xi_X$ is bounded by $r$ if and only if the ramification of $\mathcal{P}$ at $\xi_Y$ is bounded by $r$.
\end{lemma}
\begin{proof}
The boundedness of ramification is determined by the behavior of the torsor over the completion of the local rings at the generic points. 
Let $A = \mathcal{O}_{Y, \xi}$ and $B = \mathcal{O}_{X, \xi_X}$ be the discrete valuation rings at the generic points, with fraction fields 
$K$ and $L$ respectively. Since $f$ is étale at $\xi_X$, the map $A \to B$ is a flat, unramified (hence weakly unramified) local homomorphism. 
Consequently, the extension of completions $\widehat{L} / \widehat{K}$ is a finite unramified extension of complete discretely valued fields.
We finish by noticing that the upper numbering filtration on the absolute Galois group is compatible with unramified base change.\footnote{
 Specifically, let $G_K = \operatorname{Gal}(K^{sep}/K)$ and $G_L = \operatorname{Gal}(L^{sep}/L)$. For an unramified extension, the Herbrand function is the identity, which implies that for any $r \geq 0$:
\[ G_L^r = G_K^r \cap G_L \]}
\end{proof}

% \subsection{The Local Ring \texorpdfstring{$\Oo_{\eta_\ndls}$}{at the generic point of the blowup}}
% We remind the settings of this section: $C$ is a projective smooth geometrically connected curve over $k$.
% $\mdls = \sum n_i P_i$ is a fixed modulus. $U=C \setminus \mdls$.
% $\ndls \subset \mdls$ is a submodulus, $Z_\ndls$ is the closed subscheme of $C^{(\deg \ndls)} = \Sym_{k}^{\deg \ndls}(C)$ 
% defined by $\ndls$ as a point of $C^{(\deg \ndls)}$.
% $X_\ndls$ is the blowup of $C^{(\deg \ndls)}$ along $Z_\ndls$, $E_\ndls = Z_\ndls \times_{C^{(d)}} X_\ndls$ 
% is the exceptional divisor of this blowup, and $\eta_\ndls$ is the generic point of $E_\ndls$. 

% Diagrammatically:
% \begin{equation}\label{equation:blowup_diagram_ndls}
%   \begin{tikzcd}
% \overline{\{\eta_\ndls \}} =  E_\ndls \arrow[r, hook] \arrow[d] & X_\ndls \arrow[d, "\pi_\ndls"] \\
% Z_\ndls \arrow[r, hook]           & C^{(\deg \ndls)}          
% \end{tikzcd}   
% \end{equation}

% We now assume $P=P_i$, $n = n_i$ and $\ndls = n P$.

% Let $x \in \Oo_{C, P}$ be a local coordinate (uniformizer) at $P$.
% $C^{(\deg \ndls)}$ is a smooth variety of dimension $n$, 
% and $\ndls \in C^{(\deg \ndls)}$ is the point "serving as the origin" in this local context.

% Near $\ndls$, the coordinate ring of $C^{(\deg \ndls)}$ is generated by the elementary symmetric polynomials in 
% the local coordinates of the $n$ copies of $C$. If $x_1, \dots, x_n$ are the pullbacks of $x$ to $C^n$, 
% the local coordinates for $X$ at $\ndls$ are:
% $$e_1 = \sum_{i=1}^n x_i, \quad e_2 = \sum_{1 \le i < j \le n} x_i x_j, \dots, \quad e_n = x_1 x_2 \dots x_n$$


% Recall that $X_{\ndls} \subset C^{(\deg \ndls)} \times_k \mathbb{P}^{n-1}_k$ is defined locally by the equations $e_i u_j = e_j u_i$, where $[u_1 : \dots : u_n]$ are the homogeneous coordinates of $\mathbb{P}^{n-1}_k$.
% The exceptional divisor $E_{\ndls} \subset X_{\ndls}$ is the fiber over $\ndls$, $E_{\ndls} = \{ (\ndls, [u_1 : \dots : u_n]) \}$, which is isomorphic to $\mathbb{P}^{n-1}_k$. 
% Let $R = \mathcal{O}_{X_{\ndls}, \eta_{\ndls}}$. This ring $R$ is a discrete valuation ring (DVR) with fraction field $K = K(X) = k(C^n)^{S_n}$.
% Its residue field is  $\kappa(\eta_{\ndls}) = k(\frac{e_2}{e_1}, \dots, \frac{e_n}{e_1})$, 
% and the completion of $R$ with respect to its maximal ideal is:
% $$\hat{R} = \kappa(\eta_{\ndls})[[e_1]]$$
% In this local ring, the first elementary symmetric polynomial $e_1$ is a uniformizer. Any other coordinate $e_i$ is also a uniformizer as $e_i = (\frac{u_i}{u_1}) e_1$ and $\frac{u_i}{u_1}$ is a unit in $R$.

% For any symmetric polynomial $f \in k[x_1, \dots, x_n]^{S_n} = \Oo{C^{(n), \ndls}}$, we can write $f$ as a polynomial $G(e_1, \dots, e_n)$. If we expand $G$ into its homogeneous parts with respect to the $e_i$ coordinates, $G = G_d + G_{d+1} + \dots$, where $G_i$ is homogeneous of degree $i$ in $(e_1, \dots, e_n)$, the valuation $\nu_E$ associated with $E$ satisfies:
% $$\nu_E(f) = d$$
% This $d$ corresponds to the order of vanishing of the symmetric function $f$ at the point $\ndls$ in the symmetric product.


% \subsubsection*{The Affine Case}
% Let $X = \mathbb{A}_k^n$ be the affine $n$-space over a field $k$, and let $0 \in X$ be the origin. 
% Let $\tilde{X} = \text{Bl}_0(X) = X_{\ndls}$ be the blowup of $X$ at the origin. 
% Recall that $\tilde{X} \subset X \times_k \mathbb{P}^{n-1}_k$ is defined by the equations $x_i u_j = x_j u_i$, where $[u_1 : \dots : u_n]$ are the homogeneous coordinates of 
% $\mathbb{P}^{n-1}_k$.
% The exceptional divisor $E \subset \tilde{X}$ is the fiber over the origin, $E = \{ (0, [u_1 : \dots : u_n]) \}$, which is of codimension 1 in $\tilde{X}$. 
% Let $\eta \in E$ be the generic point of $E$, and let $R = \mathcal{O}_{\tilde{X}, \eta}$ be the associated local ring. 
% This ring $R$ is a discrete valuation ring (DVR) with fraction field $K = K(X) = k(x_1, \dots, x_n)$.

% On the affine chart $U_1$ where $u_1 \neq 0$, we have $x_i = \frac{u_i}{u_1} x_1$. 
% The coordinate ring is:$$\mathcal{O}_{\tilde{X}}(U_1) = k\left[x_1, \frac{u_2}{u_1}, \dots, \frac{u_n}{u_1}\right]$$
% In this chart, the generic point $\eta$ corresponds to the prime ideal $\mathfrak{p}_1 = (x_1)$. Thus, the local ring is $R = k[x_1, \frac{u_2}{u_1}, \dots, \frac{u_n}{u_1}]_{(x_1)}$. 
% The residue field is $\kappa(\eta) = k(\frac{u_2}{u_1}, \dots, \frac{u_n}{u_1})$, and the completion of $R$ with respect to its maximal ideal is:
% $$\hat{R} = \kappa(\eta)[[x_1]]$$
% In this local ring, $x_1$ is a uniformizer. Note that any $x_i$ (for $i > 1$) can also serve as a uniformizer, 
% as $x_i = (\frac{u_i}{u_1}) x_1$ and $\frac{u_i}{u_1}$ is a unit in $R$.

% For any monomial $M = x_1^{a_1} \dots x_n^{a_n} \in K$, we can write:
% $$M = x_1^{\sum a_i} \left( \frac{u_2}{u_1} \right)^{a_2} \dots \left( \frac{u_n}{u_1} \right)^{a_n}$$
% Since the term in parentheses is a unit in $R$, the valuation $\nu_E$ associated with $E$ satisfies:
% $$\nu_E(M) = \sum a_i = \deg M$$
% Consequently, for any polynomial $f = f_d + f_{d+1} + \dots + f_l$, where $f_i$ is the homogeneous part of degree $i$, we have $\nu_E(f) = d$ (the order of vanishing at the origin).

%\subsubsection*{The Symmetric Product Case}

% \textcolor{red}{in the above replace $n$ with $d$}
% \subsection{Proof of Theorem \ref{theorem:SymmetricPowerOfSheavesIsTamelyRamifiedReduction}}
% We assume $k$ is algebraically closed (We can base change by the unramified $\Spec{(k^{sep})} \to \Spec{(k)}$)
% We begin with:

% \begin{theorem}
%     Assume $G=\Z/p\Z$ and let $\cP \to U$ be a $G$-torsor which is wildly ramified at $P$ with ramification bounded by $d$.
%     Let $\ndls = dP$, the $G$-torsor $\cP^{(d)} \to U^{(d)}$ is unramified at $\eta_\ndls$- the generic point of 
%     $E_\mdls \subset X_{\mdls}$.
% \end{theorem}
% \begin{proof}
%     In order to be constructive, and present simpler proofs, we assume that 
%     $C=\bP_k^1$, $\mdls = d \cdot 0$. $U$

%     The computation that we will present is formal--local at $P$, , 
%     Once you replace $\mathbf{P}^1$ by an arbitrary smooth curve $C$, 
%     nothing essential changes at the generic point of the exceptional divisor, 
%     because (after choosing a uniformizer $x$ at $P$) the completed local ring $\widehat{\mathcal{O}}_{C,P}$ 
%     is still $k[[x]]$, and all the ``$d$--fold / symmetric / blowup'' constructions you use 
%     are compatible with \'etale (indeed formal) localization.

%     the general result will then be by 
% $\bG_m = U \subset U' = C \setminus \mdls$
%     Let $x$ be a local coordinate for $\Oo_{C, P}$, then 
%     $R=\widehat{\Oo_{C, P}} = k[[x]]$ and $K=\Frac{R}=k((x))$
%     Since $\Z/p\Z$ is simple, $\cP$ is either connected or a finite union of identity morphisms. The latter case being redundant. 
%     So, by \Cref{theorem:artin_schreier_ramification} $\cP|_{K} = \Spec(F)$, 
%     Where $F=\frac{K[X]}{(X^p - X - f(x))}$ and $f(x) = c x^{-m} + a_{-m+1}x^{-m+1} + \dots a_{-1}x^{-1} + a_0 \in k((x))$ with $m < d$.
%     Thus, The ring corresponding to $\widehat{O_{C^{(d)}, (P, \dots P)}}$ is 
%     $R' = k[[x_1, \dots, x_n]]$ with fraction field $K' = k((x_1, \dots, x_n))$.
%     And the pullback of field.    $p_1^{-1}(\cP) \times \dots \times p_d^{-1}(\cP)$ on $C^{d}$
%     Is 



%     In the previous section, we saw that when $x$ is local coordinate for $\Oo_{C, P}$ (a uniformizer) at $P$. 
%     Then $$\hat{R} = k(\frac{e_2}{e_1}, \dots, \frac{e_n}{e_1})[[e_1]]$$ where $R = \mathcal{O}_{X_{\ndls}, \eta_{\ndls}}$.
%     Thus if $K=Frac(R)$ then:
%     \begin{equation}\label{eq:complete_field_at_generic_point}
%           \hat{K}= k(\frac{e_1}{e_d}, \dots, \frac{e_{d-1}}{u_d})((e_d))
%      \end{equation}
    
    
% \end{proof}



\subsection{Proof of Theorem \ref{theorem:SymmetricPowerOfSheavesIsTamelyRamifiedReduction}}

We argue its enough to assume $C=\bP^1_k$ and $\cP \to \bG_m \subset \bP_k^1$ is our $G$-torsor. This is essentially by employing the following principle: 

\textbf{Invaraince of Local Invariants under Formal isomorphism}:
\textit{If two schemes have isomorphic completed local rings at given points, then any construction obtained by functorial algebraic operations (products, quotients by finite groups, blowups, finite covers) produces isomorphic completed local rings at the corresponding points, and hence identical ramification behavior.}

Which applies for all of our constructions- blowups, symmetric products, etc.
So in what follows, we make our life simpler by assuming the above. 
So, let $G$ be a finite abelian group and assume $C=\bP_k^1$, $\mdls = d \cdot 0$ and 
$\bG_m = U \subset U' = C \setminus \mdls$. Then $\deg \mdls = d$
We also assume $k$ is algebraicly closed. (We can etale base change, and this doesn't change ramification.)

We start by analyzing the blowup $\tilde{X} = \text{Bl}_0(X)$ of $X = \mathbb{A}_k^n$
where $0 \in X$ be the origin. 
Recall that $\tilde{X} \subset X \times_k \mathbb{P}^{n-1}_k$ is defined by the equations $x_i u_j = x_j u_i$, where $[u_1 : \dots : u_n]$ are the homogeneous coordinates of 
$\mathbb{P}^{n-1}_k$.
The exceptional divisor $E \subset \tilde{X}$ is the fiber over the origin, $E = \{ (0, [u_1 : \dots : u_n]) \}$, which is of codimension 1 in $\tilde{X}$. 
Let $\eta \in E$ be the generic point of $E$, and let $R = \mathcal{O}_{\tilde{X}, \eta}$ be the associated local ring. 
This ring $R$ is a discrete valuation ring (DVR) with fraction field $K = K(X) = k(x_1, \dots, x_n)$.

On the affine chart $U_1$ where $u_1 \neq 0$, we have $x_i = \frac{u_i}{u_1} x_1$. 
The coordinate ring is:$$\mathcal{O}_{\tilde{X}}(U_1) = k\left[x_1, \frac{u_2}{u_1}, \dots, \frac{u_n}{u_1}\right]$$
In this chart, the generic point $\eta$ corresponds to the prime ideal $\mathfrak{p}_1 = (x_1)$. Thus, the local ring is $R = k[x_1, \frac{u_2}{u_1}, \dots, \frac{u_n}{u_1}]_{(x_1)}$. 
The residue field is $\kappa(\eta) = k(\frac{u_2}{u_1}, \dots, \frac{u_n}{u_1})$, and the completion of $R$ with respect to its maximal ideal is:
$$\hat{R} = \kappa(\eta)[[x_1]]$$
In this local ring, $x_1$ is a uniformizer. Note that any $x_i$ (for $i > 1$) can also serve as a uniformizer, 
as $x_i = (\frac{u_i}{u_1}) x_1$ and $\frac{u_i}{u_1}$ is a unit in $R$.

For any monomial $M = x_1^{a_1} \dots x_n^{a_n} \in K$, we can write:
$$M = x_1^{\sum a_i} \left( \frac{u_2}{u_1} \right)^{a_2} \dots \left( \frac{u_n}{u_1} \right)^{a_n}$$
Since the term in parentheses is a unit in $R$, the valuation $\nu_E$ associated with $E$ satisfies:
$$\nu_E(M) = \sum a_i = \deg M$$
Consequently, for any polynomial $f = f_d + f_{d+1} + \dots + f_l$, where $f_i$ is the homogeneous part of degree $i$, we have $\nu_E(f) = d$ (the order of vanishing at the origin).



Our first result is:
\begin{theorem}
     Let $\cP \to \bG_m \subset \bP_k^1$ be a $G$ torsor which is either
     \begin{enumerate}
          \item tamely ramified at $0$ 
          \item totally ramified at $0$ with $G=\Z/p\Z$ and ramificaiton bounded by $d$.
     \end{enumerate}
     Then the ramification of the $G$-torsor $\cP^{(d)} \to \bG_m^{(d)}$ at $\eta_\mdls$ the generic point of $E_\mdls \subset X_{\mdls}$ is  
     \begin{enumerate}
          \item tamely ramified if $\cP$ was tamely ramified
          \item unramified if $\cP$ was totally ramified with $G=\Z/p\Z$ and ramification bounded by $d$
     \end{enumerate}
\end{theorem}
\begin{proof}
    The local ring at the generic point of the excptional divisor of the blowup of the affine space at 0 point is
     $R = k[x_d, \frac{u_1}{u_d}, \dots, \frac{u_{d-1}}{u_d}]_{(x_d)}$. 
     Where for every  $i < d$, we have $x_i = \frac{u_i}{u_d} x_d$ are all uniformizers.
     The residue field is $\kappa(\eta) = k(\frac{u_1}{u_d}, \dots, \frac{u_{d-1}}{u_d})$
     and the completion of $R$ with respect to its maximal ideal is:
     $$\hat{R} = \kappa(\eta)[[x_d]]$$
     In our situation, when we take symmetric product of the affine space, the situation is similiar with different coordinates
     if we let $e_1, \dots, e_d$ be the symmetric polynomials in $x_1, \dots x_d$ then:
     The local ring is 
     $R = k[e_d, \frac{u_2}{u_d}, \dots, \frac{u_{d-1}}{u_d}]_{(e_d)}$. 
     $e_i = \frac{u_i}{u_d} e_d$ are all uniformizers.
     The residue field being:
     $\kappa(\eta) = k(\frac{u_1}{u_d}, \dots, \frac{u_{d-1}}{u_d})$
     and the completion: $\hat{R} = \kappa(\eta)[[s_d]]$
     Note that from  $e_i = \frac{u_i}{u_d} e_d$
     We get  $\frac{e_i}{e_d} = \frac{u_i}{u_d}$ in the fracion field. hence

     \begin{equation}\label{eq:complete_field_at_generic_point}
          \hat{K}= k(\frac{e_1}{e_d}, \dots, \frac{e_{d-1}}{u_d})((e_d))
     \end{equation}

     We compute directly the extesntion of complete valued fields over the complete valued field at the generic point. 
     Note that by  \Cref{theorem:kummer_ramification} and \Cref{theorem:artin_schreier_ramification} 
     We can assume $\cP = \Spec k[x, x^{-1}][X]/(X^n - a)$ for $a \in k[x, x^{-1}]$ or $\cP = \Spec k[x, x^{-1}][X]/(X^p + X - f(x, x^{-1}))$
     where $f(x,x^{-1}) = c x^{-m} + a_{-m + 1} x^{-m + 1} + \cdots a_{-1}x^{-1} + a_0 = c x^{-m} + f_{-m+1}(x^{-1})$ where $m < d$.
     
     We deal with each case separately. 
     
     \textbf{Artin-Schreier Extensions:}

     Set $R=k[x, x^{-1}]$, and $S = \Spec k[x, x^{-1}][X]/(X^p + X - f(x^{-1}))$
     we have 
     $$
          \begin{aligned}
          R^{\otimes_k d} &= k[x_1, x_1^{-1}, \dots, x_d, x_d^{-1}] \\
          S^{\otimes_k d} &= k[x_1, x_1^{-1}, \dots, x_d, x_d^{-1}][X_1, \dots, X_d] / (X_1^p - X_1 - f(x_1), \dots, X_d^p - X_d - f(x_d)) \\
          &= R^{\otimes_k d}[X_1, \dots, X_d] / (X_1^p - X_1 - f(x_1), \dots, X_d^p - X_d - f(x_d))
          \end{aligned}
     $$

     Next, we want to understand the ring corresponding to $p_1^{-1}(\cP) \otimes \dots \otimes p_d^{-1}(\cP)$ on $C^{d}$ - 
     the $d$'th-contracted product of the $G=\Z/p\Z$-torsors  $p_1^{-1}(\cP), \dots, p_d^{-1}(\cP)$ on $U^{d}$. 
     It corresponds to qoutient:
     $\left(p_1^{-1}(\cP) \times \dots \times p_d^{-1}(\cP)\right) / G^{d-1}$ where the action of $G^{d-1}$ on the product is:
     $$(g_1, \dots, g_{d-1}) \cdot (p_1, p_2, \dots, p_{d-1}, p_{d}) = (g_1(p_1), g_1^{-1}g_2(p_2), \dots, 
     g_{d-2}^{-1}g_{d-1}(p_{d-1}), g_{d-1}^{-1}(p_d))$$
     
     The affine ring corrsponding to the contracted product is $\left(S^{\otimes_k d}\right)^{G^{d-1}}$
    
     Recall that the action of $g \in G=\Z/p\Z$ on $X$ is $g(X) = X + g$ ($g$ correspond to a number $0 \leq g \leq p-1)$.
     So, the action of $(g_1, \dots, g_{d-1})$ on the generators $(X_1, X_2, \dots, X_{d-1}, X_{d})$
     Is $X_1 \mapsto X_1 + g_1$, $X_i \mapsto X_i - g_{i-1} + g_i$ for $1<i<d$ and $X_d \mapsto X_d - g_{d-1}$.
     So we see that $Y=X_1 + \dots + X_d$ is invariant.
     Moreover $Y^P - Y - \sum_{i=1}^d f(x_i)=0$ is irreducible degree $p$ equation for $Y$,
     Since we are quotienting a rank $p^{d}$ extesntion by a group of order $p^{d-1}$ the resulting invaraint subring must
     have rank $p$ over $R^{\otimes_k d}$, So we conclude:
     \[
          \left(S^{\otimes_k d}\right)^{G^{d-1}} \cong R^{\otimes_k d}[Y] / (Y^p - Y - \sum_{i=1}^d f(x_i))
     \]  
     
     The group $S_d$ acts on 
     $ R^{\otimes_k d} = k[x_1^{\pm 1}, x_2^{\pm 1}, \dots, x_d^{\pm 1}]$
      by permuting the variables $\{x_i\}_{i=1}^d$.
     And since $Y=\sum_i^d X_i$ it leaves $Y$ invaraint. 
     The invaraint subring $(R^{\otimes_k d})^{S_d}$ is simply $k[e_1, e_2, \dots, e_d, e_d^{-1}]$
     where $\{ e_i \}$  are the symmetric polynomials in $x_1, \dots, x_d$:
     $$e_k = \sum_{1 \le j_1 < j_2 < \dots < j_k \le d} x_{j_1} x_{j_2} \dots x_{j_k}$$
     
     To find $\left(R^{\otimes_k d}[Y] / (Y^p - Y - \sum_{i=1}^d f(x_i))\right)^{S_d}$ its enough to express
     $\sum_{i=1}^d f(x_i)$ in $e_1, \dots, e_d$, this can be done with the newton polynomials, moreover, we claim the following:

     \begin{lemma}
          Let $f(x) = c x^{-m} + a_{-m + 1} x^{-m + 1} + \cdots a_{-1}x^{-1} + a_0$, 
          define  $\alpha(x_1, ..., x_d) = \sum_1^d f(x_i)$, and deonte by $e_1, ..., e_d$ the elementary symmetric polynomials 
          in $x_1, ..., x_d$. If $m < d$ then $\alpha(x_1, ..., x_d) \in k(e_1/e_d, e_2/e_d, ..., e_{d-1}/e_d)$
     \end{lemma}
     \begin{proof}
          Changing variables $y_i=x_i^{-1}$ for each $i \in \{1, \dots, d\}$
          We get 
          $$\alpha = \sum_{i=1}^d f(x_i) = \sum_{i=1}^d \left( c y_i^m + a_{-m+1} y_i^{m-1} + \dots + a_{-1} y_i + a_0 \right)$$
          Rearranging the sums, we get
          $$\alpha = c \sum_{i=1}^d y_i^m + a_{-m+1} \sum_{i=1}^d y_i^{m-1} + \dots + a_{-1} \sum_{i=1}^d y_i + d a_0$$
          Let $p_k(y_1, \dots, y_d) = \sum_{i=1}^d y_i^k$ be the $k$-th power sum symmetric polynomial. 
          The expression for $\alpha$ is a linear combination of these power sums:
          $$\alpha = c p_m(y) + a_{-m+1} p_{m-1}(y) + \dots + a_{-1} p_1(y) + d a_0$$
         
          According to the \textit{Fundamental Theorem of Symmetric Polynomials}, any symmetric polynomial in $y_1, \dots, y_d$ 
          can be expressed as a polynomial in the elementary symmetric polynomials $e_k(y_1, \dots, y_d)$. 
          Since $m < d$, $\alpha$ is a polynomial in $e_1(y), e_2(y), \dots, e_m(y)$. ($y=(y_1, \dots, y_d)$)
          The elementary symmetric polynomials in $y_i = 1/x_i$ are related to the elementary symmetric polynomials in $x_i$ as follows: 
          $$e_k(y_1, \dots, y_d) = \sum_{1 \le i_1 < \dots < i_k \le d} \frac{1}{x_{i_1} \dots x_{i_k}} = \frac{\sum_{1 \le j_1 < \dots < j_{d-k} \le d} x_{j_1} \dots x_{j_{d-k}}}{x_1 x_2 \dots x_d}$$
          Thus,
          $$e_k(y_1, \dots, y_d) = \frac{e_{d-k}(x_1, \dots, x_d)}{e_d(x_1, \dots, x_d)}$$
          which concludes the proof.
     \end{proof}

Finally, restricting $\cP^{(d)}$ to $\spec {\hat{K}}$ we get by \Cref{eq:complete_field_at_generic_point}
and \Cref{theorem:artin_schreier_ramification} the result. (that $\cP$ is unramified at the generic point of the exceptional divisor of the blowup).

\textbf{Kummer Extensions:}
Few things are different in that case,

     Set $R=k[x, x^{-1}]$, and $S=R[X]/(X^n - f)$ where $f= f(x, 1/x) \in R$
     In this case we have $char k = p$ and $gcd(p, n)=1$.

     We have 
     $$
          \begin{aligned}
          R^{\otimes_k d} &= k[x_1, x_1^{-1}, \dots, x_d, x_d^{-1}] \\
          S^{\otimes_k d} &= k[x_1, x_1^{-1}, \dots, x_d, x_d^{-1}][X_1, \dots, X_d] / (X_1^n - f_1, \dots, X_d^n - f_d) \\
          &= R^{\otimes_k d}[X_1, \dots, X_d] / (X_1^n - f_1, \dots, X_d^n - f_d)
          \end{aligned}
     $$
     Where $f_i = f(x_i, x_i^{-1})$

     Next, we want to figure out 
      $\left(S^{\otimes_k d}\right)^{G^{d-1}}$
    

     Recall that the action of $g \in G=\Z/n\Z$ on $X$ is $g(X) = \zeta^{g} X$ ($g$ correspond to a number $0 \leq g \leq n-1)$.
     So, the action of $(g_1, \dots, g_{d-1})$ on the generators $(X_1, X_2, \dots, X_{d-1}, X_{d})$
     Is $X_1 \mapsto \zeta^{g_1} X_1$, $X_i \mapsto \zeta^{g_i - g_{i-1}}X_i$ for $1<i<d$ and $X_d \mapsto \zeta^{-g_{d-1}} X_d$.

     So we see that $Y=X_1 X_2 \dots X_d$ is invariant.
     And $Y^n - \prod_{i=1}^d f_i$ is irreducible degree $n$ equation for $Y$,
     So, like before, we conclude:
     \[
          \left(S^{\otimes_k d}\right)^{G^{d-1}} \cong R^{\otimes_k d}[Y] / (Y^n - \prod_{i=1}^d f_i)
     \]  
     
     The group $S_d$ acts on $R^{\otimes_k d}[Y] / (Y^n - \prod_{i=1}^d f_i)$ by permuting the indices.
     on the variables $x_i$, 
     On $Y=\prod_1^d X_i$ it is invaraint. 
     The invaraint subring $(R^{\otimes_k d})^{S_d}$ is simply $k[e_1, e_2, \dots, e_d, e_d^{-1}]$ like before.
     

     The polynomial $F=\prod_{i=1}^d f_i$ is symmetric in $\{x_i\}_1^d$ so it can be expressed as a polynomial
     $\tilde{F}(e_1, \dots, e_d)$ in the elementary symmetric variables. 
     Hence the qoutient ring is:
     $$\left( \frac{k[x_1^{\pm 1}, \dots, t_d^{\pm 1}][Y]}{(Y^n - \prod_{i=1}^d f_i)} \right)^{S_d} \cong 
     \frac{k[e_1, \dots, e_d, e_d^{-1}][Y]}{(Y^n - \tilde{F}(e_1, \dots, e_d))}$$
     So we see again, that  restricting $\cP^{(d)}$ to $\spec {\hat{K}}$ we get by \Cref{eq:complete_field_at_generic_point} 
     and \Cref{theorem:kummer_ramification}, that $\cP$ is tamely ramified at the generic point of the exceptional divisor of the blowup.
\end{proof}

So, we conclude 
\begin{cor}
    In the general settings of a curve $C$, and $\cP \to U=C \setminus \mdls$ a $G$-torsor with ramification bounded by $\ndls = n P \subset \mdls$ at $P$, we have:
    \begin{enumerate}
        \item If $G=\Z/p\Z$ then $\cP^{(\deg \ndls)}$ is unramifeid at $\eta_\ndls$
        \item If $\cP$ is tamely ramified $\ndls$ then $\cP^{(\deg \ndls)}$ is tamely ramified at $\eta_\ndls$
    \end{enumerate}
\end{cor}

Next, we prove the following.
\begin{theorem}
    Assume $G=\Z/p^r\Z$. Then $\cP^{(\deg (\ndls))}$ is unramified at $\eta_{\ndls}$
\end{theorem}
\begin{proof}
    Again, we can assume $k$ is algebraically closed. 
    $\cP^{(\deg \ndls)}$ has bounded ramification at $n P$ by $n$. Let $R = \widehat{\Oo_{C, P}}$ and let $K=Frac(R)$
    Then $\cP^{(\deg \ndls)}|_{\Spec K}$ correspond to $\rho: \Gal(K^{sep}/K) \to G$.
    Let $H = \rho(G^{0})$ be the image of the inertia subgroup. 
    Decompose $\cP \to U$ to $\cP \to \cP_{G/H} \to U$ as in \Cref{{prop:quotient_torsors}}, where $\cP \to \cP_{G/H}$ is an $H$-torsor and $\cP_{G/H} \to U$ is a $G/H$ torsor.
    Then $\cP_{G/H}$ is unramified at $P$. Thus $\cP_{G/H}$ extends to $U' = U \cup \{ P \}$, 
    and taking any point $Q \in \cP_{G/H}$ over $P$, results in isomorphism $\widehat{\Oo_{\cP_{G/H}, Q}} \cong \widehat{\Oo_{C, P}}$.
    But because $k$ is algebraically closed, we even have equality, 
    and deduce that locally the $G/H$ action on $\widehat{\Oo_{C, P}}$ is trivial.
    Thus, we can assume $G=H$ and that $\cP \to U$ is totally ramified at $P$.
    
    \textcolor{red}{Not clear how to finish here without analyzing the ramification even further.}
    Thus, take a subgroup $H \subset G$ such that $G/H \cong \Z/p \Z$ and decompose $\cP \to U$ as:
\end{proof}

Finally, \textbf{Proof of Theorem \cref{theorem:torsors_after_blowup_is_tame}}:
Idea: decompose $G$ be the structure theorem, and finish by the above. 



\subsection{Extending To Product of Blowups}\label{subsection:ram_prod_analysis}
We conclude this sesction by proving sume auxilary results that would be useful in \Cref{section:gcft}.


%%% %%% %%% %%% %%% %%% %%% %%% %%% %%% %%% %%% %%% %%% %%% %%%
%       Behavior of Ramification under Product of Blowups
%%% %%% %%% %%% %%% %%% %%% %%% %%% %%% %%% %%% %%% %%% %%% %%%
%\subsubsection{Behavior of Ramification under Product of Blowups}

Let $X$ and $Y$ be smooth schemes over a field $k$, and let $x \in X$ and $y \in Y$ be closed points. 
We denote the blowups of these schemes at the given points by 
$\pi_X: \Bl_x(X) \to X$ and $\pi_Y: \Bl_y(Y) \to Y$. Furthermore, let $\pi_{X \times Y}: \Bl_{(x,y)}(X \times_k Y) \to X \times_k Y$ 
be the blowup of the product scheme at the point $(x,y)$.
We denote by $E_X, E_Y$, and $E_{X \times Y}$ the respective exceptional divisors, and let $\eta_X, \eta_Y$, and $\eta_{X \times Y}$ be their generic points.

In this section, we establish the following result concerning the stability of ramification bounds under the external product of torsors.

\begin{prop}\label{prop:ramification-external-product}
Let $G$ be a finite abelian group. Suppose $\cG_X$ and $\cG_Y$ are $G$-torsors defined on open subsets $U_X \subset \Bl_x(X)$ and $U_Y \subset \Bl_y(Y)$ that are disjoint 
from the exceptional divisors. 
If the ramification of $\cG_X$ at $\eta_X$ and $\cG_Y$ at $\eta_Y$ is bounded by $r$, then the external product torsors 
\[
\cG_{X \times Y} := \text{pr}_1^{-1} \cG_X \otimes \text{pr}_2^{-1} \cG_Y
\]
has ramification bounded by $r$ at the generic point $\eta_{X\times Y}$ of the exceptional divisor in the product blowup.
\end{prop}

The proposition is purely local in nature, it suffices to consider the case where $X$ and $Y$ are affine. 
More precisely, by the smoothness of $X$ and $Y$, we may restrict our attention to open neighborhoods of $x$ and $y$ that are etale over affine spaces. 
The rest of this section treats that case.

\subsubsection*{The Affine Case}
Let $X = \mathbb{A}_k^n$ be the affine $n$-space over a field $k$, and let $0 \in X$ be the origin. 
Let $\tilde{X} = \text{Bl}_0(X)$ be the blowup of $X$ at the origin. 
Recall that $\tilde{X} \subset X \times_k \mathbb{P}^{n-1}_k$ is defined by the equations $x_i u_j = x_j u_i$, where $[u_1 : \dots : u_n]$ are the homogeneous coordinates of 
$\mathbb{P}^{n-1}_k$.
The exceptional divisor $E \subset \tilde{X}$ is the fiber over the origin, $E = \{ (0, [u_1 : \dots : u_n]) \}$, which is of codimension 1 in $\tilde{X}$. 
Let $\eta \in E$ be the generic point of $E$, and let $R = \mathcal{O}_{\tilde{X}, \eta}$ be the associated local ring. 
This ring $R$ is a discrete valuation ring (DVR) with fraction field $K = K(X) = k(x_1, \dots, x_n)$.

On the affine chart $U_1$ where $u_1 \neq 0$, we have $x_i = \frac{u_i}{u_1} x_1$. 
The coordinate ring is:$$\mathcal{O}_{\tilde{X}}(U_1) = k\left[x_1, \frac{u_2}{u_1}, \dots, \frac{u_n}{u_1}\right]$$
In this chart, the generic point $\eta$ corresponds to the prime ideal $\mathfrak{p}_1 = (x_1)$. Thus, the local ring is $R = k[x_1, \frac{u_2}{u_1}, \dots, \frac{u_n}{u_1}]_{(x_1)}$. 
The residue field is $\kappa(\eta) = k(\frac{u_2}{u_1}, \dots, \frac{u_n}{u_1})$, and the completion of $R$ with respect to its maximal ideal is:
$$\hat{R} = \kappa(\eta)[[x_1]]$$
In this local ring, $x_1$ is a uniformizer. Note that any $x_i$ (for $i > 1$) can also serve as a uniformizer, 
as $x_i = (\frac{u_i}{u_1}) x_1$ and $\frac{u_i}{u_1}$ is a unit in $R$.

For any monomial $M = x_1^{a_1} \dots x_n^{a_n} \in K$, we can write:
$$M = x_1^{\sum a_i} \left( \frac{u_2}{u_1} \right)^{a_2} \dots \left( \frac{u_n}{u_1} \right)^{a_n}$$
Since the term in parentheses is a unit in $R$, the valuation $\nu_E$ associated with $E$ satisfies:
$$\nu_E(M) = \sum a_i = \deg M$$
Consequently, for any polynomial $f = f_d + f_{d+1} + \dots + f_l$, where $f_i$ is the homogeneous part of degree $i$, we have $\nu_E(f) = d$ (the order of vanishing at the origin).

\subsubsection*{The Product Case}
Now, let $X = \mathbb{A}^n$ and $Y = \mathbb{A}^m$ with origins $x=0$ and $y=0$. 
As before, $\text{Bl}_0(X) \subset X \times \mathbb{P}^{n-1}$ and $\text{Bl}_0(Y) \subset Y \times \mathbb{P}^{m-1}$ have exceptional divisors $E_X$ and $E_Y$ respectively.
Consider the product $X \times_k Y \cong \mathbb{A}_k^{n+m}$. 
The blowup of the product at the origin $(0,0)$, denoted $\text{Bl}_{(0,0)}(X \times_k Y)$, is a subscheme of $(X \times Y) \times \mathbb{P}^{n+m-1}$ defined by:

$$\begin{cases} 
x_i w_j = x_j w_i & 1 \le i, j \le n \\
y_k w_{n+l} = y_l w_{n+k} & 1 \le k, l \le m \\
x_i w_{n+j} = y_j w_i & 1 \le i \le n, \,\, 1 \le j \le m 
\end{cases}$$
where $[w_1 : \dots : w_{n+m}]$ are the homogeneous coordinates of $\mathbb{P}^{n+m-1}$. The exceptional divisor $E_{X \times Y}$ is isomorphic to $\mathbb{P}^{n+m-1}$.

\subsubsection*{Comparison of Blowups}

Both $\text{Bl}_0(X) \times_k \text{Bl}_0(Y)$ and $\text{Bl}_{(0,0)}(X \times_k Y)$ are birational to $X \times_k Y$. 
They share a common dense open set $\tilde{U}$ defined by the condition that neither the $X$-coordinates nor the $Y$-coordinates vanish simultaneously in the projective space:
$$\tilde{U} = \{ ((x,y), [w_1: \dots : w_{n+m}]) \mid (w_1, \dots, w_n) \neq 0 \text{ and } (w_{n+1}, \dots, w_{n+m}) \neq 0 \}$$This yields a diagram of open immersions:
$$ 
   \begin{tikzcd}
        & \tilde{U} \arrow[dl, "f_1"'] \arrow[dr, "f_2", hook] & \\
        \text{Bl}_0(X) \times_k \text{Bl}_0(Y) & & \text{Bl}_{(0,0)}(X \times_k Y)
    \end{tikzcd}
$$
    
    where $f_1$ maps the coordinates to the respective projectivizations $[w_1: \dots : w_n]$ and $[w_{n+1}: \dots : w_{n+m}]$.
    Also note that the generic point $\eta_{X \times Y}$  of $E_{X \times_k Y}$ is inside $\tilde{U}$

\subsubsection*{Extensions of DVRs}
Let $S$ be the local ring of the generic point $\eta_{X\times Y}$ of $E_{X \times Y}$ in $\tilde{U}$. 
On the chart where $w_1 \neq 0$ and $w_{n+1} \neq 0$, we have 
$x_1 = (\frac{w_1}{w_{n+1}}) y_1$. 
Since $\frac{w_1}{w_{n+1}}$ is a unit in this chart, $x_1$ and $y_1$ are equivalent as uniformizers.
We have:
$$
S=k\left[x_1, \frac{w_2}{w_1} \cdots, \frac{w_n}{w_1}, \frac{w_{n+1}}{w_1} \cdots \frac{w_{n+m}}{w_1}\right]_{(x_1)} = 
k\left[\frac{w_1}{w_{n+1}}, \frac{w_2}{w_{n+1}} \cdots \frac{w_n}{w_{n+1}}, y_1 \cdots \frac{w_{n+m}}{w_{n+1}}\right]_{(y_1)}
$$

$$k(\eta_{X \times Y}) = k\left(\frac{w_2}{w_1} \cdots, \frac{w_n}{w_1}, \frac{w_{n+1}}{w_1} \cdots \frac{w_{n+m}}{w_1}\right)$$

Let $R_X$ be the local ring of the exceptional divisor $E_X$ in $\text{Bl}_0(X)$. 
The pullback of $E_X \times_k \text{Bl}_0(Y)$ along $f_1$ induces an extension of DVRs $R_X \hookrightarrow S$. Which is:
\begin{enumerate}
    \item Weakly Unramified: $x_1$ is a uniformizer in both $R_X$ and $S$, so the ramification index is $e=1$.
    \item Residually Transcendental: The residue field extension $\kappa(\eta_X) \subset \kappa(\eta)$ is:
    $$k\left(\frac{w_2}{w_1}, \dots, \frac{w_n}{w_1}\right) \subset k\left(\frac{w_2}{w_1}, \dots, \frac{w_n}{w_1}, \frac{w_{n+1}}{w_1}, \dots, \frac{w_{n+m}}{w_1}\right)$$
    Hence separable.
\end{enumerate}
Since this extension is generated by transcendental elements, it is separable and formally smooth at the maximal ideal (\stackstag{09E7}).

\subsubsection*{Ramification of $G$-Torsors}
Let $G$ be a finite abelian group. 
Let $\mathcal{Q}$ be a $G$-torsor on an open $U \subset \text{Bl}_0(X)$ disjoint from $E_X$. 
%Suppose $\mathcal{Q}$ has ramification bounded by $r$ at the generic point $\eta_X$ of $E_X$.
Let $V \subset \text{Bl}_0(Y)$ be an open subscheme, and let $\pi_X: \text{Bl}_0(X) \times_k V \to \text{Bl}_0(X)$ 
be the projection onto the first factor. 
By restricting this projection to $U \times_k V$, we obtain the pullback $G$-torsor:
$$\pi_X^{-1}(\mathcal{Q}) \cong \mathcal{Q} \times_k V$$
which is defined on the open subset $U \times_k V \subset \text{Bl}_0(X) \times_k \text{Bl}_0(Y)$.
The extension of local rings $R_X \to S$ is weakly unramified (the ramification index $e=1$) and residually transcendental with seprable residue field exntesion. 
Under these conditions the ramification filtration is preserved. 
Therefore, the pullback $\pi_X^{-1}(\mathcal{Q})$ has ramification bounded by $r$ at the generic point $\eta_{X \times Y}$ of the exceptional divisor $E_{X \times Y}$ if and only if 
the original torsor $\mathcal{Q}$ has ramification bounded by $r$ at the generic point $\eta_X$ of $E_X$

And we finish by \Cref{lemma:bounding_ramification_of_contracted_product}.



