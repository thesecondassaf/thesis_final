\section{Geometric Class Field Theory}\label{section:gcft}

\subsection{Etale Fundamental Groups and Tame Fundamental Groups}
We recall the definition and basic properties of the etale fundamental group, following stacks project 
\stackstag{0BQ6}
\begin{prop}[\stackstag{0C0J}]
Let $f : X \to S$ be a flat proper morphism of finite presentation whose geometric fibres are connected and reduced. Assume $S$ is connected and let $\overline{s}$ be a geometric point of $S$. Then there is an exact sequence

\[ \pi _1(X_{\overline{s}}) \to \pi _1(X) \to \pi _1(S) \to 1 \]
of fundamental groups.
    
\end{prop}

\begin{cor}\label{cor:etale_fundamental_group_smooth_proper}
    Let $f : X \to S$ be a proper smooth morphism of finite presentation whose geometric fibres are connected. Assume $S$ is connected and let $\overline{s}$ be a geometric point of $S$. Then there is an exact sequence

    \[ \pi _1(X_{\overline{s}}) \to \pi _1(X) \to \pi _1(S) \to 1 \]
    of fundamental groups.
\end{cor}
\textcolor{red}{add about tameness?}


%
% Tame Ramification
%



\subsection{Generalized Picard Scheme}
In this section, we recall the notion of generalized Jacobian varieties and study their fundamental properties. The material presented here is primarily adapted from \cite{Guignard2018} and \cite{takeuchi2019blow}. For further background on the general theory of abelian varieties and Jacobians, the reader may also consult \cite{milneAV}.
Let $S$ be a scheme and let $C$ be a projective smooth $S$-scheme whose geometric fibers are connected and of dimension 1. Let $\mdls$ be a modulus on $C$, defined as an effective Cartier divisor of $C/S$ 
(i.e., a closed subscheme of $C$ which is finite flat of finite presentation over $S$).
We denote the projection $C \times_S T \to T$ by $\text{pr}$ for any $S$-scheme $T$.

\subsubsection*{The Functor of Points}
Let $d$ be an integer. For an $S$-scheme $T$, we consider the set of data $(\cL, \psi)$ where:
\begin{itemize}
    \item $\cL$ is an invertible sheaf of degree $d$ on $C_T$.
    \item $\psi: \Oo_{\mdls_T} \xrightarrow{\sim} \cL|_{\mdls_T}$ is a trivialization of $\cL$ along the modulus.
\end{itemize}
Two such pairs $(\cL, \psi)$ and $(\cL', \psi')$ are said to be isomorphic if there exists an isomorphism of invertible sheaves $f : \cL \to \cL'$ such that the following diagram commutes:
\[
\begin{tikzcd}
& \Oo_{\mdls_T} \arrow[dl, "\psi'"'] \arrow[dr, "\psi"] & \\
\cL'|_{\mdls_T} \arrow[rr, "f|_{\mdls_T}"] & & \cL|_{\mdls_T}
\end{tikzcd}
\]
We define the presheaf $\text{Pic}_{C, \mdls}^{d, \text{pre}}$ on $\text{Sch}/S$ by assigning to $T$ the set of isomorphism classes of such pairs. Let $\PicCm[d]$ denote the \'etale sheafification of this presheaf.

\subsubsection*{Representability and Structure}
The fundamental properties of this functor are as follows:
\begin{enumerate}
    \item $\text{Pic}_{C, \mdls}^{d}$ is represented by an $S$-scheme. (Note: If $\mdls$ is faithfully flat over $S$, the presheaf is already a \'etale sheaf).
    \item $\text{Pic}_{C, \mdls}^{0}$ is a smooth commutative group $S$-scheme with geometrically connected fibers, referred to as the \textit{generalized Jacobian variety} of $C$ with modulus $\mdls$.
    \item For any $d$, $\text{Pic}_{C, \mdls}^{d}$ is a $\text{Pic}_{C, \mdls}^{0}$-torsor.
\end{enumerate}
In the case where $\mdls = 0$, we recover the standard Jacobian variety, denoted simply as $\text{Pic}_C^d$.
\subsubsection*{Relation to the Standard Jacobian}
We now examine the behavior of the generalized Picard scheme under the variation of the modulus. By viewing the structure along the modulus as an additional rigidification, we obtain natural transition maps corresponding to the inclusion of moduli.

Let $\mdls_1$ and $\mdls_2$ be moduli such that $\mdls_1 \subset \mdls_2$. There exists a natural map 
\[
\text{Pic}_{C, \mdls_2}^d \to \text{Pic}_{C, \mdls_1}^d
\]
obtained by restricting the isomorphism $\psi$. Since $\mdls_2$ is a finite $S$-scheme, this map is a surjection as a morphism of \'etale sheaves.
In particular, for any modulus $\mdls$, there is a natural surjective morphism of \'etale sheaves:
\[
\text{Pic}_{C, \mdls}^d \to \text{Pic}_C^d.
\]


\subsubsection*{Local Freeness and Base Change}
%\textcolor{red}{there is ${\mdls}$ here - should fix that}
Let $\mdls$ be a modulus which is everywhere strictly positive. Let $g$ denote the genus of $C$, which is a locally constant function on $S$. We restrict our attention to degrees $d$ satisfying the condition:
\begin{equation} \label{equation:degree-condition}
    d \geq \max\{2g - 1 + \deg \mdls, \deg \mdls\}.
\end{equation}

Assuming $S$ is quasi-compact, such a $d$ always exists.

Fix an integer $d$ satisfying the condition above. Let $T$ be an $S$-scheme and let $\cL$ be an invertible sheaf of degree $d$ on $C_T$. One can show that the pushforwards $\mathrm{pr}_* \cL$ and $\mathrm{pr}_* \cL(-{\mdls})$ are locally free sheaves and their formations commute with any base change. Explicitly, for any morphism of $S$-schemes $f : T' \to T$, the base change morphisms are isomorphisms:
\[
    f^* \mathrm{pr}_* \cL \xrightarrow{\sim} \mathrm{pr}_* f^* \cL
\]
and
\[
    f^* \mathrm{pr}_* (\cL(-{\mdls})) \xrightarrow{\sim} \mathrm{pr}_* f^* (\cL(-{\mdls})).
\]
In particular, following \cite{Guignard2018}, if $\cL$ is invertible $\cO_C$-module with degree $d$ satisfying \ref{equation:degree-condition} on each fiber of $f$ then, $\mathrm{pr}_* \cL$ is a locally free $\mathcal{O}_S$-module of rank $d-g+1$.

For further background and verification of these constructions, we refer the reader to Milne's notes on abelian Varieties (\cite{milneAV}).

\subsection{The Abel-Jacobi Morphism and its Fibers}\label{section:AbelJacobi}


Let $U = C \setminus \mdls$ be the complement of the modulus in $C$.
The effective cartier divisors of degree $d$ which are prime to $\mdls$ are parameterized by the symmetric power $\Sym_S^d (U) = U^{(d)}$ over $S$ (See \cite{Guignard2018} Proposition 4.12, \cite{milneAV} Theorem 3.13).
For any such divisor $D \in U^{(d)}$, the associated line bundle $\Oo_C(D)$ admits a canonical trivialization along $\mdls$. 
Specifically, the canonical section $1_D$ is regular and non-vanishing on $\mdls$ because $\operatorname{supp}(D) \cap \operatorname{supp}(\mdls) = \emptyset$. 
This section restricts to a nowhere-vanishing section on the subscheme $\mathfrak{m}$, thereby determining a trivialization $\psi_D^{-1}: \Oo_C(D)|_{\mdls} \xrightarrow{\sim} \Oo_{\mdls}$.
This is done functorially in families, yielding a morphism from the symmetric power to the generalized Picard scheme (over $S$):
\begin{equation}
    \Phi_{d}: U^{(d)} \to \Pic^d_{C, \mathfrak{m}}, \quad D \mapsto [(\mathcal{O}_C(D), \psi_D)],
\end{equation}

When $\mdls = 0$, $d \geq max \{2g - 1, 0\}$ and $C$ admits a section over $S$, $C^{(d)}$ is a projective space bundle over $\PicC[d]$, 
It is proper, surjective with geometrically connected fibers.

Guignard (\cite{Guignard2018} Theorem 4.14) proves that for $\mdls > 0$ and $d$ satisfying \Cref{equation:degree-condition}, the Abel-Jacobi morphism $\Phi_d$ is 
surjective smooth of relative dimension $d - \deg \mdls - g + 1$, with geometrically connected fibers. 

When $S=\Spec (k)$, the geometric-fibers of $\Phi_d$ are well understood:
\begin{theorem}
    Assuming $S=\Spec (k)$ and $d \geq max \{2g - 1 + \deg \mdls, \deg \mdls\}$. Then, the geometric-fibers of the Abel-Jacobi morphism $$ \Phi_d : U^{(d)} \to \PicCm[d] $$ over any point are isomorphic to
$$
\begin{cases}
\mathbb{A}_{k^{sep}}^{d - \deg \mdls - g + 1} & \text{if } m > 0 \\
\mathbb{P}_{k^{sep}}^{d - g} & \text{if } m = 0
\end{cases}
$$
In both cases $\Phi_d$ is a fibration in affine spaces or projective spaces, depending on whether $\mdls$ is non-zero or zero.
\end{theorem}
\begin{proof}
see \cite{tendler2015geometricclassfieldtheory} Propositions 3.13-3.14, or \cite{Toth2011} Prop 2.1.4:    

\end{proof}


\subsection{Compactification of Blowup of Symmetric Powers of a Curve}\label{section:BlowupOfSymmetricPowerOfCurves}
We recall that our objective is to descend the local system $\cF^{(d)}$ from $U^{(d)}$ to $\PicCm[d]$ along the 
Abel-Jacobi map $\Phi_d$:

\[
\begin{tikzcd}
 \cF^{(d)} \arrow[d, purple]\\
 U^{(d)} \arrow[r, "\Phi_d"] & \PicCm[d]      
\end{tikzcd}
\]


(Here, the purple arrow emphasizes that the morphism is of sheaves on the étale site).

However, we encounter an obstruction: in the case we are considering ($\mdls > 0$), 
the fibers of $\Phi_d$ are affine spaces (of the same degree) rather than the better-behaved projective spaces. 
This hint that a solution to this problem is to compactify the morphism to yield projective fibers. 
 
This section describes the result of the compactification constructed by \cite{takeuchi2019blow} 
via the method of blowup.

Let $\mdls = \sum_{i=1}^n k_P P$ with $\deg P = d_P$ be a modulus on $C$, and let $d$ satisfy 
\Cref{equation:degree-condition}. 
Takeuchi (\cite{takeuchi2019blow}) defines $Z_0=Z_0(\mathfrak{m}, d)$ as the closed subscheme of 
$C^{(d)}$ defined by the map $C^{(d - \deg \mathfrak{m})} \to C^{(d)}$ adding $\mathfrak{m}$. 
He also defines $X_{\mathfrak{m}, d}$ as the blowup of $C^{(d)}$ along $Z_0$.
Let $E_0 = E_{\mathfrak{m},d} = Z_0(\mathfrak{m}, d) \times_{C^{(d)}} X_{\mathfrak{m}, d}$ be the exceptional divisor of the blowup. It is irreducible of codimension 1, and we let $\eta_0=\eta_{\mathfrak{m}, d}$ be its generic point.

%This section describes the compactification constructed by Takeuchi \cite{takeuchi2019blow} using the blowup method

% Let $\mdls = \sum_{i=1}^n k_P P$ with $\deg P = d_P$ be a modulus on $C$. Let $d$ satisfy \Cref{equation:degree-condition}.
% Takeuchi (\cite{takeuchi2019blow}) defines $Z_0=Z_0(\mdls, d)$ as the closed subscheme of $C^{(d)}$ defined by the map $C^{(d - \deg \mdls)} \to C^{(d)}$ adding $\mdls$.
% He also defines $X_{\mdls, d}$ as the blowup of $C^{(d)}$ along $Z_0$. 
% Let $E_0 = E_{\mdls,d} = Z_0(\mdls, d) \times_{C^{(d)}} X_{\mdls, d}$ 
% be the exceptional divisor of the blowup, it is irreducible of codimension 1, let $\eta_0=\eta_{\mdls, d}$ be its generic point.

Diagrammatically:
\[\begin{tikzcd}
\overline{\{\eta_0 \}} =  E_0 \arrow[r] \arrow[d] & X_{\mdls, d} \arrow[d, "\pi"] \\
Z_0 \arrow[r, "c.i"]           & C^{(d)}          
\end{tikzcd}\] 

Incoporating $U^{(d)}$, the local system $\cF^{(d)}$ and the Abel-Jacobi map, we have:
\[
\begin{tikzcd}
    & & \cF^{(d)} \arrow[d, purple]\\
\overline{\{\eta_0 \}} =  E_0 \arrow[r] \arrow[d] & 
X_{\mdls, d} \arrow[d, "\pi"]  & \arrow[l, hook'] \arrow[dl, hook'] U^{(d)} \arrow[r, "\Phi_d"] & \PicCm[d] \\
Z_0 \arrow[r, "c.i"]           & C^{(d)}          
\end{tikzcd}
\] 

In Section 3 of \cite{takeuchi2019blow} Takeuchi constructs, for large enough $d$ a compactification denoted by $\Cmod{d}$ and proves the following:
\textcolor{red}{exactly determined the fate of that d}

\begin{theorem}[Takeuchi]\label{theorem:takeuchi-compactification-theorem}
    The scheme $\Cmod{d}$ is an open subscheme of $X_{\mdls,d}$ containing $U^{(d)}$. 
    The morphism $\Phi_d: U^{(d)} \to \PicCm[d]$ extends to a morphism $\tilde{\Phi}_d: \Cmod{d} \to \PicCm[d]$ which makes 
    $\Cmod{d}$ a projective space bundle over $\PicCm[d]$. 
    Furthermore, the complement of $U^{(d)}$ in $\Cmod{d}$ is isomorphic to the fiber product 
    $E_0 \times_{C^{(d)}} \Cmod{d}$.
\end{theorem}

\begin{proof}
    \textcolor{red}{Add outline of construction and proofs}
\end{proof}

Diagrammatically we have:
\[
\begin{tikzcd}
    &  & \cF^{(d)} \arrow[d, purple]\\
    E_0 \times_{C^{(d)}} \Cmod{d} \arrow[d] \arrow[r]  & \Cmod{d} \arrow[d, hook] \arrow[rd, "\tilde{\Phi}_d"]  & U^{(d)} \arrow[l, hook'] \arrow[d, "\Phi_d"] \\
\overline{\{\eta_0 \}} =  E_0 \arrow[r] \arrow[d] & 
X_{\mdls, d} \arrow[d, "\pi"]  &  \PicCm[d] &  \\
Z_0 \arrow[r, "c.i"]           & C^{(d)}          
\end{tikzcd}
\] 

Also note that $ E_0 \times_{C^{(d)}} \Cmod{d} = Z_0 \times_{C^{(d)}} \Cmod{d}$

