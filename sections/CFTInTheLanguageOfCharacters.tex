\section{Class Field Theory In the language of Characters}
The character formulation of Class Field Theory provides a correspondence between characters of the idele class group and characters of the Galois group of the maximal abelian extension of a global field. 

%From \cref{prop:local-global-characters} 

\subsection{About Characters}
We need to make precise what we mean by characters on both sides of the correspondence.

\begin{definition}
    Let $G$ be an abelian group. 
    \begin{enumerate}
        \item A character $\rho: G_K \to G$ is unramified at a place $v$ if it is trivial on the 
        inertia group $I_v \subseteq G_K$. $\rho$ is called unramified if it is unramified at all places $v$ of $K$.
        \item A character $\rho: G_K \to G$ has ramification bounded by a modulus $\mdls = \sum_v n_v v$ 
        if for each place $v \in \mdls$, the restriction of $\rho$ to the higher ramification group $G_v^{n_v}$ is trivial.
    \end{enumerate}

    Note that since $G$ is abelian, the value of $\rho$ on the Frobenius element $\text{Fr}_v$ is well-defined for unramified places $v$.
\end{definition}

A useful theorem about characters is as follows:

\begin{theorem}\label{thm:duality-finite-abelian-groups}
 Let $G$ be an abelian group such that for every $n \in \N$,
    the $n$-torsion subgroup $G[n] = \{g \in G | ng = 0\}$ is cyclic of order $n$.
    Then for every finite abelian group $A$ denote by $\hat{A} = \Hom(A, G)$ the group of characters from $A$ to $G$. 

    Then the functor $A \mapsto \hat{A}$ is a contravariant equivalence of categories between the category of finite abelian groups and itself.
    Moreover the natural map $A \to \hat{\hat{A}}$ is an isomorphism.  
\end{theorem}

\subsection{The Main Theorems}
One formulation of Global Class Field Theory in terms of characters is as follows:
\begin{theorem}[Character Formulation of Unramified Global Class Field Theory]
\label{theorem:character-formulation-of-unramified-global-class-field-theory}
\begin{enumerate}
    \item For each character $\xi : K^\times \backslash \mathbb{A}_K^\times / \bO_K^\times \to \bar{\mathbb{Q}}_\ell^\times$ there exists a unique continuous unramified character $\rho : G_K \to \bar{\mathbb{Q}}_\ell^\times$ such that $\rho(\text{Fr}_v) = \xi(\pi_v)$ for all $v$.
    \item For each continuous unramified character $\rho : G_K \to \bar{\mathbb{Q}}_\ell^\times$ there exists a unique character $\xi : K^\times \backslash \mathbb{A}_K^\times / \bO_K^\times \to \bar{\mathbb{Q}}_\ell^\times$ such that $\rho(\text{Fr}_v) = \xi(\pi_v)$ for all $v$.
\end{enumerate}

Where $\bO_K^\times = \bO_{0}$
\end{theorem}


\begin{theorem}[Character Formulation of Ramified class field theory] \label{theorem:character-formulation-of-ramified-class-field-theory}
In the above notations:
\begin{enumerate}
    \item For each character $\xi : K^\times \backslash \mathbb{A}_K^\times / \mathcal{O}_{\mdls}^\times \to \bar{\mathbb{Q}}_\ell^\times$ there exists a unique continuous character $\rho : G_K \to \bar{\mathbb{Q}}_\ell^\times$ with $\ram(\rho) \subseteq \mdls$ and $\rho(Fr_v) = \xi(\pi_v)$ for all primes $v \notin \Supp(\mdls)$.

    \item For each continuous character $\rho : G_K \to \bar{\mathbb{Q}}_\ell^\times$ with $\ram(\rho) \subseteq \mdls$ there exists a unique character $\xi : K^\times \backslash \mathbb{A}_K^\times / \mathcal{O}_{\mdls}^\times \to \bar{\mathbb{Q}}_\ell^\times$ such that $\rho(Fr_v) = \xi(\pi_v)$ for all primes $v \notin \Supp(\mdls)$.
\end{enumerate}
\end{theorem}



In fact, for the above theorems, we can replace $\bar{\mathbb{Q}}_\ell^\times$ by any finite abelian group $G$ with the discrete topology, and the theorems would still hold.

\begin{theorem}\label{theorem:characters-theorem-finite-abelian-groups}
    Assume \ref{theorem:character-formulation-of-unramified-global-class-field-theory}, \ref{theorem:character-formulation-of-ramified-class-field-theory} 
    are true for all finite abelian groups $G$ with the discrete topology (as values of the characters).
    Then \ref{theorem:character-formulation-of-unramified-global-class-field-theory}, \ref{theorem:character-formulation-of-ramified-class-field-theory}
    are true as stated. 
    \begin{proof}
    Indeed, assume such theorem would be true for such cases, then by varying $G$ over $(\Z / l^n \Z)^\times$ (for all $n \in \N$), we will get a compatible system of characters and a corresponding isomorphism of character groups 
    with values in $\Z_l^\times$. And since $Tors(\Z_l^\times) \cong Tors({\mathbb{Q}}_\ell^\times)$, this is equivilant to the theorems for characters with values in ${\mathbb{Q}}_\ell^\times$. 
    Similiarly, every finite extension $\Q_l \subset F$ comes as inverse limit of its finite subgroups of units, so the same argument applies to characters with values in $F^\times$. 
    We have compatability between those characters (from uniqueness) for all finite ${\mathbb{Q}_l} \subset F$ and by going to the colimit (and since all groups involved are fintely generated, hence in \textbf{Ab} $Hom(A, -) $ preserve colimtis) we get the result for characters with values in $\bar{\mathbb{Q}}_\ell^\times$ as well.
        
    \end{proof}
\end{theorem}



\begin{theorem}\label{theorem:equivalence-of-character-and-ideal-formulations}
    Let $\mdls$ be a modulus of $K$. Assume \ref{theorem:character-formulation-of-unramified-global-class-field-theory} and \ref{theorem:character-formulation-of-ramified-class-field-theory} are true. Then the Artin map induces isomorphisms:
\begin{align*}
& Hom_{cont}(C_\mdls, \bar{\mathbb{Q}}_\ell^\times) \cong Hom_{cont}(K^\times \backslash \mathbb{A}_K^\times / \bO_{\mdls}^\times, \bar{\mathbb{Q}}_\ell^\times) \\
&\cong Hom_{cont,ram\leq \mdls}(G_K, \bar{\mathbb{Q}}_\ell^\times) \cong Hom_{(cont,ram \leq m)}(G_K^{ab}, \bar{\mathbb{Q}}_\ell^\times) \\
& \cong Hom_{cont}(Gal(L_\mdls/K), \bar{\mathbb{Q}}_\ell^\times)
\end{align*}
Where $L_\mdls$ is the maximal abelian extension of $K$ with ramification bounded by $\mdls$.
Hence by \cref{thm:duality-finite-abelian-groups} we get that the artin map induces isomorphism $C_\mdls \cong Gal(L_\mdls/K)$, which implies the statement of \cref{theorem:ArtinReciprocity} and \cref{theorem:ExistenceTheorem}.
(Details omitted, like how to go from $C_\mdls$ to every congurgence subgroups, etc.)
\end{theorem}


\textcolor{red}{
    \begin{enumerate}
        \item Is this formulation *equivilant* to adeles language? is it dervied from it?
        \item Give amichai  reference for this formulation
        \item Over what field are we working? what is $l$, what is $p$?
        \item Fix the qoutient of adeles no match the subgroup 
        \item $\mdls$ vs $\mdls$ notation for divisors
        \item maybe make \cref{theorem:characters-theorem-finite-abelian-groups} more precise
        \item maybe make \cref{theorem:equivalence-of-character-and-ideal-formulations} more precise
        \item replace $l$ by $\ell$ everywhere
    \end{enumerate}
}
\textcolor{red}{See milne, amichai, for more details.}