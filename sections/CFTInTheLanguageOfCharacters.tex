\section{Class Field Theory In the language of Characters}
The character formulation of Class Field Theory provides a correspondence between characters of the idele class group and characters of the Galois group of the maximal abelian extension of a global field. 


\subsubsection*{The Main Theorems}

\begin{theorem}[Character Formulation of Unramified Global Class Field Theory]
\end{theorem}
\begin{enumerate}
    \item For each character $\xi : K^\times \backslash \mathbb{A}_K^\times / \bO_K^\times \to \bar{\mathbb{Q}}_\ell^\times$ there exists a unique continuous unramified character $\rho : G_K \to \bar{\mathbb{Q}}_\ell^\times$ such that $\rho(\text{Fr}_v) = \xi(\pi_v)$ for all $v$.
    \item For each continuous unramified character $\rho : G_K \to \bar{\mathbb{Q}}_\ell^\times$ there exists a unique character $\xi : K^\times \backslash \mathbb{A}_K^\times / \bO_K^\times \to \bar{\mathbb{Q}}_\ell^\times$ such that $\rho(\text{Fr}_v) = \xi(\pi_v)$ for all $v$.
\end{enumerate}

Where $\bO_K^\times = \bO_{0}$


\begin{theorem}[Character Formulation of Ramified class field theory]
In the above notations:
\begin{enumerate}
    \item For each character $\xi : K^\times \backslash \mathbb{A}_K^\times / \mathcal{O}_{\mdls}^\times \to \mathbb{Q}_\ell^\times$ there exists a unique continuous character $\rho : G_K \to \mathbb{Q}_\ell^\times$ with $\ram(\rho) \subseteq \mdls$ and $\rho(Fr_v) = \xi(\pi_v)$ for all primes $v \notin \Supp(\mdls)$.

    \item For each continuous character $\rho : G_K \to \mathbb{Q}_\ell^\times$ with $\ram(\rho) \subseteq \mdls$ there exists a unique character $\xi : K^\times \backslash \mathbb{A}_K^\times / \mathcal{O}_{\mdls}^\times \to \mathbb{Q}_\ell^\times$ such that $\rho(Fr_v) = \xi(\pi_v)$ for all primes $v \notin \Supp(\mdls)$.
\end{enumerate}
\end{theorem}
Where
The term unramified character, resp. character with ramification bounded by $\mdls$, means that the character is trivial on the correspoding inertia group or higher ramification group of $G_K$ corresponding to the relevant primes. 


\textcolor{red}{See milne, amichai, for more details.}

We want to show how this formulation is equivalent to the adeles formulation given in the previous section.


\textcolor{red}{
    \begin{enumerate}
        \item Is this formulation *equivilant* to adeles language? is it dervied from it?
        \item Give amichai  reference for this formulation
        \item Over what field are we working? what is $l$, what is $p$?
        \item Fix the qoutient of adeles no match the subgroup 
        \item $\mdls$ vs $\mdls$ notation for divisors
    \end{enumerate}
}