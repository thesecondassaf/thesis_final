\section{Preliminaries}
In this section we recall the necessariy work, including work from \cite{Guignard2018}, \cite{tendler2015geometricclassfieldtheory} and \cite{takeuchi2019blow}.

\subsection{Generalities}

\subsubsection{Equivalence between Torsors and Invertible Modules}
The following proposition establishes the fundamental dictionary between the geometric theory
of principal homogeneous spaces and the algebraic theory of invertible modules.
This equivalence allows us to transport the monoidal structure from the category of modules
(the tensor product) to the category of torsors (the contracted product),
strictly within the categorical framework.

\begin{prop}\label{prop:torsor_module_equivalence}
    Let $\mathcal{E}$ be a topos and let $\Lambda$ be a ring object in $\mathcal{E}$. Let $G = \Lambda^\times$ denote the internal group object of units of $\Lambda$.

    There is a canonical equivalence of monoidal categories between the category of $G$-torsors in $\mathcal{E}$ and the category of locally free $\Lambda$-modules of rank 1 in $\mathcal{E}$:
    \[
        \Phi: \mathbf{Tors}(\mathcal{E}, \Lambda^\times) \xrightarrow{\sim} \mathbf{Pic}(\mathcal{E}, \Lambda)
    \]
    The equivalence is defined by the associated module functor:
    \[
        P \longmapsto P \times^{\Lambda^\times} \Lambda := \Lambda^\times \backslash (\Lambda \times P)
    \]
    where the quotient is taken with respect to the diagonal action of $\Lambda^\times$ on $\Lambda \times P$. The inverse functor associates to an invertible module $L$ its sheaf of basis frames $\underline{\mathrm{Isom}}_\Lambda(\Lambda, L)$.
\end{prop}

In light of this canonical equivalence, we will pass freely between the language of $G$-torsors and that of locally free $\Lambda$-modules throughout the text.

For a topos $\cE$, a group object $G$ in $\cE$ and an object $X$ in $\cE$, there is a canonical identification
between $({G\cE})_{/X}$ and $G(\cE_{/X})$, given by endowing $X$ with the trivial $G$-action.

We denote by $\mathbf{Tors}(X, G)$ the category of $G$-torsors over $X$ in $G \cE_{/X}$.
Similarly, for a ring object $\Lambda$ in $\cE$, we denote by $\mathbf{Pic}(X, \Lambda)$ the category of locally free $\Lambda$-modules of rank 1 over $X$ in $\cE_{/X}$.
The above equivilance of categories becomes 
\[
    \Phi_X: \mathbf{Tors}(X, \Lambda^\times) \xrightarrow{\sim} \mathbf{Pic}(X, \Lambda)
\].


\subsubsection{Ramification of Sheaves}
\textcolor{red}{Define and discuss a bit}

\subsubsection{Symmetric Powers of Schemes and Torsors}
This section reviews the construction of quotients for schemes and torsors under finite group actions, specifically focusing on symmetric powers. 
To ensure these quotients exist as schemes, we utilize the framework of admissible actions from \cite{sga1}. 
Our treatment here closely follows the exposition in \cite{Guignard2018}
The definitions and results presented below are adapted from their work. 
This foundation provides the necessary criteria for admissibility and base change required to define the symmetric powers of a scheme $X$
 and a $G$-torsor $\mathcal{P}$.


Let $S$ be a scheme.

\begin{definition}[(\cite{sga1}, V.1.7).]
    \noindent
    \begin{itemize}
        \item Let $T$ be an object of a category $\mathcal{C}$ endowed with a right action of a group $\Gamma$. We say that \textbf{the quotient $T/\Gamma$ exists} in $\mathcal{C}$ if the covariant functor
    \[
    \begin{aligned}
    \mathcal{C} &\to \text{Sets} \\
    U &\mapsto \text{Hom}_{\mathcal{C}}(T, U)^{\Gamma}
    \end{aligned}
    \]
    is representable by an object of $\mathcal{C}$.
    \item Let $T$ be an $S$-scheme. An action of a finite group $\Gamma$ on $T$ is \textbf{admissible} if there exists an affine $\Gamma$-invariant morphism $f : T \to T'$ such that the canonical morphism $\mathcal{O}_{T'} \to f_* \mathcal{O}_T$ induces an isomorphism from $\mathcal{O}_{T'}$ to $(f_* \mathcal{O}_T)^{\Gamma}$.
\end{itemize}
\end{definition}

\begin{prop}
    The following holds:
    \begin{enumerate}
        \item \textbf{(\cite{sga1} V.1.3)}. Let $T$ be an $S$-scheme endowed with an admissible right action of a finite group $\Gamma$. If $f : T \to T'$ is an affine $\Gamma$-invariant morphism such that the canonical morphism $\mathcal{O}_{T'} \to f_* \mathcal{O}_T$ induces an isomorphism from $\mathcal{O}_{T'}$ to $(f_* \mathcal{O}_T)^{\Gamma}$, then the quotient $T/\Gamma$ exists and is isomorphic to $T'$.
        \item \textbf{(\cite{sga1}, V.1.8)}. Let $T$ be an $S$-scheme endowed with a right action of a finite group $\Gamma$. Then, the action of $\Gamma$ on $T$ is admissible if and only if $T$ is covered by $\Gamma$-invariant affine open subsets.
        \item \textbf{(\cite{sga1}, V.1.9)}. Let $T$ be an $S$-scheme endowed with an admissible right action of a finite group $\Gamma$, and let $S'$ be a flat $S$-scheme. Then, the action of $\Gamma$ on the $S'$-scheme $T \times_S S'$ is admissible, and the canonical morphism
            \[
            (T \times_S S')/\Gamma \to (T/\Gamma) \times_S S'
            \]
            is an isomorphism.
    \end{enumerate}
\end{prop}

\begin{prop}[{[SGA1]}, IX.5.8]
Let $G$ be a finite abelian group, let $\cP$ be a $G$-torsor over an $S$-scheme $X$ in $S_{\text{Ét}}$. 
Assume that $\cP$ and $X$ are endowed with right actions from a finite group $\Gamma$ 
such that the morphism $\cP \to X$ is $\Gamma$-equivariant, and that the following properties hold:
\begin{enumerate}
    \item[(a)] The right $\Gamma$-action on $\cP$ commutes with the left $G$-action.
    \item[(b)] The right $\Gamma$-action on $X$ is admissible, and the quotient morphism $X \to X/\Gamma$ is finite.
    \item[(c)] For any geometric point $\bar{x}$ of $X$, the action of the stabilizer $\Gamma_{\bar{x}}$ of $\bar{x}$ in $\Gamma$ on the fiber $\cP_{\bar{x}}$ of $\cP$ at $\bar{x}$ is trivial.
\end{enumerate}
Then the action of $\Gamma$ on $\cP$ is admissible, and $\cP/\Gamma$ is a $G$-torsor over $X/\Gamma$ in $S_{\text{Ét}}$.
\end{prop}


\subsubsection*{Symmetric Powers of Schemes}
Let $X$ be an $S$-scheme and let $d \geq 0$ be an integer. The group ${S}_d$ of permutations of $\llbracket 1, d \rrbracket$ acts on the right on the $S$-scheme $X^{\times_S d} = X \times_S \dots \times_S X$ by the formula
\[
(x_i)_{i \in \llbracket 1, d \rrbracket} \cdot \sigma = (x_{\sigma(i)})_{i \in \llbracket 1, d \rrbracket}.
\]

\begin{prop}[\cite{Guignard2018} Proposition 2.27]
    If $X$ is a scheme, Zariski locally quasi-projective over $S$, then the right action of the symmetric group $S_d$ on the $d$-fold fiber product 
    $X^{\times_S d}$ is admissibleq. 
    Consequently, the quotient $\mathrm{Sym}_S^d(X) = X^{\times_S d}/S_d$ exists as a scheme over $S$.
\end{prop}
\begin{rem*}
    When the base $S$ is understood from context, this quotient is also denoted by $X^{(d)}$.
\end{rem*}

Guingard shows that when $X=\Spec(B)$ and $S=\Spec(A)$ then $\Sym_{S}^d(X)$ is representable by an affine $S$-scheme (See \cite{Guignard2018} Remark 2.28).

\begin{prop}[\cite{Guignard2018} Proposition 2.28]
If $X$ is flat and Zariski-locally quasi-projective over $S$, then $\text{Sym}_S^d(X)$ is flat over $S$. Moreover, for any $S$-scheme $S'$, the canonical morphism
\[
\text{Sym}_{S'}^d(X \times_S S') \to \text{Sym}_S^d(X) \times_S S'
\]
is an isomorphism.
\end{prop}

\subsubsection*{Symmetric Powers of Torsors}
\textcolor{red}{change below the exposition to be more accurate... }
Let $S$ be a scheme, let $X$ be an $S$-scheme and let $d \geq 1$ be an integer. 
Let $G$ be a finite abelian group, and let $\cP \to X$ be a $G$-torsor over $X$ in $S_{\text{Ét}}$. 
It is easy to show that the sheaf $\cP$ is representable by a finite étale $X$-scheme. (For example \cite{Guignard2018} Proposition 2.12)

For each $i \in \llbracket 1, d \rrbracket$ let $p_i : X^{\times_S d} \to X$ be the projection on $i$-th factor, and let us consider the $G$-torsor
\[
p_1^{-1}P \otimes \cdots \otimes p_d^{-1}P = G_d \backslash \cP^{\times_S d}
\]
over $X^{\times_S d}$, where $G_d \subseteq G^d$ is the kernel of the multiplication morphism $G^d \to G$. 
The object $G_d \backslash \cP^{\times_S d}$ of $S_{\text{Ét}}$ is too representable by an $S$-scheme which is finite étale over $X^{\times_S d}$. 
The group ${S}_d$ acts on the right on $G_d \backslash \cP^{\times_S d}$ by the formula
\[
(p_i)_{i \in \llbracket 1, d \rrbracket} \cdot \sigma = (p_{\sigma(i)})_{i \in \llbracket 1, d \rrbracket}.
\]
This action of ${S}_d$ commutes with the left action of $G$ on $G_d \backslash \cP^{\times_S d}$.

\medskip

\begin{prop}[\cite{Guignard2018} Proposition 2.32.]
If $X$ is Zariski-locally quasi-projective on $S$, then the right action of ${S}_d$ on $G_d \backslash \cP^{\times_S d}$ is admissible, 
so that the quotient $\cP^{(d)}$ of $G_d \backslash \cP^{\times_S d}$ by ${S}_d$ exists in $\mathrm{Sch}_{/S}$. 
Moreover, the canonical morphism $\cP^{(d)} \to \mathrm{Sym}_S^d(X)$ is a $G$-torsor, 
and the morphism
\[
p_1^{-1}\cP \otimes \cdots \otimes p_d^{-1}\cP \to r^{-1}\cP^{[d]}
\]
where $r : X^{\times_S d} \to \mathrm{Sym}_S^d(X)$ is the canonical projection, is an isomorphism of $G$-torsors over $X^{\times_S d}$.
\end{prop}
\textcolor{red}{consider replacing $\cP$ with $P$ because it is a scheme}
\textcolor{red}{Add proposition about how it is being a scheme}



\subsubsection{Etale Fundamental Groups and Tame Fundamental Groups}
We recall the definition and basic properties of the etale fundamental group, following stacks project 
\stackstag{0BQ6}
\begin{prop}[\stackstag{0C0J}]
Let $f : X \to S$ be a flat proper morphism of finite presentation whose geometric fibres are connected and reduced. Assume $S$ is connected and let $\overline{s}$ be a geometric point of $S$. Then there is an exact sequence

\[ \pi _1(X_{\overline{s}}) \to \pi _1(X) \to \pi _1(S) \to 1 \]
of fundamental groups.
    
\end{prop}

\begin{cor}\label{cor:etale_fundamental_group_smooth_proper}
    Let $f : X \to S$ be a proper smooth morphism of finite presentation whose geometric fibres are connected. Assume $S$ is connected and let $\overline{s}$ be a geometric point of $S$. Then there is an exact sequence

    \[ \pi _1(X_{\overline{s}}) \to \pi _1(X) \to \pi _1(S) \to 1 \]
    of fundamental groups.
\end{cor}
\textcolor{red}{add about tameness?}

\subsection{Symmetric Powers of Local Systems on Curves}
Let $X$ be Zariski locally quasi-projective over a scheme $S$.
And let $G$ be a finite abelian group. 
Let $\cP \to X$ be a $G$ a $G$-torsor

\subsection{Generalized Picard Scheme}
In this section, we recall the notion of generalized Jacobian varieties and study their fundamental properties. The material presented here is primarily adapted from \cite{Guignard2018} and \cite{takeuchi2019blow}. For further background on the general theory of abelian varieties and Jacobians, the reader may also consult \cite{milneAV}.
Let $S$ be a scheme and let $C$ be a projective smooth $S$-scheme whose geometric fibers are connected and of dimension 1. Let $\mdls$ be a modulus on $C$, defined as an effective Cartier divisor of $C/S$ 
(i.e., a closed subscheme of $C$ which is finite flat of finite presentation over $S$).
We denote the projection $C \times_S T \to T$ by $\text{pr}$ for any $S$-scheme $T$.

\subsubsection*{The Functor of Points}
Let $d$ be an integer. For an $S$-scheme $T$, we consider the set of data $(\cL, \psi)$ where:
\begin{itemize}
    \item $\cL$ is an invertible sheaf of degree $d$ on $C_T$.
    \item $\psi: \Oo_{\mdls_T} \xrightarrow{\sim} \cL|_{\mdls_T}$ is a trivialization of $\cL$ along the modulus.
\end{itemize}
Two such pairs $(\cL, \psi)$ and $(\cL', \psi')$ are said to be isomorphic if there exists an isomorphism of invertible sheaves $f : \cL \to \cL'$ such that the following diagram commutes:
\[
\begin{tikzcd}
& \Oo_{\mdls_T} \arrow[dl, "\psi'"'] \arrow[dr, "\psi"] & \\
\cL'|_{\mdls_T} \arrow[rr, "f|_{\mdls_T}"] & & \cL|_{\mdls_T}
\end{tikzcd}
\]
We define the presheaf $\text{Pic}_{C, \mdls}^{d, \text{pre}}$ on $\text{Sch}/S$ by assigning to $T$ the set of isomorphism classes of such pairs. Let $\PicCm[d]$ denote the \'etale sheafification of this presheaf.

\subsubsection*{Representability and Structure}
The fundamental properties of this functor are as follows:
\begin{enumerate}
    \item $\text{Pic}_{C, \mdls}^{d}$ is represented by an $S$-scheme. (Note: If $\mdls$ is faithfully flat over $S$, the presheaf is already an \'etale sheaf).
    \item $\text{Pic}_{C, \mdls}^{0}$ is a smooth commutative group $S$-scheme with geometrically connected fibers, referred to as the \textit{generalized Jacobian variety} of $C$ with modulus $\mdls$.
    \item For any $d$, $\text{Pic}_{C, \mdls}^{d}$ is a $\text{Pic}_{C, \mdls}^{0}$-torsor.
\end{enumerate}
In the case where $\mdls = 0$, we recover the standard Jacobian variety, denoted simply as $\text{Pic}_C^d$.
\subsubsection*{Relation to the Standard Jacobian}
We now examine the behavior of the generalized Picard scheme under the variation of the modulus. By viewing the structure along the modulus as an additional rigidification, we obtain natural transition maps corresponding to the inclusion of moduli.

Let $\mdls_1$ and $\mdls_2$ be moduli such that $\mdls_1 \subset \mdls_2$. There exists a natural map 
\[
\text{Pic}_{C, \mdls_2}^d \to \text{Pic}_{C, \mdls_1}^d
\]
obtained by restricting the isomorphism $\psi$. Since $\mdls_2$ is a finite $S$-scheme, this map is a surjection as a morphism of \'etale sheaves.
In particular, for any modulus $\mdls$, there is a natural surjective morphism of \'etale sheaves:
\[
\text{Pic}_{C, \mdls}^d \to \text{Pic}_C^d.
\]


\subsubsection*{Local Freeness and Base Change}
%\textcolor{red}{there is ${\mdls}$ here - should fix that}
Let $\mdls$ be a modulus which is everywhere strictly positive. Let $g$ denote the genus of $C$, which is a locally constant function on $S$. We restrict our attention to degrees $d$ satisfying the condition:
\begin{equation} \label{equation:degree-condition}
    d \geq \max\{2g - 1 + \deg \mdls, \deg \mdls\}.
\end{equation}

Assuming $S$ is quasi-compact, such a $d$ always exists.

Fix an integer $d$ satisfying the condition above. Let $T$ be an $S$-scheme and let $\cL$ be an invertible sheaf of degree $d$ on $C_T$. One can show that the pushforwards $\mathrm{pr}_* \cL$ and $\mathrm{pr}_* \cL(-{\mdls})$ are locally free sheaves and their formations commute with any base change. Explicitly, for any morphism of $S$-schemes $f : T' \to T$, the base change morphisms are isomorphisms:
\[
    f^* \mathrm{pr}_* \cL \xrightarrow{\sim} \mathrm{pr}_* f^* \cL
\]
and
\[
    f^* \mathrm{pr}_* (\cL(-{\mdls})) \xrightarrow{\sim} \mathrm{pr}_* f^* (\cL(-{\mdls})).
\]
In particular, following \cite{Guignard2018}, if $\cL$ is invertible $\cO_C$-module with degree $d$ satisfying \ref{equation:degree-condition} on each fiber of $f$ then, $\mathrm{pr}_* \cL$ is a locally free $\mathcal{O}_S$-module of rank $d-g+1$.

For further background and verification of these constructions, we refer the reader to Milne's notes on Abelian Varieties (\cite{milneAV}).

\subsection{The Abel-Jacobi Morphism and its Fibers}\label{section:AbelJacobi}


Let $U = C \setminus \mdls$ be the complement of the modulus in $C$.
The effective cartier divisors of degree $d$ which are prime to $\mdls$ are parameterized by the symmetric power $\Sym_S^d (U) = U^{(d)}$ over $S$ (See \cite{Guignard2018} Proposition 4.12, \cite{milneAV} Theorem 3.13).
For any such divisor $D \in U^{(d)}$, the associated line bundle $\Oo_C(D)$ admits a canonical trivialization along $\mdls$. 
Specifically, the canonical section $1_D$ is regular and non-vanishing on $\mdls$ because $\operatorname{supp}(D) \cap \operatorname{supp}(\mdls) = \emptyset$. 
This section restricts to a nowhere-vanishing section on the subscheme $\mathfrak{m}$, thereby determining a trivialization $\psi_D^{-1}: \Oo_C(D)|_{\mdls} \xrightarrow{\sim} \Oo_{\mdls}$.
This is done functorially in families, yielding a morphism from the symmetric power to the generalized Picard scheme (over $S$):
\begin{equation}
    \Phi_{d}: U^{(d)} \to \Pic^d_{C, \mathfrak{m}}, \quad D \mapsto [(\mathcal{O}_C(D), \psi_D)],
\end{equation}

When $\mdls = 0$, $d \geq max \{2g - 1, 0\}$ and $C$ admits a section over $S$, $C^{(d)}$ is a projective space bundle over $\PicC[d]$, 
It is proper, surjective with geometrically connected fibers.

Guignard (\cite{Guignard2018} Theorem 4.14) proves that for $\mdls > 0$ and $d$ satisfying \Cref{equation:degree-condition}, the Abel-Jacobi morphism $\Phi_d$ is 
surjective smooth of relative dimension $d - \deg \mdls - g + 1$, with geometrically connected fibers. 

When $S=\spec (k)$, the geometric-fibers of $\Phi_d$ are well understood:
\begin{theorem}
    Assuming $S=\spec (k)$ and $d \geq max \{2g - 1 + \deg \mdls, \deg \mdls\}$. Then, the geometric-fibers of the Abel-Jacobi morphism $$ \Phi_d : U^{(d)} \to \PicCm[d] $$ over any point are isomorphic to
$$
\begin{cases}
\mathbb{A}_{k^{sep}}^{d - \deg \mdls - g + 1} & \text{if } m > 0 \\
\mathbb{P}_{k^{sep}}^{d - g} & \text{if } m = 0
\end{cases}
$$
In both cases $\Phi_d$ is a fibration in affine spaces or projective spaces, depending on whether $\mdls$ is non-zero or zero.
\end{theorem}
\begin{proof}
see \cite{tendler2015geometricclassfieldtheory} Propositions 3.13-3.14, or \cite{Toth2011} Prop 2.1.4:    

\end{proof}

%\subsection{Kummer Theory}
%A kummer extension is a field extension $L/K$ where for some given $n \in \N$ we have:
\begin{enumerate}
    \item $K$ contains all $n$'th roots of unity
    \item $L/K$ has abelian galois group of exponent $n$.
\end{enumerate}

A group $G$ has exponent $n$ if $g^n = 1$ for all $g \in G$.

\begin{example}
    Quadratic extensions are kummer extensions. Multi quadratic Extensions, etc...
\end{example}

\begin{example}
    When $K$ contains n distinct $n$'th roots of unity (hence $char(K) \nmid n$) then the extension $L=K(a^\frac{1}{n})$ is a Kummer extension of degree $m \mid n$, for any element $a \in K$.
    The galois group $G$ is cyclic of order $m$, and acts as multiplication by root of unity of order $m$.% on $a^\frac{1}{n}$.
\end{example}

Kummer theory gives us the converse of the above example:
Let $K$ be a field containing $n$ distinct $n$'th roots of unity, then we have a bijection:
\begin{equation*}
    \{ \text{Kummer extensions } L/K \text{ of exponent dividing } n \} \quad \longleftrightarrow \quad \{ \text{Subgroups } H \text{ of } K^\times/(K^\times)^n \}
\end{equation*}
This bijection is given by the maps:
\begin{align*}
    L              & \longmapsto (K^\times \cap (L^\times)^n) / (K^\times)^n \\
    K(\sqrt[n]{H}) & \longmapsfrom H
\end{align*}

Where $(K^\times)^n$ is the group of $n$'th powers in $K^\times$.
And, $K(\sqrt[n]{H}) := \{ \sqrt[n]{a} \mid a \in K^\times,  a \cdot (K^\times)^n \in H \}$

In the latter case we have:
\begin{align*}
    H & \cong \text{Hom}_c(\text{Gal}(L/K), \mu_n)                                                                                                            \\
    a & \longmapsto \left(\sigma \mapsto \frac{\sigma(\alpha)}{\alpha}\right) \quad \text{where } \alpha \text{ is any } n\text{'th root of } a \text{ in } L
\end{align*}.

Also note that if $K$ contains all roots of unity then every finite abelian extension of $K$ is a kummer extension.
In this paper we will mainly be intersested in kummer extensions of degree $l^m - 1$ with galois group $G=(\Z/l^m \Z)^\times$ where $l$ is a prime number and $char K \neq l$.

\subsubsection{Ramification In Kummer extensions}
Next, we turn to a brief discussion of ramification in kummer extensions.

We begin by the following useful lemma proved in \cite{lang2005algebra}[9.1]
\textcolor{red}{Check if the lang bib entry is correct(i generated it using ai)}

\begin{lemma}
    Let $K$ be a field, and let $2 \leq n \in \N$ be a natural number. Let $0 \neq a \in K$ be a element of $K$.
    Assume that for every prime $p$ dividing $n$ we have $a \notin K^p$, and that if $4 \mid n$ then $a \not in -4K^4$.
    Then $X^n - a$ is irreducible in $K[X]$.
\end{lemma}+

Using this lemma we can show the following proposition:
\textcolor{red}{Something about the proof here doesn't work, complete it}
\begin{prop}
    Let $n \in \N$ be a natural number and let $K$ be a field such that $gcd(n, char K) = 1$.
    For $b \in K^\times$, $X^n - b$ is irreducible in $K[X]$ if and only if $ord(\bar{b}) = n$ in $K^\times/(K^\times)^n$.
\end{prop}
\begin{proof}
    In one direction, note that $ord(\bar{b}) = n$, if and only if $b^k \notin (K^\times)^n$ for every $k \mid n, k < n$, if and only if, $b^{p^r} \notin (K^\times)^n$ for every prime $p \mid n$ (\textcolor{red}{What to do in the case $n = p$?}) and every $r \leq ord_p(n)$.
    Hence we can use the lemma above:
    \begin{enumerate}
        \item For every $p \mid n$, $b^p \notin K^n$ (otherwise $b^{\frac{n}{p}} \in (K^\times)^n$).
        \item If $4 \mid n$ then $i \in K$ ($i^2 = -1$), hence $b \notin -4K^4$ (because $-4 = (2i)^2$).
    \end{enumerate}
    Hence $X^n - b$ is irreducible in $K[X]$.
    In the other direction, assume $X^n - b$ is irreducible in $K[X]$. Then if $p \mid n$  we can not have $b \in K^p$ (By factoring $X^n - b$)

\end{proof}

From now on we will focus on cyclic Kummer extensions.
Those are of the form $L=K(\sqrt[n]{a})$ where $a \in K$ and $n \in \N$ is a natural number.
Their galois group is cyclic of order $n$, and acts as multiplication by root of unity of order $n$ on $a^{\frac{1}{n}}$.
\begin{equation*}
    \text{Gal}(L/K) \cong \Z/n\Z
    \quad
    \sigma \mapsto \sigma(a^{\frac{1}{n}}) = \zeta_n \sigma(a^{\frac{1}{n}})
\end{equation*}




\textcolor{red}{
    \begin{enumerate}
        \item Switch $a^{\frac{1}{n}}$ to $\sqrt[n]{a}$ or $\alpha$ in the galois group description
    \end{enumerate}
}


\subsection{Blowup of Smooth Schemes}
\textcolor{red}{complete this section}
In this seciton we prove some auxliary lemmas and propositions about blowups and local systems. 

\begin{prop}\label{prop:blowup-prop}
    Let $X,Y$ be smooth over $k$. $x\in X, y\in Y$ closed points. Let $\Bl_x(X), \Bl_y(Y), \allowbreak Bl_{(x,y)}(X\times_k Y) $ be the respective blowups. 
    Let $\eta_X, \eta_Y,\eta_{X\times Y}$ be the generic points of the exceptional divisor of the respective blowups. 
    Then, there exists a scheme $\tilde{U}$ and maps $f_1, f_2$ making the diagram commute:
    
    \begin{diagram}
         \begin{tikzcd}
            {} & \tilde{U} \arrow[dl, "f_1"] \arrow[dr, hook, "f_2"] & {} \\
            \Bl_x(X)\times_k \Bl_y(Y) & {} & \Bl_{(x,y)}(X\times_k Y)
        \end{tikzcd}
    \end{diagram}
    such that:
    \begin{enumerate}
        \item $f_2$ is open immersion
        \item $f_1$ is open map (?open immersion?)
        \item $\eta_{X\times Y} \in \tilde{U}$
        \item $\eta_{X\times Y} \overset{f_1}{\mapsto} \eta_X \times \eta_Y$ 
    \end{enumerate}
\end{prop}

Let $P^{[d]} \to U^{(d)}$ be the corresponding $G$-torsor over $U^{(d)}$.

A conlusion maybe:
\begin{lemma}\label{lemma:boxplus}
Let $p: C^{(d)} \times_k C^{(n-d)} \xrightarrow[]{+} C^{(n)}$ be the plus map, restricting $p$ to $U^{(d)} \times_k U^{(n-d)}$ we get a map 
$$p: U^{(d)} \times_k U^{(n-d)} \xrightarrow[]{p} U^{(n)}$$
Then, $$p^* (\cP^{(n)}) \cong \boxPp {\cP^{[d]}}{\cP^{(n-d)}}[k]$$
\end{lemma}


\textcolor{red}{define box product somewhere}


\begin{lemma}
    \textcolor{red}{rephrase this lemma}
     Let $\ndls_1, \ndls_2 \subset \mdls$ be two moduli of the form $\ndls_1 = k_1 P_1$, $\ndls_2 = k_2 P_2$ where $P_1, P_2$ are distinct points.
     Let $\cF_1, \cF_2$ be local systems on $U^{(\deg \ndls_1)}$, $U^{(\deg \ndls_2)}$, at most tamely ramified at $\eta_{\ndls_1}$, $\eta_{\ndls_2}$ respectively.
     Then the local system $\boxP{\cF_1}{\cF_2} := p_1^{-1}(\cF_1) \otimes p_2^{-1}(\cF_2)$ on is at most tamely ramified at the point $\eta_{\ndls_1} \times \eta_{\ndls_2}$ of 
\end{lemma}

\subsection{Compactification of Blowup of Symmetric Powers of a Curve}\label{section:BlowupOfSymmetricPowerOfCurves}
We recall that our objective is to descend the local system $\cF^{(d)}$ from $U^{(d)}$ to $\PicCm[d]$ along the 
Abel-Jacobi map $\Phi_d$:

\[
\begin{tikzcd}
 \cF^{(d)} \arrow[d, purple]\\
 U^{(d)} \arrow[r, "\Phi_d"] & \PicCm[d]      
\end{tikzcd}
\]


(Here, the purple arrow emphasizes that the morphism is of sheaves on the étale site).

However, we encounter an obstruction: in the case we are considering ($\mdls > 0$), 
the fibers of $\Phi_d$ are affine spaces (of the same degree) rather than the better-behaved projective spaces. 
This hint that a solution to this problem is to compactify the morphism to yield projective fibers. 
 
This section describes the result of the compactification constructed by \cite{takeuchi2019blow} 
via the method of blowup.

Let $\mdls = \sum_{i=1}^n k_P P$ with $\deg P = d_P$ be a modulus on $C$, and let $d$ satisfy 
\Cref{equation:degree-condition}. 
Takeuchi (\cite{takeuchi2019blow}) defines $Z_0=Z_0(\mathfrak{m}, d)$ as the closed subscheme of 
$C^{(d)}$ defined by the map $C^{(d - \deg \mathfrak{m})} \to C^{(d)}$ adding $\mathfrak{m}$. 
He also defines $X_{\mathfrak{m}, d}$ as the blowup of $C^{(d)}$ along $Z_0$.
Let $E_0 = E_{\mathfrak{m},d} = Z_0(\mathfrak{m}, d) \times_{C^{(d)}} X_{\mathfrak{m}, d}$ be the exceptional divisor of the blowup. It is irreducible of codimension 1, and we let $\eta_0=\eta_{\mathfrak{m}, d}$ be its generic point.

%This section describes the compactification constructed by Takeuchi \cite{takeuchi2019blow} using the blowup method

% Let $\mdls = \sum_{i=1}^n k_P P$ with $\deg P = d_P$ be a modulus on $C$. Let $d$ satisfy \Cref{equation:degree-condition}.
% Takeuchi (\cite{takeuchi2019blow}) defines $Z_0=Z_0(\mdls, d)$ as the closed subscheme of $C^{(d)}$ defined by the map $C^{(d - \deg \mdls)} \to C^{(d)}$ adding $\mdls$.
% He also defines $X_{\mdls, d}$ as the blowup of $C^{(d)}$ along $Z_0$. 
% Let $E_0 = E_{\mdls,d} = Z_0(\mdls, d) \times_{C^{(d)}} X_{\mdls, d}$ 
% be the exceptional divisor of the blowup, it is irreducible of codimension 1, let $\eta_0=\eta_{\mdls, d}$ be its generic point.

Diagrammatically:
\[\begin{tikzcd}
\overline{\{\eta_0 \}} =  E_0 \arrow[r] \arrow[d] & X_{\mdls, d} \arrow[d, "\pi"] \\
Z_0 \arrow[r, "c.i"]           & C^{(d)}          
\end{tikzcd}\] 

Incoporating $U^{(d)}$, the local system $\cF^{(d)}$ and the Abel-Jacobi map, we have:
\[
\begin{tikzcd}
    & & \cF^{(d)} \arrow[d, purple]\\
\overline{\{\eta_0 \}} =  E_0 \arrow[r] \arrow[d] & 
X_{\mdls, d} \arrow[d, "\pi"]  & \arrow[l, hook'] \arrow[dl, hook'] U^{(d)} \arrow[r, "\Phi_d"] & \PicCm[d] \\
Z_0 \arrow[r, "c.i"]           & C^{(d)}          
\end{tikzcd}
\] 

In Section 3 of \cite{takeuchi2019blow} Takeuchi constructs, for large enough $d$ a compactification denoted by $\Cmod{d}$ and proves the following:
\textcolor{red}{exactly determined the fate of that d}

\begin{theorem}[Takeuchi]\label{theorem:takeuchi-compactification-theorem}
    The scheme $\Cmod{d}$ is an open subscheme of $X_{\mdls,d}$ containing $U^{(d)}$. 
    The morphism $\Phi_d: U^{(d)} \to \PicCm[d]$ extends to a morphism $\tilde{\Phi}_d: \Cmod{d} \to \PicCm[d]$ which makes 
    $\Cmod{d}$ a projective space bundle over $\PicCm[d]$. 
    Furthermore, the complement of $U^{(d)}$ in $\Cmod{d}$ is isomorphic to the fiber product 
    $E_0 \times_{C^{(d)}} \Cmod{d}$.
\end{theorem}

\begin{proof}
    \textcolor{red}{Add outline of construction and proofs}
\end{proof}

Diagrammatically we have:
\[
\begin{tikzcd}
    &  & \cF^{(d)} \arrow[d, purple]\\
    E_0 \times_{C^{(d)}} \Cmod{d} \arrow[d] \arrow[r]  & \Cmod{d} \arrow[d, hook] \arrow[rd, "\tilde{\Phi}_d"]  & U^{(d)} \arrow[l, hook'] \arrow[d, "\Phi_d"] \\
\overline{\{\eta_0 \}} =  E_0 \arrow[r] \arrow[d] & 
X_{\mdls, d} \arrow[d, "\pi"]  &  \PicCm[d] &  \\
Z_0 \arrow[r, "c.i"]           & C^{(d)}          
\end{tikzcd}
\] 

Also note that $ E_0 \times_{C^{(d)}} \Cmod{d} = Z_0 \times_{C^{(d)}} \Cmod{d}$


% \begin{equation}
% \begin{tikzcd}
%     % Row 1
%     & 
%     & U^{(d)} \arrow[d, hook, blue, dashed]  
%     \arrow[dd, hook, blue, dashed, bend right=25]
%      \arrow[ddd, hook, blue, dashed, bend right=40]
%       \arrow[dr, "\Phi_d"]
%     & 
%     \\
%     % Row 2
%     &  
%     &  \arrow[r, red] \arrow[d,  hook, red] \arrow[dr, phantom, "\square"]
%     & \PicCm[d] \arrow[d, hook, red] \\
%     % Row 3
%     & Z_0 \times_{C^{(d)}} X_{\mdls, d}  \arrow[r] \arrow[d]
%     & X_{\mdls, d} \arrow[r, red] \arrow[d, red]    
%     & P_{\mdls}^{d} \arrow[d, red] \\
%     % Row 4
%     & Z_0 \arrow[r]
%     & C^{(d)}
%     & \PicC[d]
% \end{tikzcd}
% \end{equation}

% %We remind, following the introduction, is that our mission is to decsend $\cF^{(d)}$ to $\PicCm[d]$ via $\Phi_d$. For this we don't consider the blowup. 

% In this section we define the blowup of $C^{(d)}$ by $Z_0$ and give its basic properties. As outlined in \cite{takeuchi2019blow}
% Let $C^{(d)}$ be the $d$'th symmetric product of $C$, which parametrizes effective Cartier
% divisor of $deg = d$ on $C$. Let $Z_0$ be the closed subscheme of $C^{(d)}$ defined by the map 
% And let $X_{\mdls, d}$ be the blowup of $C^{(d)}$ along $Z_0$.
% In \cite{takeuchi2019blow}[Section 3] Takeuchi proves:
% \begin{theorem}
%     There exists a commutative diagram:
%     \begin{equation}\label{diag:main-blowup-diag}
% \begin{tikzcd}
%     % Row 1
%     & 
%     & U^{(d)} \arrow[d, hook, blue, dashed]  \arrow[dd, hook, blue, dashed, bend right=25] \arrow[ddd, hook, blue, dashed, bend right=40] \arrow[dr, "\Phi_d"]
%     & 
%     \\
%     % Row 2
%     & Z_0 \times_{C^{(d)}} \Cmod{d}  \arrow[r] \arrow[d]
%     & \Cmod{d} \arrow[r, red] \arrow[d,  hook, red] \arrow[dr, phantom, "\square"]
%     & \PicCm[d] \arrow[d, hook, red] \\
%     % Row 3
%     & Z_0 \times_{C^{(d)}} X_{\mdls, d}  \arrow[r] \arrow[d]
%     & X_{\mdls, d} \arrow[r, red] \arrow[d, red]    
%     & P_{\mdls}^{d} \arrow[d, red] \\
%     % Row 4
%     & Z_0 \arrow[r]
%     & C^{(d)}
%     & \PicC[d]
% \end{tikzcd}
% \end{equation}
% Where
% \begin{enumerate}
%     \item \cite{takeuchi2019blow}[Lemma 3.1.] $P_\mdls^{d}$ is an etale sheaf on $Sch/S$ defined by:
%     $$P_{\mdls}^{d}(T) = \left\{ (\cL, \phi) \;\middle|\; \cL \in \operatorname{Pic}^{d}(C_T), \; \phi: \mathcal{O}_T \hookrightarrow pr_*(\cL/\cL(-\mdls)) \text{ s.t. } \operatorname{coker}(\phi) \text{ is loc. free} \right\} \Big/ \!\cong$$
%     (Isomorphism between two $(\cL, \phi), (\cL', \phi')$ is an isomorphism $f: \cL \xrightarrow{\cong} \cL'$ such that $pr_*(f) \circ \phi = \phi'$)
%     Moreover, it is represented by a proper smooth $S$-algebraic space. Assuming $C \to S$ has a section (\textcolor{red}{This seems rather imporant, it is also used to show $Pic_C^d$ as explicit expression as a sheaf, how do I refer to it }),
%     And letting $\cL'$ be a representative invertible sheaf of the universal class, \cite{takeuchi2019blow} shows that as sheaves on $Sch/\PicCm[d]$, $P_\mdls^d$ is isomorphic to the projectivization $\bP(pr_*(\cL'/\cL'(-\mdls)))$.
%     \item $\PicCm[d] \rightarrow P_\mdls^{d}$ is defined by $(\cL, \psi) \mapsto (\cL, \phi)$.  Where $\phi$ is defined as the composition
%     $$\Oo_T \to pr_* \Oo_{\mdls_T} \xrightarrow[]{pr_* \psi} pr_* (\cL/\cL(-\mdls)) $$. \cite{takeuchi2019blow}[Lemma 3.5] Shows this map is an open immersion.
%     \item The map $P_\mdls^d \to \PicCm[d]$ is by forgetting $\phi$
%     \item The definition of $X_{\mdls, d} \to P_\mdls$ is more technically involved, so we refer the reader to \cite{takeuchi2019blow}. But an imporant feature it is that
%     $X_{\mdls, d}$ is a projective space bundle over $P_\mdls^d$ via that map. 
%     \item $\Cmod{d}$ is an $S$-scheme, defined as the fibered product
%     \begin{tikzcd}
%     \Cmod{d} \arrow[r] \arrow[d, hook] 
%     & \mathrm{Pic}_{C,\mathfrak{m}}^{d} \arrow[d, hook] \\
%     X_{\mathfrak{m}} \arrow[r] 
%     & P_{\mathfrak{m}}^{d},
% \end{tikzcd}
% Hence, The $S$-Scheme $\Cmod{d}$ is a projective space bundle on $\PicCm[d]$ and an open subscheme of $X_{\mdls, d}$.
% \end{enumerate}
% Where for a scheme $X$ and a locally free sheaf of finite rank $\mathcal{F}$ on $X$. We use a contra-Grothendieck notation for a projective space. 
% Thus the $X$-scheme $\mathbb{P}(\mathcal{F})$ parametrizes invertible subsheaves of $\mathcal{F}$.

% Now focusing on the left side of \Cref{diag:main-blowup-diag}, \cite{takeuchi2019blow} shows:

% \begin{enumerate}[start=6]
%     \item $U^{(d)} \to \Cmod{d}$ is an open immersion, and as an open subscheme, $U^{(d)}$ is the complement of $Z_0 \times_{C^{(d)}} \Cmod{d}$
% \end{enumerate}
% \end{theorem}.

