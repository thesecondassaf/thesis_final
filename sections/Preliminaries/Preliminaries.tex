\section{Preliminaries}
This section establishes the foundational definitions and theorems necessary for the remainder of this work. 
We focus specifically on the theory of $G$-torsors, which are central to our study due to their correspondence with 
locally free sheaves of rank 1. 
Subsequently, we review the relevant background on ramification theory from the existing literature. 
Finally, we conclude with several algebraic geometric remarks and notational conventions that will be employed 
implicitly throughout this thesis.

%%%% %%%% %%%% %%%% %%%% %%%% %%%% %%%%
%               TORSORS
%%%% %%%% %%%% %%%% %%%% %%%% %%%% %%%%
\subsection{Torsors}
% $X$ with etale site, but say things here work for toposes in general and flat sites in general.

In what follows, we largely adhere to the treatment of torsors and group objects found in publised notes of Alex Youcis \cite{YoucisTorsors}
Let $\sC=(\cC, J)$ be a site and let $\cE=Sh(\sC)$ be the associated topos. 
Let $\mathcal{G}$ be a group object in $\mathcal{E}$. 
We denote by $\mathcal{G}\mathcal{E}$ the category of objects in $\mathcal{E}$ endowed with a left $\mathcal{G}$-action. 
For any object $\cX \in \mathcal{E}$, there is a canonical identification between the slice category 
$(\mathcal{G}\mathcal{E})/\cX$ and the category of group objects $\mathcal{G}(\mathcal{E}/\cX)$, where $\cX$ is viewed as having the 
trivial $\mathcal{G}$-action.

\begin{definition}
    A \textbf{$\mathcal{G}$-torsor} in $\mathcal{E}$ is an object $\mathcal{P}$ of $\mathcal{G}\mathcal{E}$ satisfying the following conditions:
    \begin{enumerate}
        \item The structural morphism $\mathcal{P} \to 1$ is an epimorphism in $\mathcal{E}$ (i.e., $\mathcal{P}$ is locally non-empty).
        \item The map $\mathcal{G} \times \mathcal{P} \to \mathcal{P} \times \mathcal{P}$ defined by $(g, p) \mapsto (g \cdot p, p)$ is an isomorphism in $\mathcal{E}$ (i.e., $\mathcal{G}$ acts simply transitively on $\mathcal{P}$).
    \end{enumerate}
\end{definition}

Since $\mathcal{E}$ is the topos of sheaves on a site $\sC$, the definition can be reformulated in terms of covers. A $\mathcal{G}$-sheaf $\mathcal{P}$ is a $\mathcal{G}$-torsor if:
\begin{enumerate}
    \item For every object $X \in \mathcal{C}$, there exists a covering $\{ U_i \to X \} \in J$ such that $\mathcal{P}(U_i) \neq \emptyset$ for all $i$.
    \item For any $X \in \mathcal{C}$ where $\mathcal{P}(X)$ is non-empty, the action of $\mathcal{G}(X)$ on $\mathcal{P}(X)$ is simply transitively.
\end{enumerate}

A fundamental property of torsors is their local triviality: 
a $\mathcal{G}$-sheaf $\mathcal{P}$ is a $\mathcal{G}$-torsor if and only if it is locally isomorphic to the trivial torsor. 
Specifically, for every $X \in \mathcal{C}$, there must exist a cover $\{ U_i \to X \}$ such that the restriction $\mathcal{P}|_{U_i}$ is 
isomorphic, as a $\mathcal{G}|_{U_i}$-sheaf, to $\mathcal{G}|_{U_i}$ acting on itself by left multiplication.

A \textbf{morphism of $\mathcal{G}$-torsors} $f: \mathcal{P}_1 \to \mathcal{P}_2$ is a morphism of sheaves that is equivariant with respect to the $\mathcal{G}$-action.

It is a standard result that every morphism of $\mathcal{G}$-torsors is an isomorphism. 
Consequently, the category of $\mathcal{G}$-torsors in $\mathcal{E}$ forms a groupoid.

\begin{definition}
    We denote the groupoid of $\mathcal{G}$-torsors in $\mathcal{E}$ by $\mathbf{Tors}(\mathcal{E}, \mathcal{G})$ .  The set of isomorphism classes of $\mathcal{G}$-torsors is denoted by $\mathrm{Tors}(\mathcal{E}, \mathcal{G})$.
\end{definition}

Torsors exhibit functoriality with respect to the group object:

\begin{definition}
Let $\varphi: \mathcal{G}_1 \to \mathcal{G}_2$ be a morphism of group 
sheaves on $\sC$, and let $\mathcal{P}$ be a $\mathcal{G}_1$-torsor. We define the \textbf{contracted product} $\mathcal{G}_2 \times^{\mathcal{G}_1} \mathcal{P}$ as the quotient sheaf $(\mathcal{G}_2 \times \mathcal{P}) / \mathcal{G}_1$, where $\mathcal{G}_1$ acts on the product by:
\[ g_1 \cdot (g_2, p) = (g_2 \varphi(g_1)^{-1}, g_1 \cdot p) \]

The contracted product inherits a natural left $\mathcal{G}_2$-action given on local sections by $h \cdot [g_2, p] = [h g_2, p]$, which endows it with the structure of a $\mathcal{G}_2$-torsor. 
\end{definition}

This construction yields a functor:
\[ \varphi_*: \mathbf{Tors}(\cE, \mathcal{G}_1) \to \mathbf{Tors}(\cE, \mathcal{G}_2), \quad \mathcal{P} \mapsto \mathcal{G}_2 \times^{\mathcal{G}_1} \mathcal{P} \]
On the level of isomorphism classes, $\varphi_*$ induces a map of pointed sets $\mathrm{Tors}(\cE, \mathcal{G}_1) \to \mathrm{Tors}(\cE, \mathcal{G}_2)$, sending the class of the trivial $\mathcal{G}_1$-torsor to the class of the trivial $\mathcal{G}_2$-torsor.

When $\mathcal{G}$ is a \textbf{sheaf of abelian groups} (an \textit{abelian sheaf}), the pointed set $\mathrm{Tors}(\cE, \mathcal{G})$ 
inherits the structure of an abelian group. Let $\mathcal{P}_1$ and $\mathcal{P}_2$ be objects of $\mathbf{Tors}(\cE, \mathcal{G})$. We define their sum $[\mathcal{P}_1] + [\mathcal{P}_2]$ to be the class $[\mathcal{P}_3]$, where $\mathcal{P}_3$ is the quotient sheaf $(\mathcal{P}_1 \times \mathcal{P}_2)/\mathcal{G}$. In this construction, $\mathcal{G}$ acts on the product $\mathcal{P}_1 \times \mathcal{P}_2$ on $T$-points by:
\[ g \cdot (f_1, f_2) := (g f_1, g^{-1} f_2) \]

The group object $\mathcal{G}$ then acts on the resulting quotient via its action on the presheaf quotient, which is given on classes by:
\[ g \cdot [(f_1, f_2)] = [(g f_1, f_2)] = [(f_1, g f_2)] \]
where the square brackets denote the class in the quotient set. This structure turns $\mathrm{Tors}(\mathcal{G})$ into an abelian group, 
where the identity is the class of the trivial torsor and the inverse is obtained by the opposite action.\footnote{
    Equivantly, the sum is obtained as the contracted product of the $\cG \times \cG$-torsor $\cP_1 \times \cP_2$ along the multiplication map
$m: \cG \times \cG \to \cG$. }




%%% %%% %%% %%% %%% %%% %%% %%%
%       Torsors as Spaces
%%% %%% %%% %%% %%% %%% %%% %%%
\subsubsection*{Torsors as Flat Spaces}
\textcolor{red}{change Fl to be straight}
We now fix a scheme $X$. We start by focusing on the big fppf site. $X_{Fl}$.
Let $\cG$ be a group sheaf on $X_{Fl}$. A flat-torsor $\cG$-torsor on $X$ is just a $\cG$-torsor on the site $X_{Fl}$.

To the end of this section assume \textbf{$G$ is a flat affine-algebraic $X$-group. } (\textcolor{red}{which by descent is associated also to a group sheaf})
\begin{definition}
Define a \textit{principal $G$-bundle} (or \textit{principal homogenous space} for $G$) to be 
a flat finite presentation $X$-scheme $f : Y \to X$ with an action of $G$ satisfying the following equivalent properties:
\begin{enumerate}
    \item The morphism $Y \times_X G \to Y \times_X Y$ defined on $T$-points by sending $(y, g) \in Y(T) \times G(T)$ to $(y, gy)$ is an isomorphism of $X$-schemes.
    \item There exists an open covering $\{U_i \to X\}$ in $X_{\text{fl}}$ such that $Y_{U_i}$ is isomorphic, as a $G_{U_i}$-space, to $G_{U_i}$ with its left multiplication action.
\end{enumerate}
\end{definition}

Similarly to the case of $G$-torsors of a topos, we define a morphism of principal $G$-bundles to be a morphism of $X$-schemes commuting with the
$G$-action. We now have:
\begin{theorem} 
    The morphism sending $Y \mapsto \mathrm{Hom}_X(-, Y)$ is an equivalence of categories from the category of principal $G$-bundles 
    to the category of $G$-torsors on $X_{\mathrm{Fl}}$. 
    Similarly, the morphism sending $Y$ to $\mathrm{Hom}_X(-, Y)$ to the category of $G$-torsors on $X_{\mathrm{fl}}$ is an equivalence.
\end{theorem}

\begin{cor}
    There is a natural equivalence $\mathbf{Tors}(X_{\mathrm{fl}}, G) \cong \mathbf{Tors}(X_{\mathrm{Fl}}, G)$ inducing a bijection of pointed sets $\mathrm{Tors}(X_{\mathrm{fl}}, G) \xrightarrow{\cong} \mathrm{Tors}(X_{\mathrm{Fl}}, G)$ which is an isomorphism of abelian groups if $G$ is abelian.
    And every $G$-torsor, in each of the above sites can be realized as a scheme, which is a prinicipal $G$-bundle.
\end{cor}


%%% %%% %%% %%% %%% %%% %%% %%%
%       Torsors as Etale Spaces
%%% %%% %%% %%% %%% %%% %%% %%%
\subsubsection*{Torsors as Etale Spaces}
Let $X$ be a scheme, and let $G$ be affine algebraic $X$-group. 
For any topology $\mathcal{T}$ on $\text{Sch}/X$ coarser than the flat topology, we say that 
a flat torsor $\mathcal{P}$ for $G$ is \textit{locally trivial for the $\mathcal{T}$ topology} 
if, in fact, one can find a covering $\{U_i \to X\}$ in $\mathcal{T}$ such that $\mathcal{P}(U_i)$ 
or, equivalently, $\mathcal{P}_{U_i}$ is isomorphic to the trivial torsor for all $i$.

Then, it is immediate that $\Tors(X_{Et, G})$ is canonically isomorphic as pointed sets (abelian groups if $G$ is abelian)
to the subset of $\Tors(X_{Fl}, G)$ consisting of that flat torsors locally trivial for the etale topology.

For the small sites we have:
\begin{theorem}
    If $G$ is smooth affine $X$ group, then any $G$-torsor $\cP$ on $X_fl$ is locally trivial for the etale topology.
\end{theorem}

\begin{cor}\label{cor:tors_etale_flat_eq}
    If $G$ is smooth affine $X$ group, then there is a canonical bijection of pointed sets (abelian groups if $G$ is abelian)
    $\Tors(X_{et}, G) \cong \Tors(X_{fl}, G)$
\end{cor}

And we also have:
\begin{theorem}
    If $G$ is a smooth algebraic $X$-groupm there every flat $G$-torsor is locally trivial for the etale topology and every principal 
    $G$-bundle $Y \to X$ is smooth. In other words the inclusion $\bTors(X_{Et}, G) \to \bTors(X_{Fl}, G)$ is actuall an isomorphism. ??(of what? of categories?)
\end{theorem}

\textcolor{red}{see which of the above theorems we leave intact, cuase it doesn't seem like we need all three}

We now focus on the etale-topology which is coarser then the flat topology. 
Two theorems summarize what happends over the big and small sites:


%%% %%% %%% %%% %%% %%% %%% %%%
%       Torsors and Cohomology
%%% %%% %%% %%% %%% %%% %%% %%%

\subsubsection*{Torsors and Cohomology}
We qoute without proof:
\begin{theorem}
There is a natural bijection of pointed sets     
$\text{Tors}(\mathcal{G|_T}) \to \check{H}^1(T, \mathcal{G})$. Moreover, if $\mathcal{G}$ is an abelian group sheaf, it's an isomorphism of abelian groups.
\end{theorem}

The idea is that for every $[\mathcal{P}] \in \text{Tors}(T, \mathcal{G})$ 
We 
\begin{enumerate}
    \item Choose a covering $\{U_i \to T\}$ such that $\mathcal{P}(U_i) \neq \emptyset$ for all $i$
    \item choose sections $\alpha_i \in \mathcal{F}(U_i)$
\end{enumerate}

Then we note that for all $(i,j)$ the elements $\alpha_i |_{U_i \times_X U_j}$ and $\alpha_j |_{U_i \times_X U_j}$ differ by a unique element of $\mathcal{G}(U_{ij})$
there exists a unique $s_{ij} \in \mathcal{G}(U_{ij})$ such that $\alpha_i |_{U_i \times_X U_j} = s_{ij}(\alpha_j |_{U_i \times_X U_j})$. 
One can then easily see that $(s_{ij})$ defines an element of  $\check{H}^1(T, \mathcal{G})$
which is independent of the choice of repreantiative $\cP$, choice of covering $\{U_i \to T\}$ and choice of sections.

%%% %%% %%% %%% %%% %%% %%% %%%
%       Constant Finite Group Torsors
%%% %%% %%% %%% %%% %%% %%% %%%

\subsubsection*{Constant Finite Group Torsors}
Let $G$ be a finite group. We denote by $\underline{G}$ the constant group scheme $\underline{G}$ over $X$. Sometimes denoted by $\underline{G}_X$
and is given by $\coprod_{g \in G} X$ with the action shuffeling the $X$'s according to multiplication.

By following the definitions, one sees that if $X$ be a connected scheme and $f : Y \to X$ is a finite Galois cover with Galois group $G$. 
Then, $f : Y \to X$ is a principal $\underline{G}$-bundle.
(Recall that a \textit{finite Galois cover} is a finite étale surjection $Y \to X$ with $Y$ connected and such that 
$G = \text{Aut}(Y/X)$ acts transitively on the geometric points of $Y$ lying over any geometric point of $X$.)

On the otherhand if $f: Y \to X$ is a principal $\underline{G}$-bundle with $Y$ connected, then $Y$ is a finite Galois cover with
automorphism group $G$.

However, not all $G$-torsors are connected. If $H\subset G$ is a proper subgroup then any connected finite etale cover $f: Y \to X$ with 
Galois group $H$ gives rise to a non-connected $\underline{G}$-torsor by looking at the induced $G$-torsor $\varphi_*(Y)$ under the inclusion $\varphi: \underline{H} \to \underline{G}$.
On the otherhand, if we fix a geometric point $\overline{x} \to X$, then to give an homomorphism
$\rho \in \mathrm{Hom}_{\mathrm{cont}}(\pi_1^{\mathrm{et}}(X, x), G)$ is equivilant to give a connected pointed Galois cover
$(Y, \overline{y}) \to (X, \overline{x})$ with Galois group $H=\rho(\pi_1^{\mathrm{et}}(X, x)) \subset G$. 
Thus, pushing forward to $G$ we get a prinicipal $\underline{G}$-bundle. The choice of a different geometric point $\overline{x'} \to X$
differ the homomorphism by an inner automorphism, thus we have:

\begin{theorem}\label{theorem:equivalence_fundamental_covers}
    Let $X$ be a connected scheme and $\overline{x}$ a geometric point of $X$. Suppose in addition that $G$ is a finite abstract group. 
    Define a map
    \begin{equation}
        \mathrm{Hom}_{\mathrm{cont}}(\pi_1^{\mathrm{et}}(X, \overline{x}), G) / \mathrm{Inn}(G) \to \mathrm{Tors}(X_{\mathrm{Fl}}, G) 
    \end{equation}
by sending a homomorphism $\rho : \pi_1^{\mathrm{et}}(X, \overline{x}) \to G$ 
to the principal $G$-bundle $\varphi_*(Y)$ where $Y$ is the principal $\underline{\rho(\pi_1^{\mathrm{et}}(X, \overline{x}))}$-bundle 
obtained above and $\varphi$ is the inclusion $\rho(\pi_1^{\mathrm{et}}(X, \overline{x})) \hookrightarrow G$. Then, the map 
 is a bijection of pointed sets where the trivial homomorphisms
  (which is the only element of its $\mathrm{Inn}(G)$-orbit) is the distinguished element of the left hand side.
\end{theorem}

If $G$ is abelian then $\mathrm{Inn}(G)$ is trivial, and we obtain:
\begin{cor}\label{cor:abelian_equivilance_fundamental_torsors}
    Let $G$ be a finite abelian group, $X$ a connected scheme, and $\overline{x}$ a geometric point of $X$. Then, the map from \Cref{theorem:equivalence_fundamental_covers}
     induces an isomorphism of abelian groups
    \begin{equation}
        \mathrm{Hom}_{\mathrm{cont.}}(\pi_1^{\text{ét}}(X, \overline{x}), G) \xrightarrow{\cong} \Tors(X_{\mathrm{Fl}}, G)
    \end{equation}
\end{cor}

\textcolor{red}{Let us give a final note that, evidently, $\mathrm{Aut}(G)$ acts on $\mathrm{Hom}_{\mathrm{cont.}}(\pi_1^{\text{ét}}(X, x), G)$ on the right, and if we consider the quotient $\mathrm{Hom}_{\mathrm{cont.}}(\pi_1^{\text{ét}}(X, x), G)/\mathrm{Aut}(G)$ we get the pointed set of all connected finite Galois covers of $X$ with Galois group isomorphism to a subgroup of $G$.}

%%% %%% %%% %%% %%% %%% %%% %%% %%% %%% %%% %%% %%% %%% %%% %%%
%       Qoutient of G-torsors
%%% %%% %%% %%% %%% %%% %%% %%% %%% %%% %%% %%% %%% %%% %%% %%%

\subsubsection*{Qoutients of \texorpdfstring{$G$}{G}-Torsors}
Let $G$ be a finite abelian group, and let $H \subseteq G$ be a subgroup. 
 Consider the natural quotient homomorphism $\pi: G \to G/H$. Given a $G$-torsor $\mathcal{P} \to X$ (in the fppf or étale topology), we may associate to it a $G/H$-torsor, 
 denoted by $\pi_*(\mathcal{P})$ or $\mathcal{P}^{G/H}$, via the extension of scalars.
 The object $\mathcal{P}^{G/H}$ is defined as the contracted product:
 $$\mathcal{P}^{G/H} = \mathcal{P} \times^G (G/H)$$
 Recall that $\mathcal{P}^{G/H}$ is the quotient of the product $\mathcal{P} \times (G/H)$ by the $G$-action defined by 
 $g \cdot (p, \bar{g}') = (g \cdot p, \pi(g^{-1})\bar{g}')$. 
 Consequently, we have a canonical morphism of sheaves 
 $\phi: \mathcal{P} \to \mathcal{P}^{G/H}$ which, on local sections, acts by 
 $p \mapsto [p, \bar{e}]$, where $\bar{e}$ is the identity element in $G/H$. 
 To see that $\mathcal{P}$ carries the structure of an $H$-torsor over $\mathcal{P}^{G/H}$, 
 consider a local trivializing cover $\{U_i \to X\}$ for $\mathcal{P}$ as a $G$-torsor. 
 Over each $U_i$, we have an isomorphism $\mathcal{P}|_{U_i} \cong G_{U_i}$. 
 Under this isomorphism, the contracted product locally satisfies:
 $$(\mathcal{P} \times^G G/H)|_{U_i} \cong (G \times^G G/H)_{U_i} \cong (G/H)_{U_i}$$
 Explicitly, the local identification $(g, \bar{g}') \sim (e, \pi(g)\bar{g}')$ 
 shows that every equivalence class in the fiber has a unique representative of the form 
 $[e, \bar{g}]$. 
 The map $\phi$ locally corresponds to the quotient map $G \to G/H$. 
 Since the kernel of this map is $H$, and $G$ is a trivial $H$-torsor over $G/H$, 
 it follows by descent that $\mathcal{P}$ is an $H$-torsor over $\mathcal{P}^{G/H}$.
 We summarize this construction in the following proposition:
 \begin{prop}\label{prop:quotient_torsors}
    Let $G$ be an abelian group and $H \subset G$ a subgroup. 
    Any $G$-torsor $\mathcal{P} \to X$ admits a natural factorization:
    $$\mathcal{P} \xrightarrow{} \mathcal{P}^{G/H} \xrightarrow{} X$$
    where $\mathcal{P}^{G/H} \to X$ is a $G/H$-torsor and $\mathcal{P} \to \mathcal{P}^{G/H}$ is an $H$-torsor.
\end{prop}

%%% %%% %%% %%% %%% %%% %%% %%% %%% %%% %%% %%% %%% %%% %%% %%%
%       Equivalence between Torsors and Invertible Modules
%%% %%% %%% %%% %%% %%% %%% %%% %%% %%% %%% %%% %%% %%% %%% %%%

\subsubsection*{Equivalence between Torsors and Invertible Modules}
\textcolor{red}{Edit this completly.}
Let $\mathcal{E}$ be a topos and let $\Lambda$ be a commutative ring object in $\mathcal{E}$. Let $G = \Lambda^\times$ denote the internal group object of units of $\Lambda$.
The following proposition establishes the fundamental dictionary between the geometric theory
of principal homogeneous spaces and the algebraic theory of invertible modules.
This equivalence allows us to transport the monoidal structure from the category of modules
(with the tensor product over $\Lambda$) to the category of torsors (with the contracted product over $G$),
strictly within the categorical framework.

\begin{prop}\label{prop:torsor_module_equivalence}
    There is a canonical equivalence of monoidal categories between the category of $G$-torsors in $\mathcal{E}$ and the category of locally free $\Lambda$-modules of rank 1 in $\mathcal{E}$:
    \[
        \Phi: \mathbf{Tors}(\mathcal{E}, \Lambda^\times) \xrightarrow{\sim} \mathbf{Pic}(\mathcal{E}, \Lambda)
    \]
    The equivalence is defined by the associated module functor:
    \[
        \cP \longmapsto \cP \times^{\Lambda^\times} \Lambda := \Lambda^\times \backslash (\Lambda \times \cP)
    \]
    where the quotient is taken with respect to the diagonal action of $\Lambda^\times$ on $\Lambda \times \cP$. 
    The inverse functor associates to an invertible module $\cF$ its sheaf of basis frames $\underline{\mathrm{Isom}}_\Lambda(\Lambda, \cF)$.
\end{prop}

In light of this canonical equivalence, we will pass freely between the language of $G$-torsors and that of locally free $\Lambda$-modules throughout the text.

For a topos $\cE$, a group object $G$ in $\cE$ and an object $X$ in $\cE$, there is a canonical identification
between $({G\cE})_{/X}$ and $G(\cE_{/X})$, given by endowing $X$ with the trivial $G$-action.

We denote by $\mathbf{Tors}(X, G)$ the category of $G$-torsors over $X$ in $G \cE_{/X}$.
Similarly, for a ring object $\Lambda$ in $\cE$, we denote by $\mathbf{Pic}(X, \Lambda)$ the category of locally free $\Lambda$-modules of rank 1 over $X$ in $\cE_{/X}$.
The above equivilance of categories becomes 
\[
    \Phi_X: \mathbf{Tors}(X, \Lambda^\times) \xrightarrow{\sim} \mathbf{Pic}(X, \Lambda)
\].

For a morphism $f: Y \to X$ in $\mathcal{E}$, the equivalence is functorial with respect to:
\begin{itemize}
    \item \textbf{Torsor Pullback:} $f^{-1}P = P \times_{X} Y$ (Fiber product).
    \item \textbf{Module Pullback:} $f^*\mathcal{L} = \Lambda_Y \otimes_{f^{-1}\Lambda_X} f^{-1}\mathcal{L}$ (Extension of scalars).
\end{itemize}

The following diagram commutes up to natural isomorphism:

\[
\begin{tikzcd}[column sep=large, row sep=large]
\mathbf{Tors}(X, G) \arrow[r, "\Phi_X", "\sim"'] \arrow[d, "f^{-1}"'] & \mathbf{Pic}(X, \Lambda) \arrow[d, "f^*"] \\
\mathbf{Tors}(Y, G) \arrow[r, "\Phi_Y", "\sim"'] & \mathbf{Pic}(Y, \Lambda)
\end{tikzcd}
\]


\subsubsection*{\texorpdfstring{More about $G$-torsors}{More about G-torsors}}
We want to be more explicit about $G$-torsors, so let us recall the definition.

\subsection*{Other Theorems}
\textcolor{red}{Say something about the toposes of Etale and etale, maybe add them up in the notation.}
We recall some propositions about $G$-torsors that will be useful later.

\begin{prop}[\cite{Guignard2018}, Proposition 2.12]\label{prop:torsor_representable}
Let $G$ be a finite abelian group, let $S$ be a scheme, and let $\cP$ be a $G$-torsor over an $S$-scheme $X$ in $S_{\text{Et}}$.
Then the etale sheaf $\cP$ is representable by a finite etale $X$-scheme.
\end{prop}
\begin{cor}[\cite{Guignard2018}, Corollary 2.13]
    Let $G$ be a finite abelian group, let $S$ be a scheme, and let $X$ be an $S$-scheme.
    Then the category of $G$-torsors over $X$ in $S_{\text{Et}}$ is equivalent to the category of $G$-torsors over $X$ (the terminal object) in $X_{\text{ét}}$.
\end{cor}

Next, we recall the definition of the contracted product of torsors, which endows the category of $G$-torsors with a monoidal structure.


\subsection{Symmetric Powers of Schemes and Torsors}
This section reviews the construction of quotients for schemes and torsors under finite group actions, specifically focusing on symmetric powers. 
To ensure these quotients exist as schemes, we utilize the framework of admissible actions from \cite{sga1}. 
Our treatment here closely follows the exposition in \cite{Guignard2018}
The definitions and results presented below are adapted from their work. 
This foundation provides the necessary criteria for admissibility and base change required to define the symmetric powers of a scheme $X$
 and a $G$-torsor $\mathcal{P}$ over $X$.


Let $S$ be a scheme.

\begin{definition}[(\cite{sga1}, V.1.7).]
    \noindent
    \begin{itemize}
        \item Let $T$ be an object of a category $\mathcal{C}$ endowed with a right action of a group $\Gamma$. We say that \textbf{the quotient $T/\Gamma$ exists} in $\mathcal{C}$ if the covariant functor
    \[
    \begin{aligned}
    \mathcal{C} &\to \text{Sets} \\
    U &\mapsto \text{Hom}_{\mathcal{C}}(T, U)^{\Gamma}
    \end{aligned}
    \]
    is representable by an object of $\mathcal{C}$.
    \item Let $T$ be an $S$-scheme. An action of a finite group $\Gamma$ on $T$ is \textbf{admissible} if there exists an affine $\Gamma$-invariant morphism $f : T \to T'$ such that the canonical morphism $\mathcal{O}_{T'} \to f_* \mathcal{O}_T$ induces an isomorphism from $\mathcal{O}_{T'}$ to $(f_* \mathcal{O}_T)^{\Gamma}$.
\end{itemize}
\end{definition}

\begin{prop}
    The following holds:
    \begin{enumerate}
        \item \textbf{(\cite{sga1} V.1.3)}. Let $T$ be an $S$-scheme endowed with an admissible right action of a finite group $\Gamma$. If $f : T \to T'$ is an affine $\Gamma$-invariant morphism such that the canonical morphism $\mathcal{O}_{T'} \to f_* \mathcal{O}_T$ induces an isomorphism from $\mathcal{O}_{T'}$ to $(f_* \mathcal{O}_T)^{\Gamma}$, then the quotient $T/\Gamma$ exists and is isomorphic to $T'$.
        \item \textbf{(\cite{sga1}, V.1.8)}. Let $T$ be an $S$-scheme endowed with a right action of a finite group $\Gamma$. Then, the action of $\Gamma$ on $T$ is admissible if and only if $T$ is covered by $\Gamma$-invariant affine open subsets.
        \item \textbf{(\cite{sga1}, V.1.9)}. Let $T$ be an $S$-scheme endowed with an admissible right action of a finite group $\Gamma$, and let $S'$ be a flat $S$-scheme. Then, the action of $\Gamma$ on the $S'$-scheme $T \times_S S'$ is admissible, and the canonical morphism
            \[
            (T \times_S S')/\Gamma \to (T/\Gamma) \times_S S'
            \]
            is an isomorphism.
    \end{enumerate}
\end{prop}

\begin{prop}[{[SGA1]}, IX.5.8]
Let $G$ be a finite abelian group, let $\cP$ be a $G$-torsor over an $S$-scheme $X$ in $S_{\text{Ét}}$. 
Assume that $\cP$ and $X$ are endowed with right actions from a finite group $\Gamma$ 
such that the morphism $\cP \to X$ is $\Gamma$-equivariant, and that the following properties hold:
\begin{enumerate}
    \item[(a)] The right $\Gamma$-action on $\cP$ commutes with the left $G$-action.
    \item[(b)] The right $\Gamma$-action on $X$ is admissible, and the quotient morphism $X \to X/\Gamma$ is finite.
    \item[(c)] For any geometric point $\bar{x}$ of $X$, the action of the stabilizer $\Gamma_{\bar{x}}$ of $\bar{x}$ in $\Gamma$ on the fiber $\cP_{\bar{x}}$ of $\cP$ at $\bar{x}$ is trivial.
\end{enumerate}
Then the action of $\Gamma$ on $\cP$ is admissible, and $\cP/\Gamma$ is a $G$-torsor over $X/\Gamma$ in $S_{\text{Ét}}$.
\end{prop}


\subsubsection*{Symmetric Powers of Schemes}
Let $X$ be an $S$-scheme and let $d \geq 0$ be an integer. The group ${S}_d$ of permutations of $\llbracket 1, d \rrbracket$ acts on the right on the $S$-scheme $X^{\times_S d} = X \times_S \dots \times_S X$ by the formula
\[
(x_i)_{i \in \llbracket 1, d \rrbracket} \cdot \sigma = (x_{\sigma(i)})_{i \in \llbracket 1, d \rrbracket}.
\]

\begin{prop}[\cite{Guignard2018} Proposition 2.27]
    If $X$ is a scheme, Zariski locally quasi-projective over $S$, then the right action of the symmetric group $S_d$ on the $d$-fold fiber product 
    $X^{\times_S d}$ is admissible. 
    Consequently, the quotient $\mathrm{Sym}_S^d(X) = X^{\times_S d}/S_d$ exists as a scheme over $S$.
\end{prop}
\begin{rem*}
    When the base $S$ is understood from context, this quotient is also denoted by $X^{(d)}$.
\end{rem*}

Guingard shows that when $X=\Spec(B)$ and $S=\Spec(A)$ then $\Sym_{S}^d(X)$ is representable by an affine $S$-scheme (See \cite{Guignard2018} Remark 2.28).

\begin{prop}[\cite{Guignard2018} Proposition 2.28]\label{prop:symmetric_power_flat_base_change}
If $X$ is flat and Zariski-locally quasi-projective over $S$, then $\text{Sym}_S^d(X)$ is flat over $S$. Moreover, for any $S$-scheme $S'$, the canonical morphism
\[
\text{Sym}_{S'}^d(X \times_S S') \to \text{Sym}_S^d(X) \times_S S'
\]
is an isomorphism.
\end{prop}

\subsubsection*{Symmetric Powers of Torsors}
\textcolor{red}{change torsor tensor product to contracted product,
and the analogy with sheaves to tensor product (which it is) below the exposition to be more accurate... }


\textcolor{red}{change below the exposition to be more accurate... }
Let $S$ be a scheme, let $X$ be an $S$-scheme and let $d \geq 1$ be an integer. 
Let $G$ be a finite abelian group, and let $\cP \to X$ be a $G$-torsor over $X$ in $S_{\text{Ét}}$. 
It is easy to show that the sheaf $\cP$ is representable by a finite étale $X$-scheme. (For example \cite{Guignard2018} Proposition 2.12)

For each $i \in \llbracket 1, d \rrbracket$ let $p_i : X^{\times_S d} \to X$ be the projection on $i$-th factor, and let us consider the $G$-torsor
\[
p_1^{-1}\cP \otimes \cdots \otimes p_d^{-1}\cP = G_d \backslash \cP^{\times_S d}
\]
over $X^{\times_S d}$, where $G_d \subseteq G^d$ is the kernel of the multiplication morphism $G^d \to G$. 
The object $G_d \backslash \cP^{\times_S d}$ of $S_{\text{Ét}}$ is too representable by an $S$-scheme which is finite étale over $X^{\times_S d}$. 
The group ${S}_d$ acts on the right on $G_d \backslash \cP^{\times_S d}$ by the formula
\[
(p_i)_{i \in \llbracket 1, d \rrbracket} \cdot \sigma = (p_{\sigma(i)})_{i \in \llbracket 1, d \rrbracket}.
\]
This action of ${S}_d$ commutes with the left action of $G$ on $G_d \backslash \cP^{\times_S d}$.

\medskip

\begin{prop}[\cite{Guignard2018} Proposition 2.32.]\label{prop:symmetric_power_torsor}
If $X$ is Zariski-locally quasi-projective on $S$, then the right action of ${S}_d$ on $G_d \backslash \cP^{\times_S d}$ is admissible, 
so that the quotient $\cP^{(d)}$ of $G_d \backslash \cP^{\times_S d}$ by ${S}_d$ exists as a scheme over $S$. 
Moreover, the canonical morphism $\cP^{(d)} \to \mathrm{Sym}_S^d(X)$ is a $G$-torsor, 
and the morphism
\[
p_1^{-1}\cP \otimes \cdots \otimes p_d^{-1}\cP \to r^{-1}\cP^{(d)}
\]
where $r : X^{\times_S d} \to \mathrm{Sym}_S^d(X)$ is the canonical projection, is an isomorphism of $G$-torsors over $X^{\times_S d}$.
\end{prop}
\textcolor{red}{consider replacing $\cP$ with $P$ because it is a scheme}
\textcolor{red}{Add proposition about how it is being a scheme}



%
% SYMMETRIC POWERS OF LOCAL SYSTEMS ON CURVES
%
\subsection{Symmetric Powers of Local Systems on Curves}\label{section:symmetric_powers_on_curves}
\textcolor{red}{don't need modulus in this section, etc.}
Let $k$ be a perfect field. 
Let $C$ be a projective smooth geometrically connected curve over $k$, with genus $g$.
Let $\mdls$ be a modulus on $C$ and let $U = C \setminus \mdls$.
Let $G$ be a finite abelian group and let $\cP$ be a $G$-torsor on $U$ with ramification bounded by
$\mdls$. Let $d \geq \deg m$.

We have the following diagram:
\[
\begin{tikzcd}
U^{(d_1)} \times_k U^{(d_2)} \arrow[r, "p_1"] \arrow[d, "p_2"] & U^{(d_1)} \\
U^{(d_2)} &  {}
\end{tikzcd}
\] 

pullbacking $\cP^{(d_i)}$ along the projections we get a $G$-torsor

\[\cP ^{(d_1)} \boxtimes \cP ^{(d_2)} = p_1^{-1} \cP^{(d_1)} \otimes p_2^{-1} \cP^{(d_2)}\]
On $U^{(d_1)} \times_k U^{(d_2)}$

Note that the plus map $C^{(d_1)} \times_k C^{(d_2)} \xrightarrow{+}{} C^{(d_1 + d_2)}$
is induced from
\[
\begin{tikzcd}
C^{d_1} \times_k C^{d_2} \arrow[r, "\cong"] \arrow[d, "r_1 \times r_2"] & C^{d_1 + d_2} \arrow[d, "r"] \\
C^{(d_1)} \times_k C^{(d_2)} \arrow[r, "+"] & C^{(d_1+ d_2)} 
\end{tikzcd}
\]


Hence, by \Cref{prop:symmetric_power_torsor} (and replacing $C$ with $U$ above) we get canonical identification:
\[
 (+^{-1})(\cP^{(d_1+d_2)}) \cong \cP ^{(d_1)} \boxtimes \cP ^{(d_2)}
\]

\subsection{Algebraic Preliminaries on Ramification}
\textcolor{red}{change?}
We recall the basic definitions and properties of the ramification of discrete valuations. 
We start with the general case of discrete valuation rings and their integral closures within finite separable field extensions. 
Then, we move to the specific setting of complete discrete valuation rings within Galois extensions, where we describe the ramification filtration of the Galois group via both lower and upper numbering.
We follow \stackstag{0EXQ}, and \cite{serreLF}.

\subsubsection*{Ramification of Discrete Valuation Rings}
Let $A$ be a discrete valuation ring with fraction field $K$. Let $L/K$ be a finite separable field extension. Let $B \subset L$ be the integral closure of $A$ in $L$. Picture:

\[ \xymatrix{ B \ar[r] & L \\ A \ar[u] \ar[r] & K \ar[u] } \]
By \stackstag{032L} the ring extension $A \subset B$ is finite, hence $B$ is Noetherian. 
By \stackstag{00OK} the dimension of $B$ is $1$, hence $B$ is a Dedekind domain, see \stackstag{034X}. 
Let $\mathfrak m_1, \ldots , \mathfrak m_ n$ be the maximal ideals of $B$ (i.e., the primes lying over $\mathfrak m_ A$). We obtain extensions of discrete valuation rings

\[ A \subset B_{\mathfrak m_ i} \]
and hence ramification indices $e_ i$ and residue degrees $f_ i$. We have

\[ [L : K] = \sum \nolimits _{i = 1, \ldots , n} e_ i f_ i \]
by \stackstag{02MJ} applied to a uniformizer in $A$. We observe that $n = 1$ if $A$ is henselian (by \stackstag{04GH} and the fact that $B$ is a domain), e.g. if $A$ is complete.

\begin{definition}\label{definition:tamely_ramified_dvrs}
Let $A$ be a discrete valuation ring with fraction field $K$. Let $L/K$ be a finite separable extension. With $B$ and $\mathfrak m_ i$, $i = 1, \ldots , n$ 
as above, we say the extension $L/K$ is
\begin{enumerate}
    \item unramified with respect to $A$ if $e_ i = 1$ and the extension $\kappa (\mathfrak m_ i)/\kappa _ A$ is separable for all $i$,
    \item tamely ramified with respect to $A$ if either the characteristic of $\kappa _ A$ is $0$ or the characteristic of $\kappa _ A$ is $p > 0$, the field extensions $\kappa (\mathfrak m_ i)/\kappa _ A$ are separable, and the ramification indices $e_ i$ are prime to $p$, and
    \item totally ramified with respect to $A$ if $n = 1$ and the residue field extension $\kappa (\mathfrak m_1)/\kappa _ A$ is trivial.
\end{enumerate}
If the discrete valuation ring $A$ is clear from context, then we sometimes say $L/K$ is unramified, totally ramified, or tamely ramified for short.
\end{definition}

\subsubsection*{Structure Theorems and Some Lemmas}\label{subsubsection:structure_theorems_ramification}
Let $A$ be a complete discrete valuation ring over with uniformizer $\pi$ and residue field $\kappa$, which we assume to be perfect.
When $A$ and $\kappa$ are of the same characteristic $p > 0$, then $A$ contains a coefficient field $k \cong \kappa$ and a well known structure theorem holds: $A = k[[\pi ]] \cong k[[t]]$.
Let $K$ be the fraction field of $A$, then $K=k((\pi))$.
By \textcolor{red}{Kummer theory}, unramified extensions of $K$ correspond to separable extensions of $k$.
The maximal unramified extension of $K$ is $\overline{k}((\pi))$ where $\overline{k}$ is a separable closure of $k$.
 \textcolor{red}{Maybe add something about the above facts.}


%We will need that broad definition of tame ramification in order to deal with higher dimensional schemes, 
%however note that in the case of global and local fields (taking $A$ to be the local ring at a prime of the ring of integers), 
%this definition of tame ramification coincides with the usual one.

\subsubsection*{Classical Ramification Filtration in the Galois Case}
We now recall the classical ramification filtration in the Galois case.
Assume $A,B$ are complete DVRs. And that $L/K$ is Galois with Galois group $G$. 
In that case there is uniformizer $\pi \in B$ such that $B=A[\pi ]$ 
   
We have the ramification filtration of $G$ by lower numbering $(G_i)_{i \geq -1}$, defined by
\[ G_i = \{ \sigma \in G \mid v_B(\sigma (x) - x) \geq i + 1 \text{ for all } x \in B \} \]
where $v_B$ is the valuation on $L$ associated to $B$. 
In particular, $G_{-1} = G$ and $G_0$ is the inertia group of the extension $L/K$. 
We have that $L/K$ is unramified if and only if $G_0$ is trivial, and $L/K$ is tamely ramified if and only if $G_1$ is trivial.
It is easy exercise that in the definition of $G_i$ it is enough to check the condition for the uniformizer $\pi$ of $B$, if we define
$i^L_K(\sigma ) = v_B(\sigma (\pi ) - \pi )$ for $\sigma \in G$, then we have $G_i = \{ \sigma \in G \mid i^L_K(\sigma ) \geq i + 1 \}$.
The groups $G_i$ are normal in $G$ and are trivial for large enough $i$. In a tower of fields $K \subset E \subset L$, where $H=\Gal(L/E)$ we have
\[ G_i \cap H = H_i \] for all $i \geq -1$, which corresponds to the fact that $i^L_E = i^L_K|_{\mathrm{Gal}(L/E)}$.
Ramification groups also behave well with respect to quotients: $G_i H/H=(G/H)_j$.
where \[ j=\frac{1}{e_{L/E}}\sum_{\tau \in H} \min(i^L_K(\tau), i+1)-1 \]
i.e. the quotient of a ramification group is itself a ramification group, but with a different index.
In the literature, one reindexes the ramification groups by defining the Herbrand function $\phi_{L/K} : [-1, \infty ) \to [-1, \infty )$:
\[ \phi_{L/K}(i) = \frac{1}{e_{L/K}}\sum_{\sigma \in G} \min(i^L_K(\sigma), i+1)-1 = \int _0^i \frac{1}{[G_0 : G_t]} dt \]
It is continuous, increasing, piecewise linear function, hence a bijection.
It satisfies $\phi_{L/K}=\phi_{E/K} \circ \phi_{L/E}$ for $K \subset E \subset L$, and $G_iH/H=(G/H)_{\phi_{L/E}(i)}$.
Thus, defining the ramification groups by upper numbering as $G^i = G_{\phi_{L/K}^{-1}(i)}$, we have:
\[ G^i H/H = (G/H)^i \]
for all $i \geq -1$.


\subsection{Kummer and Artin-Schreier Theories}
We recall the basic theorms from both theories regarding cyclic extesntions and ramifications. 
Throughout this section let $K$ be a discrete valuation field with perfect residue field $\kappa$ of characteristic $p > 0$. 


\begin{theorem}[Ramification in Kummer Extensions, \cite{koch1997algebraic}, Proposition 1.83]\label{theorem:kummer_ramification}
%Let $K$ be a local field of characteristic $\text{char}(K) =p$  
Assume $K$ contains the $n$-th roots of unity $\mu_n$. 
Let $L/K$ be the extension given by the equation $X^n = a$ for some 
$a \in K^\times$ and denote by $G$ its Galois group. Then we have:

\begin{enumerate}
    \item If $v_K(a) \in n\mathbb{Z}$ and the image of $a\pi^{-v_K(a)}$ in the residue field $\kappa$ is an $n$-th power, the extension $L/K$ is trivial.
    \item If $v_K(a) \in n\mathbb{Z}$ and the image of $a\pi^{-v_K(a)}$ in the residue field $\kappa$ is not an $n$-th power, the extension $L/K$ is cyclic and unramified.
    \item If $v_K(a) \notin n\mathbb{Z}$, the extension $L/K$ is cyclic and ramified. Specifically, if $\gcd(|v_K(a)|, n) = 1$, the extension is totally ramified of degree $n$.
     Otherwise it has ramification index $\frac{n}{|gcd(v_K(a)|, n)}$
\end{enumerate}
Conversly, Kummer theory ensures that every cyclic extension of degree $n$, prime to $p$ of a field that contains $n$-th roots of unity, is of the above form.
Moreover, in the above we can always take $a \in \Oo_K$
\end{theorem}
Note that in the case of total ramification the extesntion is tamely ramified.


\begin{theorem}[Ramification in Artin-Schreier Extensions, \cite{Thomas2005}]\label{theorem:artin_schreier_ramification}
Let $\wp(x) = x^p - x$ be the Artin-Schreier opeartor.
Let $L/K$ be the extension given by the equation $X^p - X = a$ for some $a \in K$ and denote by $G$ its Galois group. 
Then we have:
\begin{enumerate}
    \item If $v_K(a) > 0$ or if $v_K(a) = 0$ and $a \in \wp(K)$, the extension $L/K$ is trivial.
    \item If $v_K(a) = 0$ and if $a \notin \wp(K)$, the extension $L/K$ is cyclic of degree $p$ and unramified.
    \item If $v_K(a) = -m < 0$ with $m \in \mathbb{Z}_{>0}$ and if $m$ is prime to $p$, the extension $L/K$ is cyclic of degree $p$ again and totally ramified. Moreover,
     its ramification groups are given by:
     $$G = G^{(-1)} = \dots = G^{(m)} \quad \text{and} \quad G^{(m+1)} = 1.$$
\end{enumerate}
Conversely, Artin-Schreier theory ensures that every cyclic extension of degree $p$ takes this form.
Moreover, in the above and under the isomorphism $K \cong k((t))$, one can always take $a$ of the form 
$c t^{-m} + a_{-m + 1} t^{-m + 1} + \cdots a_{-1}t^{-1} + a_0$. If $k$ is algebraically closed then there is a 
change of variables such that $a=u^{-m}$.
\end{theorem}



%%%% %%%% %%%% %%%% %%%% %%%% %%%% %%%%
%          ALGEBRAIC GEOMETRY
%%%% %%%% %%%% %%%% %%%% %%%% %%%% %%%%
\subsection{Algebraic Geometry}
In this section we group together some general theorems in algebraic geometry that we will be employing throughtout the text. 
All schemes are assumed to be locally of finite type. 
\begin{theorem}
    Let $f:X \to Y$ be a finite flat map between integral schemes, of finite type over a field $k$. if $Z \subset X$ is a prime divisor with generic point $\eta_Z$, then $f(Z) \subset X$
    is a prime divisor with generic point $\eta_{f(Z)}$ satisfying $f(\eta_Z)= \eta_{f(Z)}$
\end{theorem}
\begin{proof}
    $f$ is finite hence proper hence closed so $f(Z)$ is closed subset of $Y$, it is irreducible as the image of an irreducible.
    Since $Z = \overline{\{ \eta_Z \}}$ we get: 
    $$\{f(\eta_Z)\} \subset f(Z) = f(\overline{\{\eta_Z\}}) \subseteq \overline{f(\{\eta_Z\})} = \overline{\{f(\eta_Z)\}}$$
    And since $f(Z)$ is closed we get $f(Z) = \overline{\{f(\eta_Z)\}}$.

    For flat map of integral schemes we have for every $x \in X$, $y=f(x)$ the dimension formula:
    $$\text{dim}(\mathcal{O}_{X,x}) = \text{dim}(\mathcal{O}_{Y,y}) + \text{dim}(\mathcal{O}_{X_y, x})$$
    And since $\text{dim}(\mathcal{O}_{X_y, x}) = 0$ we get $\text{dim}(\mathcal{O}_{X,x}) = \text{dim}(\mathcal{O}_{Y,y})$
    concluding that $f(Z)$ is a prime divisor as well. 
\end{proof}

A known theorem states that: 
\begin{theorem}\label{theorem:int_geoint_implies_productint}
    Let $X, Y$ be two integral schemes over a field $k$. If $X$ is geometrically integral then $X \times_k Y$ 
    is integral.
    If both $X,Y$ are geometrically integral, then $X \times_k Y$ is geometrically integral. 
\end{theorem}

\begin{theorem}
    Let $C$ be smooth projective curve geometrically connected over a field $k$. Then:
    \begin{enumerate}
        \item $C^{(d)}$ is smooth
        \item For every $d$, $C^{(d)}$ is integral.
        \item For every $d$, $C^{(d)}$ is geometrically integral.
        \item The product of every finite number of $C^{(d)}$ is geometrically integral. 
    \end{enumerate}
    \begin{proof}
        \begin{enumerate}
            \item Let $t_i$ be a local parameter for $C$ at $P_i$. The local ring of the product $C^d$ at the point $(P_1, \dots, P_d)$ is isomorphic $k[[t_1, t_2, \dots, t_d]]$
            and the local ring of the qoutient at the divisor $D=\sum P_i$ is $k[[t_1, \dots, t_d]]^{S_d}$ which is 
            isomorphic to $k[[t_1, \dots, t_d]]^{S_d} \cong k[[s_1, \dots, s_d]]$ where the $s_i$ are the 
            symmetric polynomials, hence this ring is regular local ring. 
            \item $C$ is irreducible hence $C^{d}$ is irreducible hence $C^{(d)}$ is irreducible. Since $C^{(d)}$ 
            is smooth it is reduced.
            \item By \stackstag{0366}, $C$ is geometrically integral, so it follows from the above. 
            \item \Cref{theorem:int_geoint_implies_productint}
        \end{enumerate}
    \end{proof}
\end{theorem}


\subsubsection*{Blowups}

\begin{theorem}[\stackstag{0805}]
Let $X_1 \to X_2$ be a flat morphism of schemes. Let $Z_2 \subset X_2$ be a closed subscheme. Let $Z_1$ be the inverse image of $Z_2$ in $X_1$. Let $X'_ i$ be the blowup of $Z_ i$ in $X_ i$. Then there exists a cartesian diagram

\[ \xymatrix{ X_1' \ar[r] \ar[d] & X_2' \ar[d] \\ X_1 \ar[r] & X_2 } \]
of schemes.   
\end{theorem}

\begin{theorem}
If $X$ is integral then $Bl_Z(X)$ is integral. %This can be checked locally
\end{theorem}
