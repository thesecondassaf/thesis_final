\section{Preliminaries}
This section establishes the foundational definitions and theorems necessary for the remainder of this work. 
We focus specifically on the theory of $G$-torsors, which are central to our study due to their correspondence with 
locally free sheaves of rank 1. 
Subsequently, we review the relevant background on ramification theory from the existing literature. 
Finally, we conclude with several algebraic geometric remarks and notational conventions that will be employed 
implicitly throughout this thesis.

%%%% %%%% %%%% %%%% %%%% %%%% %%%% %%%%
%               TORSORS
%%%% %%%% %%%% %%%% %%%% %%%% %%%% %%%%
\subsection{Torsors}
% $X$ with etale site, but say things here work for toposes in general and flat sites in general.

In what follows, we largely adhere to the treatment of torsors and group objects found in publised notes of Alex Youcis \cite{YoucisTorsors}.
Let $\sC=(\cC, J)$ be a site and let $\cE=Sh(\sC)$ be the associated topos. 
Let $\mathcal{G}$ be a group object in $\mathcal{E}$. 
We denote by $\mathcal{G}\mathcal{E}$ the category of objects in $\mathcal{E}$ endowed with a left $\mathcal{G}$-action. 

\begin{definition}
    A \textbf{$\mathcal{G}$-torsor} in $\mathcal{E}$ is an object $\mathcal{P}$ of $\mathcal{G}\mathcal{E}$ satisfying the following conditions:
    \begin{enumerate}
        \item The structural morphism $\mathcal{P} \to 1$ is an epimorphism in $\mathcal{E}$ (i.e., $\mathcal{P}$ is locally non-empty).
        \item The map $\mathcal{G} \times \mathcal{P} \to \mathcal{P} \times \mathcal{P}$ defined by $(g, p) \mapsto (g \cdot p, p)$ is an isomorphism in $\mathcal{E}$ (i.e., $\mathcal{G}$ acts simply transitively on $\mathcal{P}$).
    \end{enumerate}
\end{definition}

Since $\mathcal{E}$ is the topos of sheaves on a site $\sC$, the definition can be reformulated in terms of covers. A $\mathcal{G}$-sheaf $\mathcal{P}$ is a $\mathcal{G}$-torsor if:
\begin{enumerate}
    \item For every object $X \in \mathcal{C}$, there exists a covering $\{ U_i \to X \} \in J$ such that $\mathcal{P}(U_i) \neq \emptyset$ for all $i$.
    \item For any $X \in \mathcal{C}$ where $\mathcal{P}(X)$ is non-empty, the action of $\mathcal{G}(X)$ on $\mathcal{P}(X)$ is simply transitively.
\end{enumerate}

A fundamental property of torsors is their local triviality: 
a $\mathcal{G}$-sheaf $\mathcal{P}$ is a $\mathcal{G}$-torsor if and only if it is locally isomorphic to the trivial torsor. 
Specifically, for every $X \in \mathcal{C}$, there must exist a cover $\{ U_i \to X \}$ such that the restriction $\mathcal{P}|_{U_i}$ is 
isomorphic, as a $\mathcal{G}|_{U_i}$-sheaf, to $\mathcal{G}|_{U_i}$ acting on itself by left multiplication.

A \textbf{morphism of $\mathcal{G}$-torsors} $f: \mathcal{P}_1 \to \mathcal{P}_2$ is a morphism of sheaves that is equivariant with respect to the $\mathcal{G}$-action.

It is a standard result that every morphism of $\mathcal{G}$-torsors is an isomorphism. 
Consequently, the category of $\mathcal{G}$-torsors in $\mathcal{E}$ forms a groupoid.

\begin{definition}
    We denote the groupoid of $\mathcal{G}$-torsors in $\mathcal{E}$ by $\mathbf{Tors}(\mathcal{E}, \mathcal{G})$ .  The set of isomorphism classes of $\mathcal{G}$-torsors is denoted by $\mathrm{Tors}(\mathcal{E}, \mathcal{G})$.
\end{definition}

Torsors exhibit functoriality with respect to the group object:

\begin{definition}
Let $\varphi: \mathcal{G}_1 \to \mathcal{G}_2$ be a morphism of group 
sheaves on $\sC$, and let $\mathcal{P}$ be a $\mathcal{G}_1$-torsor. We define the \textbf{contracted product} $\mathcal{G}_2 \times^{\mathcal{G}_1} \mathcal{P}$ as the quotient sheaf $(\mathcal{G}_2 \times \mathcal{P}) / \mathcal{G}_1$, where $\mathcal{G}_1$ acts on the product by:
\[ g_1 \cdot (g_2, p) = (g_2 \varphi(g_1)^{-1}, g_1 \cdot p) \]

The contracted product inherits a natural left $\mathcal{G}_2$-action given on local sections by $h \cdot [g_2, p] = [h g_2, p]$, which endows it with the structure of a $\mathcal{G}_2$-torsor. 
\end{definition}

This construction yields a functor:
\[ \varphi_*: \mathbf{Tors}(\cE, \mathcal{G}_1) \to \mathbf{Tors}(\cE, \mathcal{G}_2), \quad \mathcal{P} \mapsto \mathcal{G}_2 \times^{\mathcal{G}_1} \mathcal{P} \]
On the level of isomorphism classes, $\varphi_*$ induces a map of pointed sets $\mathrm{Tors}(\cE, \mathcal{G}_1) \to \mathrm{Tors}(\cE, \mathcal{G}_2)$, sending the class of the trivial $\mathcal{G}_1$-torsor to the class of the trivial $\mathcal{G}_2$-torsor.

When $\mathcal{G}$ is a \textbf{sheaf of abelian groups} (an \textit{abelian sheaf}), the pointed set $\mathrm{Tors}(\cE, \mathcal{G})$ 
inherits the structure of an abelian group. Let $\mathcal{P}_1$ and $\mathcal{P}_2$ be objects of $\mathbf{Tors}(\cE, \mathcal{G})$. We define their sum $[\mathcal{P}_1] + [\mathcal{P}_2]$ 
to be the class $[\cP_1 \otimes \cP_2]$, where $\cP_1 \otimes \cP_2$ is defined as the quotient sheaf $(\mathcal{P}_1 \times \mathcal{P}_2)/\mathcal{G}$. 
In this construction, $\mathcal{G}$ acts on the product $\mathcal{P}_1 \times \mathcal{P}_2$ on $T$-points by:
\[ g \cdot (f_1, f_2) := (g f_1, g^{-1} f_2) \]

The group object $\mathcal{G}$ then acts on the resulting quotient via its action on the presheaf quotient, which is given on classes by:
\[ g \cdot [(f_1, f_2)] = [(g f_1, f_2)] = [(f_1, g f_2)] \]
where the square brackets denote the class in the quotient set. This structure turns $\mathrm{Tors}(\mathcal{G})$ into an abelian group, 
where the identity is the class of the trivial torsor and the inverse is obtained by the opposite action.\footnote{
    Equivantly, the sum is obtained as the contracted product of the $\cG \times \cG$-torsor $\cP_1 \times \cP_2$ along the multiplication map
$m: \cG \times \cG \to \cG$. }

Let $\Lambda$ be a commutative ring object in the topos $\mathcal{E}$. 
We denote by $\mathbf{Pic}(\mathcal{E}, \Lambda)$ the category of locally free $\Lambda$-modules of rank $1$ in 
$\mathcal{E}$, also referred to as invertible modules. For the multiplicative group object $G = \Lambda^\times$, 
there exists a classical correspondence between such $\Lambda$-modules and $G$-torsors. 
Indeed, both categories are symmetric monoidal categories, $\mathbf{Pic}(\mathcal{E}, \Lambda)$ under the tensor product $\otimes_\Lambda$, with unit object $\Lambda$.
And $\mathbf{Tors}(\mathcal{E}, G)$ Under contracted product $\times^G$, with unit object $G$ acting on itself by translation.

Throughout this thesis, we shall primarily adopt the language of $G$-torsors, with the understanding that the 
associated $\Lambda$-module structure is implicitly inferred.

The following proposition formalizes this dictionary:

\begin{prop}
    \label{prop:torsor_module_equivalence}
    The associated module functor $\Phi: \mathcal{P} \mapsto \mathcal{P} \times^G \Lambda$ 
    induces a canonical equivalence of monoidal categories:
    $$\Phi: (\mathbf{Tors}(\mathcal{E}, G), \times^G, G) \xrightarrow{\sim} (\mathbf{Pic}(\mathcal{E}, \Lambda), 
    \otimes_\Lambda, \Lambda)$$
    The inverse functor
    $\Psi: \mathcal{L} \mapsto \underline{\mathrm{Isom}}_\Lambda(\Lambda, \mathcal{L})$.

\end{prop}

Lastly, for any object $\cX \in \mathcal{E}$, there is a canonical identification between the slice category 
$(\mathcal{G}\mathcal{E})/\cX$ and the category of group objects $\mathcal{G}(\mathcal{E}/\cX)$, where $\cX$ is viewed as having the 
trivial $\mathcal{G}$-action.

We denote by $\mathbf{Tors}(\cX, \cG)$ the category of $G$-torsors over $\cX$ in $\cG \cE_{/\cX}$.

%%% %%% %%% %%% %%% %%% %%% %%%
%       Torsors as Spaces
%%% %%% %%% %%% %%% %%% %%% %%%
\subsection{Torsors over the \'Etale's Sites}
Fix a scheme $X$, and let $G$ be a smooth affine $X$-group scheme. 
Denote by $X_{\text{Ét}}$ and $X_{\text{ét}}$ the big and small étale sites on $X$, respectively. 
The functor of points of $G$, which we denote by $G(\cdot) = \mathrm{Hom}_X(-, G)$, defines a group sheaf on 
$X_{\text{ét}}$.

\begin{definition} A \textit{principal $G$-bundle} (or \textit{principal homogeneous space}) for $G$ is a 
scheme $f : Y \to X$ equipped with a left $G$-action $\rho: G \times_X Y \to Y$ satisfying:
\begin{enumerate}
    \item The morphism $G \times_X Y \to Y \times_X Y$ sending $(g, y) \mapsto (gy, y)$ is an isomorphism of $X$-schemes.
    \item There exists an étale covering $\{U_i \to X\}$ such that $Y_{U_i} \cong G_{U_i}$ as $G$-schemes, 
    where $G$ acts on itself by left translation.
\end{enumerate}
\end{definition}

\begin{rem*}
    Since $G \to X$ is smooth and $Y$ is étale-locally isomorphic to $G$, 
    the morphism $f: Y \to X$ is smooth and affine. Note that $Y$ is generally not étale over $X$ unless $G$ is an 
    étale group scheme (e.g., a finite constant group).
    Similarly, if $G$ is finite then $Y \to X$ is finite.
\end{rem*}

Similarly to the case of $G$-torsors of a topos, we define a morphism of principal $G$-bundles to be a morphism of $X$-schemes commuting with the
$G$-action. We now have:

\begin{theorem}\label{theorem:torsors_are_realized_as_spaces}
    The morphism sending $Y \mapsto \mathrm{Hom}_X(-, Y)$ is an equivalence of categories from the category of principal $G$-bundles 
    to the category of $G$-torsors on $X_{\text{\'Et}}$
    Similarly, the morphism sending $Y$ to $\mathrm{Hom}_X(-, Y)$ to the category of $G$-torsors on $X_{\text{ét}}$ is an equivalence.
\end{theorem}

\begin{cor}
    There is a natural equivalence $\mathbf{Tors}(X_{\mathrm{\text{\'et}}}, G) \cong \mathbf{Tors}(X_{\text{\'Et}}, G)$ 
    inducing a bijection of pointed sets $\mathrm{Tors}(X_{\text{\'et}}, G) \xrightarrow{\cong} \mathrm{Tors}(X_{\text{\'Et}}, G)$ 
    which is an isomorphism of abelian groups if $G$ is abelian.
    And every $G$-torsor, in each of the above sites can be realized as a scheme, which is a prinicipal $G$-bundle.
\end{cor}

%%% %%% %%% %%% %%% %%% %%% %%%
%       Torsors and Cohomology
%%% %%% %%% %%% %%% %%% %%% %%%

% \subsubsection*{Torsors and Cohomology}
% We qoute without proof:
% \begin{theorem}
% There is a natural bijection of pointed sets     
% $\text{Tors}(\mathcal{G|_T}) \to \check{H}^1(T, \mathcal{G})$. Moreover, if $\mathcal{G}$ is an abelian group sheaf, it's an isomorphism of abelian groups.
% \end{theorem}

% The idea is that for every $[\mathcal{P}] \in \text{Tors}(T, \mathcal{G})$ 
% We 
% \begin{enumerate}
%     \item Choose a covering $\{U_i \to T\}$ such that $\mathcal{P}(U_i) \neq \emptyset$ for all $i$
%     \item choose sections $\alpha_i \in \mathcal{F}(U_i)$
% \end{enumerate}

% Then we note that for all $(i,j)$ the elements $\alpha_i |_{U_i \times_X U_j}$ and $\alpha_j |_{U_i \times_X U_j}$ differ by a unique element of $\mathcal{G}(U_{ij})$
% there exists a unique $s_{ij} \in \mathcal{G}(U_{ij})$ such that $\alpha_i |_{U_i \times_X U_j} = s_{ij}(\alpha_j |_{U_i \times_X U_j})$. 
% One can then easily see that $(s_{ij})$ defines an element of  $\check{H}^1(T, \mathcal{G})$
% which is independent of the choice of repreantiative $\cP$, choice of covering $\{U_i \to T\}$ and choice of sections.



%%% %%% %%% %%% %%% %%% %%% %%%
%       Constant Finite Group Torsors
%%% %%% %%% %%% %%% %%% %%% %%%

\subsubsection*{Constant Finite Group Torsors}

Let $G$ be a finite group, within this section, we denote the associated constant group scheme over $X$ by $\underline{G}$ to maintain a clear distinction between the group as a set and the group as a scheme. In subsequent sections, we shall follow the standard practice of identifying $G$ with its constant group scheme and omit the underline for brevity.

$\underline{G}$ is given by $\coprod_{g \in G} X$ with the action shuffeling the $X$'s according to multiplication.

By following the definitions, one sees that if $X$ be a connected scheme and $f : Y \to X$ is a finite Galois cover with Galois group $G$. 
Then, $f : Y \to X$ is a principal $\underline{G}$-bundle.
(Recall that a \textit{finite Galois cover} is a finite étale surjection $Y \to X$ with $Y$ connected and such that 
$G = \text{Aut}(Y/X)$ acts transitively on the geometric points of $Y$ lying over any geometric point of $X$).

On the otherhand if $f: Y \to X$ is a principal $\underline{G}$-bundle with $Y$ connected, then $Y$ is a finite Galois cover with
automorphism group $G$.

However, not all $\underline{G}$-torsors are connected. If $H\subset G$ is a proper subgroup then any connected finite \'etale cover $f: Y \to X$ with 
Galois group $H$ gives rise to a non-connected $\underline{G}$-torsor by looking at the induced $\underline{G}$-torsor $\varphi_*(Y)$ under the inclusion $\varphi: \underline{H} \to \underline{G}$.
On the otherhand, if we fix a geometric point $\overline{x} \to X$, then to give an homomorphism
$\rho \in \mathrm{Hom}_{\mathrm{cont}}(\pi_1^{\mathrm{et}}(X, x), G)$ is equivilant to give a connected pointed Galois cover
$(Y, \overline{y}) \to (X, \overline{x})$ with Galois group $H=\rho(\pi_1^{\mathrm{et}}(X, x)) \subset G$. 
Thus, pushing forward to $\underline{G}$ we get a prinicipal $\underline{G}$-bundle. The choice of a different geometric point $\overline{x'} \to X$
differ the homomorphism by an inner automorphism, thus we have:

\begin{theorem}\label{theorem:equivalence_fundamental_covers}
    Let $X$ be a connected scheme and $\overline{x}$ a geometric point of $X$. Suppose in addition that $G$ is a finite abstract group. 
    Define a map
    \begin{equation}
        \mathrm{Hom}_{\mathrm{cont}}(\pi_1^{\mathrm{et}}(X, \overline{x}), G) / \mathrm{Inn}(G) \to \mathrm{Tors}(X_{\text{\'Et}}, \underline{G}) 
    \end{equation}
by sending a homomorphism $\rho : \pi_1^{\mathrm{et}}(X, \overline{x}) \to G$ 
to the principal $\underline{G}$-bundle $\varphi_*(Y)$ where $Y$ is the principal $\underline{\rho(\pi_1^{\mathrm{et}}(X, \overline{x}))}$-bundle 
obtained above and $\varphi$ is the inclusion $\rho(\pi_1^{\mathrm{et}}(X, \overline{x})) \hookrightarrow G$. Then, the map 
 is a bijection of pointed sets where the trivial homomorphisms
  (which is the only element of its $\mathrm{Inn}(G)$-orbit) is the distinguished element of the left hand side.
\end{theorem}

If $G$ is abelian then $\mathrm{Inn}(G)$ is trivial, and we obtain:
\begin{cor}\label{cor:abelian_equivilance_fundamental_torsors}
    Let $G$ be a finite abelian group, $X$ a connected scheme, and $\overline{x}$ a geometric point of $X$. Then, the map from \Cref{theorem:equivalence_fundamental_covers}
     induces an isomorphism of abelian groups
    \begin{equation}
        \mathrm{Hom}_{\mathrm{cont.}}(\pi_1^{\text{ét}}(X, \overline{x}), G) \xrightarrow{\cong} \Tors(X_{\text{\'Et}}, \underline{G})
    \end{equation}
\end{cor}

\subsubsection*{Decomposition of Torsors}
In the case of finite groups, the functoriality of the contracted product has a concrete geometric interpretation as a tower of covers. Let $G$ be a finite abelian group and $H \subseteq G$ a subgroup. 
Let $\pi: G \to G/H$ be the natural quotient map.

\begin{prop}\label{prop:quotient_torsors_finite}\label{prop:quotient_torsors}
    Let $\mathcal{P} \to X$ be a ${G}$-torsor in $X_{\text{ét}}$. 
    There exists a natural factorization of the morphism $\mathcal{P} \to X$:
    \[ \mathcal{P} \xrightarrow{\phi} \mathcal{P}_{G/H} \xrightarrow{\psi} X \]
    where:
    \begin{enumerate}
        \item $\mathcal{P}_{G/H} := \mathcal{P} \times^{{G}} {(G/H)}$ is a 
        ${G/H}$-torsor over $X$.
        \item $\mathcal{P}$ is an ${H}$-torsor over the scheme $\mathcal{P}_{G/H}$.
    \end{enumerate}
\end{prop}

\begin{proof}
    The first assertion follows directly from the definition of the contracted product functor $\pi_*: \mathbf{Tors}(X, G) \to \mathbf{Tors}(X, G/H)$. To see the second assertion, we note that the map $\phi: \mathcal{P} \to \mathcal{P}_{G/H}$ is surjective. Locally on $X$, we may choose a cover $U \to X$ such that $\mathcal{P}|_U \cong G \times U$. 
    Over this cover, the map $\phi$ is identified with the product of the quotient map and the identity:
    \[ \pi \times \text{id}_U: G \times U \to (G/H) \times U \]
    Since $G \to G/H$ is a trivial $H$-torsor (the fiber over any element $\bar{g} \in G/H$ is a coset $gH$, which is an $H$-set isomorphic to $H$ acting on itself), $\mathcal{P}$ is locally an $H$-torsor over $\mathcal{P}_{G/H}$. By the descent of torsors, $\mathcal{P}$ is an $H$-torsor over $\mathcal{P}_{G/H}$ globally.
\end{proof}

%\textcolor{red}{Let us give a final note that, evidently, $\mathrm{Aut}(G)$ acts on $\mathrm{Hom}_{\mathrm{cont.}}(\pi_1^{\text{ét}}(X, x), G)$ on the right, and if we consider the quotient $\mathrm{Hom}_{\mathrm{cont.}}(\pi_1^{\text{ét}}(X, x), G)/\mathrm{Aut}(G)$ we get the pointed set of all connected finite Galois covers of $X$ with Galois group isomorphism to a subgroup of $G$.}

%%% %%% %%% %%% %%% %%% %%% %%% %%% %%% %%% %%% %%% %%% %%% %%%
%       Qoutient of G-torsors
%%% %%% %%% %%% %%% %%% %%% %%% %%% %%% %%% %%% %%% %%% %%% %%%

% \subsubsection*{Qoutients of \texorpdfstring{$G$}{G}-Torsors}
% Let $G$ be a finite abelian group, and let $H \subseteq G$ be a subgroup. 
%  Consider the natural quotient homomorphism $\pi: G \to G/H$. Given a $G$-torsor $\mathcal{P} \to X$ (in the fppf or étale topology), we may associate to it a $G/H$-torsor, 
%  denoted by $\pi_*(\mathcal{P})$ or $\mathcal{P}^{G/H}$, via the extension of scalars.
%  The object $\mathcal{P}^{G/H}$ is defined as the contracted product:
%  $$\mathcal{P}^{G/H} = \mathcal{P} \times^G (G/H)$$
%  Recall that $\mathcal{P}^{G/H}$ is the quotient of the product $\mathcal{P} \times (G/H)$ by the $G$-action defined by 
%  $g \cdot (p, \bar{g}') = (g \cdot p, \pi(g^{-1})\bar{g}')$. 
%  Consequently, we have a canonical morphism of sheaves 
%  $\phi: \mathcal{P} \to \mathcal{P}^{G/H}$ which, on local sections, acts by 
%  $p \mapsto [p, \bar{e}]$, where $\bar{e}$ is the identity element in $G/H$. 
%  To see that $\mathcal{P}$ carries the structure of an $H$-torsor over $\mathcal{P}^{G/H}$, 
%  consider a local trivializing cover $\{U_i \to X\}$ for $\mathcal{P}$ as a $G$-torsor. 
%  Over each $U_i$, we have an isomorphism $\mathcal{P}|_{U_i} \cong G_{U_i}$. 
%  Under this isomorphism, the contracted product locally satisfies:
%  $$(\mathcal{P} \times^G G/H)|_{U_i} \cong (G \times^G G/H)_{U_i} \cong (G/H)_{U_i}$$
%  Explicitly, the local identification $(g, \bar{g}') \sim (e, \pi(g)\bar{g}')$ 
%  shows that every equivalence class in the fiber has a unique representative of the form 
%  $[e, \bar{g}]$. 
%  The map $\phi$ locally corresponds to the quotient map $G \to G/H$. 
%  Since the kernel of this map is $H$, and $G$ is a trivial $H$-torsor over $G/H$, 
%  it follows by descent that $\mathcal{P}$ is an $H$-torsor over $\mathcal{P}^{G/H}$.
%  We summarize this construction in the following proposition:
%  \begin{prop}\label{prop:quotient_torsors}
%     Let $G$ be an abelian group and $H \subset G$ a subgroup. 
%     Any $G$-torsor $\mathcal{P} \to X$ admits a natural factorization:
%     $$\mathcal{P} \xrightarrow{} \mathcal{P}^{G/H} \xrightarrow{} X$$
%     where $\mathcal{P}^{G/H} \to X$ is a $G/H$-torsor and $\mathcal{P} \to \mathcal{P}^{G/H}$ is an $H$-torsor.
% \end{prop}

\subsection{Symmetric Powers of Schemes and Torsors}
This section reviews the construction of quotients for schemes and torsors under finite group actions, specifically focusing on symmetric powers. 
To ensure these quotients exist as schemes, we utilize the framework of admissible actions from \cite{sga1}. 
Our treatment here closely follows the exposition in \cite{Guignard2018}
The definitions and results presented below are adapted from their work. 
This foundation provides the necessary criteria for admissibility and base change required to define the symmetric powers of a scheme $X$
 and a $G$-torsor $\mathcal{P}$ over $X$.


Let $S$ be a scheme.

\begin{definition}[(\cite{sga1}, V.1.7).]
    \noindent
    \begin{itemize}
        \item Let $T$ be an object of a category $\mathcal{C}$ endowed with a right action of a group $\Gamma$. We say that \textbf{the quotient $T/\Gamma$ exists} in $\mathcal{C}$ if the covariant functor
    \[
    \begin{aligned}
    \mathcal{C} &\to \text{Sets} \\
    U &\mapsto \text{Hom}_{\mathcal{C}}(T, U)^{\Gamma}
    \end{aligned}
    \]
    is representable by an object of $\mathcal{C}$.
    \item Let $T$ be an $S$-scheme. An action of a finite group $\Gamma$ on $T$ is \textbf{admissible} if there exists an affine $\Gamma$-invariant morphism $f : T \to T'$ such that the canonical morphism $\mathcal{O}_{T'} \to f_* \mathcal{O}_T$ induces an isomorphism from $\mathcal{O}_{T'}$ to $(f_* \mathcal{O}_T)^{\Gamma}$.
\end{itemize}
\end{definition}

\begin{prop}
    The following holds:
    \begin{enumerate}
        \item \textbf{(\cite{sga1} V.1.3)}. Let $T$ be an $S$-scheme endowed with an admissible right action of a finite group $\Gamma$. If $f : T \to T'$ is an affine $\Gamma$-invariant morphism such that the canonical morphism $\mathcal{O}_{T'} \to f_* \mathcal{O}_T$ induces an isomorphism from $\mathcal{O}_{T'}$ to $(f_* \mathcal{O}_T)^{\Gamma}$, then the quotient $T/\Gamma$ exists and is isomorphic to $T'$.
        \item \textbf{(\cite{sga1}, V.1.8)}. Let $T$ be an $S$-scheme endowed with a right action of a finite group $\Gamma$. Then, the action of $\Gamma$ on $T$ is admissible if and only if $T$ is covered by $\Gamma$-invariant affine open subsets.
        \item \textbf{(\cite{sga1}, V.1.9)}. Let $T$ be an $S$-scheme endowed with an admissible right action of a finite group $\Gamma$, and let $S'$ be a flat $S$-scheme. Then, the action of $\Gamma$ on the $S'$-scheme $T \times_S S'$ is admissible, and the canonical morphism
            \[
            (T \times_S S')/\Gamma \to (T/\Gamma) \times_S S'
            \]
            is an isomorphism.
    \end{enumerate}
\end{prop}


Now let $X$ be an $S$-scheme and let $d \geq 0$ be an integer. The group ${S}_d$ of permutations of $\llbracket 1, d \rrbracket$ acts on the right on the $S$-scheme $X^{\times_S d} = X \times_S \dots \times_S X$ by the formula
\[
(x_i)_{i \in \llbracket 1, d \rrbracket} \cdot \sigma = (x_{\sigma(i)})_{i \in \llbracket 1, d \rrbracket}.
\]

\begin{prop}[\cite{Guignard2018} Proposition 2.27]
    If $X$ is a scheme, Zariski locally quasi-projective over $S$, then the right action of the symmetric group $S_d$ on the $d$-fold fiber product 
    $X^{\times_S d}$ is admissible. 
    Consequently, the quotient $\mathrm{Sym}_S^d(X) = X^{\times_S d}/S_d$ exists as a scheme over $S$.
\end{prop}

\begin{definition}
    Under the hypotheses of the proposition above, we define the \textbf{relative symmetric product} of $X$ over $S$ 
    of degree $d$ as the quotient
    $$\mathrm{Sym}_S^d(X) \coloneqq X^{\times_S d}/S_d.$$
    When the base scheme $S$ is clear from the context, we shall denote this quotient by 
    $X^{(d)}$.
\end{definition}

Guingard shows that when $X=\Spec(B)$ and $S=\Spec(A)$ then $\Sym_{S}^d(X)$ is representable by an affine $S$-scheme (See \cite{Guignard2018} Remark 2.28).

\begin{prop}[\cite{Guignard2018} Proposition 2.28]\label{prop:symmetric_power_flat_base_change}
If $X$ is flat and Zariski-locally quasi-projective over $S$, then $\text{Sym}_S^d(X)$ is flat over $S$. Moreover, for any $S$-scheme $S'$, the canonical morphism
\[
\text{Sym}_{S'}^d(X \times_S S') \to \text{Sym}_S^d(X) \times_S S'
\]
is an isomorphism.
\end{prop}


%\subsubsection*{Symmetric Powers of Torsors}
Now, let $G$ be a finite abelian group, let $\cP$ be a $G$-torsor over an $S$-scheme $X$ in $S_{\text{Ét}}$. 

\begin{prop}[{[\cite{sga1}]}, IX.5.8]
Assume that $\cP$ and $X$ are endowed with right actions from a finite group $\Gamma$ 
such that the morphism $\cP \to X$ is $\Gamma$-equivariant, and that the following properties hold:
\begin{enumerate}
    \item[(a)] The right $\Gamma$-action on $\cP$ commutes with the left $G$-action.
    \item[(b)] The right $\Gamma$-action on $X$ is admissible, and the quotient morphism $X \to X/\Gamma$ is finite.
    \item[(c)] For any geometric point $\bar{x}$ of $X$, the action of the stabilizer $\Gamma_{\bar{x}}$ of $\bar{x}$ in $\Gamma$ on the fiber $\cP_{\bar{x}}$ of $\cP$ at $\bar{x}$ is trivial.
\end{enumerate}
Then the action of $\Gamma$ on $\cP$ is admissible, and $\cP/\Gamma$ is a $G$-torsor over $X/\Gamma$ in $S_{\text{Ét}}$.
\end{prop}

By \Cref{theorem:torsors_are_realized_as_spaces}, $\cP$ is representable by a finite étale $X$-scheme.
For each $i \in  \llbracket 1, d \rrbracket$ let $p_i : X^{\times_S d} \to X$ be the projection on $i$-th factor, and let us consider the $G$-torsor
\[
p_1^{-1}\cP \otimes \cdots \otimes p_d^{-1}\cP = G_d \backslash \cP^{\times_S d}
\]
over $X^{\times_S d}$, where $G_d \subseteq G^d$ is the kernel of the multiplication morphism $G^d \to G$. 
The object $G_d \backslash \cP^{\times_S d}$ of $S_{\text{Ét}}$ is too representable by an $S$-scheme which is finite étale over $X^{\times_S d}$. 
The group ${S}_d$ acts on the right on $G_d \backslash \cP^{\times_S d}$ by the formula
\[
(p_i)_{i \in \llbracket 1, d \rrbracket} \cdot \sigma = (p_{\sigma(i)})_{i \in \llbracket 1, d \rrbracket}.
\]
This action of ${S}_d$ commutes with the left action of $G$ on $G_d \backslash \cP^{\times_S d}$.

\medskip

\begin{prop}[\cite{Guignard2018} Proposition 2.32.]\label{prop:symmetric_power_torsor}
If $X$ is Zariski-locally quasi-projective on $S$, then the right action of ${S}_d$ on $G_d \backslash \cP^{\times_S d}$ is admissible, 
so that the quotient $\cP^{(d)}$ of $G_d \backslash \cP^{\times_S d}$ by ${S}_d$ exists as a scheme over $S$. 
Moreover, the canonical morphism $\cP^{(d)} \to \mathrm{Sym}_S^d(X)$ is a $G$-torsor, 
and the morphism
\[
p_1^{-1}\cP \otimes \cdots \otimes p_d^{-1}\cP \to r^{-1}\cP^{(d)}
\]
where $r : X^{\times_S d} \to \mathrm{Sym}_S^d(X)$ is the canonical projection, is an isomorphism of $G$-torsors over $X^{\times_S d}$.
\end{prop}


%
% SYMMETRIC POWERS OF LOCAL SYSTEMS ON CURVES
%
The construction of symmetric powers is compatible with a natural addition law. 
For any integers $d_1, d_2 \geq 0$, there exists a canonical \textbf{addition morphism}
\[ +_{d_1, d_2}: X^{(d_1)} \times_S X^{(d_2)} \to X^{(d_1+d_2)} \]
induced by the equivariant isomorphism $X^{\times_S d_1} \times_S X^{\times_S d_2} \cong X^{\times_S (d_1+d_2)}$ with respect to the inclusion of the product of symmetric groups $S_{d_1} \times S_{d_2} \subseteq S_{d_1+d_2}$.

\begin{prop}\label{prop:symmetric_powers_on_curves}
    Let $\cP$ be a $G$-torsor over $X$. There is a canonical isomorphism of $G$-torsors over $X^{(d_1)} \times_S X^{(d_2)}$:
    \[
    +_{d_1, d_2}^{-1} \left( \cP^{(d_1+d_2)} \right) \cong r_1^{-1} \cP^{(d_1)} \otimes r_2^{-1} \cP^{(d_2)}.
    \]
\end{prop}
\begin{proof}
Let $\pi = r_{d_1} \times r_{d_2}$. By the commutativity of the following diagram
\[
\begin{tikzcd}
X^{\times_S d_1} \times_S X^{\times_S d_2} \arrow[r, "\sim"] \arrow[d, "\pi"'] & X^{\times_S (d_1+d_2)} \arrow[d, "r_{d_1+d_2}"] \\
X^{(d_1)} \times_S X^{(d_2)} \arrow[r, "+_{d_1, d_2}"] & X^{(d_1+d_2)}
\end{tikzcd}
\]
and applying \Cref{prop:symmetric_power_torsor}, we obtain canonical isomorphisms:
\[
\pi^{-1} \left( +_{d_1, d_2}^{-1} \cP^{(d_1+d_2)} \right) \cong r_{d_1+d_2}^{-1} \cP^{(d_1+d_2)} \cong \bigotimes_{i=1}^{d_1+d_2} p_i^{-1} \cP
\]
and
\[
\pi^{-1} (r_1^{-1} \cP^{(d_1)} \otimes r_2^{-1} \cP^{(d_2)}) \cong \left( \bigotimes_{i=1}^{d_1} p_i^{-1} \cP \right) \otimes \left( \bigotimes_{j=d_1+1}^{d_1+d_2} p_j^{-1} \cP \right).
\]
Since both pullbacks identify with $\bigotimes_{k=1}^{d_1+d_2} p_k^{-1} \cP$ in an $S_{d_1} \times S_{d_2}$-equivariant manner, the isomorphism descends to the quotient $X^{(d_1)} \times_S X^{(d_2)}$ by \Cref{prop:symmetric_power_flat_base_change}.
\end{proof}



% \begin{proof}
%     This proof follow
%     The claim follows from the fact that the construction of $\cP^{(d)}$ 
%     is compatible with base change (Proposition \ref{prop:symmetric_power_flat_base_change}) 
%     and the identification of the group action on the fiber (\Cref{prop:symmetric_power_torsor}). 
%     Specifically, the quotient of $G_{d_1+d_2} \backslash \cP^{d_1+d_2}$ by the subgroup $S_{d_1} \times S_{d_2}$ is isomorphic to the product of the individual symmetric torsors. The isomorphism is then induced by the universal property of the quotient $X^{(d_1+d_2)}$.
% \end{proof}

%
% SYMMETRIC POWERS OF LOCAL SYSTEMS ON CURVES
%
% \subsection{Symmetric Powers of Local Systems on Curves}\label{section:symmetric_powers_on_curves}
% Lastly, when $X=C$ is a 
% Let $k$ be a perfect field. 
% Let $C$ be a projective smooth geometrically connected curve over $k$, with genus $g$.
% Let $\mdls$ be a modulus on $C$ and let $U = C \setminus \mdls$.
% Let $G$ be a finite abelian group and let $\cP$ be a $G$-torsor on $U$ with ramification bounded by
% $\mdls$. Let $d \geq \deg m$.

% We have the following diagram:
% \[
% \begin{tikzcd}
% U^{(d_1)} \times_k U^{(d_2)} \arrow[r, "p_1"] \arrow[d, "p_2"] & U^{(d_1)} \\
% U^{(d_2)} &  {}
% \end{tikzcd}
% \] 

% pullbacking $\cP^{(d_i)}$ along the projections we get a $G$-torsor

% \[\cP ^{(d_1)} \boxtimes \cP ^{(d_2)} = p_1^{-1} \cP^{(d_1)} \otimes p_2^{-1} \cP^{(d_2)}\]
% On $U^{(d_1)} \times_k U^{(d_2)}$

% Note that the plus map $C^{(d_1)} \times_k C^{(d_2)} \xrightarrow{+}{} C^{(d_1 + d_2)}$
% is induced from
% \[
% \begin{tikzcd}
% C^{d_1} \times_k C^{d_2} \arrow[r, "\cong"] \arrow[d, "r_1 \times r_2"] & C^{d_1 + d_2} \arrow[d, "r"] \\
% C^{(d_1)} \times_k C^{(d_2)} \arrow[r, "+"] & C^{(d_1+ d_2)} 
% \end{tikzcd}
% \]


% Hence, by \Cref{prop:symmetric_power_torsor} (and replacing $C$ with $U$ above) we get canonical identification:
% \[
%  (+^{-1})(\cP^{(d_1+d_2)}) \cong \cP ^{(d_1)} \boxtimes \cP ^{(d_2)}
% \]

\subsection{Algebraic Preliminaries on Ramification}
\textcolor{red}{change?}
We recall the basic definitions and properties of the ramification of discrete valuations. 
We start with the general case of discrete valuation rings and their integral closures within finite separable field extensions. 
Then, we move to the specific setting of complete discrete valuation rings within Galois extensions, where we describe the ramification filtration of the Galois group via both lower and upper numbering.
We follow \stackstag{0EXQ}, and \cite{serreLF}.

\subsubsection*{Ramification of Discrete Valuation Rings}
Let $A$ be a discrete valuation ring with fraction field $K$. Let $L/K$ be a finite separable field extension. Let $B \subset L$ be the integral closure of $A$ in $L$. Picture:

\[ \xymatrix{ B \ar[r] & L \\ A \ar[u] \ar[r] & K \ar[u] } \]
By \stackstag{032L} the ring extension $A \subset B$ is finite, hence $B$ is Noetherian. 
By \stackstag{00OK} the dimension of $B$ is $1$, hence $B$ is a Dedekind domain, see \stackstag{034X}. 
Let $\mathfrak m_1, \ldots , \mathfrak m_ n$ be the maximal ideals of $B$ (i.e., the primes lying over $\mathfrak m_ A$). We obtain extensions of discrete valuation rings

\[ A \subset B_{\mathfrak m_ i} \]
and hence ramification indices $e_ i$ and residue degrees $f_ i$. We have

\[ [L : K] = \sum \nolimits _{i = 1, \ldots , n} e_ i f_ i \]
by \stackstag{02MJ} applied to a uniformizer in $A$. We observe that $n = 1$ if $A$ is henselian (by \stackstag{04GH} and the fact that $B$ is a domain), e.g. if $A$ is complete.

\begin{definition}\label{definition:tamely_ramified_dvrs}
Let $A$ be a discrete valuation ring with fraction field $K$. Let $L/K$ be a finite separable extension. With $B$ and $\mathfrak m_ i$, $i = 1, \ldots , n$ 
as above, we say the extension $L/K$ is
\begin{enumerate}
    \item unramified with respect to $A$ if $e_ i = 1$ and the extension $\kappa (\mathfrak m_ i)/\kappa _ A$ is separable for all $i$,
    \item tamely ramified with respect to $A$ if either the characteristic of $\kappa _ A$ is $0$ or the characteristic of $\kappa _ A$ is $p > 0$, the field extensions $\kappa (\mathfrak m_ i)/\kappa _ A$ are separable, and the ramification indices $e_ i$ are prime to $p$, and
    \item totally ramified with respect to $A$ if $n = 1$ and the residue field extension $\kappa (\mathfrak m_1)/\kappa _ A$ is trivial.
\end{enumerate}
If the discrete valuation ring $A$ is clear from context, then we sometimes say $L/K$ is unramified, totally ramified, or tamely ramified for short.
\end{definition}

\subsubsection*{Structure Theorems and Some Lemmas}\label{subsubsection:structure_theorems_ramification}
Let $A$ be a complete discrete valuation ring over with uniformizer $\pi$ and residue field $\kappa$, which we assume to be perfect.
When $A$ and $\kappa$ are of the same characteristic $p > 0$, then $A$ contains a coefficient field $k \cong \kappa$ and a well known structure theorem holds: $A = k[[\pi ]] \cong k[[t]]$.
Let $K$ be the fraction field of $A$, then $K=k((\pi))$.
By \textcolor{red}{Kummer theory}, unramified extensions of $K$ correspond to separable extensions of $k$.
The maximal unramified extension of $K$ is $\overline{k}((\pi))$ where $\overline{k}$ is a separable closure of $k$.
 \textcolor{red}{Maybe add something about the above facts.}


%We will need that broad definition of tame ramification in order to deal with higher dimensional schemes, 
%however note that in the case of global and local fields (taking $A$ to be the local ring at a prime of the ring of integers), 
%this definition of tame ramification coincides with the usual one.

\subsubsection*{Classical Ramification Filtration in the Galois Case}
We now recall the classical ramification filtration in the Galois case.
Assume $A,B$ are complete DVRs. And that $L/K$ is Galois with Galois group $G$. 
In that case there is uniformizer $\pi \in B$ such that $B=A[\pi ]$ 
   
We have the ramification filtration of $G$ by lower numbering $(G_i)_{i \geq -1}$, defined by
\[ G_i = \{ \sigma \in G \mid v_B(\sigma (x) - x) \geq i + 1 \text{ for all } x \in B \} \]
where $v_B$ is the valuation on $L$ associated to $B$. 
In particular, $G_{-1} = G$ and $G_0$ is the inertia group of the extension $L/K$. 
We have that $L/K$ is unramified if and only if $G_0$ is trivial, and $L/K$ is tamely ramified if and only if $G_1$ is trivial.
It is easy exercise that in the definition of $G_i$ it is enough to check the condition for the uniformizer $\pi$ of $B$, if we define
$i^L_K(\sigma ) = v_B(\sigma (\pi ) - \pi )$ for $\sigma \in G$, then we have $G_i = \{ \sigma \in G \mid i^L_K(\sigma ) \geq i + 1 \}$.
The groups $G_i$ are normal in $G$ and are trivial for large enough $i$. In a tower of fields $K \subset E \subset L$, where $H=\Gal(L/E)$ we have
\[ G_i \cap H = H_i \] for all $i \geq -1$, which corresponds to the fact that $i^L_E = i^L_K|_{\mathrm{Gal}(L/E)}$.
Ramification groups also behave well with respect to quotients: $G_i H/H=(G/H)_j$.
where \[ j=\frac{1}{e_{L/E}}\sum_{\tau \in H} \min(i^L_K(\tau), i+1)-1 \]
i.e. the quotient of a ramification group is itself a ramification group, but with a different index.
In the literature, one reindexes the ramification groups by defining the Herbrand function $\phi_{L/K} : [-1, \infty ) \to [-1, \infty )$:
\[ \phi_{L/K}(i) = \frac{1}{e_{L/K}}\sum_{\sigma \in G} \min(i^L_K(\sigma), i+1)-1 = \int _0^i \frac{1}{[G_0 : G_t]} dt \]
It is continuous, increasing, piecewise linear function, hence a bijection.
It satisfies $\phi_{L/K}=\phi_{E/K} \circ \phi_{L/E}$ for $K \subset E \subset L$, and $G_iH/H=(G/H)_{\phi_{L/E}(i)}$.
Thus, defining the ramification groups by upper numbering as $G^i = G_{\phi_{L/K}^{-1}(i)}$, we have:
\[ G^i H/H = (G/H)^i \]
for all $i \geq -1$.


\subsection{Kummer and Artin-Schreier Theories}
We recall the basic theorms from both theories regarding cyclic extesntions and ramifications. 
Throughout this section let $K$ be a discrete valuation field with perfect residue field $\kappa$ of characteristic $p > 0$. 


\begin{theorem}[Ramification in Kummer Extensions, \cite{koch1997algebraic}, Proposition 1.83]\label{theorem:kummer_ramification}
%Let $K$ be a local field of characteristic $\text{char}(K) =p$  
Assume $K$ contains the $n$-th roots of unity $\mu_n$. 
Let $L/K$ be the extension given by the equation $X^n = a$ for some 
$a \in K^\times$ and denote by $G$ its Galois group. Then we have:

\begin{enumerate}
    \item If $v_K(a) \in n\mathbb{Z}$ and the image of $a\pi^{-v_K(a)}$ in the residue field $\kappa$ is an $n$-th power, the extension $L/K$ is trivial.
    \item If $v_K(a) \in n\mathbb{Z}$ and the image of $a\pi^{-v_K(a)}$ in the residue field $\kappa$ is not an $n$-th power, the extension $L/K$ is cyclic and unramified.
    \item If $v_K(a) \notin n\mathbb{Z}$, the extension $L/K$ is cyclic and ramified. Specifically, if $\gcd(|v_K(a)|, n) = 1$, the extension is totally ramified of degree $n$.
     Otherwise it has ramification index $\frac{n}{|gcd(v_K(a)|, n)}$
\end{enumerate}
Conversly, Kummer theory ensures that every cyclic extension of degree $n$, prime to $p$ of a field that contains $n$-th roots of unity, is of the above form.
Moreover, in the above we can always take $a \in \Oo_K$
\end{theorem}
Note that in the case of total ramification the extesntion is tamely ramified.


\begin{theorem}[Ramification in Artin-Schreier Extensions, \cite{Thomas2005}]\label{theorem:artin_schreier_ramification}
Let $\wp(x) = x^p - x$ be the Artin-Schreier opeartor.
Let $L/K$ be the extension given by the equation $X^p - X = a$ for some $a \in K$ and denote by $G$ its Galois group. 
Then we have:
\begin{enumerate}
    \item If $v_K(a) > 0$ or if $v_K(a) = 0$ and $a \in \wp(K)$, the extension $L/K$ is trivial.
    \item If $v_K(a) = 0$ and if $a \notin \wp(K)$, the extension $L/K$ is cyclic of degree $p$ and unramified.
    \item If $v_K(a) = -m < 0$ with $m \in \mathbb{Z}_{>0}$ and if $m$ is prime to $p$, the extension $L/K$ is cyclic of degree $p$ again and totally ramified. Moreover,
     its ramification groups are given by:
     $$G = G^{(-1)} = \dots = G^{(m)} \quad \text{and} \quad G^{(m+1)} = 1.$$
\end{enumerate}
Conversely, Artin-Schreier theory ensures that every cyclic extension of degree $p$ takes this form.
Moreover, in the above and under the isomorphism $K \cong k((t))$, one can always take $a$ of the form 
$c t^{-m} + a_{-m + 1} t^{-m + 1} + \cdots a_{-1}t^{-1} + a_0$. If $k$ is algebraically closed then there is a 
change of variables such that $a=u^{-m}$.
\end{theorem}



%%%% %%%% %%%% %%%% %%%% %%%% %%%% %%%%
%          ALGEBRAIC GEOMETRY
%%%% %%%% %%%% %%%% %%%% %%%% %%%% %%%%
\subsection{Algebraic Geometry}
In this section we group together some general theorems in algebraic geometry that we will be employing throughtout the text. 
All schemes are assumed to be locally of finite type. 
\begin{theorem}
    Let $f:X \to Y$ be a finite flat map between integral schemes, of finite type over a field $k$. if $Z \subset X$ is a prime divisor with generic point $\eta_Z$, then $f(Z) \subset X$
    is a prime divisor with generic point $\eta_{f(Z)}$ satisfying $f(\eta_Z)= \eta_{f(Z)}$
\end{theorem}
\begin{proof}
    $f$ is finite hence proper hence closed so $f(Z)$ is closed subset of $Y$, it is irreducible as the image of an irreducible.
    Since $Z = \overline{\{ \eta_Z \}}$ we get: 
    $$\{f(\eta_Z)\} \subset f(Z) = f(\overline{\{\eta_Z\}}) \subseteq \overline{f(\{\eta_Z\})} = \overline{\{f(\eta_Z)\}}$$
    And since $f(Z)$ is closed we get $f(Z) = \overline{\{f(\eta_Z)\}}$.

    For flat map of integral schemes we have for every $x \in X$, $y=f(x)$ the dimension formula:
    $$\text{dim}(\mathcal{O}_{X,x}) = \text{dim}(\mathcal{O}_{Y,y}) + \text{dim}(\mathcal{O}_{X_y, x})$$
    And since $\text{dim}(\mathcal{O}_{X_y, x}) = 0$ we get $\text{dim}(\mathcal{O}_{X,x}) = \text{dim}(\mathcal{O}_{Y,y})$
    concluding that $f(Z)$ is a prime divisor as well. 
\end{proof}

A known theorem states that: 
\begin{theorem}\label{theorem:int_geoint_implies_productint}
    Let $X, Y$ be two integral schemes over a field $k$. If $X$ is geometrically integral then $X \times_k Y$ 
    is integral.
    If both $X,Y$ are geometrically integral, then $X \times_k Y$ is geometrically integral. 
\end{theorem}

\begin{theorem}
    Let $C$ be smooth projective curve geometrically connected over a field $k$. Then:
    \begin{enumerate}
        \item $C^{(d)}$ is smooth
        \item For every $d$, $C^{(d)}$ is integral.
        \item For every $d$, $C^{(d)}$ is geometrically integral.
        \item The product of every finite number of $C^{(d)}$ is geometrically integral. 
    \end{enumerate}
    \begin{proof}
        \begin{enumerate}
            \item Let $t_i$ be a local parameter for $C$ at $P_i$. The local ring of the product $C^d$ at the point $(P_1, \dots, P_d)$ is isomorphic $k[[t_1, t_2, \dots, t_d]]$
            and the local ring of the qoutient at the divisor $D=\sum P_i$ is $k[[t_1, \dots, t_d]]^{S_d}$ which is 
            isomorphic to $k[[t_1, \dots, t_d]]^{S_d} \cong k[[s_1, \dots, s_d]]$ where the $s_i$ are the 
            symmetric polynomials, hence this ring is regular local ring. 
            \item $C$ is irreducible hence $C^{d}$ is irreducible hence $C^{(d)}$ is irreducible. Since $C^{(d)}$ 
            is smooth it is reduced.
            \item By \stackstag{0366}, $C$ is geometrically integral, so it follows from the above. 
            \item \Cref{theorem:int_geoint_implies_productint}
        \end{enumerate}
    \end{proof}
\end{theorem}


\subsubsection*{Blowups}

\begin{theorem}[\stackstag{0805}]
Let $X_1 \to X_2$ be a flat morphism of schemes. Let $Z_2 \subset X_2$ be a closed subscheme. Let $Z_1$ be the inverse image of $Z_2$ in $X_1$. Let $X'_ i$ be the blowup of $Z_ i$ in $X_ i$. Then there exists a cartesian diagram

\[ \xymatrix{ X_1' \ar[r] \ar[d] & X_2' \ar[d] \\ X_1 \ar[r] & X_2 } \]
of schemes.   
\end{theorem}

\begin{theorem}
If $X$ is integral then $Bl_Z(X)$ is integral. %This can be checked locally
\end{theorem}


If $X$ is a smooth curve, we have the following result regarding its symmetric power:
    \begin{prop}
        Let $f: X \to S$ be a smooth morphism of relative dimension 1.
        Suppose $f$ is flat and Zariski-locally quasi-projective.
        Then the relative symmetric power $X_S^{(d)} = \text{Sym}^d_S(X)$ is smooth over $S$ of relative dimension $d$.
    \end{prop}
    
    \begin{proof}Since $f: X \to S$ is smooth, the $d$-fold product $X_S^d$ is smooth over $S$. By \Cref{prop:symmetric_power_flat_base_change}, the quotient $X_S^{(d)}$ is flat over $S$. Smoothness is a fiberwise property for flat morphisms of finite presentation (cf. \stackstag{01V8}); thus, it suffices to verify the smoothness of the fibers. 
        For every $s \in S$, the fiber of the symmetric power is given by:$$(\text{Sym}^d_S(X))_s \cong \text{Sym}^d_{\kappa(s)}(X_s)$$Consequently, we may reduce to the case where $S = \text{Spec}(k)$ for a field $k$. 
        Since smoothness is preserved under  base change to the algebraic closure, we may further assume $k = \bar{k}$.
        Let $z = \sum_{i=1}^{r} d_i \cdot x_i \in X^{(d)}$ be a point represented by an effective cycle of degree $d$, where the points $x_i \in X(k)$ are distinct and $\sum d_i = d$. Choose pairwise disjoint Zariski-open subsets $U_i \subset X$ such that $x_i \in U_i$. Let $W \subset X^{(d)}$ be the open subset consisting of cycles whose support is contained in $\bigcup U_i$ and which meet each $U_i$ with degree exactly $d_i$. There is a canonical isomorphism:
$$W \cong \prod_{i=1}^r U_i^{(d_i)}$$
Thus, it suffices to show that each $U_i^{(d_i)}$ is smooth of dimension $d_i$ at the point $d_i \cdot x_i$. 

We may therefore restrict our attention to the "worst" case: the diagonal point $z = d \cdot x$. Let $R = \mathcal{O}_{X,x}$ be the local ring of the curve at $x$. Since $X$ is a smooth curve over $k$, $R$ is a regular local ring of dimension 1, i.e., a discrete valuation ring. Its completion is $\hat{R} \cong k[[t]]$. The completed local ring of the product $X^d$ at the point $(x, \dots, x)$ is:
$$\hat{\mathcal{O}}_{X^d, (x, \dots, x)} \cong \hat{R} \hat{\otimes}_k \dots \hat{\otimes}_k \hat{R} \cong k[[t_1, \dots, t_d]]$$
where $t_i$ is the uniformizer for the $i$-th copy of $X$. The symmetric group $S_d$ acts on this power series ring by permuting the variables 
$t_i$. By the \textit{Fundamental Theorem of Symmetric Polynomials}, the ring of invariants is:
$$\left( k[[t_1, \dots, t_d]] \right)^{S^d} = k[[e_1, \dots, e_d]]$$
where $e_j$ is the $j$-th elementary symmetric power series in the variables $t_i$.

The ring $k[[e_1, \dots, e_d]]$ is a formal power series ring in $d$ variables. A fundamental theorem in commutative algebra states that a local ring is regular if and only if its completion is a formal power series ring over a field. Since $\hat{\mathcal{O}}_{X^{(d)}, z} \cong k[[e_1, \dots, e_d]]$, the local ring $\mathcal{O}_{X^{(d)}, z}$ is a regular local ring of dimension $d$. This implies that $X^{(d)}$ is smooth over $k$ of dimension $d$, completing the proof.
\end{proof}
