\section{Preliminaries}
In this section we recall the necessariy work, including work from \cite{Guignard2018}, \cite{tendler2015geometricclassfieldtheory} and \cite{takeuchi2019blow}.
Unless otherwise stated, we work within an arbitrary topos $\mathcal{E}$.

The following proposition establishes the fundamental dictionary between the geometric theory
of principal homogeneous spaces and the algebraic theory of invertible modules.
This equivalence allows us to transport the monoidal structure from the category of modules
(the tensor product) to the category of torsors (the contracted product),
strictly within the categorical framework.

\begin{prop}\label{prop:torsor_module_equivalence}
    Let $\mathcal{E}$ be a topos and let $\Lambda$ be a ring object in $\mathcal{E}$. Let $G = \Lambda^\times$ denote the internal group object of units of $\Lambda$.

    There is a canonical equivalence of monoidal categories between the category of $G$-torsors in $\mathcal{E}$ and the category of locally free $\Lambda$-modules of rank 1 in $\mathcal{E}$:
    \[
        \Phi: \mathbf{Tors}(\mathcal{E}, \Lambda^\times) \xrightarrow{\sim} \mathbf{Pic}(\mathcal{E}, \Lambda)
    \]
    The equivalence is defined by the associated module functor:
    \[
        P \longmapsto P \times^{\Lambda^\times} \Lambda := \Lambda^\times \backslash (\Lambda \times P)
    \]
    where the quotient is taken with respect to the diagonal action of $\Lambda^\times$ on $\Lambda \times P$. The inverse functor associates to an invertible module $L$ its sheaf of basis frames $\underline{\mathrm{Isom}}_\Lambda(\Lambda, L)$.
\end{prop}

In light of this canonical equivalence, we will pass freely between the language of $G$-torsors and that of locally free $\Lambda$-modules throughout the text.

For a topos $\cE$, a group object $G$ in $\cE$ and an object $X$ in $\cE$, there is a canonical identification
between $({G\cE})_{/X}$ and $G(\cE_{/X})$, given by endowing $X$ with the trivial $G$-action.

We denote by $\mathbf{Tors}(X, G)$ the category of $G$-torsors over $X$ in $G \cE_{/X}$.
Similarly, for a ring object $\Lambda$ in $\cE$, we denote by $\mathbf{Pic}(X, \Lambda)$ the category of locally free $\Lambda$-modules of rank 1 over $X$ in $\cE_{/X}$.
The above equivilance of categories becomes 
    \[
        \Phi_X: \mathbf{Tors}(X, \Lambda^\times) \xrightarrow{\sim} \mathbf{Pic}(X, \Lambda)
    \]


    \textcolor{purple}{
\cite{Guignard2018} introduces the notion of multiplicative locally free $\Lambda$-modules of rank 1 over a commutative 
semigroup object in $\cE$.
\begin{definition}[2.5 in \cite{Guignard2018}]
    Let $G$ be an abelian group of $\cE$ and let $Q$ be a commutative semigroup of $\cE$ (with or without identity). 
    Let $m : Q \times Q \to Q$ be the multiplication morphism of $Q$. 
    A \textbf{multiplicative $G$-torsor} over $Q$ is a $G$-torsor $P \to Q$, together with an isomorphism 
    $\theta : p_1^{-1} P \otimes p_2^{-1} P \to m^{-1} P$ of $G$-torsors over $Q \times Q$ where $p_1$ and $p_2$ are 
    the canonical projections, which satisfifing the cocycle conditions correspoding to symmetricity and associativity.    
\end{definition}
The category of multiplicative $G$-torsors is fibered in groupoids over the category of commutative semigroups of $\cE$. 
We denote by $\mathrm{Tors}^\otimes(Q, G)$ the groupoid of multiplicative $G$-torsors over a commutative semigroup $Q$
 of $\cE$.
}

\textcolor{purple}{
 If $G = \Lambda^\times$ for some ring $\Lambda$ in $\cE$, we use the term \textbf{multiplicative locally free $\Lambda$-module of rank $1$}
  as a synonym for \textbf{multiplicative $G$-torsor}, when we want to emphasize the locally free $\Lambda$-module of rank $1$ corresponding
   to a given $G$-torsor, rather than the $G$-torsor itself (cf. 2.3).
}

There are several important theorems that repeat themselves throughout deligne approach to class field theory,
one of them, is that the symmetric power of $C$ over $k$, denoted by $C^{(d)}$, is a scheme over $k$,
Guigard proves this in the most general setting so we qoute his result here:
\begin{prop}
    If $X$ is Zariski locally quasi-projective over a scheme $S$, the the right action of $S_d$ on $X^{\times_S d}$ 
    is admissible, and the quotient $X^{(d)} = X^{\times_S d}/S_d$ exists as a scheme over $S$.
\end{prop}

\subsection{G-torsors}
\input{sections/Preliminaries/Gtorsors.tex}

\subsection{Symmetric Powers}
\input{sections/Preliminaries/SymmetricPowers.tex}

\subsection{Kummer Theory}
A kummer extension is a field extension $L/K$ where for some given $n \in \N$ we have:
\begin{enumerate}
    \item $K$ contains all $n$'th roots of unity
    \item $L/K$ has abelian galois group of exponent $n$.
\end{enumerate}

A group $G$ has exponent $n$ if $g^n = 1$ for all $g \in G$.

\begin{example}
    Quadratic extensions are kummer extensions. Multi quadratic Extensions, etc...
\end{example}

\begin{example}
    When $K$ contains n distinct $n$'th roots of unity (hence $char(K) \nmid n$) then the extension $L=K(a^\frac{1}{n})$ is a Kummer extension of degree $m \mid n$, for any element $a \in K$.
    The galois group $G$ is cyclic of order $m$, and acts as multiplication by root of unity of order $m$.% on $a^\frac{1}{n}$.
\end{example}

Kummer theory gives us the converse of the above example:
Let $K$ be a field containing $n$ distinct $n$'th roots of unity, then we have a bijection:
\begin{equation*}
    \{ \text{Kummer extensions } L/K \text{ of exponent dividing } n \} \quad \longleftrightarrow \quad \{ \text{Subgroups } H \text{ of } K^\times/(K^\times)^n \}
\end{equation*}
This bijection is given by the maps:
\begin{align*}
    L              & \longmapsto (K^\times \cap (L^\times)^n) / (K^\times)^n \\
    K(\sqrt[n]{H}) & \longmapsfrom H
\end{align*}

Where $(K^\times)^n$ is the group of $n$'th powers in $K^\times$.
And, $K(\sqrt[n]{H}) := \{ \sqrt[n]{a} \mid a \in K^\times,  a \cdot (K^\times)^n \in H \}$

In the latter case we have:
\begin{align*}
    H & \cong \text{Hom}_c(\text{Gal}(L/K), \mu_n)                                                                                                            \\
    a & \longmapsto \left(\sigma \mapsto \frac{\sigma(\alpha)}{\alpha}\right) \quad \text{where } \alpha \text{ is any } n\text{'th root of } a \text{ in } L
\end{align*}.

Also note that if $K$ contains all roots of unity then every finite abelian extension of $K$ is a kummer extension.
In this paper we will mainly be intersested in kummer extensions of degree $l^m - 1$ with galois group $G=(\Z/l^m \Z)^\times$ where $l$ is a prime number and $char K \neq l$.

\subsubsection{Ramification In Kummer extensions}
Next, we turn to a brief discussion of ramification in kummer extensions.

We begin by the following useful lemma proved in \cite{lang2005algebra}[9.1]
\textcolor{red}{Check if the lang bib entry is correct(i generated it using ai)}

\begin{lemma}
    Let $K$ be a field, and let $2 \leq n \in \N$ be a natural number. Let $0 \neq a \in K$ be a element of $K$.
    Assume that for every prime $p$ dividing $n$ we have $a \notin K^p$, and that if $4 \mid n$ then $a \not in -4K^4$.
    Then $X^n - a$ is irreducible in $K[X]$.
\end{lemma}+

Using this lemma we can show the following proposition:
\textcolor{red}{Something about the proof here doesn't work, complete it}
\begin{prop}
    Let $n \in \N$ be a natural number and let $K$ be a field such that $gcd(n, char K) = 1$.
    For $b \in K^\times$, $X^n - b$ is irreducible in $K[X]$ if and only if $ord(\bar{b}) = n$ in $K^\times/(K^\times)^n$.
\end{prop}
\begin{proof}
    In one direction, note that $ord(\bar{b}) = n$, if and only if $b^k \notin (K^\times)^n$ for every $k \mid n, k < n$, if and only if, $b^{p^r} \notin (K^\times)^n$ for every prime $p \mid n$ (\textcolor{red}{What to do in the case $n = p$?}) and every $r \leq ord_p(n)$.
    Hence we can use the lemma above:
    \begin{enumerate}
        \item For every $p \mid n$, $b^p \notin K^n$ (otherwise $b^{\frac{n}{p}} \in (K^\times)^n$).
        \item If $4 \mid n$ then $i \in K$ ($i^2 = -1$), hence $b \notin -4K^4$ (because $-4 = (2i)^2$).
    \end{enumerate}
    Hence $X^n - b$ is irreducible in $K[X]$.
    In the other direction, assume $X^n - b$ is irreducible in $K[X]$. Then if $p \mid n$  we can not have $b \in K^p$ (By factoring $X^n - b$)

\end{proof}

From now on we will focus on cyclic Kummer extensions.
Those are of the form $L=K(\sqrt[n]{a})$ where $a \in K$ and $n \in \N$ is a natural number.
Their galois group is cyclic of order $n$, and acts as multiplication by root of unity of order $n$ on $a^{\frac{1}{n}}$.
\begin{equation*}
    \text{Gal}(L/K) \cong \Z/n\Z
    \quad
    \sigma \mapsto \sigma(a^{\frac{1}{n}}) = \zeta_n \sigma(a^{\frac{1}{n}})
\end{equation*}




\textcolor{red}{
    \begin{enumerate}
        \item Switch $a^{\frac{1}{n}}$ to $\sqrt[n]{a}$ or $\alpha$ in the galois group description
    \end{enumerate}
}

\subsection{\texorpdfstring{The equivalence between $G$-torsors and locally free $\Lambda$-modules}{The equivalence between G-torsors and locally free Lambda-modules}}

The following proposition establishes the fundamental dictionary between the geometric theory
of principal homogeneous spaces and the algebraic theory of invertible modules.
This equivalence allows us to transport the monoidal structure from the category of modules
(the tensor product) to the category of torsors (the contracted product),
strictly within the categorical framework.


\begin{prop}[\label{prop:torsor_module_equivalence_2}]
    Let $\mathcal{E}$ be a topos and let $\Lambda$ be a ring object in $\mathcal{E}$. Let $G = \Lambda^\times$ denote the internal group object of units of $\Lambda$.

    There is a canonical equivalence of monoidal categories between the category of $G$-torsors in $\mathcal{E}$ and the category of locally free $\Lambda$-modules of rank 1 in $\mathcal{E}$:
    \[
        \Phi: \mathbf{Tors}(\mathcal{E}, \Lambda^\times) \xrightarrow{\sim} \mathbf{Pic}(\mathcal{E}, \Lambda)
    \]
    The equivalence is defined by the associated module functor:
    \[
        P \longmapsto P \times^{\Lambda^\times} \Lambda := \Lambda^\times \backslash (\Lambda \times P)
    \]
    where the quotient is taken with respect to the diagonal action of $\Lambda^\times$ on $\Lambda \times P$. The inverse functor associates to an invertible module $L$ its sheaf of basis frames $\underline{\mathrm{Isom}}_\Lambda(\Lambda, L)$.
\end{prop}

In light of this canonical equivalence, we will pass freely between the language of $G$-torsors and that of locally free $\Lambda$-modules throughout the text.


\subsection{Ramification after Blowup}
For $P \to U$ a $G$-torsor. We denote by $P^{[d]} \to U^{(d)}$ the corresponding $G$-torsor over $U^{(d)}$.
Let $\eta^i$ be the generic point of the exceptional divisor $E^i$ of the blowup of $C^{(d_i n_i)}$ by $n_i P_i$
(Where $deg P_i = d_i$).
Where $P_i \in C \setminus U$
We want to prove:
\begin{prop}
    If $ram P^{[d_1n_1]}_{\eta^1} \leq k_1 $ and $ram P^{[d_2n_2]}_{\eta^2} \leq k_2 $
    Then $ram P^{[d_1n_1 + d_2n_2]}_{\eta} = max(k_1, k_2)$ where $\eta$ is the generic point of the exceptional divisor $E$ of the blowup of $C^{(d_1 n_1 + d_2n_2)}$ by $\mdls= n_1 P_1 + n_2P_2$
\end{prop}

We will work this out along an example:
Let $X=\bG_m=\spec R[t, t^{-1}]$ and Let $\cP=\bG_m \xrightarrow[]{(\cdot)^{n}} G_m$ be the $n$'th power map.
It is a $G=\Z/n\Z$ torsor.
The ring map is $R[t, t^{-1}] \xrightarrow[]{t \mapsto t^n} R[t, t^{-1}]$
which corresponds to ring extension: $R[t, t^{-1}] \to R[t^{\frac{1}{n}}, t^{-\frac{1}{n}}]$.
The field of fractions of $\bG_m$ is $K(t)$ and the corresponding map between fields of $\cP \to X$
is $K(t) \xrightarrow[]{t \to t^n} K(t)$. which corresponds to the field extension:
$K(t) \hookrightarrow K(t^{\frac{1}{n}}) = K(t)[X]/(X^n -t)$.
Where $K = Frac R$

The points $0, \infty \in \bP^1$ correspond to the local rings
$\Oo_{0} = K[t]_{(t)}$ and $\Oo_{\infty} = K[\frac{1}{t}]_{(\frac{1}{t})}$ of $K(t)$,
which are DVRs. The corresponding valuations of $K(t)$ are given by:
\[
    v_0\left(\frac{f}{g}\right) = \text{maximal exponent } n \text{ s.t. } t^{n} \mid \frac{f}{g}, \quad v_\infty\left(\frac{f}{g}\right) = \deg g - \deg f
\]

So the diagram of the G-torsor $\cP=\bG_m$ over $\bG_m$ is:
\begin{equation}
    \begin{tikzcd}
        & \cP=\mathbb{G}_m \ar[d, "(\cdot)^n"] \\
        \bP^1 & \arrow[l] \bG_m
    \end{tikzcd}
\end{equation}

The bounded ramification condition is given by:
\begin{equation}
    \text{ram } P_{\eta} \leq k
\end{equation}

We wish to understand


In the general case of a $G$ torsor $P \to C$ we have similarly:
$K(C) \cong K(t)$ the function ring of $C$ for some variable $t$ and a finite field extension $K/\F_p(t')$.
If the basefield $K$ contains $n$'th roots of unity, then the torsor is the same... and continue here:
$\Oo_{P_1}$
the same..
and continue.



This follows by the following:
Start talking about local rings, and take everything locally and prove it locally.
\begin{prop}
    Let
\end{prop}sss


\textbf{Proposition 6 is proved in takeuchi's paper \cite{takeuchi2019blow}.}
We show here a simpler proof for a simpler case that we would need.

Let $\cP$ be a $G$-torsor over $U$.
Let $\mdls = n_1 P_1 + n_2 P_2 = \mdls_1 + \mdls_2$ be a modulus on $C$.
Let $X_{\mdls_1}, X_{\mdls_2}$ be the blowups of $C^{(\deg \mdls_1)}, C^{(\deg \mdls_2)}$ by $\mdls_1, \mdls_2$ respectively.  
Let $E_1, E_2$ be the respective exceptional divisors, and let $\eta_1$, $\eta_2$ be their generic points respectively.

\begin{prop}\label{prop:box-tame-ram}
Assume $\cP^{(\deg \mdls_1)}$, $\cP^{(\deg \mdls_2)}$ are tamely ramified on $\eta_1, \eta_2$  respectively.
Then, $\boxP{\cP^{(\deg \mdls_1)}}{\cP^{(\deg \mdls_2)}}$ is tamely ramified on $\eta_1 \times \eta_2 \in C^{(\deg \mdls_1)} \times C^{(\deg \mdls_2)}$
\end{prop}
\begin{proof}
    \textcolor{red}{Write proof here}
\end{proof}

Next, we prove:
\begin{prop}\label{prop:blowup-prop}
    Let $X,Y$ be smooth over $k$. $x\in X, y\in Y$ closed points. Let $\Bl_x(X), \Bl_y(Y), \allowbreak Bl_{(x,y)}(X\times_k Y) $ be the respective blowups. 
    Let $\eta_X, \eta_Y,\eta_{X\times Y}$ be the generic points of the exceptional divisor of the respective blowups. 
    Then, there exists a scheme $\tilde{U}$ and maps $f_1, f_2$ making the diagram commute:
    
    \begin{diagram}
         \begin{tikzcd}
            {} & \tilde{U} \arrow[dl, "f_1"] \arrow[dr, hook, "f_2"] & {} \\
            \Bl_x(X)\times_k \Bl_y(Y) & {} & \Bl_{(x,y)}(X\times_k Y)
        \end{tikzcd}
    \end{diagram}
    such that:
    \begin{enumerate}
        \item $f_2$ is open immersion
        \item $f_1$ is open map (?open immersion?)
        \item $\eta_{X\times Y} \in \tilde{U}$
        \item $\eta_{X\times Y} \overset{f_1}{\mapsto} \eta_X \times \eta_Y$ 
    \end{enumerate}
\end{prop}

Let $P^{[d]} \to U^{(d)}$ be the corresponding $G$-torsor over $U^{(d)}$.


\begin{lemma}\label{lemma:boxplus}
Let $p: C^{(d)} \times_k C^{(n-d)} \xrightarrow[]{+} C^{(n)}$ be the plus map, restricting $p$ to $U^{(d)} \times_k U^{(n-d)}$ we get a map 
$$p: U^{(d)} \times_k U^{(n-d)} \xrightarrow[]{p} U^{(n)}$$
Then, $$p^* (\cP^{(n)}) \cong \boxPp {\cP^{[d]}}{\cP^{(n-d)}}[k]$$
\end{lemma}

\begin{prop}
    Combinining \Cref{prop:box-tame-ram}, \Cref{prop:blowup-prop} and \Cref{lemma:boxplus} we get that if $P^{[d_1 n_1]}$, $P^{[d_2 n_2]}$ are tamely ramified on $\eta_1, \eta_2$ respectively, then $P^{[d_1 n_1 + d_2 n_2]}$ is tamely ramified on $\eta$ the generic point of the exceptional divisor of the blowup of $C^{(d_1 n_1 + d_2 n_2)}$ by $\mdls = n_1 P_1 + n_2 P_2$.
\end{prop}

\textbf{A few notes}
\begin{enumerate}
    \item takeuchi's paper \cite{takeuchi2019blow} proves a more general version of \Cref{prop:blowup-prop} for arbitrary ramificiation, not necessarily smooth over a field.
    more generally, he proves:
        \begin{prop}\label{prop:ram-blowup}
        If $ram P^{[d_1n_1]}_{\eta^1} = k_1 $ and $ram P^{[d_2n_2]}_{\eta^2} = k_2 $
        Then $ram P^{[d_1n_1 + d_2n_2]}_{\eta} = max(k_1, k_2)$ where $\eta$ is the generic point of the exceptional divisor $E$ of the blowup of $C^{(d_1 n_1 + d_2n_2)}$ by $\mdls= n_1 P_1 + n_2P_2$
        \end{prop}

    \item more generally, the above works for any finite number of points $P_i$ with multiplicities $n_i$. the generalizaiton is easy.
    
\end{enumerate}
\textcolor{red}{ass somewhere the defintiion of box product}
