
The following proposition establishes the fundamental dictionary between the geometric theory
of principal homogeneous spaces and the algebraic theory of invertible modules.
This equivalence allows us to transport the monoidal structure from the category of modules
(the tensor product) to the category of torsors (the contracted product),
strictly within the categorical framework.


\begin{prop}[\label{prop:torsor_module_equivalence_2}]
    Let $\mathcal{E}$ be a topos and let $\Lambda$ be a ring object in $\mathcal{E}$. Let $G = \Lambda^\times$ denote the internal group object of units of $\Lambda$.

    There is a canonical equivalence of monoidal categories between the category of $G$-torsors in $\mathcal{E}$ and the category of locally free $\Lambda$-modules of rank 1 in $\mathcal{E}$:
    \[
        \Phi: \mathbf{Tors}(\mathcal{E}, \Lambda^\times) \xrightarrow{\sim} \mathbf{Pic}(\mathcal{E}, \Lambda)
    \]
    The equivalence is defined by the associated module functor:
    \[
        P \longmapsto P \times^{\Lambda^\times} \Lambda := \Lambda^\times \backslash (\Lambda \times P)
    \]
    where the quotient is taken with respect to the diagonal action of $\Lambda^\times$ on $\Lambda \times P$. The inverse functor associates to an invertible module $L$ its sheaf of basis frames $\underline{\mathrm{Isom}}_\Lambda(\Lambda, L)$.
\end{prop}

In light of this canonical equivalence, we will pass freely between the language of $G$-torsors and that of locally free $\Lambda$-modules throughout the text.
