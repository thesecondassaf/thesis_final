\section{Class Field Theory In the language of Ideals}
In this section we describe the main results of classical class field theory for global fields, following 
\cite{milneCFT}.
We copy most of the content here from Milne. 

\subsection{Ideals, Moduli and Ray Class Groups}
Let $K$ be a global field of $\ch (K) = p$. A modulus $\mdls$ is a formal sum of places of $K$ with non-negative integer 
coefficients. 
Let $S(K, \mdls) = S(\mdls) = \{ v \in \mdls \}$ be the set of places appearing in $\mdls$ with non-zero coefficient.

Define $K_{\mdls, 1} = \{ x \in K^\times \mid v(x - 1) \geq n_v \text{ for all } v \in S(\mdls) \}$ where $n_v$ is the coefficient of $v$ in $\mdls$.

For every set of primes $S$ we define
\[I_K^S = \{ \text{ fractional ideals of } K \text{ generated by primes not in } S \}\]
There is a natural map $i: K_{\mdls, 1} \to I_K^{S(\mdls)}$ sending $x \mapsto (x)$

The qoutient \[C_\mdls = I_K^{S(\mdls)} / i(K_{\mdls, 1})\]
is called the \textbf{(ray) class group} of $K$ modulo $\mdls$.

Let $S$ be a finite set of primes of $K$. And $G$ a finite abelian group.
We shall say that a homomorphism $\psi: I^S \to G$ \textbf{admits a modulus} if there exists a modulus $\mathfrak{m}$ with $S(\mathfrak{m}) \supset S$ 
such that $\psi(i(K_{\mathfrak{m}, 1})) = 0$. 
Thus $\psi$ admits a modulus if and only if it factors through $C_\mathfrak{m}$ for some $\mathfrak{m}$ with 
$S(\mathfrak{m}) \supset S$.

\textcolor{red}{maybe we don't need this}
Milne states and prove a known theorem: 
\begin{theorem}
    For every modulus $\mdls$ of $K$ there is an exact sequence:
    \[
    0 \to \Oo_K^\times/ \Oo_K^\times \cap K_{\mdls_,1} \to K_\mdls / K_{\mdls, 1} \to C_\mdls \to C \to 0
    \]
    Where
    \[
        K_\mdls = \{ x \in K^\times \mid v(x) = 0 \text{ for all } v \in S(\mdls) \} 
    \]
    And $C$ is the usual class group of $K$.
\end{theorem}

\subsection{The Main Theorems}
\begin{theorem}[Artin Reciprocity Law] \label{theorem:ArtinReciprocity}
    Let $L$ be a finite abelian extension of a global field $K$. and let $S$ be the set of primes of $K$
    ramifying in $L$. Then the Artin map \textcolor{red}{add here reference of the definition to milne} .
    $\psi: I^S \to Gal(L/K)$ admits a modulus $\mdls$ with $S(\mdls) = S$ and it defines an isomorphism:
    \[I^S / \left(i(K_{\mdls, 1}) \cdot N_{L/K}(I_L^{S(\mdls)}) \right) \to Gal(L/K)\]
\end{theorem}

A modulus $\mdls$ as in the statement of the theorem is called a definning modulus for $L$. 
Next, we write $I_K^{\mdls}$ for the group of $S(\mdls)$-ideals in $K$, and $I_L^{\mdls}$ for the group of $S(\mdls)'$-ideals in $L$ 
where $S(\mdls)'$ is the set of primes of $L$ lying above primes in $S(\mdls)$.
Call a subgroup $H$ of $I_K^{\mdls}$ a \textbf{congruence subgroup} modulo $\mdls$ if it contains $i(K_{\mdls, 1})$.
    

\begin{theorem}\label{theorem:ExistenceTheorem}
[Existence Theorem of Class Field Theory]
For every congurence subgroup $H$ modulo $\mdls$ there exists a unique finite abelian extension $L/K$, unramified at all primes not in $S(\mdls)$, such that the Artin map induces an isomorphism:
\[I^{S(\mdls)} / H \to Gal(L/K)\].
\end{theorem}

More of the idealic class field theory in Milne. 

Theorems \Cref{theorem:ArtinReciprocity} and \Cref{theorem:ExistenceTheorem} show that there is a canonical group isomorphism:
\begin{equation}\label{eq:idealClassFieldTheory}
 \lim_{\longleftarrow \mdls} C_\mdls \to \operatorname{Gal}(K^{\mathrm{ab}} / K).    
\end{equation}
 

Rather than studying $ \limn{\longleftarrow m}{C_m}$ directly, it turns out to be more natural to introduce another group that has it as a quotient - this is the idele class group.
\textcolor{red}{replace very where idele with ide'le}

