In this thesis, we give an elementary proof of a certain imporant geometric theorem occuring in Deligne's approach to geometric class field theory.
One of the main geometric ingredients in the approach, is showing why a local system $\cF$ with ramification bounded by a modulus $\mdls$ on $U = C \setminus \mdls$ descends via the Abel-Jacobi $\Phi: U \to \PicCm[]$ to $\PicCm[]$. 
The approach, innovated by Deligne, relies on analyzing the symmetric powers $\cF^{(d)}$ of $\cF$ on the symmetric powers $U^{(d)}$ of $U$,
and showing that for sufficiently large $d$, $\cF^{(d)}$ descends to $\PicCm[d]$ via the degree $d$ Abel-Jacobi map $\Phi_d : U^{(d)} \to \PicCm[d]$.
The fibers of $\Phi_d$ (for $d \geq 2g - 1$) over any point are isomorphic to 
$$
\begin{cases}
\mathbb{A}_k^{d - \deg \mdls - g + 1} & \text{if } \mdls > 0 \\
\mathbb{P}_k^{d - g} & \text{if } \mdls = 0
\end{cases}
$$
Where $g$ is the genus of the curve $C$.
The unramified case ($\mdls = 0$) is relatively simple, as the Abel-Jacobi map is proper and smooth, which follows from the fact that it is a fibration in projective spaces.
Thus, by using the homotopy exact sequence for the etale fundamental group, one gets an isomorphism between the etale fundamental group of $U^{(d)} (=C^{(d)})$ and that of $\PicCm[d] (=\PicC[d])$.

The ramified case ($\mdls > 0$) is more subtle, as the Abel-Jacobi map is not proper anymore, and one needs to analyze the ramification of $\cF^{(d)}$ "along the boundary" of $U^{(d)}$ in $C^{(d)}$.

Previous work has generalized Deligne's approach to the ramified case, 
most notably by Guignard \cite{Guignard2018} and Takeuchi \cite{takeuchi2019blow}. 
Their approaches differ. 
To descend, Guignard proves that the restriction of $\cF^{(d)}$ to any line in the fiber of the degree $d$ Abel-Jacobi map is a constant 
étale sheaf. He achieves this by demonstrating that the restriction is at most tamely ramified and invoking the triviality of the tame fundamental group 
of $\mathbb{A}^1_k$. His analysis relies on local geometric class field theory. 
It is also worth noting that Guignard's method generalizes to relative curves over arbitrary base schemes.
Takeuchi, on the other hand, constructs a compactification of $U^{(d)}$ by blowing up $C^{(d)}$ along certain well-chosen centers.
This compactification, denoted by $\tilde{C}^{(d)}_\mdls$, has $U^{(d)}$ as an open subscheme with a codimension 1 closed subscheme $H$ as complement. 
He then shows that the Abel-Jacobi map extends to a proper morphism from $\tilde{C}^{(d)}_\mdls$ to $\PicCm[d]$,
which is a fibration in projective spaces. Thus, by the homotopy exact sequence for the etale fundamental group, one gets an isomorphism between the etale fundamental group of
$\tilde{C}^{(d)}_\mdls$ and that of $\PicCm[d]$. 
To conclude the descent, Takeuchi analyzes the ramification of $\cF^{(d)}$ along the boundary $H$ of $\tilde{C}^{(d)}_\mdls$,
showing that it is tamely ramified there, which suffices. His methods relies on the theory of Witt vectors and refined Swan conductors.

For an account of these approaches, see \cite{Guignard2018} and \cite{takeuchi2019blow}.
For a full approach following Deligne's method in the unramified case, and the tamely ramified case see \cite{tendler2015geometricclassfieldtheory}, and \cite{Toth2011}.

In this thesis, we combine techniques and ideas from the approaches, and from \cite{tendler2015geometricclassfieldtheory}, to give an elementary proof of the ramified case of Deligne's approach to geometric class field theory.
We follow Takeuchi's construction of the compactification $\tilde{C}^{(d)}_\mdls$ of $U^{(d)}$ by blowing up $C^{(d)}$ and calculate the ramification of $\cF^{(d)}$ along the boundary $H$ of $\tilde{C}^{(d)}_\mdls$ directly, avoiding 
the use of Swan conductors. 

In the rest of the introduction, we state the main theorem of geometric class field theory \cref{thm:GCFT}, and its reduction to \cref{thm:GCFT_reduced}, which we prove in this thesis.

Let $k$ be a perfect field, and let $C$ be a projective smooth geometrically connected curve over $k$, with genus $g$. 
Geometric class field theory gives a geometric description of abelian coverings of $C$ by relating it to isogenies of the generalized picard schemes.

Fix a modulus $\mdls$, i.e.\ an effective Cartier divisor of $C$ and let $U$ be its complement in $C$. 
The pairs $(\cL, \alpha)$, where $\cL$ is an invertible $\cO_C$-module and $\alpha$ is a rigidification of $\cL$ along $\mdls$, are parametrized by a $k$-group scheme $\PicCm$, called the rigidified Picard scheme.
The Abel-Jacobi morphism $$ \Phi : U \to \PicCm $$
is the morphism which sends a section $x$ of $U$ to the pair $(\cO(x), 1)$ . 
We prove the following version of geometric class field theory: 
\begin{theorem}[Geometric Class Field Theory]\label{thm:GCFT}
    Let $\Lambda$ be a finite ring of cardinality invertible in $k$, and let $\cF$ be an \'etale sheaf of $\Lambda$-modules, locally free of rank 1 on $U$, with ramification bounded by $\mdls$. 
    Then, there exists a unique (up to isomorphism) \textcolor{purple}{\underline{\textit{multiplicative}}} \'etale sheaf of $\Lambda$-modules $\cG$ on $\PicCm$, locally free of rank 1, such that the pullback of $\cG$ by $\Phi$ is isomorphic to $\cF$.
\end{theorem}

\textcolor{purple}{The notion of a multiplicative locally free $\Lambda$-module of rank 1 is due to \cite{Guignard2018} 
and corresponds to isogenies $G \to \PicCm$ with constant kernel $\Lambda^\times$. 
This concept corresponds to multiplicative characters of $H^1(\PicCm[], \Q/\Z)$ in the formulation of \cite{takeuchi2019blow}, 
and generalizes Hecke eigensheaves in the context of \cite{tendler2015geometricclassfieldtheory}.}

Let $d$ be a positive integer. 
We denote by $U^{(d)}$ the $d$-th symmetric power of $U$ over $k$. 
For an \'etale sheaf $\cF$ on $U$, we denote by $\cF^{(d)}$ the $d$-th symmetric power of $\cF$ on $U^{(d)}$.
The degree $d$ Abel-Jacobi morphism is defined as the map
$$ \Phi_d : U^{(d)} \to \PicCm[d] $$
which sends a section $x_1 + \cdots + x_d$ of $U^{(d)}$ to the pair $(\cO(x_1 + \cdots + x_d), 1)$ .

The method of descent shows that to prove \cref{thm:GCFT}, 
it suffices to prove the following reduced version (see the last page of \cite{Guignard2018}, 
Section 8.3 of \cite{tendler2015geometricclassfieldtheory}, or the proof of Theorem 1.2 in \cite{takeuchi2019blow} for details on this reduction):

\begin{theorem}\label{thm:GCFT_reduced}
    Let $\Lambda$ be a finite ring of cardinality invertible in $k$, and let $\cF$ be an \'etale sheaf of $\Lambda$-modules, locally free of rank 1 on $U$, with ramification bounded by $\mdls$. 
    Then, for sufficiently large integer $d$, there exists a unique (up to isomorphism) \'etale sheaf of $\Lambda$-modules $\cG_d$ on $\PicCm[d]$, locally free of rank 1, such that the pullback of $\cG_d$ by $\Phi_d$ is isomorphic to $\cF^{(d)}$.
\end{theorem}

Using the equivilance between $G$-torsors and locally free $\Lambda$-modules of rank 1 ($G=\Lambda^\times$, 
see \cref{prop:torsor_module_equivalence}), Theorem \ref{thm:GCFT_reduced} can be reformulated in terms of $G$-torsors as follows:
\begin{theorem}\label{thm:GCFT_torsors}
    Let $G=\Lambda^\times$ be a finite abelian group ($\Lambda$ as before), and let $\cP$ be a $G$-torsor on $U$, with ramification bounded by $\mdls$. 
    Then, for sufficiently large integer $d$, there exists a unique (up to isomorphism) $G$-torsor $\cQ_d$ on $\PicCm[d]$, such that the pullback of $\cQ_d$ by $\Phi_d$ is isomorphic to $\cP^{(d)}$.
\end{theorem}

To prove Theorem \ref{thm:GCFT_torsors} we follow the work of \cite{takeuchi2019blow}, there he analyzed 
the ramification of $\cP^{(d)}$ after blowing up $C^{(d)}$, we analyze this ramification using elementary methods, drawing techniques and ideas
from the works of \cite{Guignard2018} and \cite{takeuchi2019blow}, and \cite{tendler2015geometricclassfieldtheory}.


\textbf{Notation and conventions}. 


\textcolor{red}{
 \begin{enumerate}
    \item A word about the decompositon of the Picard scheme into connected components indexed by degree.
    \item Definition of the abel jacobi-map, and its properties, its fibers, see how amichai does it. 
    \item Definition of multiplicative sheaves?? and maybe its not needed here, or we are not exactly right here...
    \item Say something about the ramification condition.
 \end{enumerate}
}

