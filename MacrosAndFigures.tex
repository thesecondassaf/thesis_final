% -------------------- TikZ Patterns --------------------
\usetikzlibrary{patterns}
\tikzset{
  hatch distance/.store in=\hatchdistance,
  hatch distance=10pt,
  hatch thickness/.store in=\hatchthickness,
  hatch thickness=1pt
}
\makeatletter
\pgfdeclarepatternformonly[\hatchthickness]{my horizontal lines}
{\pgfpointorigin}{\pgfqpoint{100pt}{1pt}}{\pgfqpoint{100pt}{3pt}}{
  \pgfsetcolor{\tikz@pattern@color}
  \pgfsetlinewidth{\hatchthickness}
  \pgfpatheoremoveto{\pgfqpoint{0pt}{0.5pt}}
  \pgfpathlineto{\pgfqpoint{100pt}{0.5pt}}
  \pgfusepath{stroke}
}
\pgfdeclarepatternformonly[\hatchthickness]{my vertical lines}
{\pgfpointorigin}{\pgfqpoint{1pt}{100pt}}{\pgfqpoint{3pt}{100pt}}{
  \pgfsetcolor{\tikz@pattern@color}
  \pgfsetlinewidth{\hatchthickness}
  \pgfpatheoremoveto{\pgfqpoint{0.5pt}{0pt}}
  \pgfpathlineto{\pgfqpoint{0.5pt}{100pt}}
  \pgfusepath{stroke}
}
\pgfdeclarepatternformonly[\hatchthickness]{my north east lines}
{\pgfqpoint{-1pt}{-1pt}}{\pgfqpoint{4pt}{4pt}}{\pgfqpoint{3pt}{3pt}}{
  \pgfsetcolor{\tikz@pattern@color}
  \pgfsetlinewidth{1pt}
  \pgfpatheoremoveto{\pgfqpoint{0pt}{0pt}}
  \pgfpathlineto{\pgfqpoint{3.1pt}{3.1pt}}
  \pgfusepath{stroke}
}
\pgfdeclarepatternformonly[\hatchthickness]{my north west lines}
{\pgfqpoint{-1pt}{-1pt}}{\pgfqpoint{4pt}{4pt}}{\pgfqpoint{3pt}{3pt}}{
  \pgfsetcolor{\tikz@pattern@color}
  \pgfsetlinewidth{1.2pt}
  \pgfpatheoremoveto{\pgfqpoint{0pt}{3pt}}
  \pgfpathlineto{\pgfqpoint{3.1pt}{-0.1pt}}
  \pgfusepath{stroke}
}
\makeatother

% -------------------- Color --------------------
\colorlet{grayX}{black!26!white}
\colorlet{darkgrayX}{black!47!white}

% -------------------- Cross-reference shortcuts --------------------
\makeatletter
\AtBeginDocument{\providecommand\corref[1]{\ref{cor:#1}}}
\AtBeginDocument{\providecommand\theoremref[1]{\ref{theorem:#1}}}
\AtBeginDocument{\providecommand\lemref[1]{\ref{lem:#1}}}
\AtBeginDocument{\providecommand\figref[1]{\ref{fig:#1}}}
\makeatother
\providecommand{\tabularnewline}{\\}

% -------------------- Theorem Environments --------------------

% Base counter:
\newtheorem{theorem}{Theorem}
\newtheorem*{theorem*}{Theorem}

% --- Definition ---
\newaliascnt{definition}{theorem}
\newtheorem{definition}[definition]{Definition}
\aliascntresetthe{definition}
\crefname{definition}{definition}{definitions}
\Crefname{definition}{Definition}{Definitions}

% --- Corollary ---
\newaliascnt{cor}{theorem}
\newtheorem{cor}[cor]{Corollary}
\aliascntresetthe{cor}
\crefname{cor}{corollary}{corollaries}
\Crefname{cor}{Corollary}{Corollaries}

% --- Lemma ---
\newaliascnt{lemma}{theorem}
\newtheorem{lemma}[lemma]{Lemma}
\aliascntresetthe{lemma}
\crefname{lemma}{lemma}{lemmas}
\Crefname{lemma}{Lemma}{Lemmas}

% --- Proposition ---
\newaliascnt{prop}{theorem}
\newtheorem{prop}[prop]{Proposition}
\aliascntresetthe{prop}
\crefname{prop}{proposition}{propositions}
\Crefname{prop}{Proposition}{Propositions}

% --- Proposition* (unnumbered) ---
\newtheorem*{prop*}{Proposition}

% --- Observation ---
\newaliascnt{obs}{theorem}
\newtheorem{obs}[obs]{Observation}
\aliascntresetthe{obs}
\crefname{obs}{observation}{observations}
\Crefname{obs}{Observation}{Observations}

% --- Claim ---
\newaliascnt{claim}{theorem}
\newtheorem{claim}[claim]{Claim}
\aliascntresetthe{claim}
\crefname{claim}{claim}{claims}
\Crefname{claim}{Claim}{Claims}

% --- Problem (independent counter, not sharing with theorem) ---
\newtheorem{problem}{Problem}
\crefname{problem}{problem}{problems}
\Crefname{problem}{Problem}{Problems}

% --- Remark-style (unnumbered) ---
\theoremstyle{remark}
\newtheorem*{note*}{Note}
\newtheorem*{rem*}{Remark}
\newtheorem*{notation*}{Notation}
\newtheorem*{context*}{Context}

% -------------------- Example Environment --------------------
\theoremstyle{definition}
\newtheorem{example}[theorem]{\protect\examplename}
\providecommand{\examplename}{Example}

% -------------------- Cleveref labels --------------------
\crefname{equation}{}{}
\crefname{diagram}{Diagram}{Diagrams}

% -------------------- Main and Front Matter --------------------
\makeatletter
\newif\if@mainmatter \@mainmattertrue
\newcommand\frontmatter{%
  \cleardoublepage
  \@mainmatterfalse
  \pagenumbering{Roman}}
\newcommand\mainmatter{%
  \cleardoublepage
  \@mainmattertrue
  \pagenumbering{arabic}}
\makeatother

% -------------------- Fonts and symbols --------------------
\def\wasyfamily{\fontencoding{U}\fontfamily{wasy}\selectfont}
\DeclareTextFontCommand{\textwasy}{\wasyfamily}
\DeclareSymbolFont{wasy}{U}{wasy}{m}{n}
\SetSymbolFont{wasy}{bold}{U}{wasy}{b}{n}
%\def\hexagon    {\mbox{\wasyfamily\char55}}

% -------------------- Standard Math Shortcuts --------------------
\newcommand{\FF}{\mathsf{F}}
\newcommand{\ff}{\mathsf{f}}
\newcommand{\N}{\mathbb{N}}
\newcommand{\bP}{\mathbb{P}}
\newcommand{\bG}{\mathbb{G}}


\newcommand{\Z}{\mathbb{Z}}
\newcommand{\F}{\mathbb{F}}
\newcommand{\A}{\mathbb{A}}
\newcommand{\bO}{\mathbb{O}}

\newcommand{\R}{\mathbb{R}}
\newcommand{\Q}{\mathbb{Q}}
\newcommand{\C}{\mathbb{C}}
\newcommand{\Pro}{\mathbb{P}}

\newcommand{\Qp}{\mathbb{Q}_{p}}
\newcommand{\Fqt}{\mathbb{F}_q(t)}
\newcommand{\Fpt}{\mathbb{F}_p(t)}

\newcommand{\Ff}[2][F]{#1({#2})}
\newcommand{\Ffl}[2][F]{#1(({#2}))}
\DeclarePairedDelimiter\abs{\lvert}{\rvert}%

% -------------------- Mathcal and Mathfrak --------------------
\newcommand{\cF}{\mathcal{F}}
\newcommand{\cE}{\mathcal{E}}
\newcommand{\cG}{\mathcal{G}}
\newcommand{\cC}{\mathcal{C}}
\newcommand{\cL}{\mathcal{L}}
\newcommand{\cX}{\mathcal{X}}
\newcommand{\cO}{\mathcal{O}}
\newcommand{\cP}{\mathcal{P}}
\newcommand{\cQ}{\mathcal{Q}}


\newcommand{\cY}{\mathcal{Y}}
\newcommand{\cZ}{\mathcal{Z}}
\newcommand{\cA}{\mathcal{A}}
\newcommand{\cD}{\mathcal{D}}
\newcommand{\Sh}{\mathcal{\mathop{\mathit{Sh}}}}
\newcommand{\fC}{\mathbf{C}}

\newcommand{\vp}{\varphi}

% -------------------- Math Operators --------------------
\DeclareMathOperator{\lip}{Lip}
\DeclareMathOperator{\pic}{Pic}
\DeclareMathOperator{\Pic}{Pic}

\DeclareMathOperator{\Spl}{Spl}

\DeclareMathOperator{\Frob}{Frob}
\DeclareMathOperator{\Gal}{Gal}
\DeclareMathOperator{\ram}{ram}
\DeclareMathOperator{\Supp}{Supp}
\DeclareMathOperator{\ch}{char}
\DeclareMathOperator{\codim}{codim}
\DeclareMathOperator{\Sym}{Sym}


\DeclareMathOperator{\id}{id}

\newcommand{\PicCm}[1][]{\operatorname{Pic}^{#1}_{C, \mdls}}
\newcommand{\PicC}[1][]{\operatorname{Pic}^{#1}_{C}}

\newcommand{\spec}{\operatorname{spec}}
\newcommand{\Spec}{\operatorname{Spec}}
\newcommand{\Hom}{\operatorname{Hom}}

\newcommand{\Bl}{\operatorname{Bl}}

\newcommand{\Frac}{\operatorname{Frac}}
\newcommand{\Tors}{\operatorname{Tors}}


% -------------------- Ideals and Valuation Notation --------------------
\newcommand{\ind}{\mathbbm{1}}
\newcommand{\norm}[1]{\left\lVert#1\right\rVert}
\newcommand{\numberthis}{\addtocounter{equation}{1}\tag{\theequation}}

\newcommand{\hypa}[1]{\textbf{A}$(#1)$}
\newcommand{\hyp}[1]{\textbf{B}$(#1)$}

\newcommand{\D}{\Delta}
\renewcommand{\arraystretch}{1.3}

\newcommand{\OK}{\mathcal{O}_K}
\newcommand{\OL}{\mathcal{O}_L}
\newcommand{\Oo}{\mathcal{O}}
\newcommand{\Onu}{\mathcal{O}_\nu}

\newcommand{\mnu}{\mathfrak{m}_\nu}
\newcommand{\mdls}{\mathfrak{m}}
\newcommand{\ndls}{\mathfrak{n}}
\newcommand{\cdls}{\mathfrak{c}}
\newcommand{\f}{\mathfrak{f}}
\newcommand{\fdls}{\mathfrak{f}}

\newcommand{\pp}{\mathfrak{p}}
\newcommand{\Pp}{\mathfrak{P}}
\newcommand{\qq}{\mathfrak{q}}
\newcommand{\Qq}{\mathfrak{Q}}

\newcommand{\I}{\mathbb{I}}
\newcommand{\Ak}{\mathbb{A}_K}
\newcommand{\Il}{\mathbb{I}_L}
\newcommand{\Idl}{\mathbb{I}_L}

\newcommand{\Fbox}[2]{\mathcal{\cF}^{(#1)} \boxtimes \mathcal{\cF}^{(#2)}}
\newcommand{\boxP}[2]{{#1} \boxtimes #2}

\NewDocumentCommand{\boxPp}{m m o}{%
  \IfValueTF{#3}%
    {#1 \boxtimes_{#3} #2}% If the 3rd argument exists
    {#1 \boxtimes #2}%      If the 3rd argument is missing
}

\newcommand{\Cmod}[1]{\tilde{C}_{\mdls}^{(#1)}}

% -------------------- TikZ: arrow labels --------------------
\tikzset{
  symbol/.style={
      draw=none,
      every to/.append style={
          edge node={node [sloped, allow upside down, auto=false]{$#1$}}}
    }
}

% -------------------- Function arrows --------------------
\newcommand{\function}[5]{%
  \begin{tikzcd}[
      column sep=2em,
      row sep=1ex,
      ampersand replacement=\&
    ]
    #1\colon \&[-3em]
    #2\vphantom{#3} \arrow[r] \&
    #3\vphantom{#2} \\
    \&
    #4\vphantom{#5}  \arrow[r,mapsto] \&
    #5\vphantom{#4}
  \end{tikzcd}%
}

\newcommand{\isomto}{\stackrel{\sim}{\smash{\longrightarrow}\rule{0pt}{0.4ex}}}
\newcommand{\isoto}{\overset{\sim}{\to}}

% -------------------- Restrictions --------------------
\newcommand\restr[2]{%
\left.\kern-\nulldelimiterspace #1 \littletaller \right|_{#2}%
}
\newcommand{\littletaller}{\mathchoice{\vphantom{\big|}}{}{}{}}

% -------------------- Equation & Diagram Numbering --------------------
\newcounter{diagram}
\renewcommand{\thediagram}{\arabic{diagram}}
\newenvironment{diagram}
{\refstepcounter{diagram}\begin{equation}\tag*{\textnormal{(\thediagram})}}
    {\end{equation}}
\newcommand{\diagramref}[1]{\textnormal{Diagram~\ref{#1}}}

% -------------------- Stacks Tag Shortcut --------------------
\newcommand{\stackstag}[1]{%
  \cite[\href{https://stacks.math.columbia.edu/tag/#1}{Tag~#1}]{stacks-project}%
}
\newcommand{\nlabpage}[2]{%
  \cite[\href{https://ncatlab.org/nlab/revision/#1/#2}{#1, Rev.~#2}]{nlab}%
}


% -------------------- Provide missing names --------------------
\providecommand{\claimname}{Claim}
\providecommand{\corollaryname}{Corollary}
\providecommand{\definitionname}{Definition}
\providecommand{\lemmaname}{Lemma}
\providecommand{\probname}{Problem}
\providecommand{\notationname}{Notation}
\providecommand{\notename}{Note}
\providecommand{\propositionname}{Proposition}
\providecommand{\remarkname}{Remark}
\providecommand{\theoremname}{Theorem}
\providecommand{\observationname}{Observation}
\providecommand{\contextnname}{Context}

% Other Shortcuts
\newcommand{\SheavesSetsEquiv}{
  \[
    \left\{
    \begin{matrix}
      \text{sheaves of sets on } \Spec(L)_{\acute{e}tale}
    \end{matrix}
    \right\}
    \overset{\sim}{\longrightarrow}
    \left\{
    \begin{matrix}
      \text{sets with left continuous action of } G_L
    \end{matrix}
    \right\}
  \]
}

\newcommand{\FiniteLocConstSheavesEquiv}{
  \[\left\{
    \begin{matrix}
      \text{finite locally constant} \\
      \text{sheaves of sets on } \Spec(L)_{\acute{e}tale}
    \end{matrix}
    \right\}
    \overset{\sim}{\longrightarrow}
    \textit{Finite-}G_L\textit{-Sets} \]
}

\newcommand{\limn}[2]{\lim\limits_{#1} #2}