\documentclass[11pt]{article}
\usepackage{geometry}\geometry{margin=1in}
\usepackage[T1]{fontenc}
% if using luatex to compile, the following is unnecessary 
%\usepackage[latin9]{inputenc}
\usepackage{ifthen}

%%%% this is incharge of spacing between paragraphs.... fix later
\setlength{\parindent}{1em}
\setlength{\parskip}{0.5em}

\usepackage{titling}
\usepackage{color}
\usepackage[english]{babel}
\usepackage{refstyle}
\usepackage{float}
\usepackage{amsmath}
\usepackage{amsthm}
\usepackage{makecell}
\usepackage{bbm}
\usepackage{amssymb}
\usepackage{stmaryrd}
\usepackage{enumitem}
\usepackage{setspace}
\usepackage{comment}
\usepackage{mathtools}
\usepackage{subcaption}
\usepackage{caption}
\usepackage{tikz}
\usepackage{rotating} %for "turn"
\usepackage{fontawesome} %for the shield icons
\usepackage[misc]{ifsym} %for the fire icons
\usepackage{ifthen}
\usepackage{graphicx}
\usepackage{tikz-3dplot}
\usepackage{xstring}
\usepackage{xcolor}
\usepackage[colorinlistoftodos]{todonotes}
\usepackage{aliascnt}
\usepackage[unicode=true,pdfusetitle,bookmarks=true,bookmarksnumbered=false,bookmarksopen=false,breaklinks=false,pdfborder={0 0 1},backref=false,colorlinks=true,linkcolor=blue]{hyperref}
\usepackage[nameinlink]{cleveref}
% \usepackage{natbib}
\usepackage{verbatim}
\usepackage{tikz-cd}
\usepackage{url}
\usepackage{hyperref}
% \usepackage{emoji}
\usepackage{xparse}
\usepackage[all]{xy}
\usepackage{mathrsfs}
\usepackage{thmtools} 
\usepackage{thm-restate}




\usetikzlibrary{calc}
\pgfmathsetmacro{\scaling}{0.4}

\delimitershortfall-1sp
\usepackage{mleftright}
\mleftright % make \left & \right behave like \mleft & \mright


% -------------------- TikZ Patterns --------------------
\usetikzlibrary{patterns}
\tikzset{
  hatch distance/.store in=\hatchdistance,
  hatch distance=10pt,
  hatch thickness/.store in=\hatchthickness,
  hatch thickness=1pt
}
\makeatletter
\pgfdeclarepatternformonly[\hatchthickness]{my horizontal lines}
{\pgfpointorigin}{\pgfqpoint{100pt}{1pt}}{\pgfqpoint{100pt}{3pt}}{
  \pgfsetcolor{\tikz@pattern@color}
  \pgfsetlinewidth{\hatchthickness}
  \pgfpatheoremoveto{\pgfqpoint{0pt}{0.5pt}}
  \pgfpathlineto{\pgfqpoint{100pt}{0.5pt}}
  \pgfusepath{stroke}
}
\pgfdeclarepatternformonly[\hatchthickness]{my vertical lines}
{\pgfpointorigin}{\pgfqpoint{1pt}{100pt}}{\pgfqpoint{3pt}{100pt}}{
  \pgfsetcolor{\tikz@pattern@color}
  \pgfsetlinewidth{\hatchthickness}
  \pgfpatheoremoveto{\pgfqpoint{0.5pt}{0pt}}
  \pgfpathlineto{\pgfqpoint{0.5pt}{100pt}}
  \pgfusepath{stroke}
}
\pgfdeclarepatternformonly[\hatchthickness]{my north east lines}
{\pgfqpoint{-1pt}{-1pt}}{\pgfqpoint{4pt}{4pt}}{\pgfqpoint{3pt}{3pt}}{
  \pgfsetcolor{\tikz@pattern@color}
  \pgfsetlinewidth{1pt}
  \pgfpatheoremoveto{\pgfqpoint{0pt}{0pt}}
  \pgfpathlineto{\pgfqpoint{3.1pt}{3.1pt}}
  \pgfusepath{stroke}
}
\pgfdeclarepatternformonly[\hatchthickness]{my north west lines}
{\pgfqpoint{-1pt}{-1pt}}{\pgfqpoint{4pt}{4pt}}{\pgfqpoint{3pt}{3pt}}{
  \pgfsetcolor{\tikz@pattern@color}
  \pgfsetlinewidth{1.2pt}
  \pgfpatheoremoveto{\pgfqpoint{0pt}{3pt}}
  \pgfpathlineto{\pgfqpoint{3.1pt}{-0.1pt}}
  \pgfusepath{stroke}
}
\makeatother

% -------------------- Color --------------------
\colorlet{grayX}{black!26!white}
\colorlet{darkgrayX}{black!47!white}

% -------------------- Cross-reference shortcuts --------------------
\makeatletter
\AtBeginDocument{\providecommand\corref[1]{\ref{cor:#1}}}
\AtBeginDocument{\providecommand\theoremref[1]{\ref{theorem:#1}}}
\AtBeginDocument{\providecommand\lemref[1]{\ref{lem:#1}}}
\AtBeginDocument{\providecommand\figref[1]{\ref{fig:#1}}}
\makeatother
\providecommand{\tabularnewline}{\\}

% -------------------- Theorem Environments --------------------

% Base counter:
\newtheorem{theorem}{Theorem}
\newtheorem*{theorem*}{Theorem}

% --- Definition ---
\newaliascnt{definition}{theorem}
\newtheorem{definition}[definition]{Definition}
\aliascntresetthe{definition}
\crefname{definition}{definition}{definitions}
\Crefname{definition}{Definition}{Definitions}

% --- Corollary ---
\newaliascnt{cor}{theorem}
\newtheorem{cor}[cor]{Corollary}
\aliascntresetthe{cor}
\crefname{cor}{corollary}{corollaries}
\Crefname{cor}{Corollary}{Corollaries}

% --- Lemma ---
\newaliascnt{lemma}{theorem}
\newtheorem{lemma}[lemma]{Lemma}
\aliascntresetthe{lemma}
\crefname{lemma}{lemma}{lemmas}
\Crefname{lemma}{Lemma}{Lemmas}

% --- Proposition ---
\newaliascnt{prop}{theorem}
\newtheorem{prop}[prop]{Proposition}
\aliascntresetthe{prop}
\crefname{prop}{proposition}{propositions}
\Crefname{prop}{Proposition}{Propositions}

% --- Proposition* (unnumbered) ---
\newtheorem*{prop*}{Proposition}

% --- Observation ---
\newaliascnt{obs}{theorem}
\newtheorem{obs}[obs]{Observation}
\aliascntresetthe{obs}
\crefname{obs}{observation}{observations}
\Crefname{obs}{Observation}{Observations}

% --- Claim ---
\newaliascnt{claim}{theorem}
\newtheorem{claim}[claim]{Claim}
\aliascntresetthe{claim}
\crefname{claim}{claim}{claims}
\Crefname{claim}{Claim}{Claims}

% --- Problem (independent counter, not sharing with theorem) ---
\newtheorem{problem}{Problem}
\crefname{problem}{problem}{problems}
\Crefname{problem}{Problem}{Problems}

% --- Remark-style (unnumbered) ---
\theoremstyle{remark}
\newtheorem*{note*}{Note}
\newtheorem*{rem*}{Remark}
\newtheorem*{notation*}{Notation}
\newtheorem*{context*}{Context}

% -------------------- Example Environment --------------------
\theoremstyle{definition}
\newtheorem{example}[theorem]{\protect\examplename}
\providecommand{\examplename}{Example}

% -------------------- Cleveref labels --------------------
\crefname{equation}{}{}
\crefname{diagram}{Diagram}{Diagrams}

% -------------------- Main and Front Matter --------------------
\makeatletter
\newif\if@mainmatter \@mainmattertrue
\newcommand\frontmatter{%
  \cleardoublepage
  \@mainmatterfalse
  \pagenumbering{Roman}}
\newcommand\mainmatter{%
  \cleardoublepage
  \@mainmattertrue
  \pagenumbering{arabic}}
\makeatother

% -------------------- Fonts and symbols --------------------
\def\wasyfamily{\fontencoding{U}\fontfamily{wasy}\selectfont}
\DeclareTextFontCommand{\textwasy}{\wasyfamily}
\DeclareSymbolFont{wasy}{U}{wasy}{m}{n}
\SetSymbolFont{wasy}{bold}{U}{wasy}{b}{n}
%\def\hexagon    {\mbox{\wasyfamily\char55}}

% -------------------- Standard Math Shortcuts --------------------
\newcommand{\FF}{\mathsf{F}}
\newcommand{\ff}{\mathsf{f}}
\newcommand{\N}{\mathbb{N}}
\newcommand{\bP}{\mathbb{P}}
\newcommand{\bG}{\mathbb{G}}


\newcommand{\Z}{\mathbb{Z}}
\newcommand{\F}{\mathbb{F}}
\newcommand{\A}{\mathbb{A}}
\newcommand{\bO}{\mathbb{O}}

\newcommand{\R}{\mathbb{R}}
\newcommand{\Q}{\mathbb{Q}}
\newcommand{\C}{\mathbb{C}}
\newcommand{\Pro}{\mathbb{P}}

\newcommand{\Qp}{\mathbb{Q}_{p}}
\newcommand{\Fqt}{\mathbb{F}_q(t)}
\newcommand{\Fpt}{\mathbb{F}_p(t)}

\newcommand{\Ff}[2][F]{#1({#2})}
\newcommand{\Ffl}[2][F]{#1(({#2}))}
\DeclarePairedDelimiter\abs{\lvert}{\rvert}%

% -------------------- Mathcal and Mathfrak --------------------
\newcommand{\cF}{\mathcal{F}}
\newcommand{\cE}{\mathcal{E}}
\newcommand{\cG}{\mathcal{G}}
\newcommand{\cC}{\mathcal{C}}
\newcommand{\cL}{\mathcal{L}}
\newcommand{\cX}{\mathcal{X}}
\newcommand{\cO}{\mathcal{O}}
\newcommand{\cP}{\mathcal{P}}
\newcommand{\cQ}{\mathcal{Q}}


\newcommand{\cY}{\mathcal{Y}}
\newcommand{\cZ}{\mathcal{Z}}
\newcommand{\cA}{\mathcal{A}}
\newcommand{\cD}{\mathcal{D}}
\newcommand{\Sh}{\mathcal{\mathop{\mathit{Sh}}}}
\newcommand{\fC}{\mathbf{C}}

\newcommand{\vp}{\varphi}

% -------------------- Math Operators --------------------
\DeclareMathOperator{\lip}{Lip}
\DeclareMathOperator{\pic}{Pic}
\DeclareMathOperator{\Pic}{Pic}

\DeclareMathOperator{\Spl}{Spl}

\DeclareMathOperator{\Frob}{Frob}
\DeclareMathOperator{\Gal}{Gal}
\DeclareMathOperator{\ram}{ram}
\DeclareMathOperator{\Supp}{Supp}
\DeclareMathOperator{\ch}{char}
\DeclareMathOperator{\codim}{codim}
\DeclareMathOperator{\Sym}{Sym}


\DeclareMathOperator{\id}{id}

\newcommand{\PicCm}[1][]{\operatorname{Pic}^{#1}_{C, \mdls}}
\newcommand{\PicC}[1][]{\operatorname{Pic}^{#1}_{C}}

\newcommand{\spec}{\operatorname{spec}}
\newcommand{\Spec}{\operatorname{Spec}}
\newcommand{\Hom}{\operatorname{Hom}}

\newcommand{\Bl}{\operatorname{Bl}}

\newcommand{\Frac}{\operatorname{Frac}}
\newcommand{\Tors}{\operatorname{Tors}}


% -------------------- Ideals and Valuation Notation --------------------
\newcommand{\ind}{\mathbbm{1}}
\newcommand{\norm}[1]{\left\lVert#1\right\rVert}
\newcommand{\numberthis}{\addtocounter{equation}{1}\tag{\theequation}}

\newcommand{\hypa}[1]{\textbf{A}$(#1)$}
\newcommand{\hyp}[1]{\textbf{B}$(#1)$}

\newcommand{\D}{\Delta}
\renewcommand{\arraystretch}{1.3}

\newcommand{\OK}{\mathcal{O}_K}
\newcommand{\OL}{\mathcal{O}_L}
\newcommand{\Oo}{\mathcal{O}}
\newcommand{\Onu}{\mathcal{O}_\nu}

\newcommand{\mnu}{\mathfrak{m}_\nu}
\newcommand{\mdls}{\mathfrak{m}}
\newcommand{\ndls}{\mathfrak{n}}
\newcommand{\cdls}{\mathfrak{c}}
\newcommand{\f}{\mathfrak{f}}
\newcommand{\fdls}{\mathfrak{f}}

\newcommand{\pp}{\mathfrak{p}}
\newcommand{\Pp}{\mathfrak{P}}
\newcommand{\qq}{\mathfrak{q}}
\newcommand{\Qq}{\mathfrak{Q}}

\newcommand{\I}{\mathbb{I}}
\newcommand{\Ak}{\mathbb{A}_K}
\newcommand{\Il}{\mathbb{I}_L}
\newcommand{\Idl}{\mathbb{I}_L}

\newcommand{\Fbox}[2]{\mathcal{\cF}^{(#1)} \boxtimes \mathcal{\cF}^{(#2)}}
\newcommand{\boxP}[2]{{#1} \boxtimes #2}

\NewDocumentCommand{\boxPp}{m m o}{%
  \IfValueTF{#3}%
    {#1 \boxtimes_{#3} #2}% If the 3rd argument exists
    {#1 \boxtimes #2}%      If the 3rd argument is missing
}

\newcommand{\Cmod}[1]{\tilde{C}_{\mdls}^{(#1)}}

% -------------------- TikZ: arrow labels --------------------
\tikzset{
  symbol/.style={
      draw=none,
      every to/.append style={
          edge node={node [sloped, allow upside down, auto=false]{$#1$}}}
    }
}

% -------------------- Function arrows --------------------
\newcommand{\function}[5]{%
  \begin{tikzcd}[
      column sep=2em,
      row sep=1ex,
      ampersand replacement=\&
    ]
    #1\colon \&[-3em]
    #2\vphantom{#3} \arrow[r] \&
    #3\vphantom{#2} \\
    \&
    #4\vphantom{#5}  \arrow[r,mapsto] \&
    #5\vphantom{#4}
  \end{tikzcd}%
}

\newcommand{\isomto}{\stackrel{\sim}{\smash{\longrightarrow}\rule{0pt}{0.4ex}}}
\newcommand{\isoto}{\overset{\sim}{\to}}

% -------------------- Restrictions --------------------
\newcommand\restr[2]{%
\left.\kern-\nulldelimiterspace #1 \littletaller \right|_{#2}%
}
\newcommand{\littletaller}{\mathchoice{\vphantom{\big|}}{}{}{}}

% -------------------- Equation & Diagram Numbering --------------------
\newcounter{diagram}
\renewcommand{\thediagram}{\arabic{diagram}}
\newenvironment{diagram}
{\refstepcounter{diagram}\begin{equation}\tag*{\textnormal{(\thediagram})}}
    {\end{equation}}
\newcommand{\diagramref}[1]{\textnormal{Diagram~\ref{#1}}}

% -------------------- Stacks Tag Shortcut --------------------
\newcommand{\stackstag}[1]{%
  \cite[\href{https://stacks.math.columbia.edu/tag/#1}{Tag~#1}]{stacks-project}%
}
\newcommand{\nlabpage}[2]{%
  \cite[\href{https://ncatlab.org/nlab/revision/#1/#2}{#1, Rev.~#2}]{nlab}%
}


% -------------------- Provide missing names --------------------
\providecommand{\claimname}{Claim}
\providecommand{\corollaryname}{Corollary}
\providecommand{\definitionname}{Definition}
\providecommand{\lemmaname}{Lemma}
\providecommand{\probname}{Problem}
\providecommand{\notationname}{Notation}
\providecommand{\notename}{Note}
\providecommand{\propositionname}{Proposition}
\providecommand{\remarkname}{Remark}
\providecommand{\theoremname}{Theorem}
\providecommand{\observationname}{Observation}
\providecommand{\contextnname}{Context}

% Other Shortcuts
\newcommand{\SheavesSetsEquiv}{
  \[
    \left\{
    \begin{matrix}
      \text{sheaves of sets on } \Spec(L)_{\acute{e}tale}
    \end{matrix}
    \right\}
    \overset{\sim}{\longrightarrow}
    \left\{
    \begin{matrix}
      \text{sets with left continuous action of } G_L
    \end{matrix}
    \right\}
  \]
}

\newcommand{\FiniteLocConstSheavesEquiv}{
  \[\left\{
    \begin{matrix}
      \text{finite locally constant} \\
      \text{sheaves of sets on } \Spec(L)_{\acute{e}tale}
    \end{matrix}
    \right\}
    \overset{\sim}{\longrightarrow}
    \textit{Finite-}G_L\textit{-Sets} \]
}

\newcommand{\limn}[2]{\lim\limits_{#1} #2}

\usepackage{csquotes}
\usepackage[
backend=biber,
style=alphabetic,
sorting=ynt
]{biblatex}
\bibliography{Bib}  % or use \addbibresource{mybibfile.bib} with biblatex
\addbibresource{Bib.bib}

\newboolean{ThesisIntro}
\setboolean{ThesisIntro}{false}

\begin{document}

\title{Geometric Class Field Theory}
\author{Assaf Marzan}
\date{\today}
\maketitle

\tableofcontents

\section{Introduction}
Throughout this work we will be working over fields with characteristic $\neq 0$ unless otherwise stated.

\section{Class Field Theory In the language of Ideals}
In this section we describe the main results of classical class field theory for global fields, following 
\cite{milneCFT}.
We copy most of the content here from Milne. 

\subsection{Ideals, Moduli and Ray Class Groups}
Let $K$ be a global field of $\ch (K) = p$. A modulus $\mdls$ is a formal sum of places of $K$ with non-negative integer 
coefficients. 
Let $S(K, \mdls) = S(\mdls) = \{ v \in \mdls \}$ be the set of places appearing in $\mdls$ with non-zero coefficient.

Define $K_{\mdls, 1} = \{ x \in K^\times \mid v(x - 1) \geq n_v \text{ for all } v \in S(\mdls) \}$ where $n_v$ is the coefficient of $v$ in $\mdls$.

For every set of primes $S$ we define
\[I_K^S = \{ \text{ fractional ideals of } K \text{ generated by primes not in } S \}\]
There is a natural map $i: K_{\mdls, 1} \to I_K^{S(\mdls)}$ sending $x \mapsto (x)$

The qoutient \[C_\mdls = I_K^{S(\mdls)} / i(K_{\mdls, 1})\]
is called the \textbf{(ray) class group} of $K$ modulo $\mdls$.

Let $S$ be a finite set of primes of $K$. And $G$ a finite abelian group.
We shall say that a homomorphism $\psi: I^S \to G$ \textbf{admits a modulus} if there exists a modulus $\mathfrak{m}$ with $S(\mathfrak{m}) \supset S$ 
such that $\psi(i(K_{\mathfrak{m}, 1})) = 0$. 
Thus $\psi$ admits a modulus if and only if it factors through $C_\mathfrak{m}$ for some $\mathfrak{m}$ with 
$S(\mathfrak{m}) \supset S$.

\textcolor{red}{maybe we don't need this}
Milne states and prove a known theorem: 
\begin{theorem}
    For every modulus $\mdls$ of $K$ there is an exact sequence:
    \[
    0 \to \Oo_K^\times/ \Oo_K^\times \cap K_{\mdls_,1} \to K_\mdls / K_{\mdls, 1} \to C_\mdls \to C \to 0
    \]
    Where
    \[
        K_\mdls = \{ x \in K^\times \mid v(x) = 0 \text{ for all } v \in S(\mdls) \} 
    \]
    And $C$ is the usual class group of $K$.
\end{theorem}

\subsection{The Main Theorems}
\begin{theorem}[Artin Reciprocity Law] \label{theorem:ArtinReciprocity}
    Let $L$ be a finite abelian extension of a global field $K$. and let $S$ be the set of primes of $K$
    ramifying in $L$. Then the Artin map \textcolor{red}{add here reference of the definition to milne} .
    $\psi: I^S \to Gal(L/K)$ admits a modulus $\mdls$ with $S(\mdls) = S$ and it defines an isomorphism:
    \[I^S / \left(i(K_{\mdls, 1}) \cdot N_{L/K}(I_L^{S(\mdls)}) \right) \to Gal(L/K)\]
\end{theorem}

A modulus $\mdls$ as in the statement of the theorem is called a definning modulus for $L$. 
Next, we write $I_K^{\mdls}$ for the group of $S(\mdls)$-ideals in $K$, and $I_L^{\mdls}$ for the group of $S(\mdls)'$-ideals in $L$ 
where $S(\mdls)'$ is the set of primes of $L$ lying above primes in $S(\mdls)$.
Call a subgroup $H$ of $I_K^{\mdls}$ a \textbf{congruence subgroup} modulo $\mdls$ if it contains $i(K_{\mdls, 1})$.
    

\begin{theorem}\label{theorem:ExistenceTheorem}
[Existence Theorem of Class Field Theory]
For every congurence subgroup $H$ modulo $\mdls$ there exists a unique finite abelian extension $L/K$, unramified at all primes not in $S(\mdls)$, such that the Artin map induces an isomorphism:
\[I^{S(\mdls)} / H \to Gal(L/K)\].
\end{theorem}

More of the idealic class field theory in Milne. 

Theorems \Cref{theorem:ArtinReciprocity} and \Cref{theorem:ExistenceTheorem} show that there is a canonical group isomorphism:
\begin{equation}\label{eq:idealClassFieldTheory}
 \lim_{\longleftarrow \mdls} C_\mdls \to \operatorname{Gal}(K^{\mathrm{ab}} / K).    
\end{equation}
 

Rather than studying $ \limn{\longleftarrow m}{C_m}$ directly, it turns out to be more natural to introduce another group that has it as a quotient - this is the idele class group.
\textcolor{red}{replace very where idele with ide'le}




\section{Class Field Theory In the language of Adeles and Ideles}
\textcolor{red}{Can we already say we are only considering function fields here?}
The modern formulation of Global Class Field Theory is given in terms of the adele and idele groups of a global field. In this chapter we will define these objects and state the main theorems of Class Field Theory in this language.
\subsection{Adeles and Ideles}
Let $K$ be a global field. For each place $v$ of $K$, we denote:
\begin{enumerate}
    \item $K_v$ = the completion of $K$ at $v$
    \item $\pp_v$ = the corresponding prime ideal in the ring of integers $\Oo_K$ of $K$
    \item $\Oo_v$ = the ring of integers of $K_v$ 
    \item $\hat{\pp}_v$ = the completion of $\pp_v$ = the maximal ideal of $\Oo_v$
\end{enumerate}
 We define the \textbf{adele ring} of $K$ as the restricted direct product
\[\A_K = \prod_v' K_v
\]
where the restriction is taken with respect to the rings of integers $\Oo_v$ of $K_v$ for all \textcolor{red}{non-archimedean (IS IT NECESSEARY TO STATE HERE? WE WORK OVER P ANYWAY)}  places $v$. In other words, an adele is a tuple $(x_v)_v$ with $x_v \in K_v$ such that $x_v \in \Oo_v$ for all but finitely many non-archimedean places $v$.
The \textbf{idele group} of $K$ is defined as the group of units of the adele ring:
\[\I_K = \A_K^\times = \prod_v' K_v^\times
\]
where the restriction is taken with respect to the unit groups $\Oo_v^\times$ of the rings of integers $\Oo_v$ for all non-archimedean places $v$. An idele is thus a tuple $(x_v)_v$ with $x_v \in K_v^\times$ such that $x_v \in \Oo_v^\times$ for all but finitely many non-archimedean places $v$.  

The field $K$ embeds diagonally into $\mathbb{A}_K$, and thus $K^\times$ embeds diagonally into $\mathbb{I}_K$ as the subgroup of principal ideles. The \textbf{idele class group} $\fC_K$ is the quotient:
\[\fC_K = \I_K / K^\times\]

There is a natural isomorphism between certain quotients of the idele group and the ideal group of $K$, which ultimately follows by understanding
ideles as thickening of ideals:
There is a canonical surjective homomophism $\id$:

\begin{align*}
    &\id: \I_K \to I_K \\
    &(x_v)_v \mapsto \prod_v \pp_v^{v(x_v)}
\end{align*}

Thus, composing with $I_K \to C$ gives a surjective homomorphism $\I_K \to C$, noting that $K^\times \to \I_K \to C$ is 0,
we realize $C=I_K/i(K^\times)$ as a quotient of $\fC_K=\I_K/K^\times$.

The same thing is true for $C_\mdls$:
Let $\mdls=\sum n_v \pp_v$ be a modulus of $K$, set: 


\[W_{\mdls}(v) = \begin{cases}
\Oo_v^\times & v \notin \Supp(\mdls)  \\
1 + \hat{\pp_v}^{n_v} & v \in \Supp(\mdls)
\end{cases}
\]

And define 
\[
\I_{\mdls} = \left(\prod_{v \notin \Supp(\mdls)}^{}K_v^\times \times \prod_{v \in \Supp(\mdls)}^{}
 W_{\mdls}(v)  \right) \bigcap \I_K
\]

And \[
\bO_{\mdls}^\times = \prod_v W_{\mdls}(v)
\]
 
Note that:
\[
K_{\mdls,1} = K^\times \cap \prod_{v \in \mdls}^{} W_{\mdls}(v) \qquad \text{Intersection inside } \prod_{v \in \mdls}^{} K_v^\times
\]

and that 
\[
K_{\mdls,1} = K^\times \cap \I_{\mdls} \qquad \text{Intersection inside } \I_K
\]

Milne \textcolor{red}{shows} the following proposition:
\begin{prop} Let $\mdls$ be a modulus of $K$.
\begin{enumerate}
    \item The map $\id: \I_\mdls \to I^{S(\mdls)}_K$ defines an isomorphism
    \[ \I_\mdls / K_{\mdls,1} \bO_\mdls^\times \isomto I^{S(\mdls)}_K / i(K_{\mdls,1}) = C_\mdls \]
    \item The inclusion $\I_\mdls \hookrightarrow \I_K$ defines an isomorphism
    \[ \I_\mdls / K_{\mdls,1} \isomto \I_K / K^\times \]
\end{enumerate}
\end{prop}

\textcolor{red}{Taking the qoutient into character form?}


\subsubsection{Topology on Adeles and Ideles}
We state quickly the topology on the adele ring and the idele group. More can be found in \textcolor{red}{add reference}%\cite{milneCFT}.
Recall that, for all $v$, $K_v$ is locally compact more over, $\Oo_v$ is a compact neighborhood of 0. Similarly 
$K_v^\times$ is locally compact, in fact:
\[
1 + \hat{\pp}_v \supset 1 + \hat{\pp}_v^2 \supset 1 + \hat{\pp}_v^3\ldots
\]
is a fundamental system of neighborhoods of 1 consisting of compact open subgroups of $K_v^\times$.

For every finite set $S$ of places of $K$, define:
\[
\I_S = \prod_{v \in S} K_v^\times \times \prod_{v \notin S} \Oo_v^\times
\]
with the product topology. $\I_S$ is locally compact and as sets we have:
\[\I_K = \bigcup_{S} \I_S
\]
where the union is taken over all finite sets of places of $K$.
We define a topology on $I_K$ by giving a basis for the open sets $\prod_v V_v$ with 
$V_v \subseteq K_v^\times$ open for all $v$ and and $V_v = \Oo_v^\times$ for all but finitely many $v$.
This makes $\I_K$ a locally compact topological group, such that each $\I_S$ is open in $\I_K$, and inherits 
the product topology. The following sets form a fundamental system of neighborhoods of 1: for each 
finite set of primes $S$ and $n > 0$, define 
\[
U_{S,n} = \left\{ (x_v)_v \in \I_K \mid v(x_v - 1) > n \text{ for all } v \in S, x_v \in \Oo_v^\times \text{ for } v \notin S \right\}    
\]

Note that the embedding $K^\times \to \I_K$ is discrete and thus the idele class group $\fC_K = \I_K/K^\times$ is a locally compact topological group as well.
Moreover the canonical injective homomorphism 
\begin{align}
    &K_v^\times \to \I_K \\
    & x \mapsto (1,\ldots,1,x,1,\ldots, 1) \qquad \text{($x$ in the $v$-th position)}
\end{align}
is a topological embedding for each place $v$ of $K$.

\subsubsection{Characters of ideals and of ideles}
in \cite{milneCFT}, Milne proves:
\begin{prop}\label{prop:local-global-characters}
    Let $G$ be a finite abelian group.
    If $\psi: I^S \to G$ admits a modulus, then there exists a unique homomorphism $\phi: \mathbb{I} \to G$ such that
    \begin{enumerate}
        \item $\phi$ is continuous ($G$ with the discrete topology)
        \item $\phi(K^\times) = 1$;
        \item $\phi(\mathbf{a}) = \psi(\text{id}(\mathbf{a}))$, all $\mathbf{a} \in \mathbb{I}^S \stackrel{\text{def}}{=} \{\mathbf{a} \mid a_v = 1 \text{ all } v \in S\}$.
    \end{enumerate}
    Moreover, every continuous homomorphism $\phi: \mathbb{I} \to G$ satisfying (2) arises from a $\psi$.
    More over, $\phi$ and $\psi$ fit in the following chain:
\begin{equation}\label{eq:local-global-characters-diagram}
\begin{tikzcd}
& I^\mathfrak{m} \arrow[r] & C_\mathfrak{m} \arrow[r, "\psi"] & G \\
& \mathbb{I}_\mathfrak{m} / K_{\mathfrak{m},1} \arrow[r] \arrow[d, "\cong"] & \mathbb{I}_\mathfrak{m} / K_{\mathfrak{m},1} \bO^\times_\mdls \arrow[u, "\cong"] & \\
\mathbb{I} \arrow[r] \arrow[uurrr, bend right=50, "\phi"] & \mathbb{I}_K / K^\times & &
\end{tikzcd}
\end{equation}

\end{prop}

\subsubsection{Norms of ideles}
Let $L$ be a finite extension of the number field $K$.

For an idèle $\mathbf{a} = (a_w) \in \mathbb{I}_L$, define $\text{Nm}_{L/K}(\mathbf{a})$ to be the idèle $\mathbf{b} \in \mathbb{I}_K$ with $b_v = \prod_{w|v} \text{Nm}_{L_w/K_v} a_w$. 
Then, one can show that the following diagram commutes:
\[
\begin{tikzcd}
L^\times \arrow[r] \arrow[d, "\text{Nm}_{L/K}"] & \mathbb{I}_L \arrow[r, "\text{id}"] \arrow[d, "\text{Nm}_{L/K}"] & I_L \arrow[d, "\text{Nm}_{L/K}"] \\
K^\times \arrow[r] & \mathbb{I}_K \arrow[r, "\text{id}"] & I_K.
\end{tikzcd}
\]
Thus getting a commutative diagram:
\[
\begin{tikzcd}
\mathbf{C}_L \arrow[r] \arrow[d, "\text{Nm}_{L/K}"] & C_L \arrow[d, "\text{Nm}_{L/K}"] \\
\mathbf{C}_K \arrow[r] & C_K
\end{tikzcd}
\]
(where $C_L, C_K$ are the ideal class groups of $L$ and $K$ respectively).

\subsection{The main theorems}
The theory establishes a fundamental connection between the idele class group $\fC_K$ and the Galois group of the maximal abelian extension of $K$, denoted $K^{ab}$.
\begin{theorem}[Reciprocity Law]\label{theorem:idele-reciprocity-law}
    There exists a unique continious homomorphism $\phi_K: \I_K \to \text{Gal}(K^{ab}/K)$ called the \textbf{Artin map} with the following properties:
    \begin{enumerate}
        \item $\phi_K(K^\times) = 1$;
        \item For every finite abelian extension $L/K$, $\phi_K$ defines an isomorphism:
        \[ 
        \phi_{L/K}: \I_K/(K^\times \cdot \text{Nm}_{L/K}(\I_L)) \isomto \text{Gal}(L/K)
        \]
        or, equivalently, an isomorphism:
        \[
        \fC_K / \text{Nm}_{L/K}(\fC_L) \isomto \text{Gal}(L/K)
        \]
        \item $\phi_{L/K} $ arises from the global $\I_K \to \Gal(L/K)$ coming from the ideal-theoetic global artin map, as in \Cref{prop:local-global-characters}. 
    \end{enumerate}    
\end{theorem}

\begin{theorem}[The Existence Theorem]\label{theorem:idele-existence-theorem}
    There is a one-to-one, inclusion-reversing correspondence between the set of finite abelian extensions of $K$ and the set of open subgroups of finite index in the idele class group $\fC_K$.
    $$
    \begin{Bmatrix}
    \text{Finite abelian} \\
    \text{extensions } L/K
    \end{Bmatrix}
    \quad \longleftrightarrow \quad
    \begin{Bmatrix}
    \text{Open subgroups } H \subseteq \fC_K \\
    \text{of finite index}
    \end{Bmatrix}
    $$
    Under this correspondence, an extension $L$ corresponds to the subgroup $H = N_{L/K}(\fC_L)$.
    
\end{theorem}

\begin{theorem}
    Ideal-Theoetic and Idele-Theoetic formulations of CFT are equivilant through \Cref{prop:local-global-characters}. 
\end{theorem}

\textcolor{red}{
    \begin{enumerate}
        \item Above is finite adelic formulation of CFT, state somethign about the infinite extension CFT theorem
        \item State something about the topology on the idele class group, Say how the fintie implies the infinte by taking inverse limits. 
        \item Find sources for the above. for exampel milne? maybe other?
        \item The restriction is for the non-archimedean, are you sure?
        \item What is the topology of the "continious" homomorphism?
    \end{enumerate}
    }


\section{Class Field Theory In the language of Characters}
The character formulation of Class Field Theory provides a correspondence between characters of the idele class group and characters of the Galois group of the maximal abelian extension of a global field. 


\subsubsection*{The Main Theorems}

\begin{theorem}[Character Formulation of Unramified Global Class Field Theory]
\end{theorem}
\begin{enumerate}
    \item For each character $\xi : K^\times \backslash \mathbb{A}_K^\times / \bO_K^\times \to \bar{\mathbb{Q}}_\ell^\times$ there exists a unique continuous unramified character $\rho : G_K \to \bar{\mathbb{Q}}_\ell^\times$ such that $\rho(\text{Fr}_v) = \xi(\pi_v)$ for all $v$.
    \item For each continuous unramified character $\rho : G_K \to \bar{\mathbb{Q}}_\ell^\times$ there exists a unique character $\xi : K^\times \backslash \mathbb{A}_K^\times / \bO_K^\times \to \bar{\mathbb{Q}}_\ell^\times$ such that $\rho(\text{Fr}_v) = \xi(\pi_v)$ for all $v$.
\end{enumerate}

Where $\bO_K^\times = \bO_{0}$


\begin{theorem}[Character Formulation of Ramified class field theory]
In the above notations:
\begin{enumerate}
    \item For each character $\xi : K^\times \backslash \mathbb{A}_K^\times / \mathcal{O}_{\mdls}^\times \to \mathbb{Q}_\ell^\times$ there exists a unique continuous character $\rho : G_K \to \mathbb{Q}_\ell^\times$ with $\ram(\rho) \subseteq \mdls$ and $\rho(Fr_v) = \xi(\pi_v)$ for all primes $v \notin \Supp(\mdls)$.

    \item For each continuous character $\rho : G_K \to \mathbb{Q}_\ell^\times$ with $\ram(\rho) \subseteq \mdls$ there exists a unique character $\xi : K^\times \backslash \mathbb{A}_K^\times / \mathcal{O}_{\mdls}^\times \to \mathbb{Q}_\ell^\times$ such that $\rho(Fr_v) = \xi(\pi_v)$ for all primes $v \notin \Supp(\mdls)$.
\end{enumerate}
\end{theorem}
Where
The term unramified character, resp. character with ramification bounded by $\mdls$, means that the character is trivial on the correspoding inertia group or higher ramification group of $G_K$ corresponding to the relevant primes. 


\textcolor{red}{See milne, amichai, for more details.}

We want to show how this formulation is equivalent to the adeles formulation given in the previous section.


\textcolor{red}{
    \begin{enumerate}
        \item Is this formulation *equivilant* to adeles language? is it dervied from it?
        \item Give amichai  reference for this formulation
        \item Over what field are we working? what is $l$, what is $p$?
        \item Fix the qoutient of adeles no match the subgroup 
        \item $\mdls$ vs $\mdls$ notation for divisors
    \end{enumerate}
}

\input{sections/IntroductionToGeometricCFT.tex}

\input{sections/ConnectionBetweenCharacterCFTsAndGeometricCFT.tex}

state in term of finite $G$ , show it implies character formulation, etc\dots

Proof of geometric CFT






\medskip

\printbibliography

%\bibliography{Bib}
%\bibliographystyle{plainnat}
%https://ivanfesenko.org/wp-content/uploads/2021/10/rapg.pdf


\end{document}


