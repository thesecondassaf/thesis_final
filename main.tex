\documentclass[11pt]{article}
\usepackage{geometry}\geometry{margin=1in}
\usepackage[T1]{fontenc}
% if using luatex to compile, the following is unnecessary 
%\usepackage[latin9]{inputenc}
\usepackage{ifthen}

%%%% this is incharge of spacing between paragraphs.... fix later
\setlength{\parindent}{1em}
\setlength{\parskip}{0.5em}

\usepackage{titling}
\usepackage{color}
\usepackage[english]{babel}
\usepackage{refstyle}
\usepackage{float}
\usepackage{amsmath}
\usepackage{amsthm}
\usepackage{makecell}
\usepackage{bbm}
\usepackage{amssymb}
\usepackage{stmaryrd}
\usepackage{enumitem}
\usepackage{setspace}
\usepackage{comment}
\usepackage{mathtools}
\usepackage{subcaption}
\usepackage{caption}
\usepackage{tikz}
\usepackage{rotating} %for "turn"
\usepackage{fontawesome} %for the shield icons
\usepackage[misc]{ifsym} %for the fire icons
\usepackage{ifthen}
\usepackage{graphicx}
\usepackage{tikz-3dplot}
\usepackage{xstring}
\usepackage{xcolor}
\usepackage[colorinlistoftodos]{todonotes}
\usepackage{aliascnt}
\usepackage[unicode=true,pdfusetitle,bookmarks=true,bookmarksnumbered=false,bookmarksopen=false,breaklinks=false,pdfborder={0 0 1},backref=false,colorlinks=true,linkcolor=blue]{hyperref}
\usepackage[nameinlink]{cleveref}
% \usepackage{natbib}
\usepackage{verbatim}
\usepackage{tikz-cd}
\usepackage{url}
\usepackage{hyperref}
% \usepackage{emoji}
\usepackage{xparse}
\usepackage[all]{xy}
\usepackage{mathrsfs}
\usepackage{thmtools} 
\usepackage{thm-restate}




\usetikzlibrary{calc}
\pgfmathsetmacro{\scaling}{0.4}

\delimitershortfall-1sp
\usepackage{mleftright}
\mleftright % make \left & \right behave like \mleft & \mright


% -------------------- TikZ Patterns --------------------
\usetikzlibrary{patterns}
\tikzset{
  hatch distance/.store in=\hatchdistance,
  hatch distance=10pt,
  hatch thickness/.store in=\hatchthickness,
  hatch thickness=1pt
}
\makeatletter
\pgfdeclarepatternformonly[\hatchthickness]{my horizontal lines}
{\pgfpointorigin}{\pgfqpoint{100pt}{1pt}}{\pgfqpoint{100pt}{3pt}}{
  \pgfsetcolor{\tikz@pattern@color}
  \pgfsetlinewidth{\hatchthickness}
  \pgfpatheoremoveto{\pgfqpoint{0pt}{0.5pt}}
  \pgfpathlineto{\pgfqpoint{100pt}{0.5pt}}
  \pgfusepath{stroke}
}
\pgfdeclarepatternformonly[\hatchthickness]{my vertical lines}
{\pgfpointorigin}{\pgfqpoint{1pt}{100pt}}{\pgfqpoint{3pt}{100pt}}{
  \pgfsetcolor{\tikz@pattern@color}
  \pgfsetlinewidth{\hatchthickness}
  \pgfpatheoremoveto{\pgfqpoint{0.5pt}{0pt}}
  \pgfpathlineto{\pgfqpoint{0.5pt}{100pt}}
  \pgfusepath{stroke}
}
\pgfdeclarepatternformonly[\hatchthickness]{my north east lines}
{\pgfqpoint{-1pt}{-1pt}}{\pgfqpoint{4pt}{4pt}}{\pgfqpoint{3pt}{3pt}}{
  \pgfsetcolor{\tikz@pattern@color}
  \pgfsetlinewidth{1pt}
  \pgfpatheoremoveto{\pgfqpoint{0pt}{0pt}}
  \pgfpathlineto{\pgfqpoint{3.1pt}{3.1pt}}
  \pgfusepath{stroke}
}
\pgfdeclarepatternformonly[\hatchthickness]{my north west lines}
{\pgfqpoint{-1pt}{-1pt}}{\pgfqpoint{4pt}{4pt}}{\pgfqpoint{3pt}{3pt}}{
  \pgfsetcolor{\tikz@pattern@color}
  \pgfsetlinewidth{1.2pt}
  \pgfpatheoremoveto{\pgfqpoint{0pt}{3pt}}
  \pgfpathlineto{\pgfqpoint{3.1pt}{-0.1pt}}
  \pgfusepath{stroke}
}
\makeatother

% -------------------- Color --------------------
\colorlet{grayX}{black!26!white}
\colorlet{darkgrayX}{black!47!white}

% -------------------- Cross-reference shortcuts --------------------
\makeatletter
\AtBeginDocument{\providecommand\corref[1]{\ref{cor:#1}}}
\AtBeginDocument{\providecommand\theoremref[1]{\ref{theorem:#1}}}
\AtBeginDocument{\providecommand\lemref[1]{\ref{lem:#1}}}
\AtBeginDocument{\providecommand\figref[1]{\ref{fig:#1}}}
\makeatother
\providecommand{\tabularnewline}{\\}

% -------------------- Theorem Environments --------------------

% Base counter:
\newtheorem{theorem}{Theorem}
\newtheorem*{theorem*}{Theorem}

% --- Definition ---
\newaliascnt{definition}{theorem}
\newtheorem{definition}[definition]{Definition}
\aliascntresetthe{definition}
\crefname{definition}{definition}{definitions}
\Crefname{definition}{Definition}{Definitions}

% --- Corollary ---
\newaliascnt{cor}{theorem}
\newtheorem{cor}[cor]{Corollary}
\aliascntresetthe{cor}
\crefname{cor}{corollary}{corollaries}
\Crefname{cor}{Corollary}{Corollaries}

% --- Lemma ---
\newaliascnt{lemma}{theorem}
\newtheorem{lemma}[lemma]{Lemma}
\aliascntresetthe{lemma}
\crefname{lemma}{lemma}{lemmas}
\Crefname{lemma}{Lemma}{Lemmas}

% --- Proposition ---
\newaliascnt{prop}{theorem}
\newtheorem{prop}[prop]{Proposition}
\aliascntresetthe{prop}
\crefname{prop}{proposition}{propositions}
\Crefname{prop}{Proposition}{Propositions}

% --- Proposition* (unnumbered) ---
\newtheorem*{prop*}{Proposition}

% --- Observation ---
\newaliascnt{obs}{theorem}
\newtheorem{obs}[obs]{Observation}
\aliascntresetthe{obs}
\crefname{obs}{observation}{observations}
\Crefname{obs}{Observation}{Observations}

% --- Claim ---
\newaliascnt{claim}{theorem}
\newtheorem{claim}[claim]{Claim}
\aliascntresetthe{claim}
\crefname{claim}{claim}{claims}
\Crefname{claim}{Claim}{Claims}

% --- Problem (independent counter, not sharing with theorem) ---
\newtheorem{problem}{Problem}
\crefname{problem}{problem}{problems}
\Crefname{problem}{Problem}{Problems}

% --- Remark-style (unnumbered) ---
\theoremstyle{remark}
\newtheorem*{note*}{Note}
\newtheorem*{rem*}{Remark}
\newtheorem*{notation*}{Notation}
\newtheorem*{context*}{Context}

% -------------------- Example Environment --------------------
\theoremstyle{definition}
\newtheorem{example}[theorem]{\protect\examplename}
\providecommand{\examplename}{Example}

% -------------------- Cleveref labels --------------------
\crefname{equation}{}{}
\crefname{diagram}{Diagram}{Diagrams}

% -------------------- Main and Front Matter --------------------
\makeatletter
\newif\if@mainmatter \@mainmattertrue
\newcommand\frontmatter{%
  \cleardoublepage
  \@mainmatterfalse
  \pagenumbering{Roman}}
\newcommand\mainmatter{%
  \cleardoublepage
  \@mainmattertrue
  \pagenumbering{arabic}}
\makeatother

% -------------------- Fonts and symbols --------------------
\def\wasyfamily{\fontencoding{U}\fontfamily{wasy}\selectfont}
\DeclareTextFontCommand{\textwasy}{\wasyfamily}
\DeclareSymbolFont{wasy}{U}{wasy}{m}{n}
\SetSymbolFont{wasy}{bold}{U}{wasy}{b}{n}
%\def\hexagon    {\mbox{\wasyfamily\char55}}

% -------------------- Standard Math Shortcuts --------------------
\newcommand{\FF}{\mathsf{F}}
\newcommand{\ff}{\mathsf{f}}
\newcommand{\N}{\mathbb{N}}
\newcommand{\bP}{\mathbb{P}}
\newcommand{\bG}{\mathbb{G}}


\newcommand{\Z}{\mathbb{Z}}
\newcommand{\F}{\mathbb{F}}
\newcommand{\A}{\mathbb{A}}
\newcommand{\bO}{\mathbb{O}}

\newcommand{\R}{\mathbb{R}}
\newcommand{\Q}{\mathbb{Q}}
\newcommand{\C}{\mathbb{C}}
\newcommand{\Pro}{\mathbb{P}}

\newcommand{\Qp}{\mathbb{Q}_{p}}
\newcommand{\Fqt}{\mathbb{F}_q(t)}
\newcommand{\Fpt}{\mathbb{F}_p(t)}

\newcommand{\Ff}[2][F]{#1({#2})}
\newcommand{\Ffl}[2][F]{#1(({#2}))}
\DeclarePairedDelimiter\abs{\lvert}{\rvert}%

% -------------------- Mathcal and Mathfrak --------------------
\newcommand{\cF}{\mathcal{F}}
\newcommand{\cE}{\mathcal{E}}
\newcommand{\cG}{\mathcal{G}}
\newcommand{\cC}{\mathcal{C}}
\newcommand{\cL}{\mathcal{L}}
\newcommand{\cX}{\mathcal{X}}
\newcommand{\cO}{\mathcal{O}}
\newcommand{\cP}{\mathcal{P}}
\newcommand{\cQ}{\mathcal{Q}}


\newcommand{\cY}{\mathcal{Y}}
\newcommand{\cZ}{\mathcal{Z}}
\newcommand{\cA}{\mathcal{A}}
\newcommand{\cD}{\mathcal{D}}
\newcommand{\Sh}{\mathcal{\mathop{\mathit{Sh}}}}
\newcommand{\fC}{\mathbf{C}}

\newcommand{\vp}{\varphi}

% -------------------- Math Operators --------------------
\DeclareMathOperator{\lip}{Lip}
\DeclareMathOperator{\pic}{Pic}
\DeclareMathOperator{\Pic}{Pic}

\DeclareMathOperator{\Spl}{Spl}

\DeclareMathOperator{\Frob}{Frob}
\DeclareMathOperator{\Gal}{Gal}
\DeclareMathOperator{\ram}{ram}
\DeclareMathOperator{\Supp}{Supp}
\DeclareMathOperator{\ch}{char}
\DeclareMathOperator{\codim}{codim}
\DeclareMathOperator{\Sym}{Sym}


\DeclareMathOperator{\id}{id}

\newcommand{\PicCm}[1][]{\operatorname{Pic}^{#1}_{C, \mdls}}
\newcommand{\PicC}[1][]{\operatorname{Pic}^{#1}_{C}}

\newcommand{\spec}{\operatorname{spec}}
\newcommand{\Spec}{\operatorname{Spec}}
\newcommand{\Hom}{\operatorname{Hom}}

\newcommand{\Bl}{\operatorname{Bl}}

\newcommand{\Frac}{\operatorname{Frac}}
\newcommand{\Tors}{\operatorname{Tors}}


% -------------------- Ideals and Valuation Notation --------------------
\newcommand{\ind}{\mathbbm{1}}
\newcommand{\norm}[1]{\left\lVert#1\right\rVert}
\newcommand{\numberthis}{\addtocounter{equation}{1}\tag{\theequation}}

\newcommand{\hypa}[1]{\textbf{A}$(#1)$}
\newcommand{\hyp}[1]{\textbf{B}$(#1)$}

\newcommand{\D}{\Delta}
\renewcommand{\arraystretch}{1.3}

\newcommand{\OK}{\mathcal{O}_K}
\newcommand{\OL}{\mathcal{O}_L}
\newcommand{\Oo}{\mathcal{O}}
\newcommand{\Onu}{\mathcal{O}_\nu}

\newcommand{\mnu}{\mathfrak{m}_\nu}
\newcommand{\mdls}{\mathfrak{m}}
\newcommand{\ndls}{\mathfrak{n}}
\newcommand{\cdls}{\mathfrak{c}}
\newcommand{\f}{\mathfrak{f}}
\newcommand{\fdls}{\mathfrak{f}}

\newcommand{\pp}{\mathfrak{p}}
\newcommand{\Pp}{\mathfrak{P}}
\newcommand{\qq}{\mathfrak{q}}
\newcommand{\Qq}{\mathfrak{Q}}

\newcommand{\I}{\mathbb{I}}
\newcommand{\Ak}{\mathbb{A}_K}
\newcommand{\Il}{\mathbb{I}_L}
\newcommand{\Idl}{\mathbb{I}_L}

\newcommand{\Fbox}[2]{\mathcal{\cF}^{(#1)} \boxtimes \mathcal{\cF}^{(#2)}}
\newcommand{\boxP}[2]{{#1} \boxtimes #2}

\NewDocumentCommand{\boxPp}{m m o}{%
  \IfValueTF{#3}%
    {#1 \boxtimes_{#3} #2}% If the 3rd argument exists
    {#1 \boxtimes #2}%      If the 3rd argument is missing
}

\newcommand{\Cmod}[1]{\tilde{C}_{\mdls}^{(#1)}}

% -------------------- TikZ: arrow labels --------------------
\tikzset{
  symbol/.style={
      draw=none,
      every to/.append style={
          edge node={node [sloped, allow upside down, auto=false]{$#1$}}}
    }
}

% -------------------- Function arrows --------------------
\newcommand{\function}[5]{%
  \begin{tikzcd}[
      column sep=2em,
      row sep=1ex,
      ampersand replacement=\&
    ]
    #1\colon \&[-3em]
    #2\vphantom{#3} \arrow[r] \&
    #3\vphantom{#2} \\
    \&
    #4\vphantom{#5}  \arrow[r,mapsto] \&
    #5\vphantom{#4}
  \end{tikzcd}%
}

\newcommand{\isomto}{\stackrel{\sim}{\smash{\longrightarrow}\rule{0pt}{0.4ex}}}
\newcommand{\isoto}{\overset{\sim}{\to}}

% -------------------- Restrictions --------------------
\newcommand\restr[2]{%
\left.\kern-\nulldelimiterspace #1 \littletaller \right|_{#2}%
}
\newcommand{\littletaller}{\mathchoice{\vphantom{\big|}}{}{}{}}

% -------------------- Equation & Diagram Numbering --------------------
\newcounter{diagram}
\renewcommand{\thediagram}{\arabic{diagram}}
\newenvironment{diagram}
{\refstepcounter{diagram}\begin{equation}\tag*{\textnormal{(\thediagram})}}
    {\end{equation}}
\newcommand{\diagramref}[1]{\textnormal{Diagram~\ref{#1}}}

% -------------------- Stacks Tag Shortcut --------------------
\newcommand{\stackstag}[1]{%
  \cite[\href{https://stacks.math.columbia.edu/tag/#1}{Tag~#1}]{stacks-project}%
}
\newcommand{\nlabpage}[2]{%
  \cite[\href{https://ncatlab.org/nlab/revision/#1/#2}{#1, Rev.~#2}]{nlab}%
}


% -------------------- Provide missing names --------------------
\providecommand{\claimname}{Claim}
\providecommand{\corollaryname}{Corollary}
\providecommand{\definitionname}{Definition}
\providecommand{\lemmaname}{Lemma}
\providecommand{\probname}{Problem}
\providecommand{\notationname}{Notation}
\providecommand{\notename}{Note}
\providecommand{\propositionname}{Proposition}
\providecommand{\remarkname}{Remark}
\providecommand{\theoremname}{Theorem}
\providecommand{\observationname}{Observation}
\providecommand{\contextnname}{Context}

% Other Shortcuts
\newcommand{\SheavesSetsEquiv}{
  \[
    \left\{
    \begin{matrix}
      \text{sheaves of sets on } \Spec(L)_{\acute{e}tale}
    \end{matrix}
    \right\}
    \overset{\sim}{\longrightarrow}
    \left\{
    \begin{matrix}
      \text{sets with left continuous action of } G_L
    \end{matrix}
    \right\}
  \]
}

\newcommand{\FiniteLocConstSheavesEquiv}{
  \[\left\{
    \begin{matrix}
      \text{finite locally constant} \\
      \text{sheaves of sets on } \Spec(L)_{\acute{e}tale}
    \end{matrix}
    \right\}
    \overset{\sim}{\longrightarrow}
    \textit{Finite-}G_L\textit{-Sets} \]
}

\newcommand{\limn}[2]{\lim\limits_{#1} #2}

\usepackage{csquotes}
\usepackage[
backend=biber,
style=alphabetic,
sorting=ynt
]{biblatex}
\bibliography{Bib} 
\addbibresource{Bib.bib}

\newboolean{ThesisIntro}
\setboolean{ThesisIntro}{false}

\begin{document}

\title{Geometric Class Field Theory}
\author{Assaf Marzan}
\date{\today}
\maketitle
\begin{abstract}
We study the ramification of the symmetric product $\cF^{(\deg \mdls)}$ of a local system $\cF$ on a curve $C \setminus \mdls$. 
Assuming the ramification of $\cF$ is bounded by $\mdls$, we prove that the symmetric product $\cF^{(\deg \mdls)}$ is at most tamely ramified at the generic point of 
the exceptional divisor $E_\mdls$ of the blowup of $C^{(\deg \mdls)}$ at $\mdls$. As a primary application, 
we utilize this result to prove Geometric Class Field Theory. 
Our approach builds upon the geometric framework for the unramified case originally established by Deligne for the rank-one Langlands correspondence, 
following the subsequent extensions to the ramified case developed by Takeuchi and Guignard.
\end{abstract}

\newpage
\tableofcontents
\newpage
\section{Introduction}

In this thesis, we give an elementary proof of a certain imporant geometric theorem occuring in Deligne's approach to geometric class field theory.
One of the main geometric ingredients in the approach, is showing why a local system $\cF$ with ramification bounded by a modulus $\mdls$ on $U = C \setminus \mdls$ descends via the Abel-Jacobi $\Phi: U \to \PicCm[]$ to $\PicCm[]$. 
The approach, innovated by Deligne, relies on analyzing the symmetric powers $\cF^{(d)}$ of $\cF$ on the symmetric powers $U^{(d)}$ of $U$,
and showing that for sufficiently large $d$, $\cF^{(d)}$ descends to $\PicCm[d]$ via the degree $d$ Abel-Jacobi map $\Phi_d : U^{(d)} \to \PicCm[d]$.
The fibers of $\Phi_d$ (for $d \geq 2g - 1$) over any point are isomorphic to 
$$
\begin{cases}
\mathbb{A}_k^{d - \deg \mdls - g + 1} & \text{if } \mdls > 0 \\
\mathbb{P}_k^{d - g} & \text{if } \mdls = 0
\end{cases}
$$
Where $g$ is the genus of the curve $C$.
The unramified case ($\mdls = 0$) is relatively simple, as the Abel-Jacobi map is proper and smooth, which follows from the fact that it is a fibration in projective spaces.
Thus, by using the homotopy exact sequence for the etale fundamental group, one gets an isomorphism between the etale fundamental group of $U^{(d)} (=C^{(d)})$ and that of $\PicCm[d] (=\PicC[d])$.

The ramified case ($\mdls > 0$) is more subtle, as the Abel-Jacobi map is not proper anymore, and one needs to analyze the ramification of $\cF^{(d)}$ "along the boundary" of $U^{(d)}$ in $C^{(d)}$.

Previous work has generalized Deligne's approach to the ramified case, 
most notably by Guignard \cite{Guignard2018} and Takeuchi \cite{takeuchi2019blow}. 
Their approaches differ. 
To descend, Guignard proves that the restriction of $\cF^{(d)}$ to any line in the fiber of the degree $d$ Abel-Jacobi map is a constant 
étale sheaf. He achieves this by demonstrating that the restriction is at most tamely ramified and invoking the triviality of the tame fundamental group 
of $\mathbb{A}^1_k$. His analysis relies on local geometric class field theory. 
It is also worth noting that Guignard's method generalizes to relative curves over arbitrary base schemes.
Takeuchi, on the other hand, constructs a compactification of $U^{(d)}$ by blowing up $C^{(d)}$ along certain well-chosen centers.
This compactification, denoted by $\tilde{C}^{(d)}_\mdls$, has $U^{(d)}$ as an open subscheme with a codimension 1 closed subscheme $H$ as complement. 
He then shows that the Abel-Jacobi map extends to a proper morphism from $\tilde{C}^{(d)}_\mdls$ to $\PicCm[d]$,
which is a fibration in projective spaces. Thus, by the homotopy exact sequence for the etale fundamental group, one gets an isomorphism between the etale fundamental group of
$\tilde{C}^{(d)}_\mdls$ and that of $\PicCm[d]$. 
To conclude the descent, Takeuchi analyzes the ramification of $\cF^{(d)}$ along the boundary $H$ of $\tilde{C}^{(d)}_\mdls$,
showing that it is tamely ramified there, which suffices. His methods relies on the theory of Witt vectors and refined Swan conductors.

For an account of these approaches, see \cite{Guignard2018} and \cite{takeuchi2019blow}.
For a full approach following Deligne's method in the unramified case, and the tamely ramified case see \cite{tendler2015geometricclassfieldtheory}, and \cite{Toth2011}.

In this thesis, we combine techniques and ideas from the approaches, and from \cite{tendler2015geometricclassfieldtheory}, to give an elementary proof of the ramified case of Deligne's approach to geometric class field theory.
We follow Takeuchi's construction of the compactification $\tilde{C}^{(d)}_\mdls$ of $U^{(d)}$ by blowing up $C^{(d)}$ and calculate the ramification of $\cF^{(d)}$ along the boundary $H$ of $\tilde{C}^{(d)}_\mdls$ directly, avoiding 
the use of Swan conductors. 

In the rest of the introduction, we state the main theorem of geometric class field theory \cref{thm:GCFT}, and its reduction to \cref{thm:GCFT_reduced}, which we prove in this thesis.

Let $k$ be a perfect field, and let $C$ be a projective smooth geometrically connected curve over $k$, with genus $g$. 
Geometric class field theory gives a geometric description of abelian coverings of $C$ by relating it to isogenies of the generalized picard schemes.

Fix a modulus $\mdls$, i.e.\ an effective Cartier divisor of $C$ and let $U$ be its complement in $C$. 
The pairs $(\cL, \alpha)$, where $\cL$ is an invertible $\cO_C$-module and $\alpha$ is a rigidification of $\cL$ along $\mdls$, are parametrized by a $k$-group scheme $\PicCm$, called the rigidified Picard scheme.
The Abel-Jacobi morphism $$ \Phi : U \to \PicCm $$
is the morphism which sends a section $x$ of $U$ to the pair $(\cO(x), 1)$ . 
We prove the following version of geometric class field theory: 
\begin{theorem}[Geometric Class Field Theory]\label{thm:GCFT}
    Let $\Lambda$ be a finite ring of cardinality invertible in $k$, and let $\cF$ be an \'etale sheaf of $\Lambda$-modules, locally free of rank 1 on $U$, with ramification bounded by $\mdls$. 
    Then, there exists a unique (up to isomorphism) \textcolor{purple}{\underline{\textit{multiplicative}}} \'etale sheaf of $\Lambda$-modules $\cG$ on $\PicCm$, locally free of rank 1, such that the pullback of $\cG$ by $\Phi$ is isomorphic to $\cF$.
\end{theorem}

\textcolor{purple}{The notion of a multiplicative locally free $\Lambda$-module of rank 1 is due to \cite{Guignard2018} 
and corresponds to isogenies $G \to \PicCm$ with constant kernel $\Lambda^\times$. 
This concept corresponds to multiplicative characters of $H^1(\PicCm[], \Q/\Z)$ in the formulation of \cite{takeuchi2019blow}, 
and generalizes Hecke eigensheaves in the context of \cite{tendler2015geometricclassfieldtheory}.}

Let $d$ be a positive integer. 
We denote by $U^{(d)}$ the $d$-th symmetric power of $U$ over $k$. 
For an \'etale sheaf $\cF$ on $U$, we denote by $\cF^{(d)}$ the $d$-th symmetric power of $\cF$ on $U^{(d)}$.
The degree $d$ Abel-Jacobi morphism is defined as the map
$$ \Phi_d : U^{(d)} \to \PicCm[d] $$
which sends a section $x_1 + \cdots + x_d$ of $U^{(d)}$ to the pair $(\cO(x_1 + \cdots + x_d), 1)$ .

The method of descent shows that to prove \cref{thm:GCFT}, 
it suffices to prove the following reduced version (see the last page of \cite{Guignard2018}, 
Section 8.3 of \cite{tendler2015geometricclassfieldtheory}, or the proof of Theorem 1.2 in \cite{takeuchi2019blow} for details on this reduction):

\begin{theorem}\label{thm:GCFT_reduced}
    Let $\Lambda$ be a finite ring of cardinality invertible in $k$, and let $\cF$ be an \'etale sheaf of $\Lambda$-modules, locally free of rank 1 on $U$, with ramification bounded by $\mdls$. 
    Then, for sufficiently large integer $d$, there exists a unique (up to isomorphism) \'etale sheaf of $\Lambda$-modules $\cG_d$ on $\PicCm[d]$, locally free of rank 1, such that the pullback of $\cG_d$ by $\Phi_d$ is isomorphic to $\cF^{(d)}$.
\end{theorem}

Using the equivilance between $G$-torsors and locally free $\Lambda$-modules of rank 1 ($G=\Lambda^\times$, 
see \cref{prop:torsor_module_equivalence}), Theorem \ref{thm:GCFT_reduced} can be reformulated in terms of $G$-torsors as follows:
\begin{theorem}\label{thm:GCFT_torsors}
    Let $G=\Lambda^\times$ be a finite abelian group ($\Lambda$ as before), and let $\cP$ be a $G$-torsor on $U$, with ramification bounded by $\mdls$. 
    Then, for sufficiently large integer $d$, there exists a unique (up to isomorphism) $G$-torsor $\cQ_d$ on $\PicCm[d]$, such that the pullback of $\cQ_d$ by $\Phi_d$ is isomorphic to $\cP^{(d)}$.
\end{theorem}

To prove Theorem \ref{thm:GCFT_torsors} we follow the work of \cite{takeuchi2019blow}, there he analyzed 
the ramification of $\cP^{(d)}$ after blowing up $C^{(d)}$, we analyze this ramification using elementary methods, drawing techniques and ideas
from the works of \cite{Guignard2018} and \cite{takeuchi2019blow}, and \cite{tendler2015geometricclassfieldtheory}.


\textbf{Notation and conventions}. 


\textcolor{red}{
 \begin{enumerate}
    \item A word about the decompositon of the Picard scheme into connected components indexed by degree.
    \item Definition of the abel jacobi-map, and its properties, its fibers, see how amichai does it. 
    \item Definition of multiplicative sheaves?? and maybe its not needed here, or we are not exactly right here...
    \item Say something about the ramification condition.
 \end{enumerate}
}




\section{Preliminaries}
This section establishes the foundational definitions and theorems necessary for the remainder of this work. 
We focus specifically on the theory of $G$-torsors, which are central to our study due to their correspondence with 
locally free sheaves of rank 1. 
Subsequently, we review the relevant background on ramification theory from the existing literature. 
Finally, we conclude with several algebraic geometric remarks and notational conventions that will be employed 
implicitly throughout this thesis.

%%%% %%%% %%%% %%%% %%%% %%%% %%%% %%%%
%               TORSORS
%%%% %%%% %%%% %%%% %%%% %%%% %%%% %%%%
\subsection{Torsors}
% $X$ with etale site, but say things here work for toposes in general and flat sites in general.

In what follows, we largely adhere to the treatment of torsors and group objects found in publised notes of Alex Youcis \cite{YoucisTorsors}
Let $\sC=(\cC, J)$ be a site and let $\cE=Sh(\sC)$ be the associated topos. 
Let $\mathcal{G}$ be a group object in $\mathcal{E}$. 
We denote by $\mathcal{G}\mathcal{E}$ the category of objects in $\mathcal{E}$ endowed with a left $\mathcal{G}$-action. 
For any object $\cX \in \mathcal{E}$, there is a canonical identification between the slice category 
$(\mathcal{G}\mathcal{E})/\cX$ and the category of group objects $\mathcal{G}(\mathcal{E}/\cX)$, where $\cX$ is viewed as having the 
trivial $\mathcal{G}$-action.

\begin{definition}
    A \textbf{$\mathcal{G}$-torsor} in $\mathcal{E}$ is an object $\mathcal{P}$ of $\mathcal{G}\mathcal{E}$ satisfying the following conditions:
    \begin{enumerate}
        \item The structural morphism $\mathcal{P} \to 1$ is an epimorphism in $\mathcal{E}$ (i.e., $\mathcal{P}$ is locally non-empty).
        \item The map $\mathcal{G} \times \mathcal{P} \to \mathcal{P} \times \mathcal{P}$ defined by $(g, p) \mapsto (g \cdot p, p)$ is an isomorphism in $\mathcal{E}$ (i.e., $\mathcal{G}$ acts simply transitively on $\mathcal{P}$).
    \end{enumerate}
\end{definition}

Since $\mathcal{E}$ is the topos of sheaves on a site $\sC$, the definition can be reformulated in terms of covers. A $\mathcal{G}$-sheaf $\mathcal{P}$ is a $\mathcal{G}$-torsor if:
\begin{enumerate}
    \item For every object $X \in \mathcal{C}$, there exists a covering $\{ U_i \to X \} \in J$ such that $\mathcal{P}(U_i) \neq \emptyset$ for all $i$.
    \item For any $X \in \mathcal{C}$ where $\mathcal{P}(X)$ is non-empty, the action of $\mathcal{G}(X)$ on $\mathcal{P}(X)$ is simply transitively.
\end{enumerate}

A fundamental property of torsors is their local triviality: 
a $\mathcal{G}$-sheaf $\mathcal{P}$ is a $\mathcal{G}$-torsor if and only if it is locally isomorphic to the trivial torsor. 
Specifically, for every $X \in \mathcal{C}$, there must exist a cover $\{ U_i \to X \}$ such that the restriction $\mathcal{P}|_{U_i}$ is 
isomorphic, as a $\mathcal{G}|_{U_i}$-sheaf, to $\mathcal{G}|_{U_i}$ acting on itself by left multiplication.

A \textbf{morphism of $\mathcal{G}$-torsors} $f: \mathcal{P}_1 \to \mathcal{P}_2$ is a morphism of sheaves that is equivariant with respect to the $\mathcal{G}$-action.

It is a standard result that every morphism of $\mathcal{G}$-torsors is an isomorphism. 
Consequently, the category of $\mathcal{G}$-torsors in $\mathcal{E}$ forms a groupoid.

\begin{definition}
    We denote the groupoid of $\mathcal{G}$-torsors in $\mathcal{E}$ by $\mathbf{Tors}(\mathcal{E}, \mathcal{G})$ .  The set of isomorphism classes of $\mathcal{G}$-torsors is denoted by $\mathrm{Tors}(\mathcal{E}, \mathcal{G})$.
\end{definition}

Torsors exhibit functoriality with respect to the group object:

\begin{definition}
Let $\varphi: \mathcal{G}_1 \to \mathcal{G}_2$ be a morphism of group 
sheaves on $\sC$, and let $\mathcal{P}$ be a $\mathcal{G}_1$-torsor. We define the \textbf{contracted product} $\mathcal{G}_2 \times^{\mathcal{G}_1} \mathcal{P}$ as the quotient sheaf $(\mathcal{G}_2 \times \mathcal{P}) / \mathcal{G}_1$, where $\mathcal{G}_1$ acts on the product by:
\[ g_1 \cdot (g_2, p) = (g_2 \varphi(g_1)^{-1}, g_1 \cdot p) \]

The contracted product inherits a natural left $\mathcal{G}_2$-action given on local sections by $h \cdot [g_2, p] = [h g_2, p]$, which endows it with the structure of a $\mathcal{G}_2$-torsor. 
\end{definition}

This construction yields a functor:
\[ \varphi_*: \mathbf{Tors}(\cE, \mathcal{G}_1) \to \mathbf{Tors}(\cE, \mathcal{G}_2), \quad \mathcal{P} \mapsto \mathcal{G}_2 \times^{\mathcal{G}_1} \mathcal{P} \]
On the level of isomorphism classes, $\varphi_*$ induces a map of pointed sets $\mathrm{Tors}(\cE, \mathcal{G}_1) \to \mathrm{Tors}(\cE, \mathcal{G}_2)$, sending the class of the trivial $\mathcal{G}_1$-torsor to the class of the trivial $\mathcal{G}_2$-torsor.

When $\mathcal{G}$ is a \textbf{sheaf of abelian groups} (an \textit{abelian sheaf}), the pointed set $\mathrm{Tors}(\cE, \mathcal{G})$ 
inherits the structure of an abelian group. Let $\mathcal{P}_1$ and $\mathcal{P}_2$ be objects of $\mathbf{Tors}(\cE, \mathcal{G})$. We define their sum $[\mathcal{P}_1] + [\mathcal{P}_2]$ to be the class $[\mathcal{P}_3]$, where $\mathcal{P}_3$ is the quotient sheaf $(\mathcal{P}_1 \times \mathcal{P}_2)/\mathcal{G}$. In this construction, $\mathcal{G}$ acts on the product $\mathcal{P}_1 \times \mathcal{P}_2$ on $T$-points by:
\[ g \cdot (f_1, f_2) := (g f_1, g^{-1} f_2) \]

The group object $\mathcal{G}$ then acts on the resulting quotient via its action on the presheaf quotient, which is given on classes by:
\[ g \cdot [(f_1, f_2)] = [(g f_1, f_2)] = [(f_1, g f_2)] \]
where the square brackets denote the class in the quotient set. This structure turns $\mathrm{Tors}(\mathcal{G})$ into an abelian group, 
where the identity is the class of the trivial torsor and the inverse is obtained by the opposite action.\footnote{
    Equivantly, the sum is obtained as the contracted product of the $\cG \times \cG$-torsor $\cP_1 \times \cP_2$ along the multiplication map
$m: \cG \times \cG \to \cG$. }




%%% %%% %%% %%% %%% %%% %%% %%%
%       Torsors as Spaces
%%% %%% %%% %%% %%% %%% %%% %%%
\subsubsection*{Torsors as Flat Spaces}
\textcolor{red}{change Fl to be straight}
We now fix a scheme $X$. We start by focusing on the big fppf site. $X_{Fl}$.
Let $\cG$ be a group sheaf on $X_{Fl}$. A flat-torsor $\cG$-torsor on $X$ is just a $\cG$-torsor on the site $X_{Fl}$.

To the end of this section assume \textbf{$G$ is a flat affine-algebraic $X$-group. } (\textcolor{red}{which by descent is associated also to a group sheaf})
\begin{definition}
Define a \textit{principal $G$-bundle} (or \textit{principal homogenous space} for $G$) to be 
a flat finite presentation $X$-scheme $f : Y \to X$ with an action of $G$ satisfying the following equivalent properties:
\begin{enumerate}
    \item The morphism $Y \times_X G \to Y \times_X Y$ defined on $T$-points by sending $(y, g) \in Y(T) \times G(T)$ to $(y, gy)$ is an isomorphism of $X$-schemes.
    \item There exists an open covering $\{U_i \to X\}$ in $X_{\text{fl}}$ such that $Y_{U_i}$ is isomorphic, as a $G_{U_i}$-space, to $G_{U_i}$ with its left multiplication action.
\end{enumerate}
\end{definition}

Similarly to the case of $G$-torsors of a topos, we define a morphism of principal $G$-bundles to be a morphism of $X$-schemes commuting with the
$G$-action. We now have:
\begin{theorem} 
    The morphism sending $Y \mapsto \mathrm{Hom}_X(-, Y)$ is an equivalence of categories from the category of principal $G$-bundles 
    to the category of $G$-torsors on $X_{\mathrm{Fl}}$. 
    Similarly, the morphism sending $Y$ to $\mathrm{Hom}_X(-, Y)$ to the category of $G$-torsors on $X_{\mathrm{fl}}$ is an equivalence.
\end{theorem}

\begin{cor}
    There is a natural equivalence $\mathbf{Tors}(X_{\mathrm{fl}}, G) \cong \mathbf{Tors}(X_{\mathrm{Fl}}, G)$ inducing a bijection of pointed sets $\mathrm{Tors}(X_{\mathrm{fl}}, G) \xrightarrow{\cong} \mathrm{Tors}(X_{\mathrm{Fl}}, G)$ which is an isomorphism of abelian groups if $G$ is abelian.
    And every $G$-torsor, in each of the above sites can be realized as a scheme, which is a prinicipal $G$-bundle.
\end{cor}


%%% %%% %%% %%% %%% %%% %%% %%%
%       Torsors as Etale Spaces
%%% %%% %%% %%% %%% %%% %%% %%%
\subsubsection*{Torsors as Etale Spaces}
Let $X$ be a scheme, and let $G$ be affine algebraic $X$-group. 
For any topology $\mathcal{T}$ on $\text{Sch}/X$ coarser than the flat topology, we say that 
a flat torsor $\mathcal{P}$ for $G$ is \textit{locally trivial for the $\mathcal{T}$ topology} 
if, in fact, one can find a covering $\{U_i \to X\}$ in $\mathcal{T}$ such that $\mathcal{P}(U_i)$ 
or, equivalently, $\mathcal{P}_{U_i}$ is isomorphic to the trivial torsor for all $i$.

Then, it is immediate that $\Tors(X_{Et, G})$ is canonically isomorphic as pointed sets (abelian groups if $G$ is abelian)
to the subset of $\Tors(X_{Fl}, G)$ consisting of that flat torsors locally trivial for the etale topology.

For the small sites we have:
\begin{theorem}
    If $G$ is smooth affine $X$ group, then any $G$-torsor $\cP$ on $X_fl$ is locally trivial for the etale topology.
\end{theorem}

\begin{cor}\label{cor:tors_etale_flat_eq}
    If $G$ is smooth affine $X$ group, then there is a canonical bijection of pointed sets (abelian groups if $G$ is abelian)
    $\Tors(X_{et}, G) \cong \Tors(X_{fl}, G)$
\end{cor}

And we also have:
\begin{theorem}
    If $G$ is a smooth algebraic $X$-groupm there every flat $G$-torsor is locally trivial for the etale topology and every principal 
    $G$-bundle $Y \to X$ is smooth. In other words the inclusion $\bTors(X_{Et}, G) \to \bTors(X_{Fl}, G)$ is actuall an isomorphism. ??(of what? of categories?)
\end{theorem}

\textcolor{red}{see which of the above theorems we leave intact, cuase it doesn't seem like we need all three}

We now focus on the etale-topology which is coarser then the flat topology. 
Two theorems summarize what happends over the big and small sites:


%%% %%% %%% %%% %%% %%% %%% %%%
%       Torsors and Cohomology
%%% %%% %%% %%% %%% %%% %%% %%%

\subsubsection*{Torsors and Cohomology}
We qoute without proof:
\begin{theorem}
There is a natural bijection of pointed sets     
$\text{Tors}(\mathcal{G|_T}) \to \check{H}^1(T, \mathcal{G})$. Moreover, if $\mathcal{G}$ is an abelian group sheaf, it's an isomorphism of abelian groups.
\end{theorem}

The idea is that for every $[\mathcal{P}] \in \text{Tors}(T, \mathcal{G})$ 
We 
\begin{enumerate}
    \item Choose a covering $\{U_i \to T\}$ such that $\mathcal{P}(U_i) \neq \emptyset$ for all $i$
    \item choose sections $\alpha_i \in \mathcal{F}(U_i)$
\end{enumerate}

Then we note that for all $(i,j)$ the elements $\alpha_i |_{U_i \times_X U_j}$ and $\alpha_j |_{U_i \times_X U_j}$ differ by a unique element of $\mathcal{G}(U_{ij})$
there exists a unique $s_{ij} \in \mathcal{G}(U_{ij})$ such that $\alpha_i |_{U_i \times_X U_j} = s_{ij}(\alpha_j |_{U_i \times_X U_j})$. 
One can then easily see that $(s_{ij})$ defines an element of  $\check{H}^1(T, \mathcal{G})$
which is independent of the choice of repreantiative $\cP$, choice of covering $\{U_i \to T\}$ and choice of sections.

%%% %%% %%% %%% %%% %%% %%% %%%
%       Constant Finite Group Torsors
%%% %%% %%% %%% %%% %%% %%% %%%

\subsubsection*{Constant Finite Group Torsors}
Let $G$ be a finite group. We denote by $\underline{G}$ the constant group scheme $\underline{G}$ over $X$. Sometimes denoted by $\underline{G}_X$
and is given by $\coprod_{g \in G} X$ with the action shuffeling the $X$'s according to multiplication.

By following the definitions, one sees that if $X$ be a connected scheme and $f : Y \to X$ is a finite Galois cover with Galois group $G$. 
Then, $f : Y \to X$ is a principal $\underline{G}$-bundle.
(Recall that a \textit{finite Galois cover} is a finite étale surjection $Y \to X$ with $Y$ connected and such that 
$G = \text{Aut}(Y/X)$ acts transitively on the geometric points of $Y$ lying over any geometric point of $X$.)

On the otherhand if $f: Y \to X$ is a principal $\underline{G}$-bundle with $Y$ connected, then $Y$ is a finite Galois cover with
automorphism group $G$.

However, not all $G$-torsors are connected. If $H\subset G$ is a proper subgroup then any connected finite etale cover $f: Y \to X$ with 
Galois group $H$ gives rise to a non-connected $\underline{G}$-torsor by looking at the induced $G$-torsor $\varphi_*(Y)$ under the inclusion $\varphi: \underline{H} \to \underline{G}$.
On the otherhand, if we fix a geometric point $\overline{x} \to X$, then to give an homomorphism
$\rho \in \mathrm{Hom}_{\mathrm{cont}}(\pi_1^{\mathrm{et}}(X, x), G)$ is equivilant to give a connected pointed Galois cover
$(Y, \overline{y}) \to (X, \overline{x})$ with Galois group $H=\rho(\pi_1^{\mathrm{et}}(X, x)) \subset G$. 
Thus, pushing forward to $G$ we get a prinicipal $\underline{G}$-bundle. The choice of a different geometric point $\overline{x'} \to X$
differ the homomorphism by an inner automorphism, thus we have:

\begin{theorem}\label{theorem:equivalence_fundamental_covers}
    Let $X$ be a connected scheme and $\overline{x}$ a geometric point of $X$. Suppose in addition that $G$ is a finite abstract group. 
    Define a map
    \begin{equation}
        \mathrm{Hom}_{\mathrm{cont}}(\pi_1^{\mathrm{et}}(X, \overline{x}), G) / \mathrm{Inn}(G) \to \mathrm{Tors}(X_{\mathrm{Fl}}, G) 
    \end{equation}
by sending a homomorphism $\rho : \pi_1^{\mathrm{et}}(X, \overline{x}) \to G$ 
to the principal $G$-bundle $\varphi_*(Y)$ where $Y$ is the principal $\underline{\rho(\pi_1^{\mathrm{et}}(X, \overline{x}))}$-bundle 
obtained above and $\varphi$ is the inclusion $\rho(\pi_1^{\mathrm{et}}(X, \overline{x})) \hookrightarrow G$. Then, the map 
 is a bijection of pointed sets where the trivial homomorphisms
  (which is the only element of its $\mathrm{Inn}(G)$-orbit) is the distinguished element of the left hand side.
\end{theorem}

If $G$ is abelian then $\mathrm{Inn}(G)$ is trivial, and we obtain:
\begin{cor}\label{cor:abelian_equivilance_fundamental_torsors}
    Let $G$ be a finite abelian group, $X$ a connected scheme, and $\overline{x}$ a geometric point of $X$. Then, the map from \Cref{theorem:equivalence_fundamental_covers}
     induces an isomorphism of abelian groups
    \begin{equation}
        \mathrm{Hom}_{\mathrm{cont.}}(\pi_1^{\text{ét}}(X, \overline{x}), G) \xrightarrow{\cong} \Tors(X_{\mathrm{Fl}}, G)
    \end{equation}
\end{cor}

\textcolor{red}{Let us give a final note that, evidently, $\mathrm{Aut}(G)$ acts on $\mathrm{Hom}_{\mathrm{cont.}}(\pi_1^{\text{ét}}(X, x), G)$ on the right, and if we consider the quotient $\mathrm{Hom}_{\mathrm{cont.}}(\pi_1^{\text{ét}}(X, x), G)/\mathrm{Aut}(G)$ we get the pointed set of all connected finite Galois covers of $X$ with Galois group isomorphism to a subgroup of $G$.}

%%% %%% %%% %%% %%% %%% %%% %%% %%% %%% %%% %%% %%% %%% %%% %%%
%       Qoutient of G-torsors
%%% %%% %%% %%% %%% %%% %%% %%% %%% %%% %%% %%% %%% %%% %%% %%%

\subsubsection*{Qoutients of \texorpdfstring{$G$}{G}-Torsors}
Let $G$ be a finite abelian group, and let $H \subseteq G$ be a subgroup. 
 Consider the natural quotient homomorphism $\pi: G \to G/H$. Given a $G$-torsor $\mathcal{P} \to X$ (in the fppf or étale topology), we may associate to it a $G/H$-torsor, 
 denoted by $\pi_*(\mathcal{P})$ or $\mathcal{P}^{G/H}$, via the extension of scalars.
 The object $\mathcal{P}^{G/H}$ is defined as the contracted product:
 $$\mathcal{P}^{G/H} = \mathcal{P} \times^G (G/H)$$
 Recall that $\mathcal{P}^{G/H}$ is the quotient of the product $\mathcal{P} \times (G/H)$ by the $G$-action defined by 
 $g \cdot (p, \bar{g}') = (g \cdot p, \pi(g^{-1})\bar{g}')$. 
 Consequently, we have a canonical morphism of sheaves 
 $\phi: \mathcal{P} \to \mathcal{P}^{G/H}$ which, on local sections, acts by 
 $p \mapsto [p, \bar{e}]$, where $\bar{e}$ is the identity element in $G/H$. 
 To see that $\mathcal{P}$ carries the structure of an $H$-torsor over $\mathcal{P}^{G/H}$, 
 consider a local trivializing cover $\{U_i \to X\}$ for $\mathcal{P}$ as a $G$-torsor. 
 Over each $U_i$, we have an isomorphism $\mathcal{P}|_{U_i} \cong G_{U_i}$. 
 Under this isomorphism, the contracted product locally satisfies:
 $$(\mathcal{P} \times^G G/H)|_{U_i} \cong (G \times^G G/H)_{U_i} \cong (G/H)_{U_i}$$
 Explicitly, the local identification $(g, \bar{g}') \sim (e, \pi(g)\bar{g}')$ 
 shows that every equivalence class in the fiber has a unique representative of the form 
 $[e, \bar{g}]$. 
 The map $\phi$ locally corresponds to the quotient map $G \to G/H$. 
 Since the kernel of this map is $H$, and $G$ is a trivial $H$-torsor over $G/H$, 
 it follows by descent that $\mathcal{P}$ is an $H$-torsor over $\mathcal{P}^{G/H}$.
 We summarize this construction in the following proposition:
 \begin{prop}\label{prop:quotient_torsors}
    Let $G$ be an abelian group and $H \subset G$ a subgroup. 
    Any $G$-torsor $\mathcal{P} \to X$ admits a natural factorization:
    $$\mathcal{P} \xrightarrow{} \mathcal{P}^{G/H} \xrightarrow{} X$$
    where $\mathcal{P}^{G/H} \to X$ is a $G/H$-torsor and $\mathcal{P} \to \mathcal{P}^{G/H}$ is an $H$-torsor.
\end{prop}

%%% %%% %%% %%% %%% %%% %%% %%% %%% %%% %%% %%% %%% %%% %%% %%%
%       Equivalence between Torsors and Invertible Modules
%%% %%% %%% %%% %%% %%% %%% %%% %%% %%% %%% %%% %%% %%% %%% %%%

\subsubsection*{Equivalence between Torsors and Invertible Modules}
\textcolor{red}{Edit this completly.}
Let $\mathcal{E}$ be a topos and let $\Lambda$ be a commutative ring object in $\mathcal{E}$. Let $G = \Lambda^\times$ denote the internal group object of units of $\Lambda$.
The following proposition establishes the fundamental dictionary between the geometric theory
of principal homogeneous spaces and the algebraic theory of invertible modules.
This equivalence allows us to transport the monoidal structure from the category of modules
(with the tensor product over $\Lambda$) to the category of torsors (with the contracted product over $G$),
strictly within the categorical framework.

\begin{prop}\label{prop:torsor_module_equivalence}
    There is a canonical equivalence of monoidal categories between the category of $G$-torsors in $\mathcal{E}$ and the category of locally free $\Lambda$-modules of rank 1 in $\mathcal{E}$:
    \[
        \Phi: \mathbf{Tors}(\mathcal{E}, \Lambda^\times) \xrightarrow{\sim} \mathbf{Pic}(\mathcal{E}, \Lambda)
    \]
    The equivalence is defined by the associated module functor:
    \[
        \cP \longmapsto \cP \times^{\Lambda^\times} \Lambda := \Lambda^\times \backslash (\Lambda \times \cP)
    \]
    where the quotient is taken with respect to the diagonal action of $\Lambda^\times$ on $\Lambda \times \cP$. 
    The inverse functor associates to an invertible module $\cF$ its sheaf of basis frames $\underline{\mathrm{Isom}}_\Lambda(\Lambda, \cF)$.
\end{prop}

In light of this canonical equivalence, we will pass freely between the language of $G$-torsors and that of locally free $\Lambda$-modules throughout the text.

For a topos $\cE$, a group object $G$ in $\cE$ and an object $X$ in $\cE$, there is a canonical identification
between $({G\cE})_{/X}$ and $G(\cE_{/X})$, given by endowing $X$ with the trivial $G$-action.

We denote by $\mathbf{Tors}(X, G)$ the category of $G$-torsors over $X$ in $G \cE_{/X}$.
Similarly, for a ring object $\Lambda$ in $\cE$, we denote by $\mathbf{Pic}(X, \Lambda)$ the category of locally free $\Lambda$-modules of rank 1 over $X$ in $\cE_{/X}$.
The above equivilance of categories becomes 
\[
    \Phi_X: \mathbf{Tors}(X, \Lambda^\times) \xrightarrow{\sim} \mathbf{Pic}(X, \Lambda)
\].

For a morphism $f: Y \to X$ in $\mathcal{E}$, the equivalence is functorial with respect to:
\begin{itemize}
    \item \textbf{Torsor Pullback:} $f^{-1}P = P \times_{X} Y$ (Fiber product).
    \item \textbf{Module Pullback:} $f^*\mathcal{L} = \Lambda_Y \otimes_{f^{-1}\Lambda_X} f^{-1}\mathcal{L}$ (Extension of scalars).
\end{itemize}

The following diagram commutes up to natural isomorphism:

\[
\begin{tikzcd}[column sep=large, row sep=large]
\mathbf{Tors}(X, G) \arrow[r, "\Phi_X", "\sim"'] \arrow[d, "f^{-1}"'] & \mathbf{Pic}(X, \Lambda) \arrow[d, "f^*"] \\
\mathbf{Tors}(Y, G) \arrow[r, "\Phi_Y", "\sim"'] & \mathbf{Pic}(Y, \Lambda)
\end{tikzcd}
\]


\subsubsection*{\texorpdfstring{More about $G$-torsors}{More about G-torsors}}
We want to be more explicit about $G$-torsors, so let us recall the definition.

\subsection*{Other Theorems}
\textcolor{red}{Say something about the toposes of Etale and etale, maybe add them up in the notation.}
We recall some propositions about $G$-torsors that will be useful later.

\begin{prop}[\cite{Guignard2018}, Proposition 2.12]\label{prop:torsor_representable}
Let $G$ be a finite abelian group, let $S$ be a scheme, and let $\cP$ be a $G$-torsor over an $S$-scheme $X$ in $S_{\text{Et}}$.
Then the etale sheaf $\cP$ is representable by a finite etale $X$-scheme.
\end{prop}
\begin{cor}[\cite{Guignard2018}, Corollary 2.13]
    Let $G$ be a finite abelian group, let $S$ be a scheme, and let $X$ be an $S$-scheme.
    Then the category of $G$-torsors over $X$ in $S_{\text{Et}}$ is equivalent to the category of $G$-torsors over $X$ (the terminal object) in $X_{\text{ét}}$.
\end{cor}

Next, we recall the definition of the contracted product of torsors, which endows the category of $G$-torsors with a monoidal structure.


\subsection{Symmetric Powers of Schemes and Torsors}
This section reviews the construction of quotients for schemes and torsors under finite group actions, specifically focusing on symmetric powers. 
To ensure these quotients exist as schemes, we utilize the framework of admissible actions from \cite{sga1}. 
Our treatment here closely follows the exposition in \cite{Guignard2018}
The definitions and results presented below are adapted from their work. 
This foundation provides the necessary criteria for admissibility and base change required to define the symmetric powers of a scheme $X$
 and a $G$-torsor $\mathcal{P}$ over $X$.


Let $S$ be a scheme.

\begin{definition}[(\cite{sga1}, V.1.7).]
    \noindent
    \begin{itemize}
        \item Let $T$ be an object of a category $\mathcal{C}$ endowed with a right action of a group $\Gamma$. We say that \textbf{the quotient $T/\Gamma$ exists} in $\mathcal{C}$ if the covariant functor
    \[
    \begin{aligned}
    \mathcal{C} &\to \text{Sets} \\
    U &\mapsto \text{Hom}_{\mathcal{C}}(T, U)^{\Gamma}
    \end{aligned}
    \]
    is representable by an object of $\mathcal{C}$.
    \item Let $T$ be an $S$-scheme. An action of a finite group $\Gamma$ on $T$ is \textbf{admissible} if there exists an affine $\Gamma$-invariant morphism $f : T \to T'$ such that the canonical morphism $\mathcal{O}_{T'} \to f_* \mathcal{O}_T$ induces an isomorphism from $\mathcal{O}_{T'}$ to $(f_* \mathcal{O}_T)^{\Gamma}$.
\end{itemize}
\end{definition}

\begin{prop}
    The following holds:
    \begin{enumerate}
        \item \textbf{(\cite{sga1} V.1.3)}. Let $T$ be an $S$-scheme endowed with an admissible right action of a finite group $\Gamma$. If $f : T \to T'$ is an affine $\Gamma$-invariant morphism such that the canonical morphism $\mathcal{O}_{T'} \to f_* \mathcal{O}_T$ induces an isomorphism from $\mathcal{O}_{T'}$ to $(f_* \mathcal{O}_T)^{\Gamma}$, then the quotient $T/\Gamma$ exists and is isomorphic to $T'$.
        \item \textbf{(\cite{sga1}, V.1.8)}. Let $T$ be an $S$-scheme endowed with a right action of a finite group $\Gamma$. Then, the action of $\Gamma$ on $T$ is admissible if and only if $T$ is covered by $\Gamma$-invariant affine open subsets.
        \item \textbf{(\cite{sga1}, V.1.9)}. Let $T$ be an $S$-scheme endowed with an admissible right action of a finite group $\Gamma$, and let $S'$ be a flat $S$-scheme. Then, the action of $\Gamma$ on the $S'$-scheme $T \times_S S'$ is admissible, and the canonical morphism
            \[
            (T \times_S S')/\Gamma \to (T/\Gamma) \times_S S'
            \]
            is an isomorphism.
    \end{enumerate}
\end{prop}

\begin{prop}[{[SGA1]}, IX.5.8]
Let $G$ be a finite abelian group, let $\cP$ be a $G$-torsor over an $S$-scheme $X$ in $S_{\text{Ét}}$. 
Assume that $\cP$ and $X$ are endowed with right actions from a finite group $\Gamma$ 
such that the morphism $\cP \to X$ is $\Gamma$-equivariant, and that the following properties hold:
\begin{enumerate}
    \item[(a)] The right $\Gamma$-action on $\cP$ commutes with the left $G$-action.
    \item[(b)] The right $\Gamma$-action on $X$ is admissible, and the quotient morphism $X \to X/\Gamma$ is finite.
    \item[(c)] For any geometric point $\bar{x}$ of $X$, the action of the stabilizer $\Gamma_{\bar{x}}$ of $\bar{x}$ in $\Gamma$ on the fiber $\cP_{\bar{x}}$ of $\cP$ at $\bar{x}$ is trivial.
\end{enumerate}
Then the action of $\Gamma$ on $\cP$ is admissible, and $\cP/\Gamma$ is a $G$-torsor over $X/\Gamma$ in $S_{\text{Ét}}$.
\end{prop}


\subsubsection*{Symmetric Powers of Schemes}
Let $X$ be an $S$-scheme and let $d \geq 0$ be an integer. The group ${S}_d$ of permutations of $\llbracket 1, d \rrbracket$ acts on the right on the $S$-scheme $X^{\times_S d} = X \times_S \dots \times_S X$ by the formula
\[
(x_i)_{i \in \llbracket 1, d \rrbracket} \cdot \sigma = (x_{\sigma(i)})_{i \in \llbracket 1, d \rrbracket}.
\]

\begin{prop}[\cite{Guignard2018} Proposition 2.27]
    If $X$ is a scheme, Zariski locally quasi-projective over $S$, then the right action of the symmetric group $S_d$ on the $d$-fold fiber product 
    $X^{\times_S d}$ is admissible. 
    Consequently, the quotient $\mathrm{Sym}_S^d(X) = X^{\times_S d}/S_d$ exists as a scheme over $S$.
\end{prop}
\begin{rem*}
    When the base $S$ is understood from context, this quotient is also denoted by $X^{(d)}$.
\end{rem*}

Guingard shows that when $X=\Spec(B)$ and $S=\Spec(A)$ then $\Sym_{S}^d(X)$ is representable by an affine $S$-scheme (See \cite{Guignard2018} Remark 2.28).

\begin{prop}[\cite{Guignard2018} Proposition 2.28]\label{prop:symmetric_power_flat_base_change}
If $X$ is flat and Zariski-locally quasi-projective over $S$, then $\text{Sym}_S^d(X)$ is flat over $S$. Moreover, for any $S$-scheme $S'$, the canonical morphism
\[
\text{Sym}_{S'}^d(X \times_S S') \to \text{Sym}_S^d(X) \times_S S'
\]
is an isomorphism.
\end{prop}

\subsubsection*{Symmetric Powers of Torsors}
\textcolor{red}{change torsor tensor product to contracted product,
and the analogy with sheaves to tensor product (which it is) below the exposition to be more accurate... }


\textcolor{red}{change below the exposition to be more accurate... }
Let $S$ be a scheme, let $X$ be an $S$-scheme and let $d \geq 1$ be an integer. 
Let $G$ be a finite abelian group, and let $\cP \to X$ be a $G$-torsor over $X$ in $S_{\text{Ét}}$. 
It is easy to show that the sheaf $\cP$ is representable by a finite étale $X$-scheme. (For example \cite{Guignard2018} Proposition 2.12)

For each $i \in \llbracket 1, d \rrbracket$ let $p_i : X^{\times_S d} \to X$ be the projection on $i$-th factor, and let us consider the $G$-torsor
\[
p_1^{-1}\cP \otimes \cdots \otimes p_d^{-1}\cP = G_d \backslash \cP^{\times_S d}
\]
over $X^{\times_S d}$, where $G_d \subseteq G^d$ is the kernel of the multiplication morphism $G^d \to G$. 
The object $G_d \backslash \cP^{\times_S d}$ of $S_{\text{Ét}}$ is too representable by an $S$-scheme which is finite étale over $X^{\times_S d}$. 
The group ${S}_d$ acts on the right on $G_d \backslash \cP^{\times_S d}$ by the formula
\[
(p_i)_{i \in \llbracket 1, d \rrbracket} \cdot \sigma = (p_{\sigma(i)})_{i \in \llbracket 1, d \rrbracket}.
\]
This action of ${S}_d$ commutes with the left action of $G$ on $G_d \backslash \cP^{\times_S d}$.

\medskip

\begin{prop}[\cite{Guignard2018} Proposition 2.32.]\label{prop:symmetric_power_torsor}
If $X$ is Zariski-locally quasi-projective on $S$, then the right action of ${S}_d$ on $G_d \backslash \cP^{\times_S d}$ is admissible, 
so that the quotient $\cP^{(d)}$ of $G_d \backslash \cP^{\times_S d}$ by ${S}_d$ exists as a scheme over $S$. 
Moreover, the canonical morphism $\cP^{(d)} \to \mathrm{Sym}_S^d(X)$ is a $G$-torsor, 
and the morphism
\[
p_1^{-1}\cP \otimes \cdots \otimes p_d^{-1}\cP \to r^{-1}\cP^{(d)}
\]
where $r : X^{\times_S d} \to \mathrm{Sym}_S^d(X)$ is the canonical projection, is an isomorphism of $G$-torsors over $X^{\times_S d}$.
\end{prop}
\textcolor{red}{consider replacing $\cP$ with $P$ because it is a scheme}
\textcolor{red}{Add proposition about how it is being a scheme}



%
% SYMMETRIC POWERS OF LOCAL SYSTEMS ON CURVES
%
\subsection{Symmetric Powers of Local Systems on Curves}\label{section:symmetric_powers_on_curves}
\textcolor{red}{don't need modulus in this section, etc.}
Let $k$ be a perfect field. 
Let $C$ be a projective smooth geometrically connected curve over $k$, with genus $g$.
Let $\mdls$ be a modulus on $C$ and let $U = C \setminus \mdls$.
Let $G$ be a finite abelian group and let $\cP$ be a $G$-torsor on $U$ with ramification bounded by
$\mdls$. Let $d \geq \deg m$.

We have the following diagram:
\[
\begin{tikzcd}
U^{(d_1)} \times_k U^{(d_2)} \arrow[r, "p_1"] \arrow[d, "p_2"] & U^{(d_1)} \\
U^{(d_2)} &  {}
\end{tikzcd}
\] 

pullbacking $\cP^{(d_i)}$ along the projections we get a $G$-torsor

\[\cP ^{(d_1)} \boxtimes \cP ^{(d_2)} = p_1^{-1} \cP^{(d_1)} \otimes p_2^{-1} \cP^{(d_2)}\]
On $U^{(d_1)} \times_k U^{(d_2)}$

Note that the plus map $C^{(d_1)} \times_k C^{(d_2)} \xrightarrow{+}{} C^{(d_1 + d_2)}$
is induced from
\[
\begin{tikzcd}
C^{d_1} \times_k C^{d_2} \arrow[r, "\cong"] \arrow[d, "r_1 \times r_2"] & C^{d_1 + d_2} \arrow[d, "r"] \\
C^{(d_1)} \times_k C^{(d_2)} \arrow[r, "+"] & C^{(d_1+ d_2)} 
\end{tikzcd}
\]


Hence, by \Cref{prop:symmetric_power_torsor} (and replacing $C$ with $U$ above) we get canonical identification:
\[
 (+^{-1})(\cP^{(d_1+d_2)}) \cong \cP ^{(d_1)} \boxtimes \cP ^{(d_2)}
\]

\subsection{Algebraic Preliminaries on Ramification}
\textcolor{red}{change?}
We recall the basic definitions and properties of the ramification of discrete valuations. 
We start with the general case of discrete valuation rings and their integral closures within finite separable field extensions. 
Then, we move to the specific setting of complete discrete valuation rings within Galois extensions, where we describe the ramification filtration of the Galois group via both lower and upper numbering.
We follow \stackstag{0EXQ}, and \cite{serreLF}.

\subsubsection*{Ramification of Discrete Valuation Rings}
Let $A$ be a discrete valuation ring with fraction field $K$. Let $L/K$ be a finite separable field extension. Let $B \subset L$ be the integral closure of $A$ in $L$. Picture:

\[ \xymatrix{ B \ar[r] & L \\ A \ar[u] \ar[r] & K \ar[u] } \]
By \stackstag{032L} the ring extension $A \subset B$ is finite, hence $B$ is Noetherian. 
By \stackstag{00OK} the dimension of $B$ is $1$, hence $B$ is a Dedekind domain, see \stackstag{034X}. 
Let $\mathfrak m_1, \ldots , \mathfrak m_ n$ be the maximal ideals of $B$ (i.e., the primes lying over $\mathfrak m_ A$). We obtain extensions of discrete valuation rings

\[ A \subset B_{\mathfrak m_ i} \]
and hence ramification indices $e_ i$ and residue degrees $f_ i$. We have

\[ [L : K] = \sum \nolimits _{i = 1, \ldots , n} e_ i f_ i \]
by \stackstag{02MJ} applied to a uniformizer in $A$. We observe that $n = 1$ if $A$ is henselian (by \stackstag{04GH} and the fact that $B$ is a domain), e.g. if $A$ is complete.

\begin{definition}\label{definition:tamely_ramified_dvrs}
Let $A$ be a discrete valuation ring with fraction field $K$. Let $L/K$ be a finite separable extension. With $B$ and $\mathfrak m_ i$, $i = 1, \ldots , n$ 
as above, we say the extension $L/K$ is
\begin{enumerate}
    \item unramified with respect to $A$ if $e_ i = 1$ and the extension $\kappa (\mathfrak m_ i)/\kappa _ A$ is separable for all $i$,
    \item tamely ramified with respect to $A$ if either the characteristic of $\kappa _ A$ is $0$ or the characteristic of $\kappa _ A$ is $p > 0$, the field extensions $\kappa (\mathfrak m_ i)/\kappa _ A$ are separable, and the ramification indices $e_ i$ are prime to $p$, and
    \item totally ramified with respect to $A$ if $n = 1$ and the residue field extension $\kappa (\mathfrak m_1)/\kappa _ A$ is trivial.
\end{enumerate}
If the discrete valuation ring $A$ is clear from context, then we sometimes say $L/K$ is unramified, totally ramified, or tamely ramified for short.
\end{definition}

\subsubsection*{Structure Theorems and Some Lemmas}\label{subsubsection:structure_theorems_ramification}
Let $A$ be a complete discrete valuation ring over with uniformizer $\pi$ and residue field $\kappa$, which we assume to be perfect.
When $A$ and $\kappa$ are of the same characteristic $p > 0$, then $A$ contains a coefficient field $k \cong \kappa$ and a well known structure theorem holds: $A = k[[\pi ]] \cong k[[t]]$.
Let $K$ be the fraction field of $A$, then $K=k((\pi))$.
By \textcolor{red}{Kummer theory}, unramified extensions of $K$ correspond to separable extensions of $k$.
The maximal unramified extension of $K$ is $\overline{k}((\pi))$ where $\overline{k}$ is a separable closure of $k$.
 \textcolor{red}{Maybe add something about the above facts.}


%We will need that broad definition of tame ramification in order to deal with higher dimensional schemes, 
%however note that in the case of global and local fields (taking $A$ to be the local ring at a prime of the ring of integers), 
%this definition of tame ramification coincides with the usual one.

\subsubsection*{Classical Ramification Filtration in the Galois Case}
We now recall the classical ramification filtration in the Galois case.
Assume $A,B$ are complete DVRs. And that $L/K$ is Galois with Galois group $G$. 
In that case there is uniformizer $\pi \in B$ such that $B=A[\pi ]$ 
   
We have the ramification filtration of $G$ by lower numbering $(G_i)_{i \geq -1}$, defined by
\[ G_i = \{ \sigma \in G \mid v_B(\sigma (x) - x) \geq i + 1 \text{ for all } x \in B \} \]
where $v_B$ is the valuation on $L$ associated to $B$. 
In particular, $G_{-1} = G$ and $G_0$ is the inertia group of the extension $L/K$. 
We have that $L/K$ is unramified if and only if $G_0$ is trivial, and $L/K$ is tamely ramified if and only if $G_1$ is trivial.
It is easy exercise that in the definition of $G_i$ it is enough to check the condition for the uniformizer $\pi$ of $B$, if we define
$i^L_K(\sigma ) = v_B(\sigma (\pi ) - \pi )$ for $\sigma \in G$, then we have $G_i = \{ \sigma \in G \mid i^L_K(\sigma ) \geq i + 1 \}$.
The groups $G_i$ are normal in $G$ and are trivial for large enough $i$. In a tower of fields $K \subset E \subset L$, where $H=\Gal(L/E)$ we have
\[ G_i \cap H = H_i \] for all $i \geq -1$, which corresponds to the fact that $i^L_E = i^L_K|_{\mathrm{Gal}(L/E)}$.
Ramification groups also behave well with respect to quotients: $G_i H/H=(G/H)_j$.
where \[ j=\frac{1}{e_{L/E}}\sum_{\tau \in H} \min(i^L_K(\tau), i+1)-1 \]
i.e. the quotient of a ramification group is itself a ramification group, but with a different index.
In the literature, one reindexes the ramification groups by defining the Herbrand function $\phi_{L/K} : [-1, \infty ) \to [-1, \infty )$:
\[ \phi_{L/K}(i) = \frac{1}{e_{L/K}}\sum_{\sigma \in G} \min(i^L_K(\sigma), i+1)-1 = \int _0^i \frac{1}{[G_0 : G_t]} dt \]
It is continuous, increasing, piecewise linear function, hence a bijection.
It satisfies $\phi_{L/K}=\phi_{E/K} \circ \phi_{L/E}$ for $K \subset E \subset L$, and $G_iH/H=(G/H)_{\phi_{L/E}(i)}$.
Thus, defining the ramification groups by upper numbering as $G^i = G_{\phi_{L/K}^{-1}(i)}$, we have:
\[ G^i H/H = (G/H)^i \]
for all $i \geq -1$.


\subsection{Kummer and Artin-Schreier Theories}
We recall the basic theorms from both theories regarding cyclic extesntions and ramifications. 
Throughout this section let $K$ be a discrete valuation field with perfect residue field $\kappa$ of characteristic $p > 0$. 


\begin{theorem}[Ramification in Kummer Extensions, \cite{koch1997algebraic}, Proposition 1.83]\label{theorem:kummer_ramification}
%Let $K$ be a local field of characteristic $\text{char}(K) =p$  
Assume $K$ contains the $n$-th roots of unity $\mu_n$. 
Let $L/K$ be the extension given by the equation $X^n = a$ for some 
$a \in K^\times$ and denote by $G$ its Galois group. Then we have:

\begin{enumerate}
    \item If $v_K(a) \in n\mathbb{Z}$ and the image of $a\pi^{-v_K(a)}$ in the residue field $\kappa$ is an $n$-th power, the extension $L/K$ is trivial.
    \item If $v_K(a) \in n\mathbb{Z}$ and the image of $a\pi^{-v_K(a)}$ in the residue field $\kappa$ is not an $n$-th power, the extension $L/K$ is cyclic and unramified.
    \item If $v_K(a) \notin n\mathbb{Z}$, the extension $L/K$ is cyclic and ramified. Specifically, if $\gcd(|v_K(a)|, n) = 1$, the extension is totally ramified of degree $n$.
     Otherwise it has ramification index $\frac{n}{|gcd(v_K(a)|, n)}$
\end{enumerate}
Conversly, Kummer theory ensures that every cyclic extension of degree $n$, prime to $p$ of a field that contains $n$-th roots of unity, is of the above form.
Moreover, in the above we can always take $a \in \Oo_K$
\end{theorem}
Note that in the case of total ramification the extesntion is tamely ramified.


\begin{theorem}[Ramification in Artin-Schreier Extensions, \cite{Thomas2005}]\label{theorem:artin_schreier_ramification}
Let $\wp(x) = x^p - x$ be the Artin-Schreier opeartor.
Let $L/K$ be the extension given by the equation $X^p - X = a$ for some $a \in K$ and denote by $G$ its Galois group. 
Then we have:
\begin{enumerate}
    \item If $v_K(a) > 0$ or if $v_K(a) = 0$ and $a \in \wp(K)$, the extension $L/K$ is trivial.
    \item If $v_K(a) = 0$ and if $a \notin \wp(K)$, the extension $L/K$ is cyclic of degree $p$ and unramified.
    \item If $v_K(a) = -m < 0$ with $m \in \mathbb{Z}_{>0}$ and if $m$ is prime to $p$, the extension $L/K$ is cyclic of degree $p$ again and totally ramified. Moreover,
     its ramification groups are given by:
     $$G = G^{(-1)} = \dots = G^{(m)} \quad \text{and} \quad G^{(m+1)} = 1.$$
\end{enumerate}
Conversely, Artin-Schreier theory ensures that every cyclic extension of degree $p$ takes this form.
Moreover, in the above and under the isomorphism $K \cong k((t))$, one can always take $a$ of the form 
$c t^{-m} + a_{-m + 1} t^{-m + 1} + \cdots a_{-1}t^{-1} + a_0$. If $k$ is algebraically closed then there is a 
change of variables such that $a=u^{-m}$.
\end{theorem}



%%%% %%%% %%%% %%%% %%%% %%%% %%%% %%%%
%          ALGEBRAIC GEOMETRY
%%%% %%%% %%%% %%%% %%%% %%%% %%%% %%%%
\subsection{Algebraic Geometry}
In this section we group together some general theorems in algebraic geometry that we will be employing throughtout the text. 
All schemes are assumed to be locally of finite type. 
\begin{theorem}
    Let $f:X \to Y$ be a finite flat map between integral schemes, of finite type over a field $k$. if $Z \subset X$ is a prime divisor with generic point $\eta_Z$, then $f(Z) \subset X$
    is a prime divisor with generic point $\eta_{f(Z)}$ satisfying $f(\eta_Z)= \eta_{f(Z)}$
\end{theorem}
\begin{proof}
    $f$ is finite hence proper hence closed so $f(Z)$ is closed subset of $Y$, it is irreducible as the image of an irreducible.
    Since $Z = \overline{\{ \eta_Z \}}$ we get: 
    $$\{f(\eta_Z)\} \subset f(Z) = f(\overline{\{\eta_Z\}}) \subseteq \overline{f(\{\eta_Z\})} = \overline{\{f(\eta_Z)\}}$$
    And since $f(Z)$ is closed we get $f(Z) = \overline{\{f(\eta_Z)\}}$.

    For flat map of integral schemes we have for every $x \in X$, $y=f(x)$ the dimension formula:
    $$\text{dim}(\mathcal{O}_{X,x}) = \text{dim}(\mathcal{O}_{Y,y}) + \text{dim}(\mathcal{O}_{X_y, x})$$
    And since $\text{dim}(\mathcal{O}_{X_y, x}) = 0$ we get $\text{dim}(\mathcal{O}_{X,x}) = \text{dim}(\mathcal{O}_{Y,y})$
    concluding that $f(Z)$ is a prime divisor as well. 
\end{proof}

A known theorem states that: 
\begin{theorem}\label{theorem:int_geoint_implies_productint}
    Let $X, Y$ be two integral schemes over a field $k$. If $X$ is geometrically integral then $X \times_k Y$ 
    is integral.
    If both $X,Y$ are geometrically integral, then $X \times_k Y$ is geometrically integral. 
\end{theorem}

\begin{theorem}
    Let $C$ be smooth projective curve geometrically connected over a field $k$. Then:
    \begin{enumerate}
        \item $C^{(d)}$ is smooth
        \item For every $d$, $C^{(d)}$ is integral.
        \item For every $d$, $C^{(d)}$ is geometrically integral.
        \item The product of every finite number of $C^{(d)}$ is geometrically integral. 
    \end{enumerate}
    \begin{proof}
        \begin{enumerate}
            \item Let $t_i$ be a local parameter for $C$ at $P_i$. The local ring of the product $C^d$ at the point $(P_1, \dots, P_d)$ is isomorphic $k[[t_1, t_2, \dots, t_d]]$
            and the local ring of the qoutient at the divisor $D=\sum P_i$ is $k[[t_1, \dots, t_d]]^{S_d}$ which is 
            isomorphic to $k[[t_1, \dots, t_d]]^{S_d} \cong k[[s_1, \dots, s_d]]$ where the $s_i$ are the 
            symmetric polynomials, hence this ring is regular local ring. 
            \item $C$ is irreducible hence $C^{d}$ is irreducible hence $C^{(d)}$ is irreducible. Since $C^{(d)}$ 
            is smooth it is reduced.
            \item By \stackstag{0366}, $C$ is geometrically integral, so it follows from the above. 
            \item \Cref{theorem:int_geoint_implies_productint}
        \end{enumerate}
    \end{proof}
\end{theorem}


\subsubsection*{Blowups}

\begin{theorem}[\stackstag{0805}]
Let $X_1 \to X_2$ be a flat morphism of schemes. Let $Z_2 \subset X_2$ be a closed subscheme. Let $Z_1$ be the inverse image of $Z_2$ in $X_1$. Let $X'_ i$ be the blowup of $Z_ i$ in $X_ i$. Then there exists a cartesian diagram

\[ \xymatrix{ X_1' \ar[r] \ar[d] & X_2' \ar[d] \\ X_1 \ar[r] & X_2 } \]
of schemes.   
\end{theorem}

\begin{theorem}
If $X$ is integral then $Bl_Z(X)$ is integral. %This can be checked locally
\end{theorem}


\section{Ramification After Blowup Is Tame}\label{section:ramification_after_blowup_is_tame}
Throughout this section let $C$ be a projective smooth geometrically connected curve over $k$.
Let $\mdls = \sum n_i P_i$ be a fixed modulus. Let $U=C \setminus \mdls$.
For any modulus $\ndls \subset \mdls$, we define $Z_\ndls$ as the closed subscheme of $C^{(\deg \ndls)} = \Sym_{k}^{\deg \ndls}(C)$ 
defined by $\ndls$ as a point of $C^{(\deg \ndls)}$
We then define $X_\ndls$ as the blowup of $C^{(\deg \ndls)}$ at $Z_\ndls$,
and we denote by $E_\ndls = Z_\ndls \times_{C^{(d)}} X_\ndls$ the exceptional divisor of this blowup, it is irreducible of codimension 1.
We denote by $\eta_\ndls$ the generic point of $E_\ndls$. 

Diagrammatically:
\begin{equation}\label{equation:blowup_diagram_ndls}
  \begin{tikzcd}
\overline{\{\eta_\ndls \}} =  E_\ndls \arrow[r, hook] \arrow[d] & X_\ndls \arrow[d, "\pi_\ndls"] \\
Z_\ndls \arrow[r, hook]           & C^{(\deg \ndls)}          
\end{tikzcd}   
\end{equation}

Our main goal in this section is to prove \Cref{theorem:SymmetricPowerOfSheavesIsTamelyRamifiedReduction}:
\begin{theorem}\label{theorem:torsors_after_blowup_is_tame}
    Let $G$ be a finite abelian group. And let $\cP \to U$ be a $G$-torsor in $U_{\text{\'Et}}$ with ramification bounded by $\mdls = \sum n_i P_i$.
    Let $\ndls = n_i P_i \subset \mdls$ be a sub modulus of $\mdls$. Then $\cP^{(\deg \ndls)}$ is tamely ramified at $\eta_\ndls$.
\end{theorem}

Our approach proceeds in three stages. 
First, we establish the result for the cyclic case 
$G = \mathbb{Z}/p\mathbb{Z}$ and for groups $G$ where $\gcd(|G|, p) = 1$.
Next, we extend the proof to cyclic $p$-groups $G = \mathbb{Z}/p^r\mathbb{Z}$. 
Finally, for a general finite abelian group $G$, we conclude by applying the structure theorem for finite abelian groups 
and \Cref{prop:quotient_torsors} to decompose the torsor into these constituent cases.

\subsection{\texorpdfstring{Ramification of $G$-Torsors}{Ramification of G-Torsors}}

Generally speaking, assume we are given a locally Noetherian scheme $X$, 
a dense open $U \subset X$, a finite \'etale morphsm $f: Y \to U$ and a prime divisor $Z \subset X$ such that $Z \cap U = \emptyset$
and the local ring $\Oo_{X, \xi}$ of $X$ at the generic point $\xi$ of $Z$ is a discrete valuation ring. 
Setting $K_\xi = \Frac(\Oo_{X, \xi})$ we obtain a cartesian square
\[ \xymatrix{ \mathop{\mathrm{Spec}}(K_\xi ) \ar[r] \ar[d] & U \ar[d] \\ \mathop{\mathrm{Spec}}(\mathcal{O}_{X, \xi }) \ar[r] & X } \]
of schemes. 
In particular, we see that $Y \times _ U \mathop{\mathrm{Spec}}(K_\xi )$ is the spectrum of a finite separable algebra $L_\xi /K_\xi $. 
\begin{definition}\label{definition:tamely_ramified_in_codim_1}
We say: 
\begin{enumerate}
    \item $Y$ is \textit{unramified} over $X$ at $(Z, \xi)$ resp. $Y$ is \textit{tamely ramified} over $X$ at $(Z, \xi)$
        if $L_\xi /K_\xi $ is unramified, resp. tamely ramified with respect to $\mathcal{O}_{X, \xi }$
        (\Cref{definition:tamely_ramified_dvrs})
    
    \item $Y$ is unramified over $X$ in codimension $1$, resp. $Y$ is tamely ramified over $X$ in codimension $1$ if $L_\xi /K_\xi $ is unramified, resp. tamely ramified 
    with respect to $\mathcal{O}_{X, \xi }$ for every $(Z, \xi )$ as above. 
\end{enumerate}

More precisely, we decompose $L_\xi $ into a product of finite separable field extensions of $K_\xi $ and we require each of these to be unramified, resp. tamely ramified with 
respect to $\mathcal{O}_{X, \xi }$.

\end{definition}

Now back to our orginial assumptions, $C$ is projective smooth geometrically connected curve
over a field $k$ and $U=C\setminus \mdls$. Every $(Z, \xi)$ is now a closed point $P$ and $\Oo_{C, P}$ is a discrete valuation ring. 
Assume $f: Y \to U$ is actually $G$-torsor $\cP \to U$ in $U_{\text{\'Et}}$ for $G$ finite abelian group.

By passing to the completion $\widehat{\Oo}_{C,P}$, we obtain the complete local field $\widehat{K}_P$.
Restricting the torsor $\cP$ to the punctured formal neighborhood $\Spec(\widehat{K}_P)$
 allows us to study the ramification in a strictly local context. 
 This restriction yields a $G$-torsor over a field, which necessarily decomposes into a disjoint union of spectra of finite separable field extensions
 $$ \cP|_{\Spec(\widehat{K}_P)} \cong \bigsqcup \Spec(F) $$
 such that:
 \begin{enumerate}
    \item These extensions $F/\widehat{K}_P$ are all isomorphic and Galois. 
    \item The Galois group $H = \Gal(F/\widehat{K}_P)$ identifies with the stabilizer of any connected component of the pullback $\cP|_{\Spec(\widehat{K}_P)}$, 
     which is a subgroup of $G$. 
    \item  The number of such components is then given by the index $[G:H]$.
 \end{enumerate}

 \begin{definition}\label{definition:local_ramification_boundness}
    We say that the extension $F/\widehat{K}_P$ \textit{has ramification bounded by $r$} if the ramification group $H^r$ is trivial. 
    We say \textit{$\cP$ has ramification bounded by $r$ at $P$} if the corresponding local field extensions $F/\widehat{K}_P$  
    satisfy this condition.
 \end{definition}


\begin{definition}
    Let $\mdls = \sum n_i P_i$ be a modulus on $C$. We say the $G$-torsor $\cP \to U$ has 
    \textbf{ramification bounded by $\mdls$} if, for each point $P_i$, the local restriction 
    $\cP|_{\Spec(\widehat{K}_{P_i})}$ has ramification bounded by the coefficient $n_i$.
\end{definition}

%%%%%%%%%%%%%%%%%%%%%%%%%%%%%%%%%%%%%%%
This local property remains invariant under base change to the separable closure.
Let $\bar{s} = \Spec(\bar{k}) \to \Spec(k)$ 
be a geometric point. 
The higher ramification groups for $\cP|_{\Spec(\widehat{K}_P)}$ are isomorphic to those of 
the geometric restriction $\cP_{\bar{k}}$ at any point $\bar{x}$ lying over $P$. 
Thus an equivalent defintion:
\begin{definition}
    Let $\mdls = \sum n_i P_i$ be a modulus on $C$. 
    The $G$-torsor $\cP \to U$ has \textbf{ramification bounded by $\mdls$} 
    if for every geometric point $\bar{x}$ of $\mdls$ (with image $\bar{s}$ in $\Spec(k)$), 
    the restriction of $\cP$ to$$ \Spec(\widehat{\mathcal{O}}_{C_{\bar{k}},\bar{x}}) \times_{C_{\bar{k}}} U_{\bar{k}} $$
    has ramification bounded by the multiplicity of $\mdls_{\bar{s}}$ at $\bar{x}$ in the sense of 
    \Cref{definition:local_ramification_boundness}.
\end{definition}
These definitions are equivalent because $\widehat{\mathcal{O}}_{C_{\bar{k}},\bar{x}} \cong \widehat{\mathcal{O}}_{C,P} \otimes_k \bar{k}$. Furthermore, the notions of tame ramification and unramifiedness derived here coincide with the codimension-$1$ criteria in \Cref{definition:tamely_ramified_in_codim_1}.

There is another equiviliant formulation through the language of characters.
The group $G$, being a finite abelian group over $k$, is \'etale. 
Consequently, the correspondence between $G$-torsors and the fundamental group allows us to describe $G$-torsors
 via characters. Specifically, there is an isomorphism: (\Cref{cor:abelian_equivilance_fundamental_torsors})
\[ \mathrm{Hom}_{\mathrm{cont.}}(\pi_1^{\text{\'et}}(U, \overline{x}), G) \xrightarrow{\cong} \mathrm{Tors}(U_{\text{\'et}}, G) \]

In our local setting, where we consider the restriction to the complete local field $\widehat{K}_P$, 
the fundamental group identifies with the absolute Galois group 
$G_{\widehat{K}_P} := \mathrm{Gal}({\widehat{K}_P}^{\mathrm{sep}}/{\widehat{K}_P})$. 
Thus, the restricted $G$-torsor $\mathcal{P}|_{\mathrm{Spec}(\widehat{K}_P)}$ corresponds to a unique continuous homomorphism:
\[ \rho : G_{\widehat{K}_P} \to G \]

Under this correspondence, the torsor $\mathcal{P}$ has ramification bounded by $r$ at $P$ 
if and only if the homomorphism $\rho$ trivializes the $r$-th ramification group in the upper numbering:
\[ \rho(G_{\widehat{K}_P}^r) = \{ 1 \} \]

\subsubsection*{Basic Properties of Ramification of \texorpdfstring{$G$-Torsors}{G-Torsors}}
We now prove some elementary lemmas regarding the ramification of $G$-torsors.
\textcolor{red}{fix this if you have time}
\begin{lemma}\label{lemma:bounding_ramification_of_contracted_product}
    Let $G$ be a finite abelian group and $X$ be a locally Noetherian scheme over a field $k$. 
    Let $U \subset X$ be a dense open subset and let $(Z, \xi)$ be a prime divisor in the complement $X \setminus U$.
    Assume $\cP_1$ and $\cP_2$ are two $G$-torsors on $U_{\text{ét}}$, Such that $\cP_1$ has ramification bounded by $r_1$ at $(Z, \xi)$, and $\cP_2$ has ramification bounded by $r_2$ at $(Z, \xi)$.
    Then their contracted product $\cP_1 \times^G \cP_2$ has ramification bounded by $max(r_1, r_2)$ at $(Z, \xi)$.
\end{lemma}
\begin{proof}
    Let $A=\widehat{\Oo_{X, \xi}}$ and let $K=Frac(A)$.
    Let $\rho_1, \rho_2: G_K \to G$ be the associated continious homomorphisms corresponding to the $G$-torsors 
    $\cP_1|_{\Spec(K)}$, $\cP_2|_{\Spec(K)}$.
    We finish by noticing that the associated character of $(\cP_1 \times^G \cP_2)|_{\Spec(K)}$ is $\rho=\rho_1 + \rho_2$.    
\end{proof}

\begin{lemma}\label{lemma:descend_ramification_along_etale}

Let $f: X \to Y$ be a morphism of locally Noetherian schemes. Let $U_X \subset X$ and $U_Y \subset Y$ be dense open subschemes such that $f^{-1}(U_Y) \subset U_X$. 

Let $(Z_X, \xi_X)$ and $(Z_Y, \xi_Y)$ be prime divisors of $X$ and $Y$ such that $Z_X \cap U_X = \emptyset$ and $Z_Y \cap U_Y = \emptyset$. 
Suppose $f(\xi_X) = \xi_Y$ and that $f$ is étale at $\xi_X$.
Let $\mathcal{P}$ be a $G$-torsor over $U_Y$, and let $f^{-1} \mathcal{P}$ be its pullback to $f^{-1}(U_Y) \subset U_X$. 
Then, the ramification of $f^{-1} \mathcal{P}$ at $\xi_X$ is bounded by $r$ if and only if the ramification of $\mathcal{P}$ at $\xi_Y$ is bounded by $r$.
\end{lemma}
\begin{proof}
The boundedness of ramification is determined by the behavior of the torsor over the completion of the local rings at the generic points. 
Let $A = \mathcal{O}_{Y, \xi}$ and $B = \mathcal{O}_{X, \xi_X}$ be the discrete valuation rings at the generic points, with fraction fields 
$K$ and $L$ respectively. Since $f$ is étale at $\xi_X$, the map $A \to B$ is a flat, unramified (hence weakly unramified) local homomorphism. 
Consequently, the extension of completions $\widehat{L} / \widehat{K}$ is a finite unramified extension of complete discretely valued fields.
We finish by noticing that the upper numbering filtration on the absolute Galois group is compatible with unramified base change.\footnote{
 Specifically, let $G_K = \operatorname{Gal}(K^{sep}/K)$ and $G_L = \operatorname{Gal}(L^{sep}/L)$. For an unramified extension, the Herbrand function is the identity, which implies that for any $r \geq 0$:
\[ G_L^r = G_K^r \cap G_L \]}
\end{proof}

% \subsection{The Local Ring \texorpdfstring{$\Oo_{\eta_\ndls}$}{at the generic point of the blowup}}
% We remind the settings of this section: $C$ is a projective smooth geometrically connected curve over $k$.
% $\mdls = \sum n_i P_i$ is a fixed modulus. $U=C \setminus \mdls$.
% $\ndls \subset \mdls$ is a submodulus, $Z_\ndls$ is the closed subscheme of $C^{(\deg \ndls)} = \Sym_{k}^{\deg \ndls}(C)$ 
% defined by $\ndls$ as a point of $C^{(\deg \ndls)}$.
% $X_\ndls$ is the blowup of $C^{(\deg \ndls)}$ along $Z_\ndls$, $E_\ndls = Z_\ndls \times_{C^{(d)}} X_\ndls$ 
% is the exceptional divisor of this blowup, and $\eta_\ndls$ is the generic point of $E_\ndls$. 

% Diagrammatically:
% \begin{equation}\label{equation:blowup_diagram_ndls}
%   \begin{tikzcd}
% \overline{\{\eta_\ndls \}} =  E_\ndls \arrow[r, hook] \arrow[d] & X_\ndls \arrow[d, "\pi_\ndls"] \\
% Z_\ndls \arrow[r, hook]           & C^{(\deg \ndls)}          
% \end{tikzcd}   
% \end{equation}

% We now assume $P=P_i$, $n = n_i$ and $\ndls = n P$.

% Let $x \in \Oo_{C, P}$ be a local coordinate (uniformizer) at $P$.
% $C^{(\deg \ndls)}$ is a smooth variety of dimension $n$, 
% and $\ndls \in C^{(\deg \ndls)}$ is the point "serving as the origin" in this local context.

% Near $\ndls$, the coordinate ring of $C^{(\deg \ndls)}$ is generated by the elementary symmetric polynomials in 
% the local coordinates of the $n$ copies of $C$. If $x_1, \dots, x_n$ are the pullbacks of $x$ to $C^n$, 
% the local coordinates for $X$ at $\ndls$ are:
% $$e_1 = \sum_{i=1}^n x_i, \quad e_2 = \sum_{1 \le i < j \le n} x_i x_j, \dots, \quad e_n = x_1 x_2 \dots x_n$$


% Recall that $X_{\ndls} \subset C^{(\deg \ndls)} \times_k \mathbb{P}^{n-1}_k$ is defined locally by the equations $e_i u_j = e_j u_i$, where $[u_1 : \dots : u_n]$ are the homogeneous coordinates of $\mathbb{P}^{n-1}_k$.
% The exceptional divisor $E_{\ndls} \subset X_{\ndls}$ is the fiber over $\ndls$, $E_{\ndls} = \{ (\ndls, [u_1 : \dots : u_n]) \}$, which is isomorphic to $\mathbb{P}^{n-1}_k$. 
% Let $R = \mathcal{O}_{X_{\ndls}, \eta_{\ndls}}$. This ring $R$ is a discrete valuation ring (DVR) with fraction field $K = K(X) = k(C^n)^{S_n}$.
% Its residue field is  $\kappa(\eta_{\ndls}) = k(\frac{e_2}{e_1}, \dots, \frac{e_n}{e_1})$, 
% and the completion of $R$ with respect to its maximal ideal is:
% $$\hat{R} = \kappa(\eta_{\ndls})[[e_1]]$$
% In this local ring, the first elementary symmetric polynomial $e_1$ is a uniformizer. Any other coordinate $e_i$ is also a uniformizer as $e_i = (\frac{u_i}{u_1}) e_1$ and $\frac{u_i}{u_1}$ is a unit in $R$.

% For any symmetric polynomial $f \in k[x_1, \dots, x_n]^{S_n} = \Oo{C^{(n), \ndls}}$, we can write $f$ as a polynomial $G(e_1, \dots, e_n)$. If we expand $G$ into its homogeneous parts with respect to the $e_i$ coordinates, $G = G_d + G_{d+1} + \dots$, where $G_i$ is homogeneous of degree $i$ in $(e_1, \dots, e_n)$, the valuation $\nu_E$ associated with $E$ satisfies:
% $$\nu_E(f) = d$$
% This $d$ corresponds to the order of vanishing of the symmetric function $f$ at the point $\ndls$ in the symmetric product.


% \subsubsection*{The Affine Case}
% Let $X = \mathbb{A}_k^n$ be the affine $n$-space over a field $k$, and let $0 \in X$ be the origin. 
% Let $\tilde{X} = \text{Bl}_0(X) = X_{\ndls}$ be the blowup of $X$ at the origin. 
% Recall that $\tilde{X} \subset X \times_k \mathbb{P}^{n-1}_k$ is defined by the equations $x_i u_j = x_j u_i$, where $[u_1 : \dots : u_n]$ are the homogeneous coordinates of 
% $\mathbb{P}^{n-1}_k$.
% The exceptional divisor $E \subset \tilde{X}$ is the fiber over the origin, $E = \{ (0, [u_1 : \dots : u_n]) \}$, which is of codimension 1 in $\tilde{X}$. 
% Let $\eta \in E$ be the generic point of $E$, and let $R = \mathcal{O}_{\tilde{X}, \eta}$ be the associated local ring. 
% This ring $R$ is a discrete valuation ring (DVR) with fraction field $K = K(X) = k(x_1, \dots, x_n)$.

% On the affine chart $U_1$ where $u_1 \neq 0$, we have $x_i = \frac{u_i}{u_1} x_1$. 
% The coordinate ring is:$$\mathcal{O}_{\tilde{X}}(U_1) = k\left[x_1, \frac{u_2}{u_1}, \dots, \frac{u_n}{u_1}\right]$$
% In this chart, the generic point $\eta$ corresponds to the prime ideal $\mathfrak{p}_1 = (x_1)$. Thus, the local ring is $R = k[x_1, \frac{u_2}{u_1}, \dots, \frac{u_n}{u_1}]_{(x_1)}$. 
% The residue field is $\kappa(\eta) = k(\frac{u_2}{u_1}, \dots, \frac{u_n}{u_1})$, and the completion of $R$ with respect to its maximal ideal is:
% $$\hat{R} = \kappa(\eta)[[x_1]]$$
% In this local ring, $x_1$ is a uniformizer. Note that any $x_i$ (for $i > 1$) can also serve as a uniformizer, 
% as $x_i = (\frac{u_i}{u_1}) x_1$ and $\frac{u_i}{u_1}$ is a unit in $R$.

% For any monomial $M = x_1^{a_1} \dots x_n^{a_n} \in K$, we can write:
% $$M = x_1^{\sum a_i} \left( \frac{u_2}{u_1} \right)^{a_2} \dots \left( \frac{u_n}{u_1} \right)^{a_n}$$
% Since the term in parentheses is a unit in $R$, the valuation $\nu_E$ associated with $E$ satisfies:
% $$\nu_E(M) = \sum a_i = \deg M$$
% Consequently, for any polynomial $f = f_d + f_{d+1} + \dots + f_l$, where $f_i$ is the homogeneous part of degree $i$, we have $\nu_E(f) = d$ (the order of vanishing at the origin).

%\subsubsection*{The Symmetric Product Case}

% \textcolor{red}{in the above replace $n$ with $d$}
% \subsection{Proof of Theorem \ref{theorem:SymmetricPowerOfSheavesIsTamelyRamifiedReduction}}
% We assume $k$ is algebraically closed (We can base change by the unramified $\Spec{(k^{sep})} \to \Spec{(k)}$)
% We begin with:

% \begin{theorem}
%     Assume $G=\Z/p\Z$ and let $\cP \to U$ be a $G$-torsor which is wildly ramified at $P$ with ramification bounded by $d$.
%     Let $\ndls = dP$, the $G$-torsor $\cP^{(d)} \to U^{(d)}$ is unramified at $\eta_\ndls$- the generic point of 
%     $E_\mdls \subset X_{\mdls}$.
% \end{theorem}
% \begin{proof}
%     In order to be constructive, and present simpler proofs, we assume that 
%     $C=\bP_k^1$, $\mdls = d \cdot 0$. $U$

%     The computation that we will present is formal--local at $P$, , 
%     Once you replace $\mathbf{P}^1$ by an arbitrary smooth curve $C$, 
%     nothing essential changes at the generic point of the exceptional divisor, 
%     because (after choosing a uniformizer $x$ at $P$) the completed local ring $\widehat{\mathcal{O}}_{C,P}$ 
%     is still $k[[x]]$, and all the ``$d$--fold / symmetric / blowup'' constructions you use 
%     are compatible with \'etale (indeed formal) localization.

%     the general result will then be by 
% $\bG_m = U \subset U' = C \setminus \mdls$
%     Let $x$ be a local coordinate for $\Oo_{C, P}$, then 
%     $R=\widehat{\Oo_{C, P}} = k[[x]]$ and $K=\Frac{R}=k((x))$
%     Since $\Z/p\Z$ is simple, $\cP$ is either connected or a finite union of identity morphisms. The latter case being redundant. 
%     So, by \Cref{theorem:artin_schreier_ramification} $\cP|_{K} = \Spec(F)$, 
%     Where $F=\frac{K[X]}{(X^p - X - f(x))}$ and $f(x) = c x^{-m} + a_{-m+1}x^{-m+1} + \dots a_{-1}x^{-1} + a_0 \in k((x))$ with $m < d$.
%     Thus, The ring corresponding to $\widehat{O_{C^{(d)}, (P, \dots P)}}$ is 
%     $R' = k[[x_1, \dots, x_n]]$ with fraction field $K' = k((x_1, \dots, x_n))$.
%     And the pullback of field.    $p_1^{-1}(\cP) \times \dots \times p_d^{-1}(\cP)$ on $C^{d}$
%     Is 



%     In the previous section, we saw that when $x$ is local coordinate for $\Oo_{C, P}$ (a uniformizer) at $P$. 
%     Then $$\hat{R} = k(\frac{e_2}{e_1}, \dots, \frac{e_n}{e_1})[[e_1]]$$ where $R = \mathcal{O}_{X_{\ndls}, \eta_{\ndls}}$.
%     Thus if $K=Frac(R)$ then:
%     \begin{equation}\label{eq:complete_field_at_generic_point}
%           \hat{K}= k(\frac{e_1}{e_d}, \dots, \frac{e_{d-1}}{u_d})((e_d))
%      \end{equation}
    
    
% \end{proof}



\subsection{Proof of Theorem \ref{theorem:SymmetricPowerOfSheavesIsTamelyRamifiedReduction}}

We argue its enough to assume $C=\bP^1_k$ and $\cP \to \bG_m \subset \bP_k^1$ is our $G$-torsor. This is essentially by employing the following principle: 

\textbf{Invaraince of Local Invariants under Formal isomorphism}:
\textit{If two schemes have isomorphic completed local rings at given points, then any construction obtained by functorial algebraic operations (products, quotients by finite groups, blowups, finite covers) produces isomorphic completed local rings at the corresponding points, and hence identical ramification behavior.}

Which applies for all of our constructions- blowups, symmetric products, etc.
So in what follows, we make our life simpler by assuming the above. 
So, let $G$ be a finite abelian group and assume $C=\bP_k^1$, $\mdls = d \cdot 0$ and 
$\bG_m = U \subset U' = C \setminus \mdls$. Then $\deg \mdls = d$
We also assume $k$ is algebraicly closed. (We can etale base change, and this doesn't change ramification.)

We start by analyzing the blowup $\tilde{X} = \text{Bl}_0(X)$ of $X = \mathbb{A}_k^n$
where $0 \in X$ be the origin. 
Recall that $\tilde{X} \subset X \times_k \mathbb{P}^{n-1}_k$ is defined by the equations $x_i u_j = x_j u_i$, where $[u_1 : \dots : u_n]$ are the homogeneous coordinates of 
$\mathbb{P}^{n-1}_k$.
The exceptional divisor $E \subset \tilde{X}$ is the fiber over the origin, $E = \{ (0, [u_1 : \dots : u_n]) \}$, which is of codimension 1 in $\tilde{X}$. 
Let $\eta \in E$ be the generic point of $E$, and let $R = \mathcal{O}_{\tilde{X}, \eta}$ be the associated local ring. 
This ring $R$ is a discrete valuation ring (DVR) with fraction field $K = K(X) = k(x_1, \dots, x_n)$.

On the affine chart $U_1$ where $u_1 \neq 0$, we have $x_i = \frac{u_i}{u_1} x_1$. 
The coordinate ring is:$$\mathcal{O}_{\tilde{X}}(U_1) = k\left[x_1, \frac{u_2}{u_1}, \dots, \frac{u_n}{u_1}\right]$$
In this chart, the generic point $\eta$ corresponds to the prime ideal $\mathfrak{p}_1 = (x_1)$. Thus, the local ring is $R = k[x_1, \frac{u_2}{u_1}, \dots, \frac{u_n}{u_1}]_{(x_1)}$. 
The residue field is $\kappa(\eta) = k(\frac{u_2}{u_1}, \dots, \frac{u_n}{u_1})$, and the completion of $R$ with respect to its maximal ideal is:
$$\hat{R} = \kappa(\eta)[[x_1]]$$
In this local ring, $x_1$ is a uniformizer. Note that any $x_i$ (for $i > 1$) can also serve as a uniformizer, 
as $x_i = (\frac{u_i}{u_1}) x_1$ and $\frac{u_i}{u_1}$ is a unit in $R$.

For any monomial $M = x_1^{a_1} \dots x_n^{a_n} \in K$, we can write:
$$M = x_1^{\sum a_i} \left( \frac{u_2}{u_1} \right)^{a_2} \dots \left( \frac{u_n}{u_1} \right)^{a_n}$$
Since the term in parentheses is a unit in $R$, the valuation $\nu_E$ associated with $E$ satisfies:
$$\nu_E(M) = \sum a_i = \deg M$$
Consequently, for any polynomial $f = f_d + f_{d+1} + \dots + f_l$, where $f_i$ is the homogeneous part of degree $i$, we have $\nu_E(f) = d$ (the order of vanishing at the origin).



Our first result is:
\begin{theorem}
     Let $\cP \to \bG_m \subset \bP_k^1$ be a $G$ torsor which is either
     \begin{enumerate}
          \item tamely ramified at $0$ 
          \item totally ramified at $0$ with $G=\Z/p\Z$ and ramificaiton bounded by $d$.
     \end{enumerate}
     Then the ramification of the $G$-torsor $\cP^{(d)} \to \bG_m^{(d)}$ at $\eta_\mdls$ the generic point of $E_\mdls \subset X_{\mdls}$ is  
     \begin{enumerate}
          \item tamely ramified if $\cP$ was tamely ramified
          \item unramified if $\cP$ was totally ramified with $G=\Z/p\Z$ and ramification bounded by $d$
     \end{enumerate}
\end{theorem}
\begin{proof}
    The local ring at the generic point of the excptional divisor of the blowup of the affine space at 0 point is
     $R = k[x_d, \frac{u_1}{u_d}, \dots, \frac{u_{d-1}}{u_d}]_{(x_d)}$. 
     Where for every  $i < d$, we have $x_i = \frac{u_i}{u_d} x_d$ are all uniformizers.
     The residue field is $\kappa(\eta) = k(\frac{u_1}{u_d}, \dots, \frac{u_{d-1}}{u_d})$
     and the completion of $R$ with respect to its maximal ideal is:
     $$\hat{R} = \kappa(\eta)[[x_d]]$$
     In our situation, when we take symmetric product of the affine space, the situation is similiar with different coordinates
     if we let $e_1, \dots, e_d$ be the symmetric polynomials in $x_1, \dots x_d$ then:
     The local ring is 
     $R = k[e_d, \frac{u_2}{u_d}, \dots, \frac{u_{d-1}}{u_d}]_{(e_d)}$. 
     $e_i = \frac{u_i}{u_d} e_d$ are all uniformizers.
     The residue field being:
     $\kappa(\eta) = k(\frac{u_1}{u_d}, \dots, \frac{u_{d-1}}{u_d})$
     and the completion: $\hat{R} = \kappa(\eta)[[s_d]]$
     Note that from  $e_i = \frac{u_i}{u_d} e_d$
     We get  $\frac{e_i}{e_d} = \frac{u_i}{u_d}$ in the fracion field. hence

     \begin{equation}\label{eq:complete_field_at_generic_point}
          \hat{K}= k(\frac{e_1}{e_d}, \dots, \frac{e_{d-1}}{u_d})((e_d))
     \end{equation}

     We compute directly the extesntion of complete valued fields over the complete valued field at the generic point. 
     Note that by  \Cref{theorem:kummer_ramification} and \Cref{theorem:artin_schreier_ramification} 
     We can assume $\cP = \Spec k[x, x^{-1}][X]/(X^n - a)$ for $a \in k[x, x^{-1}]$ or $\cP = \Spec k[x, x^{-1}][X]/(X^p + X - f(x, x^{-1}))$
     where $f(x,x^{-1}) = c x^{-m} + a_{-m + 1} x^{-m + 1} + \cdots a_{-1}x^{-1} + a_0 = c x^{-m} + f_{-m+1}(x^{-1})$ where $m < d$.
     
     We deal with each case separately. 
     
     \textbf{Artin-Schreier Extensions:}

     Set $R=k[x, x^{-1}]$, and $S = \Spec k[x, x^{-1}][X]/(X^p + X - f(x^{-1}))$
     we have 
     $$
          \begin{aligned}
          R^{\otimes_k d} &= k[x_1, x_1^{-1}, \dots, x_d, x_d^{-1}] \\
          S^{\otimes_k d} &= k[x_1, x_1^{-1}, \dots, x_d, x_d^{-1}][X_1, \dots, X_d] / (X_1^p - X_1 - f(x_1), \dots, X_d^p - X_d - f(x_d)) \\
          &= R^{\otimes_k d}[X_1, \dots, X_d] / (X_1^p - X_1 - f(x_1), \dots, X_d^p - X_d - f(x_d))
          \end{aligned}
     $$

     Next, we want to understand the ring corresponding to $p_1^{-1}(\cP) \otimes \dots \otimes p_d^{-1}(\cP)$ on $C^{d}$ - 
     the $d$'th-contracted product of the $G=\Z/p\Z$-torsors  $p_1^{-1}(\cP), \dots, p_d^{-1}(\cP)$ on $U^{d}$. 
     It corresponds to qoutient:
     $\left(p_1^{-1}(\cP) \times \dots \times p_d^{-1}(\cP)\right) / G^{d-1}$ where the action of $G^{d-1}$ on the product is:
     $$(g_1, \dots, g_{d-1}) \cdot (p_1, p_2, \dots, p_{d-1}, p_{d}) = (g_1(p_1), g_1^{-1}g_2(p_2), \dots, 
     g_{d-2}^{-1}g_{d-1}(p_{d-1}), g_{d-1}^{-1}(p_d))$$
     
     The affine ring corrsponding to the contracted product is $\left(S^{\otimes_k d}\right)^{G^{d-1}}$
    
     Recall that the action of $g \in G=\Z/p\Z$ on $X$ is $g(X) = X + g$ ($g$ correspond to a number $0 \leq g \leq p-1)$.
     So, the action of $(g_1, \dots, g_{d-1})$ on the generators $(X_1, X_2, \dots, X_{d-1}, X_{d})$
     Is $X_1 \mapsto X_1 + g_1$, $X_i \mapsto X_i - g_{i-1} + g_i$ for $1<i<d$ and $X_d \mapsto X_d - g_{d-1}$.
     So we see that $Y=X_1 + \dots + X_d$ is invariant.
     Moreover $Y^P - Y - \sum_{i=1}^d f(x_i)=0$ is irreducible degree $p$ equation for $Y$,
     Since we are quotienting a rank $p^{d}$ extesntion by a group of order $p^{d-1}$ the resulting invaraint subring must
     have rank $p$ over $R^{\otimes_k d}$, So we conclude:
     \[
          \left(S^{\otimes_k d}\right)^{G^{d-1}} \cong R^{\otimes_k d}[Y] / (Y^p - Y - \sum_{i=1}^d f(x_i))
     \]  
     
     The group $S_d$ acts on 
     $ R^{\otimes_k d} = k[x_1^{\pm 1}, x_2^{\pm 1}, \dots, x_d^{\pm 1}]$
      by permuting the variables $\{x_i\}_{i=1}^d$.
     And since $Y=\sum_i^d X_i$ it leaves $Y$ invaraint. 
     The invaraint subring $(R^{\otimes_k d})^{S_d}$ is simply $k[e_1, e_2, \dots, e_d, e_d^{-1}]$
     where $\{ e_i \}$  are the symmetric polynomials in $x_1, \dots, x_d$:
     $$e_k = \sum_{1 \le j_1 < j_2 < \dots < j_k \le d} x_{j_1} x_{j_2} \dots x_{j_k}$$
     
     To find $\left(R^{\otimes_k d}[Y] / (Y^p - Y - \sum_{i=1}^d f(x_i))\right)^{S_d}$ its enough to express
     $\sum_{i=1}^d f(x_i)$ in $e_1, \dots, e_d$, this can be done with the newton polynomials, moreover, we claim the following:

     \begin{lemma}
          Let $f(x) = c x^{-m} + a_{-m + 1} x^{-m + 1} + \cdots a_{-1}x^{-1} + a_0$, 
          define  $\alpha(x_1, ..., x_d) = \sum_1^d f(x_i)$, and deonte by $e_1, ..., e_d$ the elementary symmetric polynomials 
          in $x_1, ..., x_d$. If $m < d$ then $\alpha(x_1, ..., x_d) \in k(e_1/e_d, e_2/e_d, ..., e_{d-1}/e_d)$
     \end{lemma}
     \begin{proof}
          Changing variables $y_i=x_i^{-1}$ for each $i \in \{1, \dots, d\}$
          We get 
          $$\alpha = \sum_{i=1}^d f(x_i) = \sum_{i=1}^d \left( c y_i^m + a_{-m+1} y_i^{m-1} + \dots + a_{-1} y_i + a_0 \right)$$
          Rearranging the sums, we get
          $$\alpha = c \sum_{i=1}^d y_i^m + a_{-m+1} \sum_{i=1}^d y_i^{m-1} + \dots + a_{-1} \sum_{i=1}^d y_i + d a_0$$
          Let $p_k(y_1, \dots, y_d) = \sum_{i=1}^d y_i^k$ be the $k$-th power sum symmetric polynomial. 
          The expression for $\alpha$ is a linear combination of these power sums:
          $$\alpha = c p_m(y) + a_{-m+1} p_{m-1}(y) + \dots + a_{-1} p_1(y) + d a_0$$
         
          According to the \textit{Fundamental Theorem of Symmetric Polynomials}, any symmetric polynomial in $y_1, \dots, y_d$ 
          can be expressed as a polynomial in the elementary symmetric polynomials $e_k(y_1, \dots, y_d)$. 
          Since $m < d$, $\alpha$ is a polynomial in $e_1(y), e_2(y), \dots, e_m(y)$. ($y=(y_1, \dots, y_d)$)
          The elementary symmetric polynomials in $y_i = 1/x_i$ are related to the elementary symmetric polynomials in $x_i$ as follows: 
          $$e_k(y_1, \dots, y_d) = \sum_{1 \le i_1 < \dots < i_k \le d} \frac{1}{x_{i_1} \dots x_{i_k}} = \frac{\sum_{1 \le j_1 < \dots < j_{d-k} \le d} x_{j_1} \dots x_{j_{d-k}}}{x_1 x_2 \dots x_d}$$
          Thus,
          $$e_k(y_1, \dots, y_d) = \frac{e_{d-k}(x_1, \dots, x_d)}{e_d(x_1, \dots, x_d)}$$
          which concludes the proof.
     \end{proof}

Finally, restricting $\cP^{(d)}$ to $\spec {\hat{K}}$ we get by \Cref{eq:complete_field_at_generic_point}
and \Cref{theorem:artin_schreier_ramification} the result. (that $\cP$ is unramified at the generic point of the exceptional divisor of the blowup).

\textbf{Kummer Extensions:}
Few things are different in that case,

     Set $R=k[x, x^{-1}]$, and $S=R[X]/(X^n - f)$ where $f= f(x, 1/x) \in R$
     In this case we have $char k = p$ and $gcd(p, n)=1$.

     We have 
     $$
          \begin{aligned}
          R^{\otimes_k d} &= k[x_1, x_1^{-1}, \dots, x_d, x_d^{-1}] \\
          S^{\otimes_k d} &= k[x_1, x_1^{-1}, \dots, x_d, x_d^{-1}][X_1, \dots, X_d] / (X_1^n - f_1, \dots, X_d^n - f_d) \\
          &= R^{\otimes_k d}[X_1, \dots, X_d] / (X_1^n - f_1, \dots, X_d^n - f_d)
          \end{aligned}
     $$
     Where $f_i = f(x_i, x_i^{-1})$

     Next, we want to figure out 
      $\left(S^{\otimes_k d}\right)^{G^{d-1}}$
    

     Recall that the action of $g \in G=\Z/n\Z$ on $X$ is $g(X) = \zeta^{g} X$ ($g$ correspond to a number $0 \leq g \leq n-1)$.
     So, the action of $(g_1, \dots, g_{d-1})$ on the generators $(X_1, X_2, \dots, X_{d-1}, X_{d})$
     Is $X_1 \mapsto \zeta^{g_1} X_1$, $X_i \mapsto \zeta^{g_i - g_{i-1}}X_i$ for $1<i<d$ and $X_d \mapsto \zeta^{-g_{d-1}} X_d$.

     So we see that $Y=X_1 X_2 \dots X_d$ is invariant.
     And $Y^n - \prod_{i=1}^d f_i$ is irreducible degree $n$ equation for $Y$,
     So, like before, we conclude:
     \[
          \left(S^{\otimes_k d}\right)^{G^{d-1}} \cong R^{\otimes_k d}[Y] / (Y^n - \prod_{i=1}^d f_i)
     \]  
     
     The group $S_d$ acts on $R^{\otimes_k d}[Y] / (Y^n - \prod_{i=1}^d f_i)$ by permuting the indices.
     on the variables $x_i$, 
     On $Y=\prod_1^d X_i$ it is invaraint. 
     The invaraint subring $(R^{\otimes_k d})^{S_d}$ is simply $k[e_1, e_2, \dots, e_d, e_d^{-1}]$ like before.
     

     The polynomial $F=\prod_{i=1}^d f_i$ is symmetric in $\{x_i\}_1^d$ so it can be expressed as a polynomial
     $\tilde{F}(e_1, \dots, e_d)$ in the elementary symmetric variables. 
     Hence the qoutient ring is:
     $$\left( \frac{k[x_1^{\pm 1}, \dots, t_d^{\pm 1}][Y]}{(Y^n - \prod_{i=1}^d f_i)} \right)^{S_d} \cong 
     \frac{k[e_1, \dots, e_d, e_d^{-1}][Y]}{(Y^n - \tilde{F}(e_1, \dots, e_d))}$$
     So we see again, that  restricting $\cP^{(d)}$ to $\spec {\hat{K}}$ we get by \Cref{eq:complete_field_at_generic_point} 
     and \Cref{theorem:kummer_ramification}, that $\cP$ is tamely ramified at the generic point of the exceptional divisor of the blowup.
\end{proof}

So, we conclude 
\begin{cor}
    In the general settings of a curve $C$, and $\cP \to U=C \setminus \mdls$ a $G$-torsor with ramification bounded by $\ndls = n P \subset \mdls$ at $P$, we have:
    \begin{enumerate}
        \item If $G=\Z/p\Z$ then $\cP^{(\deg \ndls)}$ is unramifeid at $\eta_\ndls$
        \item If $\cP$ is tamely ramified $\ndls$ then $\cP^{(\deg \ndls)}$ is tamely ramified at $\eta_\ndls$
    \end{enumerate}
\end{cor}

Next, we prove the following.
\begin{theorem}
    Assume $G=\Z/p^r\Z$. Then $\cP^{(\deg (\ndls))}$ is unramified at $\eta_{\ndls}$
\end{theorem}
\begin{proof}
    Again, we can assume $k$ is algebraically closed. 
    $\cP^{(\deg \ndls)}$ has bounded ramification at $n P$ by $n$. Let $R = \widehat{\Oo_{C, P}}$ and let $K=Frac(R)$
    Then $\cP^{(\deg \ndls)}|_{\Spec K}$ correspond to $\rho: \Gal(K^{sep}/K) \to G$.
    Let $H = \rho(G^{0})$ be the image of the inertia subgroup. 
    Decompose $\cP \to U$ to $\cP \to \cP_{G/H} \to U$ as in \Cref{{prop:quotient_torsors}}, where $\cP \to \cP_{G/H}$ is an $H$-torsor and $\cP_{G/H} \to U$ is a $G/H$ torsor.
    Then $\cP_{G/H}$ is unramified at $P$. Thus $\cP_{G/H}$ extends to $U' = U \cup \{ P \}$, 
    and taking any point $Q \in \cP_{G/H}$ over $P$, results in isomorphism $\widehat{\Oo_{\cP_{G/H}, Q}} \cong \widehat{\Oo_{C, P}}$.
    But because $k$ is algebraically closed, we even have equality, 
    and deduce that locally the $G/H$ action on $\widehat{\Oo_{C, P}}$ is trivial.
    Thus, we can assume $G=H$ and that $\cP \to U$ is totally ramified at $P$.
    
    \textcolor{red}{Not clear how to finish here without analyzing the ramification even further.}
    Thus, take a subgroup $H \subset G$ such that $G/H \cong \Z/p \Z$ and decompose $\cP \to U$ as:
\end{proof}

Finally, \textbf{Proof of Theorem \cref{theorem:torsors_after_blowup_is_tame}}:
Idea: decompose $G$ be the structure theorem, and finish by the above. 



\subsection{Extending To Product of Blowups}\label{subsection:ram_prod_analysis}
We conclude this sesction by proving sume auxilary results that would be useful in \Cref{section:gcft}.


%%% %%% %%% %%% %%% %%% %%% %%% %%% %%% %%% %%% %%% %%% %%% %%%
%       Behavior of Ramification under Product of Blowups
%%% %%% %%% %%% %%% %%% %%% %%% %%% %%% %%% %%% %%% %%% %%% %%%
%\subsubsection{Behavior of Ramification under Product of Blowups}

Let $X$ and $Y$ be smooth schemes over a field $k$, and let $x \in X$ and $y \in Y$ be closed points. 
We denote the blowups of these schemes at the given points by 
$\pi_X: \Bl_x(X) \to X$ and $\pi_Y: \Bl_y(Y) \to Y$. Furthermore, let $\pi_{X \times Y}: \Bl_{(x,y)}(X \times_k Y) \to X \times_k Y$ 
be the blowup of the product scheme at the point $(x,y)$.
We denote by $E_X, E_Y$, and $E_{X \times Y}$ the respective exceptional divisors, and let $\eta_X, \eta_Y$, and $\eta_{X \times Y}$ be their generic points.

In this section, we establish the following result concerning the stability of ramification bounds under the external product of torsors.

\begin{prop}\label{prop:ramification-external-product}
Let $G$ be a finite abelian group. Suppose $\cG_X$ and $\cG_Y$ are $G$-torsors defined on open subsets $U_X \subset \Bl_x(X)$ and $U_Y \subset \Bl_y(Y)$ that are disjoint 
from the exceptional divisors. 
If the ramification of $\cG_X$ at $\eta_X$ and $\cG_Y$ at $\eta_Y$ is bounded by $r$, then the external product torsors 
\[
\cG_{X \times Y} := \text{pr}_1^{-1} \cG_X \otimes \text{pr}_2^{-1} \cG_Y
\]
has ramification bounded by $r$ at the generic point $\eta_{X\times Y}$ of the exceptional divisor in the product blowup.
\end{prop}

The proposition is purely local in nature, it suffices to consider the case where $X$ and $Y$ are affine. 
More precisely, by the smoothness of $X$ and $Y$, we may restrict our attention to open neighborhoods of $x$ and $y$ that are etale over affine spaces. 
The rest of this section treats that case.

\subsubsection*{The Affine Case}
Let $X = \mathbb{A}_k^n$ be the affine $n$-space over a field $k$, and let $0 \in X$ be the origin. 
Let $\tilde{X} = \text{Bl}_0(X)$ be the blowup of $X$ at the origin. 
Recall that $\tilde{X} \subset X \times_k \mathbb{P}^{n-1}_k$ is defined by the equations $x_i u_j = x_j u_i$, where $[u_1 : \dots : u_n]$ are the homogeneous coordinates of 
$\mathbb{P}^{n-1}_k$.
The exceptional divisor $E \subset \tilde{X}$ is the fiber over the origin, $E = \{ (0, [u_1 : \dots : u_n]) \}$, which is of codimension 1 in $\tilde{X}$. 
Let $\eta \in E$ be the generic point of $E$, and let $R = \mathcal{O}_{\tilde{X}, \eta}$ be the associated local ring. 
This ring $R$ is a discrete valuation ring (DVR) with fraction field $K = K(X) = k(x_1, \dots, x_n)$.

On the affine chart $U_1$ where $u_1 \neq 0$, we have $x_i = \frac{u_i}{u_1} x_1$. 
The coordinate ring is:$$\mathcal{O}_{\tilde{X}}(U_1) = k\left[x_1, \frac{u_2}{u_1}, \dots, \frac{u_n}{u_1}\right]$$
In this chart, the generic point $\eta$ corresponds to the prime ideal $\mathfrak{p}_1 = (x_1)$. Thus, the local ring is $R = k[x_1, \frac{u_2}{u_1}, \dots, \frac{u_n}{u_1}]_{(x_1)}$. 
The residue field is $\kappa(\eta) = k(\frac{u_2}{u_1}, \dots, \frac{u_n}{u_1})$, and the completion of $R$ with respect to its maximal ideal is:
$$\hat{R} = \kappa(\eta)[[x_1]]$$
In this local ring, $x_1$ is a uniformizer. Note that any $x_i$ (for $i > 1$) can also serve as a uniformizer, 
as $x_i = (\frac{u_i}{u_1}) x_1$ and $\frac{u_i}{u_1}$ is a unit in $R$.

For any monomial $M = x_1^{a_1} \dots x_n^{a_n} \in K$, we can write:
$$M = x_1^{\sum a_i} \left( \frac{u_2}{u_1} \right)^{a_2} \dots \left( \frac{u_n}{u_1} \right)^{a_n}$$
Since the term in parentheses is a unit in $R$, the valuation $\nu_E$ associated with $E$ satisfies:
$$\nu_E(M) = \sum a_i = \deg M$$
Consequently, for any polynomial $f = f_d + f_{d+1} + \dots + f_l$, where $f_i$ is the homogeneous part of degree $i$, we have $\nu_E(f) = d$ (the order of vanishing at the origin).

\subsubsection*{The Product Case}
Now, let $X = \mathbb{A}^n$ and $Y = \mathbb{A}^m$ with origins $x=0$ and $y=0$. 
As before, $\text{Bl}_0(X) \subset X \times \mathbb{P}^{n-1}$ and $\text{Bl}_0(Y) \subset Y \times \mathbb{P}^{m-1}$ have exceptional divisors $E_X$ and $E_Y$ respectively.
Consider the product $X \times_k Y \cong \mathbb{A}_k^{n+m}$. 
The blowup of the product at the origin $(0,0)$, denoted $\text{Bl}_{(0,0)}(X \times_k Y)$, is a subscheme of $(X \times Y) \times \mathbb{P}^{n+m-1}$ defined by:

$$\begin{cases} 
x_i w_j = x_j w_i & 1 \le i, j \le n \\
y_k w_{n+l} = y_l w_{n+k} & 1 \le k, l \le m \\
x_i w_{n+j} = y_j w_i & 1 \le i \le n, \,\, 1 \le j \le m 
\end{cases}$$
where $[w_1 : \dots : w_{n+m}]$ are the homogeneous coordinates of $\mathbb{P}^{n+m-1}$. The exceptional divisor $E_{X \times Y}$ is isomorphic to $\mathbb{P}^{n+m-1}$.

\subsubsection*{Comparison of Blowups}

Both $\text{Bl}_0(X) \times_k \text{Bl}_0(Y)$ and $\text{Bl}_{(0,0)}(X \times_k Y)$ are birational to $X \times_k Y$. 
They share a common dense open set $\tilde{U}$ defined by the condition that neither the $X$-coordinates nor the $Y$-coordinates vanish simultaneously in the projective space:
$$\tilde{U} = \{ ((x,y), [w_1: \dots : w_{n+m}]) \mid (w_1, \dots, w_n) \neq 0 \text{ and } (w_{n+1}, \dots, w_{n+m}) \neq 0 \}$$This yields a diagram of open immersions:
$$ 
   \begin{tikzcd}
        & \tilde{U} \arrow[dl, "f_1"'] \arrow[dr, "f_2", hook] & \\
        \text{Bl}_0(X) \times_k \text{Bl}_0(Y) & & \text{Bl}_{(0,0)}(X \times_k Y)
    \end{tikzcd}
$$
    
    where $f_1$ maps the coordinates to the respective projectivizations $[w_1: \dots : w_n]$ and $[w_{n+1}: \dots : w_{n+m}]$.
    Also note that the generic point $\eta_{X \times Y}$  of $E_{X \times_k Y}$ is inside $\tilde{U}$

\subsubsection*{Extensions of DVRs}
Let $S$ be the local ring of the generic point $\eta_{X\times Y}$ of $E_{X \times Y}$ in $\tilde{U}$. 
On the chart where $w_1 \neq 0$ and $w_{n+1} \neq 0$, we have 
$x_1 = (\frac{w_1}{w_{n+1}}) y_1$. 
Since $\frac{w_1}{w_{n+1}}$ is a unit in this chart, $x_1$ and $y_1$ are equivalent as uniformizers.
We have:
$$
S=k\left[x_1, \frac{w_2}{w_1} \cdots, \frac{w_n}{w_1}, \frac{w_{n+1}}{w_1} \cdots \frac{w_{n+m}}{w_1}\right]_{(x_1)} = 
k\left[\frac{w_1}{w_{n+1}}, \frac{w_2}{w_{n+1}} \cdots \frac{w_n}{w_{n+1}}, y_1 \cdots \frac{w_{n+m}}{w_{n+1}}\right]_{(y_1)}
$$

$$k(\eta_{X \times Y}) = k\left(\frac{w_2}{w_1} \cdots, \frac{w_n}{w_1}, \frac{w_{n+1}}{w_1} \cdots \frac{w_{n+m}}{w_1}\right)$$

Let $R_X$ be the local ring of the exceptional divisor $E_X$ in $\text{Bl}_0(X)$. 
The pullback of $E_X \times_k \text{Bl}_0(Y)$ along $f_1$ induces an extension of DVRs $R_X \hookrightarrow S$. Which is:
\begin{enumerate}
    \item Weakly Unramified: $x_1$ is a uniformizer in both $R_X$ and $S$, so the ramification index is $e=1$.
    \item Residually Transcendental: The residue field extension $\kappa(\eta_X) \subset \kappa(\eta)$ is:
    $$k\left(\frac{w_2}{w_1}, \dots, \frac{w_n}{w_1}\right) \subset k\left(\frac{w_2}{w_1}, \dots, \frac{w_n}{w_1}, \frac{w_{n+1}}{w_1}, \dots, \frac{w_{n+m}}{w_1}\right)$$
    Hence separable.
\end{enumerate}
Since this extension is generated by transcendental elements, it is separable and formally smooth at the maximal ideal (\stackstag{09E7}).

\subsubsection*{Ramification of $G$-Torsors}
Let $G$ be a finite abelian group. 
Let $\mathcal{Q}$ be a $G$-torsor on an open $U \subset \text{Bl}_0(X)$ disjoint from $E_X$. 
%Suppose $\mathcal{Q}$ has ramification bounded by $r$ at the generic point $\eta_X$ of $E_X$.
Let $V \subset \text{Bl}_0(Y)$ be an open subscheme, and let $\pi_X: \text{Bl}_0(X) \times_k V \to \text{Bl}_0(X)$ 
be the projection onto the first factor. 
By restricting this projection to $U \times_k V$, we obtain the pullback $G$-torsor:
$$\pi_X^{-1}(\mathcal{Q}) \cong \mathcal{Q} \times_k V$$
which is defined on the open subset $U \times_k V \subset \text{Bl}_0(X) \times_k \text{Bl}_0(Y)$.
The extension of local rings $R_X \to S$ is weakly unramified (the ramification index $e=1$) and residually transcendental with seprable residue field exntesion. 
Under these conditions the ramification filtration is preserved. 
Therefore, the pullback $\pi_X^{-1}(\mathcal{Q})$ has ramification bounded by $r$ at the generic point $\eta_{X \times Y}$ of the exceptional divisor $E_{X \times Y}$ if and only if 
the original torsor $\mathcal{Q}$ has ramification bounded by $r$ at the generic point $\eta_X$ of $E_X$

And we finish by \Cref{lemma:bounding_ramification_of_contracted_product}.




\section{Geometric Class Field Theory}\label{section:gcft}
The main theorem of this section is
\begin{restatable}{theorem}{SymmetricPowerOfSheafIsTamelyRamified}\label{theorem:SymmetricPowerOfSheafIsTamelyRamified}
     Let $\Lambda$ be a finite ring of cardinality invertible in $k$, and let $\cF$ be an \'etale sheaf of $\Lambda$-modules, locally free of rank 1 on $U$, with ramification bounded by $\mdls$. 
     Considering $U^{(d)}$ as an open subscheme of the blowup $\tilde{C}^{(d)}_\mdls$ of $C^{(d)}$, we have that
     for sufficiently large integer $d$, $\cF^{(d)}$ is tamely ramified on $H= \tilde{C}^{(d)}_\mdls \setminus U^{(d)}=E_0 \times_{C^{(d)}} \tilde{C}^{(d)}_\mdls $.
\end{restatable}

From which we will deduce \Cref{thm:GCFT_reduced}.

We start by recalling the necessarily material, and then move to proving this theorem.

\subsection{Etale Fundamental Groups and Tame Fundamental Groups}
We recall the definition and basic properties of the etale fundamental group, following stacks project 
\stackstag{0BQ6}
\begin{prop}[\stackstag{0C0J}]
Let $f : X \to S$ be a flat proper morphism of finite presentation whose geometric fibres are connected and reduced. Assume $S$ is connected and let $\overline{s}$ be a geometric point of $S$. Then there is an exact sequence

\[ \pi _1(X_{\overline{s}}) \to \pi _1(X) \to \pi _1(S) \to 1 \]
of fundamental groups.
    
\end{prop}

\begin{cor}\label{cor:etale_fundamental_group_smooth_proper}
    Let $f : X \to S$ be a proper smooth morphism of finite presentation whose geometric fibres are connected. Assume $S$ is connected and let $\overline{s}$ be a geometric point of $S$. Then there is an exact sequence

    \[ \pi _1(X_{\overline{s}}) \to \pi _1(X) \to \pi _1(S) \to 1 \]
    of fundamental groups.
\end{cor}
%\textcolor{red}{add about tameness?}


%
% Tame Ramification
%



\subsection{Generalized Picard Scheme}
We recall the notion of generalized Jacobian varieties and study their fundamental properties. The material presented here is primarily adapted from \cite{Guignard2018} and \cite{takeuchi2019blow}. For further background on the general theory of abelian varieties and Jacobians, the reader may also consult \cite{milneAV}.
Let $S$ be a scheme and let $C$ be a projective smooth $S$-scheme whose geometric fibers are connected and of dimension 1. Let $\mdls$ be a modulus on $C$, defined as an effective Cartier divisor of $C/S$ 
(i.e., a closed subscheme of $C$ which is finite flat of finite presentation over $S$).
We denote the projection $C \times_S T \to T$ by $\text{pr}$ for any $S$-scheme $T$.

\subsubsection*{The Functor of Points}
Let $d$ be an integer. For an $S$-scheme $T$, we consider the set of data $(\cL, \psi)$ where:
\begin{itemize}
    \item $\cL$ is an invertible sheaf of degree $d$ on $C_T$.
    \item $\psi: \Oo_{\mdls_T} \xrightarrow{\sim} \cL|_{\mdls_T}$ is a trivialization of $\cL$ along the modulus.
\end{itemize}
Two such pairs $(\cL, \psi)$ and $(\cL', \psi')$ are said to be isomorphic if there exists an isomorphism of invertible sheaves $f : \cL \to \cL'$ such that the following diagram commutes:
\[
\begin{tikzcd}
& \Oo_{\mdls_T} \arrow[dl, "\psi'"'] \arrow[dr, "\psi"] & \\
\cL'|_{\mdls_T} \arrow[rr, "f|_{\mdls_T}"] & & \cL|_{\mdls_T}
\end{tikzcd}
\]
We define the presheaf $\text{Pic}_{C, \mdls}^{d, \text{pre}}$ on $\text{Sch}/S$ by assigning to $T$ the set of isomorphism classes of such pairs. Let $\PicCm[d]$ denote the \'etale sheafification of this presheaf.

\subsubsection*{Representability and Structure}
The fundamental properties of this functor are as follows:
\begin{enumerate}
    \item $\text{Pic}_{C, \mdls}^{d}$ is represented by an $S$-scheme. (Note: If $\mdls$ is faithfully flat over $S$, the presheaf is already a \'etale sheaf).
    \item $\text{Pic}_{C, \mdls}^{0}$ is a smooth commutative group $S$-scheme with geometrically connected fibers, referred to as the \textit{generalized Jacobian variety} of $C$ with modulus $\mdls$.
    \item For any $d$, $\text{Pic}_{C, \mdls}^{d}$ is a $\text{Pic}_{C, \mdls}^{0}$-torsor.
\end{enumerate}
In the case where $\mdls = 0$, we recover the standard Jacobian variety, denoted simply as $\text{Pic}_C^d$.
\subsubsection*{Relation to the Standard Jacobian}
We now examine the behavior of the generalized Picard scheme under the variation of the modulus. By viewing the structure along the modulus as an additional rigidification, we obtain natural transition maps corresponding to the inclusion of moduli.

Let $\mdls_1$ and $\mdls_2$ be moduli such that $\mdls_1 \subset \mdls_2$. There exists a natural map 
\[
\text{Pic}_{C, \mdls_2}^d \to \text{Pic}_{C, \mdls_1}^d
\]
obtained by restricting the isomorphism $\psi$. Since $\mdls_2$ is a finite $S$-scheme, this map is a surjection as a morphism of \'etale sheaves.
In particular, for any modulus $\mdls$, there is a natural surjective morphism of \'etale sheaves:
\[
\text{Pic}_{C, \mdls}^d \to \text{Pic}_C^d.
\]


\subsubsection*{Local Freeness and Base Change}
%\textcolor{red}{there is ${\mdls}$ here - should fix that}
Let $\mdls$ be a modulus which is everywhere strictly positive. Let $g$ denote the genus of $C$, which is a locally constant function on $S$. We restrict our attention to degrees $d$ satisfying the condition:
\begin{equation} \label{equation:degree-condition}
    d \geq \max\{2g - 1 + \deg \mdls, \deg \mdls\}.
\end{equation}

Assuming $S$ is quasi-compact, such a $d$ always exists.

Fix an integer $d$ satisfying the condition above. Let $T$ be an $S$-scheme and let $\cL$ be an invertible sheaf of degree $d$ on $C_T$. One can show that the pushforwards $\mathrm{pr}_* \cL$ and $\mathrm{pr}_* \cL(-{\mdls})$ are locally free sheaves and their formations commute with any base change. Explicitly, for any morphism of $S$-schemes $f : T' \to T$, the base change morphisms are isomorphisms:
\[
    f^* \mathrm{pr}_* \cL \xrightarrow{\sim} \mathrm{pr}_* f^* \cL
\]
and
\[
    f^* \mathrm{pr}_* (\cL(-{\mdls})) \xrightarrow{\sim} \mathrm{pr}_* f^* (\cL(-{\mdls})).
\]
In particular, following \cite{Guignard2018}, if $\cL$ is invertible $\cO_C$-module with degree $d$ satisfying \ref{equation:degree-condition} on each fiber of $f$ then, $\mathrm{pr}_* \cL$ is a locally free $\mathcal{O}_S$-module of rank $d-g+1$.

For further background and verification of these constructions, we refer the reader to Milne's notes on abelian Varieties (\cite{milneAV}).

\subsection{The Abel-Jacobi Morphism and its Fibers}\label{section:AbelJacobi}


Let $U = C \setminus \mdls$ be the complement of the modulus in $C$.
The effective cartier divisors of degree $d$ which are prime to $\mdls$ are parameterized by the symmetric power $\Sym_S^d (U) = U^{(d)}$ over $S$ (See \cite{Guignard2018} Proposition 4.12, \cite{milneAV} Theorem 3.13).
For any such divisor $D \in U^{(d)}$, the associated line bundle $\Oo_C(D)$ admits a canonical trivialization along $\mdls$. 
Specifically, the canonical section $1_D$ is regular and non-vanishing on $\mdls$ because $\operatorname{supp}(D) \cap \operatorname{supp}(\mdls) = \emptyset$. 
This section restricts to a nowhere-vanishing section on the subscheme $\mathfrak{m}$, thereby determining a trivialization $\psi_D^{-1}: \Oo_C(D)|_{\mdls} \xrightarrow{\sim} \Oo_{\mdls}$.
This is done functorially in families, yielding a morphism from the symmetric power to the generalized Picard scheme (over $S$):
\begin{equation}
    \Phi_{d}: U^{(d)} \to \Pic^d_{C, \mathfrak{m}}, \quad D \mapsto [(\mathcal{O}_C(D), \psi_D)],
\end{equation}

When $\mdls = 0$, $d \geq max \{2g - 1, 0\}$ and $C$ admits a section over $S$, $C^{(d)}$ is a projective space bundle over $\PicC[d]$, 
It is proper, surjective with geometrically connected fibers.

Guignard (\cite{Guignard2018} Theorem 4.14) proves that for $\mdls > 0$ and $d$ satisfying \Cref{equation:degree-condition}, the Abel-Jacobi morphism $\Phi_d$ is 
surjective smooth of relative dimension $d - \deg \mdls - g + 1$, with geometrically connected fibers. 

When $S=\Spec (k)$, the geometric-fibers of $\Phi_d$ are well understood:
\begin{theorem}
    Assuming $S=\Spec (k)$ and $d \geq max \{2g - 1 + \deg \mdls, \deg \mdls\}$. Then, the geometric-fibers of the Abel-Jacobi morphism $$ \Phi_d : U^{(d)} \to \PicCm[d] $$ over any point are isomorphic to
$$
\begin{cases}
\mathbb{A}_{k^{sep}}^{d - \deg \mdls - g + 1} & \text{if } m > 0 \\
\mathbb{P}_{k^{sep}}^{d - g} & \text{if } m = 0
\end{cases}
$$
In both cases $\Phi_d$ is a fibration in affine spaces or projective spaces, depending on whether $\mdls$ is non-zero or zero.
\end{theorem}
\begin{proof}
see \cite{tendler2015geometricclassfieldtheory} Propositions 3.13-3.14, or \cite{Toth2011} Prop 2.1.4:    

\end{proof}


\subsection{Compactification of Blowup of Symmetric Powers of a Curve}\label{section:BlowupOfSymmetricPowerOfCurves}
We recall that our objective is to descend the local system $\cF^{(d)}$ from $U^{(d)}$ to $\PicCm[d]$ along the 
Abel-Jacobi map $\Phi_d$:

\[
\begin{tikzcd}
 \cF^{(d)} \arrow[d, purple]\\
 U^{(d)} \arrow[r, "\Phi_d"] & \PicCm[d]      
\end{tikzcd}
\]


(Here, the purple arrow emphasizes that the morphism is of sheaves on the étale site).

However, we encounter an obstruction: in the case we are considering ($\mdls > 0$), 
the fibers of $\Phi_d$ are affine spaces (of the same degree) rather than the better-behaved projective spaces. 
This hint that a solution to this problem is to compactify the morphism to yield projective fibers. 
 
This section describes the result of the compactification constructed by \cite{takeuchi2019blow} 
via the method of blowup.

Let $\mdls = \sum_{i=1}^n k_P P$ with $\deg P = d_P$ be a modulus on $C$, and let $d$ satisfy 
\Cref{equation:degree-condition}. 
Takeuchi (\cite{takeuchi2019blow}) defines $Z_0=Z_0(\mathfrak{m}, d)$ as the closed subscheme of 
$C^{(d)}$ defined by the map $C^{(d - \deg \mathfrak{m})} \to C^{(d)}$ adding $\mathfrak{m}$. 
He also defines $X_{\mathfrak{m}, d}$ as the blowup of $C^{(d)}$ along $Z_0$.
Let $E_0 = E_{\mathfrak{m},d} = Z_0(\mathfrak{m}, d) \times_{C^{(d)}} X_{\mathfrak{m}, d}$ be the exceptional divisor of the blowup. It is irreducible of codimension 1, and we let $\eta_0=\eta_{\mathfrak{m}, d}$ be its generic point.

%This section describes the compactification constructed by Takeuchi \cite{takeuchi2019blow} using the blowup method

% Let $\mdls = \sum_{i=1}^n k_P P$ with $\deg P = d_P$ be a modulus on $C$. Let $d$ satisfy \Cref{equation:degree-condition}.
% Takeuchi (\cite{takeuchi2019blow}) defines $Z_0=Z_0(\mdls, d)$ as the closed subscheme of $C^{(d)}$ defined by the map $C^{(d - \deg \mdls)} \to C^{(d)}$ adding $\mdls$.
% He also defines $X_{\mdls, d}$ as the blowup of $C^{(d)}$ along $Z_0$. 
% Let $E_0 = E_{\mdls,d} = Z_0(\mdls, d) \times_{C^{(d)}} X_{\mdls, d}$ 
% be the exceptional divisor of the blowup, it is irreducible of codimension 1, let $\eta_0=\eta_{\mdls, d}$ be its generic point.

Diagrammatically:
\[\begin{tikzcd}
\overline{\{\eta_0 \}} =  E_0 \arrow[r] \arrow[d] & X_{\mdls, d} \arrow[d, "\pi"] \\
Z_0 \arrow[r, "c.i"]           & C^{(d)}          
\end{tikzcd}\] 

Incoporating $U^{(d)}$, the local system $\cF^{(d)}$ and the Abel-Jacobi map, we have:
\[
\begin{tikzcd}
    & & \cF^{(d)} \arrow[d, purple]\\
\overline{\{\eta_0 \}} =  E_0 \arrow[r] \arrow[d] & 
X_{\mdls, d} \arrow[d, "\pi"]  & \arrow[l, hook'] \arrow[dl, hook'] U^{(d)} \arrow[r, "\Phi_d"] & \PicCm[d] \\
Z_0 \arrow[r, "c.i"]           & C^{(d)}          
\end{tikzcd}
\] 

In Section 3 of \cite{takeuchi2019blow} Takeuchi constructs, for $d$ satisfying \Cref{equation:degree-condition} a compactification denoted by $\Cmod{d}$ and proves the following:

\begin{theorem}[Takeuchi]\label{theorem:takeuchi-compactification-theorem}
    The scheme $\Cmod{d}$ is an open subscheme of $X_{\mdls,d}$ containing $U^{(d)}$. 
    The morphism $\Phi_d: U^{(d)} \to \PicCm[d]$ extends to a morphism $\tilde{\Phi}_d: \Cmod{d} \to \PicCm[d]$ which makes 
    $\Cmod{d}$ a projective space bundle over $\PicCm[d]$. 
    Furthermore, the complement of $U^{(d)}$ in $\Cmod{d}$ is isomorphic to the fiber product 
    $E_0 \times_{C^{(d)}} \Cmod{d}$.
\end{theorem}

%\begin{proof}
     
    %\textcolor{red}{Add outline of construction and proofs}
%\end{proof}

Diagrammatically we have:
\[
\begin{tikzcd}
    &  & \cF^{(d)} \arrow[d, purple]\\
    E_0 \times_{C^{(d)}} \Cmod{d} \arrow[d] \arrow[r]  & \Cmod{d} \arrow[d, hook] \arrow[rd, "\tilde{\Phi}_d"]  & U^{(d)} \arrow[l, hook'] \arrow[d, "\Phi_d"] \\
\overline{\{\eta_0 \}} =  E_0 \arrow[r] \arrow[d] & 
X_{\mdls, d} \arrow[d, "\pi"]  &  \PicCm[d] &  \\
Z_0 \arrow[r, "c.i"]           & C^{(d)}          
\end{tikzcd}
\] 

Also note that $ E_0 \times_{C^{(d)}} \Cmod{d} = Z_0 \times_{C^{(d)}} \Cmod{d}$

\subsection{Proof of \Cref{theorem:SymmetricPowerOfSheafIsTamelyRamified}}
In what follows we reduce \Cref{theorem:SymmetricPowerOfSheafIsTamelyRamified} to \Cref{theorem:torsors_after_blowup_is_tame}
\subsubsection{Reduction Lemmas}

The following lemma is adapted from \cite{takeuchi2019blow} (Lemma 4.1)
\begin{lemma}\label{lemma:takeuchi_lemma}
     Let $C$ be a projective, smooth, and geometrically connected curve over a perfect field $k$. 
     Let $\mathfrak{m} = \sum_{i=1}^{r} k_i P_i$ be an effective divisor where $P_1, \dots, P_r$ are distinct closed points. 
     Let $U = C \setminus \mathfrak{m}$ and let $d \geq \deg \mathfrak{m}$.

     Suppose $\mathfrak{n}_1, \dots, \mathfrak{n}_l$ are pairwise coprime submoduli of $\mathfrak{m}$ such that $\mathfrak{m} = \sum_{j=1}^l \mathfrak{n}_j$. 
     Consider the summation morphism:
     \[
     \pi : C^{(\deg \mathfrak{n}_1)} \times_k \dots \times_k C^{(\deg \mathfrak{n}_l)} \times_k C^{(d - \deg \mathfrak{m})} \longrightarrow C^{(d)}
     \]
     defined by $(D_1, \dots, D_l, D_{extra}) \mapsto \sum_{j=1}^l D_j + D_{extra}$.

     Then $\pi$ is étale at the generic point of the closed subvariety 
     \[
     V = \{\mathfrak{n}_1\} \times_k \dots \times_k \{\mathfrak{n}_l\} \times_k C^{(d - \deg \mathfrak{m})}
     \]
     inside the domain $C^{(\deg \mathfrak{n}_1)} \times_k \dots \times_k C^{(\deg \mathfrak{n}_l)} \times_k C^{(d - \deg \mathfrak{m})}$.

\begin{proof}
     We may assume that $k$ is algebraically closed (hence $\deg P_i = 1$ for all $i$). By miracle flatness $\pi$ is flat, it is quasi-finite and projective as a map between projecive spaces.
     so we condlude $\pi$ is finite and flat. 
     It is enough to show that there exists a closed point $Q$ of $\ndls_1 + \dots \ndls_l + C^{(d - \deg \mathfrak{m})} \subset C^{(d)}$ 
     over which there are $\deg \pi$ points on $C^{(\ndls_1)} \times_k \dots \times_k C^{(\ndls_l)} \times_k C^{(d - \deg \mathfrak{m})}$. (Because it will be unramified 
     at this point and thus also at the generic point of $V$.)
     Choose $Q$ as a point corresponding to a divisor
     $\ndls_1 + \dots + \ndls_l + P_{r+1} + \dots + P_{r+d - \deg \mathfrak{m}}$, where $P_1, \dots, P_{r+d - \deg \mathfrak{m}}$ are distinct points of $U(k)$.
\end{proof}     
\end{lemma}

\begin{cor}\label{cor:etale_at_generic_point}
     The morphism $C^{(\deg \mdls)} \times_k C^{(d - \deg \mdls)} \xrightarrow[]{\pi} C^{(d)}$
     is finite flat everywhere, and étale at the generic point of the closed subvariety
     $Z_\mdls \times_k C^{(d - \deg \mdls)} \subset C^{(\deg \mdls)} \times_k C^{(d - \deg \mdls)}$.     
\end{cor}

Following from this, we look at the following diagram, coming from the flat base change

$C^{(d - \deg \mdls)} \to \Spec(k)$ (\Cref{prop:symmetric_power_flat_base_change}) of \Cref{equation:blowup_diagram_ndls}:
%(\textcolor{red}{Is this even smooth? })

\[\begin{tikzcd}
E_{\mdls} \times_k C^{(d-\deg \mdls)} \arrow[r] \arrow[d] & X_{\mdls} \times_k C^{(d-\deg \mdls)} \arrow[d] & \cF^{(\deg \mdls)} \times_k C^{(d-\deg \mdls)} \arrow[d, color=purple] \\
Z_\mdls \times_k C^{(d-\deg \mdls)} \arrow[r] & C^{(\deg \mdls)} \times_k C^{(d-\deg \mdls)} & U^{(\deg \mdls)} \times_k C^{(d-\deg \mdls)} \arrow[lu]
\end{tikzcd}\]

Note that $U^{(\deg \mdls)} \times_k C^{(d-\deg \mdls)}$ is dense open subscheme of $X_{\mdls} \times_k C^{(d-\deg \mdls)}$,
And $E_{\mdls} \times_k C^{(d-\deg \mdls)}$ is a prime divisor of $X_{\mdls} \times_k C^{(d-\deg \mdls)}$.
Hence it is well defined question according to \Cref{definition:tamely_ramified_in_codim_1} to ask whether
$\cF^{(\deg \mdls)} \times_k C^{(d-\deg \mdls)}$ is tamely ramified at the generic point $\theta$ of
$E_{\mdls} \times_k C^{(d-\deg \mdls)}$.

\begin{lemma}\label{lemma:tame_ramification_persists}
     If $\cF^{(\deg \mdls)}$ is tamely ramified at $\eta_\mdls$, then
     $\cF^{(\deg \mdls)} \times_k C^{(d-\deg \mdls)}$ is tamely ramified at the generic point $\theta$ of
     $E_{\mdls} \times_k C^{(d-\deg \mdls)}$.
\end{lemma}
\begin{proof}
    This follows from \stackstag{0EYD}
    %\textcolor{red}{make adjusments to definition and lemma, to only require that some prime divisors are with desired properties.}
\end{proof}


Replacing $C^{(d - \deg \mdls)}$ with the dense open subscheme $U^{(d - \deg \mdls)} \subset C^{(d - \deg \mdls)}$,
we get that the $G$-torsor $p_1^{-1} \cP^{(\deg \mdls)}$ ($\cP$ correspodns to $\cF$ under \Cref{prop:torsor_module_equivalence})
is tamely ramified at $\theta$ the generic point of
$E_{\mdls} \times_k U^{(d-\deg \mdls)} \subset U^{(\deg \mdls)} \times_k U^{(d - \deg \mdls)}$,
where $p_1: U^{(\deg \mdls)} \times_k U^{(d - \deg \mdls)} \to U^{(\deg \mdls)}$ is the projection to the first factor.

Looking at the second projection $p_2:X_{\mdls} \times_k U^{(d - \deg \mdls)} \to U^{(d - \deg \mdls)}$, and the fact that 
$\cP^{(d - \deg \mdls)}$ is \'etale on $U^{(d - \deg \mdls)}$ we get that 
$p_2^{-1} \cP^{(d - \deg \mdls)}= X_{\mdls} \times_k \cP^{(d - \deg \mdls)}$ is \'etale on $X_{\mdls} \times_k U^{(d - \deg \mdls)}$.
Hence, its restriction to $U^{(\deg \mdls)} \times_k U^{(d - \deg \mdls)}$ is unramified at $\theta$.

Thus, by the following lemma, we conclude that $\boxP{\cP^{(\deg \mdls)}}{\cP^{(d - \deg \mdls)}} = p_1^{-1} \cP^{(\deg \mdls)} \wedge^G p_2^{-1} \cP^{(d - \deg \mdls)}$
     is tamely ramified at $\theta$ the generic point of $E_{\mdls} \times_k U^{(d-\deg \mdls)}$.

\begin{lemma}
     Let $X \to \Spec k$ be a scheme over a field $k$, and let $\cP_1, \cP_2$ be two $G$-torsors on $U_{et}$.
     Let $\xi$ be the generic point of a prime divisor $D \subset X$.
     If $\cP_1$ is tamely ramified at $\xi$, and $\cP_2$ is unramified at $\xi$,
     then the contracted product $\cP_1 \wedge^G \cP_2$ is tamely ramified at $\xi$.
\end{lemma}
\begin{proof}
     This follows from \Cref{lemma:bounding_ramification_of_contracted_product}.
\end{proof}

Combining this with \Cref{cor:etale_at_generic_point} we get

\begin{cor}\label{cor:tame-ramification-symmetric-product}
     Let $\mdls$ be a modulus as above, and let $\eta_\mdls$ be the generic point of $E_\mdls$.
     Let $\cP$ be a $G$-torsor on $U_{et}$ with ramification bounded by $\mdls$.
     Assume $\cP^{(\deg \mdls)}$ is tamely ramified at $\eta_\mdls$.
     Then $\boxP{\cP^{(\deg \mdls)}}{\cP^{(d - \deg \mdls)}}$ is tamely ramified at the generic point $\theta$ of 
     $E_{\mdls} \times_k C^{(d-\deg \mdls)} \subset C^{(\deg \mdls)} \times_k C^{(d - \deg \mdls)}$,
     and $\cP^{(d)}$ is tamely ramified at the generic point $\eta_0=\eta_{\mdls,d}$ of $E_0=E_{\mdls,d}$
\end{cor}
\begin{proof} 
The first assertion, that $\mathcal{P}^{(\deg \mathfrak{m})} \boxtimes \mathcal{P}^{(d - \deg \mathfrak{m})}$ is tamely ramified at $\theta$, follows from the preceding discussion. Thus, it remains to show that $\mathcal{P}^{(d)}$ is tamely ramified at $\eta_0$.

Consider the blowup diagram defining $X_{\mathfrak{m}, d}$:
\[
\begin{tikzcd}
\overline{\{\eta_0 \}} =  E_0 \arrow[r] \arrow[d] & X_{\mdls, d} \arrow[d, "\pi"] \\
Z_0 \arrow[r, "c.i"]           & C^{(d)}          
\end{tikzcd}
\] 
By performing a base change along the flat addition map $+: C^{(\deg \mathfrak{m})} \times_k U^{(d-\deg \mathfrak{m})} \to C^{(d)}$, we obtain the following commutative diagram:

\[
\begin{tikzcd}
&  \left(C^{(\deg \mdls)} \times_k U^{(d-\deg \mdls)} \right) \times_{C^{(d)}} E_0 \arrow[r] \arrow[d] & \overline{\{\eta_0 \}} =  E_0 \arrow[d] \\
& \left(C^{(\deg \mdls)} \times_k U^{(d-\deg \mdls)} \right) \times_{C^{(d)}} X_{\mdls, d} \arrow[d] \arrow[r] & X_{\mdls, d} \arrow[d, "\pi"] \\
\mdls \times_k U^{(d-\deg \mdls)} \arrow[r, "c.i"] & C^{(\deg \mdls)} \times_k U^{(d-\deg \mdls)} \arrow[r, "+"]  & C^{(d)}          
\end{tikzcd}
\] 
Since blowups commute with flat base change, and the inverse image of the center $Z_0$ under the map $+$ is $\mathfrak{m} \times_k U^{(d-\deg \mathfrak{m})}$, the scheme $\left(C^{(\deg \mathfrak{m})} \times_k U^{(d-\deg \mathfrak{m})} \right) \times_{C^{(d)}} X_{\mathfrak{m}, d}$ is the blowup of $C^{(\deg \mathfrak{m})} \times_k U^{(d-\deg \mathfrak{m})}$ along $\mathfrak{m} \times_k U^{(d-\deg \mathfrak{m})}$. 
This, in turn, is isomorphic to the base change of the blowup $X_{\mathfrak{m}}$ (of $C^{(\deg \mathfrak{m})}$ along $\mathfrak{m}$) via the (flat) projection $C^{(\deg \mathfrak{m})} \times_k U^{(d - \deg \mathfrak{m})} \to C^{(\deg \mathfrak{m})}$.

Assembling these facts, we obtain the following Cartesian square:
\[
\begin{tikzcd}
&  E_{\mdls}\times U^{(d - \deg \mdls)}  \arrow[r, "\tilde{+}"] \arrow[d] & \overline{\{\eta_0 \}} =  E_0 \arrow[d] \\
& X_{\mdls} \times_k U^{(d - \deg \mdls)} \arrow[d] \arrow[r, "\tilde{+}"] & X_{\mdls, d} \arrow[d, "\pi"] \\
\mdls \times_k U^{(d-\deg \mdls)} \arrow[r, "c.i"] & C^{(\deg \mdls)} \times_k U^{(d-\deg \mdls)} \arrow[r, "+"]  & C^{(d)}          
\end{tikzcd}
\] 

The point $\theta$ defined in the Corollary is the generic point of $E_{\mathfrak{m}} \times_k U^{(d-\deg \mathfrak{m})}$.
 Let $\eta$ be the generic point of $\mathfrak{m} \times_k U^{(d-\deg \mathfrak{m})}$. 
 By \Cref{cor:etale_at_generic_point}, the map $+$ is étale at $\eta$. 
Consequently, the lifted map $\tilde{+}$ is étale at $\theta$. 
Given the isomorphism $(+^{-1})(\cP^{(d)}) \cong \cP ^{(\deg \mdls)} \boxtimes \cP ^{(d - \deg \mdls)}$ from \Cref{prop:symmetric_powers_on_curves},
the tame ramification of the box product at $\theta$ descends to the tame ramification of $\mathcal{P}^{(d)}$ at $\eta_0$ by applying \Cref{lemma:descend_ramification_along_etale}.

\end{proof}

% \begin{lemma}
%      Let $\ndls_1, \ndls_2 \subset \mdls$ be two moduli of the form $\ndls_1 = k_1 P_1$, $\ndls_2 = k_2 P_2$ where $P_1, P_2$ are distinct points.
%      Assume $\cF ^{(\deg \ndls_1)}$, $\cF ^{(\deg \ndls_2)}$ are at most tamely ramified at $\eta_{\ndls_1}$, $\eta_{\ndls_2}$ respectively.
%      Then $\cF^{(\deg \ndls_1 + \deg \ndls_2)}$ is at most tamely tamified at $\eta_{\ndls_1 + \ndls_2}$.
% \end{lemma}
% \begin{proof}
%      \textcolor{red}{Complete}
% \end{proof}

\begin{lemma}\label{lemma:moduli_reduction}
     Let $\ndls_1, \ndls_2 \subset \mdls$ be two coprime sub moduli of $\mdls$.
     Assume $\cP ^{(\deg \ndls_1)}$, $\cP ^{(\deg \ndls_2)}$ are at most tamely ramified at $\eta_{\ndls_1}$, $\eta_{\ndls_2}$ respectively.
     Then $\cP^{(\deg \ndls_1 + \deg \ndls_2)}$ is at most tamely tamified at $\eta_{\ndls_1 + \ndls_2}$.
\end{lemma}

\begin{proof}
     It follows from \Cref{prop:ramification-external-product} and \Cref{prop:symmetric_powers_on_curves}
\end{proof}

\subsection{\texorpdfstring{Proof of \Cref{theorem:SymmetricPowerOfSheafIsTamelyRamified}}{Proof of Theorem X}}
\begin{proof}[Proof of \Cref{theorem:SymmetricPowerOfSheafIsTamelyRamified}]
     Let $\cF$ be as in \Cref{theorem:SymmetricPowerOfSheafIsTamelyRamified}, $\mdls = \sum_{i=1}^n k_P P$ with $\deg P = d_P$  
     Then by \Cref{theorem:SymmetricPowerOfSheavesIsTamelyRamifiedReduction} for every $\ndls \subset \mdls$ of the form $\ndls = k_P P$, $\cF^{(\deg \ndls)}$ is at most tamely ramified at $\eta_\ndls$.
     By \Cref{lemma:moduli_reduction}, $\cF^{(\deg \mdls)}$ is then at most tamely ramified at $\eta_\mdls$. And thus by \Cref{cor:tame-ramification-symmetric-product}
     $\cF^{(d)}$ is tamely ramified at the generic point $\eta_0$  of $E_0$
\end{proof}






\section{Ramification of Sheaves after Blowup}
The main theorem of this section is
\begin{restatable}{theorem}{SymmetricPowerOfSheafIsTamelyRamified}\label{theorem:SymmetricPowerOfSheafIsTamelyRamified}
     Let $\Lambda$ be a finite ring of cardinality invertible in $k$, and let $\cF$ be an \'etale sheaf of $\Lambda$-modules, locally free of rank 1 on $U$, with ramification bounded by $\mdls$. 
     Considering $U^{(d)}$ as an open subscheme of the blowup $\tilde{C}^{(d)}_\mdls$ of $C^{(d)}$, we have that
     for sufficiently large integer $d$, $\cF^{(d)}$ is tamely ramified on $H= \tilde{C}^{(d)}_\mdls \setminus U^{(d)}=Z_0 \times_{C^{(d)}} \tilde{C}^{(d)}_\mdls $.
\end{restatable}

Recall the defintions from \Cref{section:BlowupOfSymmetricPowerOfCurves},
$\ndls \subset \mdls$ a submodulus, $Z_\ndls$ is the closed subscheme of 
$C^{(\deg \ndls)}$ defined by $\ndls$ as a point.
$X_\ndls$ as the blowup of $C^{(\deg \ndls)}$ at 
$Z_\ndls$, and $E_\ndls = Z_\ndls \times_{C^{(d)}} X_\ndls$ is exceptional divisor of this blowup with 
$\eta_\ndls$ its generic point.

\Cref{theorem:SymmetricPowerOfSheafIsTamelyRamified} easily follows from: 
\begin{theorem}
     Let $\cF$ be a local system on $U$ with ramification at $P$ bounded by $\ndls = k_P P \subset \mdls$.
     Then $\cF^{(\deg \ndls)}$ is tamely ramified on $E_\ndls$
\end{theorem}

\begin{proof}[Proof of \Cref{theorem:SymmetricPowerOfSheafIsTamelyRamified}]
     Let $\cF$ be as in \Cref{theorem:SymmetricPowerOfSheafIsTamelyRamified}, $\mdls = \sum_{i=1}^n k_P P$ with $\deg P = d_P$  
     Then for every $\ndls \subset \mdls$ of the form $\ndls = k_P P$, $\cF^{(d)}$ has ramification 
\end{proof}

We do a reduction theorem \ref{theorem:SymmetricPowerOfSheafIsTamelyRamified} to the case 
$\mdls=dP$. 

Throughout this section Let $C \to S$ be a smooth morphism of schemes of relative dimension 1, with connected geometric fibers of genus $g$, which is
\textcolor{red}{OR}

Let $C$ be a projective smooth geometrically connected curve over a perfect field $k$. Let $\mathfrak{m}$ be a modulus on $C$ and write $\mathfrak{m} = n_1 P_1 + \dots + n_r P_r$, where $P_1, \dots, P_r$ are distinct closed points of $\mathfrak{m}$. Denote the complement of $\mathfrak{m}$ in $C$ by $U$. Let $d_i := \deg P_i$. Take a positive integer $d$ so that $d \ge \deg \mathfrak{m}$.

Zariski-locally projective over $S$.

The reduction is followed by 3 lemmas:
\begin{lemma}
The morphism $\pi : C^{(n_1 d_1)} \times_k \dots \times_k C^{(n_r d_r)} \times_k C^{(d - \deg \mathfrak{m})} \to C^{(d)}$, taking the sum, is étale at the generic point of the closed subvariety $\{n_1 P_1\} \times \dots \times \{n_r P_r\} \times C^{(d - \deg \mathfrak{m})}$ of $C^{(n_1 d_1)} \times_k \dots \times_k C^{(n_r d_r)} \times_k C^{(d - \deg \mathfrak{m})}$.
\end{lemma}

\begin{proof}
We may assume that $k$ is algebraically closed (hence $d_i = 1$ for all $i$). Since the map $\pi : C^{(n_1)} \times_k \dots \times_k C^{(n_r)} \times_k C^{(d - \deg \mathfrak{m})} \to C^{(d)}$ is finite flat, it is enough to show that there exists a closed point $Q$ of $n_1 P_1 + \dots n_r P_r + C^{(d - \deg \mathfrak{m})}$ over which there are $\deg \pi$ points on $C^{(n_1)} \times_k \dots \times_k C^{(n_r)} \times_k C^{(d - \deg \mathfrak{m})}$. Choose $Q$ as a point corresponding to a divisor
$n_1 P_1 + \dots n_r P_r + P_{r+1} + \dots + P_{r+d - \deg \mathfrak{m}}$, where $P_1, \dots, P_{r+d - \deg \mathfrak{m}}$ are distinct points of $U(k)$.
\end{proof}


\begin{lemma} Denote $\mdls_1=n_1P_1$ and $\mdls_2=n_2P_2 + ... + n_rP_r$. Let $X_{\mdls_1}, X_{\mdls_2}$ be the blowups of $C^{(\deg \mdls_1)}, C^{(\deg \mdls_2)}$ by $\mdls_1$, $\mdls_2$ respectively.  
Let $E_1, E_2$ be the respective exceptional divisors (\textcolor{red}{which are irreducibele of codim 1}), and let $\eta_1$, $\eta_2$ be their generic points respectively.
Assume $\cF^{(\deg \mdls_1)}$, $\cF^{(\deg \mdls_2)}$ are tamely ramified (\textcolor{red}{is bounded ramification here good enough?}) on $\eta_1, \eta_2$  respectively.
Then $\cF^{(\deg \mdls)}$ is tamely ramified (\textcolor{red}{or any bounded ramification?}) at $\eta$ - the generic point of the exceiptional divisor of the blowup $X_\mdls$ of $C^{(\deg \mdls)}$ by $\mdls$
\end{lemma}

\begin{lemma}
    In the notations of the previous lemma, suppose $\cF^{(\deg \mdls)}$ is tamely ramified at $\eta$.
    Then  $\Fbox{\deg \mdls}{n - \deg \mdls}$ is tamely ramified at the generic point $\theta$ of $E \times_k C^{(n-\deg \mdls)}$
\end{lemma}

Combining the above we get:
\begin{prop}
    If for every $1 \leq i \leq r$, $\cF^{(d_i n_i)}$ is tamely ramified at $\eta_i$ then $\cF^{(n)}$ is tamely ramified at $\theta$
    \textcolor{red}{Make that precise and everything. notation wise etc...}
\end{prop}

\textcolor{red}{
\begin{enumerate}
    \item The first lemma copied from takeuchi, should we explain its proof? give another proof? exclude its proof and refer?
    \item Which of the above definition of $C$ are we going to use? (over $k$ or $s$)
    \item In the second lemmas, add/explain why are the exceptioanl divisors are irreudcible of codmin 1
    \item Say what is E - the exceiptional divisor of the blowup.
\end{enumerate}
}




\section{\texorpdfstring{Proof of \Cref{thm:GCFT_reduced}}{Proof of theorem 2}}
Let $\mdls$ be an effective Cartier divisor on $C$, and let $d$ be a positive integer satisfying $d \geq max \{2g-1 + \deg \mdls, \deg \mdls\}$ where $g$ is the genus of $C$. 
We denote by $C^{(d)}$ the $d$-th symmetric power of $C$ over $k$.
By \textcolor{red}{add reference}, the fibers of the map \textcolor{red}{for d large enough? under what conditions?}
$$ \Phi_d : U^{(d)} \to \PicCm[d] $$ over any point are isomorphic to
$$
\begin{cases}
\mathbb{A}_k^{d - \deg \mdls - g + 1} & \text{if } m > 0 \\
\mathbb{P}_k^{d - g} & \text{if } m = 0
\end{cases}
$$


%More accurately, for a $k$-point of $\PicCm[d]$ corresponding to a pair $(\cL, \alpha)$ where $\cL$ is an invertible $\cO_C$-module of degree $d$ and $\alpha$ is a rigidification of $\cL$ along $\mdls$, the fiber of $\Phi_d$ over this point is isomorphic to the projective space of global sections of $\cL$ which restrict to $\alpha$ along $\mdls$:
%$$ \mathbb{P}(\Gamma_\mdls(C, \cL)) = \mathbb{P}(\{ s \in \Gamma(C, \cL) \mid s|_\mdls = \alpha \}) $$

One can show this makes $\Phi_d$ into a fibration in affine spaces or projective spaces, depending on whether $\mdls$ is non-zero or zero. 

Using a fundamental theorem about etale fundamental groups:
\begin{theorem}[Homotopy Exact Sequence \stackstag{0C0J}]
Let $f : X \to S$ be a flat proper morphism of finite presentation whose geometric fibres are connected and reduced. Assume $S$ is connected and let $\overline{s}$ be a geometric point of $S$. Then there is an exact sequence

\[ \pi _1(X_{\overline{s}}) \to \pi _1(X) \to \pi _1(S) \to 1 \]
of fundamental groups.
\end{theorem}

We get, in the case when $\mdls = 0$ (Hence $\Phi_d$ is smooth and proper) that there is an isomorphism of fundamental groups:
$$ \pi_1^{et}(C^{(d)}) \cong \pi_1^{et}(\PicC[d]) $$
Implying \cref{thm:GCFT_reduced} \textcolor{red}{explain exactly how, becuase here it is fundamental groups and there it is etale sheafves of $\Lambda$-modules}. Also, we need it to be \textbf{multiplicative}!

In the case when $\mdls > 0$, $\Phi_d$ is not proper anymore, so we cannot apply the homotopy exact sequence directly.
In \cite{takeuchi2019blow} Takeuchi constructs a compactifiaction of $U^{(d)}$ by adding a boundary hyperplane, $H$, over which $\cF^{(d)}$ is tamely ramified.
This compactification is denoted by $\Cmod{d}$, it has $U^{(d)}$ as an open subscheme with complement $H$. 

\textbf{This is a theorem we are going to prove in a different way:}
\begin{theorem}\label{theorem:SymmetricPowerOfSheafIsTamelyRamified}
     Let $\Lambda$ be a finite ring of cardinality invertible in $k$, and let $\cF$ be an \'etale sheaf of $\Lambda$-modules, locally free of rank 1 on $U$, with ramification bounded by $\mdls$. 
    Then, for sufficiently large integer $d$, $\cF^{(d)}$ is tamely ramified on $H$.
\end{theorem}
\textcolor{red}{Note that the above terminology of tamely ramified on $H$ is defined exactly when H is the complement of $U^{(d)}$, so I want to say something about it, or make it clear, or somehow define it someplace else and refer to it }

More over, this compactification, denoted by $\tilde{C}^{(d)}_\mdls$ is fibered over $\PicCm[d]$ with fibers isomorphic to projective spaces. Thus we get
$$ \pi_1(\tilde{C}^{(d)}_\mdls) \cong \pi_1(\PicCm[d])$$

From here, one can go in two routes, 
\textbf{Route 1} - Showing $ \ker \{\pi^{ab}_1(U^{(d)}) \to \pi^{ab}_1(\PicCm[d]) \} $ is pro-$p$ group. 

\textbf{Route 2} - Showing $$\pi^{t, ab}_1(U^{(d)}) \to \pi^{t,ab}_1(\PicCm[d])$$ is isomorphism to its image.  \textcolor{red}{here one needs to be precise.}

In both cases, one would then get \cref{thm:GCFT_reduced}.

We start by defintions and basic proposition of everything we need. 


\printbibliography


\end{document}


